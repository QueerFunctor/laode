\documentclass{ximera}
 

\usepackage{epsfig}

\graphicspath{
  {./}
  {figures/}
}

\usepackage{morewrites}
\makeatletter
\newcommand\subfile[1]{%
\renewcommand{\input}[1]{}%
\begingroup\skip@preamble\otherinput{#1}\endgroup\par\vspace{\topsep}
\let\input\otherinput}
\makeatother

\newcommand{\includeexercises}{\directlua{dofile("/home/jim/linearAlgebra/laode/exercises.lua")}}

%\newcounter{ccounter}
%\setcounter{ccounter}{1}
%\newcommand{\Chapter}[1]{\setcounter{chapter}{\arabic{ccounter}}\chapter{#1}\addtocounter{ccounter}{1}}

%\newcommand{\section}[1]{\section{#1}\setcounter{thm}{0}\setcounter{equation}{0}}

%\renewcommand{\theequation}{\arabic{chapter}.\arabic{section}.\arabic{equation}}
%\renewcommand{\thefigure}{\arabic{chapter}.\arabic{figure}}
%\renewcommand{\thetable}{\arabic{chapter}.\arabic{table}}

%\newcommand{\Sec}[2]{\section{#1}\markright{\arabic{ccounter}.\arabic{section}.#2}\setcounter{equation}{0}\setcounter{thm}{0}\setcounter{figure}{0}}

\newcommand{\Sec}[2]{\section{#1}}

\setcounter{secnumdepth}{2}
%\setcounter{secnumdepth}{1} 

%\newcounter{THM}
%\renewcommand{\theTHM}{\arabic{chapter}.\arabic{section}}

\newcommand{\trademark}{{R\!\!\!\!\!\bigcirc}}
%\newtheorem{exercise}{}

\newcommand{\dfield}{{\sf dfield9}}
\newcommand{\pplane}{{\sf pplane9}}

\newcommand{\EXER}{\section*{Exercises}}%\vspace*{0.2in}\hrule\small\setcounter{exercise}{0}}
\newcommand{\CEXER}{}%\vspace{0.08in}\begin{center}Computer Exercises\end{center}}
\newcommand{\TEXER}{} %\vspace{0.08in}\begin{center}Hand Exercises\end{center}}
\newcommand{\AEXER}{} %\vspace{0.08in}\begin{center}Hand Exercises\end{center}}

% BADBAD: \newcommand{\Bbb}{\bf}

\newcommand{\R}{\mbox{$\Bbb{R}$}}
\newcommand{\C}{\mbox{$\Bbb{C}$}}
\newcommand{\Z}{\mbox{$\Bbb{Z}$}}
\newcommand{\N}{\mbox{$\Bbb{N}$}}
\newcommand{\D}{\mbox{{\bf D}}}
\usepackage{amssymb}
%\newcommand{\qed}{\hfill\mbox{\raggedright$\square$} \vspace{1ex}}
%\newcommand{\proof}{\noindent {\bf Proof:} \hspace{0.1in}}

\newcommand{\setmin}{\;\mbox{--}\;}
\newcommand{\Matlab}{{M\small{AT\-LAB}} }
\newcommand{\Matlabp}{{M\small{AT\-LAB}}}
\newcommand{\computer}{\Matlab Instructions}
\newcommand{\half}{\mbox{$\frac{1}{2}$}}
\newcommand{\compose}{\raisebox{.15ex}{\mbox{{\scriptsize$\circ$}}}}
\newcommand{\AND}{\quad\mbox{and}\quad}
\newcommand{\vect}[2]{\left(\begin{array}{c} #1_1 \\ \vdots \\
 #1_{#2}\end{array}\right)}
\newcommand{\mattwo}[4]{\left(\begin{array}{rr} #1 & #2\\ #3
&#4\end{array}\right)}
\newcommand{\mattwoc}[4]{\left(\begin{array}{cc} #1 & #2\\ #3
&#4\end{array}\right)}
\newcommand{\vectwo}[2]{\left(\begin{array}{r} #1 \\ #2\end{array}\right)}
\newcommand{\vectwoc}[2]{\left(\begin{array}{c} #1 \\ #2\end{array}\right)}

\newcommand{\ignore}[1]{}


\newcommand{\inv}{^{-1}}
\newcommand{\CC}{{\cal C}}
\newcommand{\CCone}{\CC^1}
\newcommand{\Span}{{\rm span}}
\newcommand{\rank}{{\rm rank}}
\newcommand{\trace}{{\rm tr}}
\newcommand{\RE}{{\rm Re}}
\newcommand{\IM}{{\rm Im}}
\newcommand{\nulls}{{\rm null\;space}}

\newcommand{\dps}{\displaystyle}
\newcommand{\arraystart}{\renewcommand{\arraystretch}{1.8}}
\newcommand{\arrayfinish}{\renewcommand{\arraystretch}{1.2}}
\newcommand{\Start}[1]{\vspace{0.08in}\noindent {\bf Section~\ref{#1}}}
\newcommand{\exer}[1]{\noindent {\bf \ref{#1}}}
\newcommand{\ans}{}
\newcommand{\matthree}[9]{\left(\begin{array}{rrr} #1 & #2 & #3 \\ #4 & #5 & #6
\\ #7 & #8 & #9\end{array}\right)}
\newcommand{\cvectwo}[2]{\left(\begin{array}{c} #1 \\ #2\end{array}\right)}
\newcommand{\cmatthree}[9]{\left(\begin{array}{ccc} #1 & #2 & #3 \\ #4 & #5 &
#6 \\ #7 & #8 & #9\end{array}\right)}
\newcommand{\vecthree}[3]{\left(\begin{array}{r} #1 \\ #2 \\
#3\end{array}\right)}
\newcommand{\cvecthree}[3]{\left(\begin{array}{c} #1 \\ #2 \\
#3\end{array}\right)}
\newcommand{\cmattwo}[4]{\left(\begin{array}{cc} #1 & #2\\ #3
&#4\end{array}\right)}

\newcommand{\Matrix}[1]{\ensuremath{\left(\begin{array}{rrrrrrrrrrrrrrrrrr} #1 \end{array}\right)}}

\newcommand{\Matrixc}[1]{\ensuremath{\left(\begin{array}{cccccccccccc} #1 \end{array}\right)}}



\renewcommand{\labelenumi}{\theenumi)}
\newenvironment{enumeratea}%
{\begingroup
 \renewcommand{\theenumi}{\alph{enumi}}
 \renewcommand{\labelenumi}{(\theenumi)}
 \begin{enumerate}}
 {\end{enumerate}\endgroup}



\newcounter{help}
\renewcommand{\thehelp}{\thesection.\arabic{equation}}

%\newenvironment{equation*}%
%{\renewcommand\endequation{\eqno (\theequation)* $$}%
%   \begin{equation}}%
%   {\end{equation}\renewcommand\endequation{\eqno \@eqnnum
%$$\global\@ignoretrue}}

%\input{psfig.tex}

\author{Martin Golubitsky and Michael Dellnitz}

%\newenvironment{matlabEquation}%
%{\renewcommand\endequation{\eqno (\theequation*) $$}%
%   \begin{equation}}%
%   {\end{equation}\renewcommand\endequation{\eqno \@eqnnum
% $$\global\@ignoretrue}}

\newcommand{\soln}{\textbf{Solution:} }
\newcommand{\exercap}[1]{\centerline{Figure~\ref{#1}}}
\newcommand{\exercaptwo}[1]{\centerline{Figure~\ref{#1}a\hspace{2.1in}
Figure~\ref{#1}b}}
\newcommand{\exercapthree}[1]{\centerline{Figure~\ref{#1}a\hspace{1.2in}
Figure~\ref{#1}b\hspace{1.2in}Figure~\ref{#1}c}}
\newcommand{\para}{\hspace{0.4in}}

\renewenvironment{solution}{\suppress}{\endsuppress}

\ifxake
\newenvironment{matlabEquation}{\begin{equation}}{\end{equation}}
\else
\newenvironment{matlabEquation}%
{\let\oldtheequation\theequation\renewcommand{\theequation}{\oldtheequation*}\begin{equation}}%
  {\end{equation}\let\theequation\oldtheequation}
\fi

\makeatother

\begin{document}

\noindent In Exercises~\ref{c14.2.6a} -- \ref{c14.2.6d} solve the given 
initial value problems by variation of parameters.
\begin{exercise}   \label{c14.2.6a}
$\dps \frac{dx}{dt} = t^2 x + t^2, \quad x(1)=1$.

\begin{solution}
\ans The solution to the initial value problem is
$x(t) = 2e^{\frac{1}{3}(t^3 - 1)} - 1$.

\soln By inspection $x(t) = -1$ is a solution to the inhomogeneous
equation.  The solution to the homogeneous equation $\dot{x}(t) = t^2x$ is
$x(t) = Ke^{H(t)}$, where
\[
H(t) = \int t^2dt = \frac{1}{3}t^3.
\]
So the general solution to the inhomogeneous equation is
\[
x(t) = Ke^{\frac{1}{3}t^3} - 1.
\]
Substitute the initial condition $x(1) = 1$ into this equation and
solve for $K = 2e^{-\frac{1}{3}}$.


\end{solution}
\end{exercise}
\begin{exercise}   \label{c14.2.6b}
$\dps \frac{dx}{dt} = x+2t, \quad x(0)=-1$.

\begin{solution}
\ans The solution to the initial value problem is
$x(t) = e^t - 2t - 2$.

\soln The solution to the homogeneous equation $\dot{x}=x$ is $x(t)=ce^t$
for some constant $c$.  The method of variation of parameters looks for 
solutions to the inhomogeneous equation of the form $x(t)=c(t)e^t$.  If this 
$x(t)$ is a solution to the inhomogeneous equation $\dot{x}=x+2t$, then
\[
x+2t=\dot{x}=\dot{c}e^t+ce^t = \dot{c}e^t+x.
\]
Therefore,
\[
\dot{c} = 2te^{-t}.
\]
Integrating by parts yields
\[
c(t) = 2\int te^{-t}dt = -2(t+1)e^{-t} + K.
\]
Thus the general solution to the inhomogeneous equation is:
\[
x(t) = -2(t+1) + Ke^t.
\]
Using the initial condition $x(0)=-1$ yields $K=1$. 

\end{solution}
\end{exercise}
\begin{exercise}   \label{c14.2.6c}
$\dps \frac{dx}{dt} = 
\frac{t}{t^2+1}x+\sin(t)\sqrt{t^2+1},\quad x(0)=2$.

\begin{solution}
\ans The solution to the initial value problem is
$x(t) = (3 - \cos t)\sqrt{t^2 + 1}$.

\soln By Theorem~\ref{thm:varpar}, the
solution to the initial value problem is $x(t) = c(t)e^{H(t)}$, where
$\frac{dx}{dt} = a(t)x + g(t)$.  So
\[
H(t) = \int_{t_0}^t a(\tau)d\tau = \int_0^t\frac{\tau}{\tau^2 + 1}d\tau
= \ln\sqrt{t^2 + 1}.
\]
\[
c(t) = \int_{t_0}^t g(\tau)e^{-H(\tau)}d\tau + x_0
= \int_0^t \sin(\tau)d\tau + 2
= 3 - \cos t.
\]

\end{solution}
\end{exercise}
\begin{exercise}   \label{c14.2.6d}
$\dps \frac{dx}{dt} = 2x+\frac{1}{t}e^{2t}, \quad x(1)=4$.

\begin{solution}
\ans The solution to the initial value problem is
\[
x(t) = e^{2t}(4e^{-2}+\ln t).
\]

\soln The solution to the homogeneous equation $\dot{x}=2x$ is $x(t)=e^{2t}$.
 Using variation of parameters we look for a solution to the inhomogeneous
equation $\dot{x}=2x+\frac{1}{t}e^{2t}$ of the form $x(t)=c(t)e^{2t}$.  If
$x(t)$ is a solution to the inhomogeneous equation, then
\[
2x+\frac{1}{t}e^{2t} =\dot{x}=\dot{c}e^{2t}+2ce^{2t} = \dot{c}e^{2t}+2x.
\]
Therefore, 
\[
\dot{c} = \frac{1}{t},
\]
and 
\[
c(t) = \ln|t| + K.
\]
The general solution to the inhomogeneous equation is:
\[
x(t) = (\ln|t| + K)e^{2t}.
\]
Using the initial condition $x(1)=4$, we find that $K=4e^{-2}$.



\end{solution}
\end{exercise}
\end{document}
