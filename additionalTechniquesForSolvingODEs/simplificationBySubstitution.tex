\documentclass{ximera}

 

\usepackage{epsfig}

\graphicspath{
  {./}
  {figures/}
}

\usepackage{morewrites}
\makeatletter
\newcommand\subfile[1]{%
\renewcommand{\input}[1]{}%
\begingroup\skip@preamble\otherinput{#1}\endgroup\par\vspace{\topsep}
\let\input\otherinput}
\makeatother

\newcommand{\includeexercises}{\directlua{dofile("/home/jim/linearAlgebra/laode/exercises.lua")}}

%\newcounter{ccounter}
%\setcounter{ccounter}{1}
%\newcommand{\Chapter}[1]{\setcounter{chapter}{\arabic{ccounter}}\chapter{#1}\addtocounter{ccounter}{1}}

%\newcommand{\section}[1]{\section{#1}\setcounter{thm}{0}\setcounter{equation}{0}}

%\renewcommand{\theequation}{\arabic{chapter}.\arabic{section}.\arabic{equation}}
%\renewcommand{\thefigure}{\arabic{chapter}.\arabic{figure}}
%\renewcommand{\thetable}{\arabic{chapter}.\arabic{table}}

%\newcommand{\Sec}[2]{\section{#1}\markright{\arabic{ccounter}.\arabic{section}.#2}\setcounter{equation}{0}\setcounter{thm}{0}\setcounter{figure}{0}}

\newcommand{\Sec}[2]{\section{#1}}

\setcounter{secnumdepth}{2}
%\setcounter{secnumdepth}{1} 

%\newcounter{THM}
%\renewcommand{\theTHM}{\arabic{chapter}.\arabic{section}}

\newcommand{\trademark}{{R\!\!\!\!\!\bigcirc}}
%\newtheorem{exercise}{}

\newcommand{\dfield}{{\sf dfield9}}
\newcommand{\pplane}{{\sf pplane9}}

\newcommand{\EXER}{\section*{Exercises}}%\vspace*{0.2in}\hrule\small\setcounter{exercise}{0}}
\newcommand{\CEXER}{}%\vspace{0.08in}\begin{center}Computer Exercises\end{center}}
\newcommand{\TEXER}{} %\vspace{0.08in}\begin{center}Hand Exercises\end{center}}
\newcommand{\AEXER}{} %\vspace{0.08in}\begin{center}Hand Exercises\end{center}}

% BADBAD: \newcommand{\Bbb}{\bf}

\newcommand{\R}{\mbox{$\Bbb{R}$}}
\newcommand{\C}{\mbox{$\Bbb{C}$}}
\newcommand{\Z}{\mbox{$\Bbb{Z}$}}
\newcommand{\N}{\mbox{$\Bbb{N}$}}
\newcommand{\D}{\mbox{{\bf D}}}
\usepackage{amssymb}
%\newcommand{\qed}{\hfill\mbox{\raggedright$\square$} \vspace{1ex}}
%\newcommand{\proof}{\noindent {\bf Proof:} \hspace{0.1in}}

\newcommand{\setmin}{\;\mbox{--}\;}
\newcommand{\Matlab}{{M\small{AT\-LAB}} }
\newcommand{\Matlabp}{{M\small{AT\-LAB}}}
\newcommand{\computer}{\Matlab Instructions}
\newcommand{\half}{\mbox{$\frac{1}{2}$}}
\newcommand{\compose}{\raisebox{.15ex}{\mbox{{\scriptsize$\circ$}}}}
\newcommand{\AND}{\quad\mbox{and}\quad}
\newcommand{\vect}[2]{\left(\begin{array}{c} #1_1 \\ \vdots \\
 #1_{#2}\end{array}\right)}
\newcommand{\mattwo}[4]{\left(\begin{array}{rr} #1 & #2\\ #3
&#4\end{array}\right)}
\newcommand{\mattwoc}[4]{\left(\begin{array}{cc} #1 & #2\\ #3
&#4\end{array}\right)}
\newcommand{\vectwo}[2]{\left(\begin{array}{r} #1 \\ #2\end{array}\right)}
\newcommand{\vectwoc}[2]{\left(\begin{array}{c} #1 \\ #2\end{array}\right)}

\newcommand{\ignore}[1]{}


\newcommand{\inv}{^{-1}}
\newcommand{\CC}{{\cal C}}
\newcommand{\CCone}{\CC^1}
\newcommand{\Span}{{\rm span}}
\newcommand{\rank}{{\rm rank}}
\newcommand{\trace}{{\rm tr}}
\newcommand{\RE}{{\rm Re}}
\newcommand{\IM}{{\rm Im}}
\newcommand{\nulls}{{\rm null\;space}}

\newcommand{\dps}{\displaystyle}
\newcommand{\arraystart}{\renewcommand{\arraystretch}{1.8}}
\newcommand{\arrayfinish}{\renewcommand{\arraystretch}{1.2}}
\newcommand{\Start}[1]{\vspace{0.08in}\noindent {\bf Section~\ref{#1}}}
\newcommand{\exer}[1]{\noindent {\bf \ref{#1}}}
\newcommand{\ans}{}
\newcommand{\matthree}[9]{\left(\begin{array}{rrr} #1 & #2 & #3 \\ #4 & #5 & #6
\\ #7 & #8 & #9\end{array}\right)}
\newcommand{\cvectwo}[2]{\left(\begin{array}{c} #1 \\ #2\end{array}\right)}
\newcommand{\cmatthree}[9]{\left(\begin{array}{ccc} #1 & #2 & #3 \\ #4 & #5 &
#6 \\ #7 & #8 & #9\end{array}\right)}
\newcommand{\vecthree}[3]{\left(\begin{array}{r} #1 \\ #2 \\
#3\end{array}\right)}
\newcommand{\cvecthree}[3]{\left(\begin{array}{c} #1 \\ #2 \\
#3\end{array}\right)}
\newcommand{\cmattwo}[4]{\left(\begin{array}{cc} #1 & #2\\ #3
&#4\end{array}\right)}

\newcommand{\Matrix}[1]{\ensuremath{\left(\begin{array}{rrrrrrrrrrrrrrrrrr} #1 \end{array}\right)}}

\newcommand{\Matrixc}[1]{\ensuremath{\left(\begin{array}{cccccccccccc} #1 \end{array}\right)}}



\renewcommand{\labelenumi}{\theenumi)}
\newenvironment{enumeratea}%
{\begingroup
 \renewcommand{\theenumi}{\alph{enumi}}
 \renewcommand{\labelenumi}{(\theenumi)}
 \begin{enumerate}}
 {\end{enumerate}\endgroup}



\newcounter{help}
\renewcommand{\thehelp}{\thesection.\arabic{equation}}

%\newenvironment{equation*}%
%{\renewcommand\endequation{\eqno (\theequation)* $$}%
%   \begin{equation}}%
%   {\end{equation}\renewcommand\endequation{\eqno \@eqnnum
%$$\global\@ignoretrue}}

%\input{psfig.tex}

\author{Martin Golubitsky and Michael Dellnitz}

%\newenvironment{matlabEquation}%
%{\renewcommand\endequation{\eqno (\theequation*) $$}%
%   \begin{equation}}%
%   {\end{equation}\renewcommand\endequation{\eqno \@eqnnum
% $$\global\@ignoretrue}}

\newcommand{\soln}{\textbf{Solution:} }
\newcommand{\exercap}[1]{\centerline{Figure~\ref{#1}}}
\newcommand{\exercaptwo}[1]{\centerline{Figure~\ref{#1}a\hspace{2.1in}
Figure~\ref{#1}b}}
\newcommand{\exercapthree}[1]{\centerline{Figure~\ref{#1}a\hspace{1.2in}
Figure~\ref{#1}b\hspace{1.2in}Figure~\ref{#1}c}}
\newcommand{\para}{\hspace{0.4in}}

\renewenvironment{solution}{\suppress}{\endsuppress}

\ifxake
\newenvironment{matlabEquation}{\begin{equation}}{\end{equation}}
\else
\newenvironment{matlabEquation}%
{\let\oldtheequation\theequation\renewcommand{\theequation}{\oldtheequation*}\begin{equation}}%
  {\end{equation}\let\theequation\oldtheequation}
\fi

\makeatother


\title{Simplification by Substitution}

\begin{document}
\begin{abstract}
\end{abstract}
\maketitle


\label{sec:SBS}

So far in this chapter --- as well as in Chapter~\ref{C:LDE} ---
we have seen how to find solutions of ordinary differential equations 
that have special forms.  (For instance, separation of variables can 
be applied to solve equations of the form $\frac{dx}{dt}=g(x) h(t)$.) 
However, ``most'' differential equations are not of the form needed to
apply one of these techniques.  But sometimes it is possible to
transform the equation into such a form.  This is accomplished
by substituting a new function for $x(t)$, so that the differential 
equation for this new function has a simpler form.  We illustrate this 
procedure in two cases.

\subsection*{Homogeneous Coefficients}
\index{homogeneous coefficients}

Differential equations of the form
\begin{equation}
\label{eq:homcoeff}
\frac{dx}{dt} = F\left(\frac{x}{t}\right),
\end{equation}
where $F:\R\to\R$ is continuous, are said to have {\em
homogeneous coefficients}\index{homogeneous coefficients}.  
In general, none of the techniques
described so far can be applied to this equation.  

Since the function $x(t)/t$ appears as the argument of $F$ in
the equation it seems plausible to write down \Ref{eq:homcoeff}
in terms of this function.  Having this in mind, define
\[
v(t) = \frac{x(t)}{t}.
\]
Then $x(t) = t v(t)$ and \Ref{eq:homcoeff} becomes
\[
v(t) + t \frac{dv}{dt}(t) = F(v(t)).
\]
When $t\not= 0$, this equation is equivalent to
\begin{equation}
\label{eq:homsep}
\frac{dv}{dt} = \frac{F(v)-v}{t}.
\end{equation}
We have arrived at an equation to which we can apply separation
of variables\index{separation of variables}.  
Once a solution $v(t)$ of \Ref{eq:homsep} is
found, we obtain a solution of \Ref{eq:homcoeff} by setting
\[
x(t) = t v(t).
\]
If an initial condition $x(t_0)=x_0$ is specified in
\Ref{eq:homcoeff}, then we have to solve \Ref{eq:homsep} with
the transformed initial condition $v(t_0)=x_0/t_0$.

\subsubsection*{An Example}
Consider the initial value problem
\begin{equation}
\label{eq:homcoex1}
\begin{array}{rcl}
\dps \frac{dx}{dt} & = & \dps \frac{2t^2+x^2}{tx} \\
x(2) & = & 6.
\end{array}
\end{equation}
{\rm Since the right hand side of this equation can be written as
\[
F\left(\frac{x}{t}\right) = 2\frac{t}{x}+\frac{x}{t},
\]
we see that \Ref{eq:homcoex1} has homogeneous 
coefficients\index{homogeneous coefficients}.  We
set $v(t) = x(t)/t$. By \Ref{eq:homsep} we have to solve the
initial value problem\index{initial value problem}
\[
\begin{array}{rcl}
\dps \frac{dv}{dt} & = & \dps \frac{F(v)-v}{t} = 
\frac{2\frac{1}{v}+v-v}{t}=\frac{2}{tv} \\
v(2) & = & 6/2 = 3.
\end{array}
\]
Separation of variables \index{separation of variables} shows that 
$v(t)$ has to satisfy
\[
\frac{v^2}{2} = \ln(t) - \ln(2) + \frac{9}{2}
\]
(see Section~\ref{sec:sov}). Therefore
\[
v(t) = \sqrt{9+2\ln\frac{t}{2}}.
\]
Finally, we obtain the solution
\[
x(t) = tv(t) = t\sqrt{9+2\ln\frac{t}{2}}.
\]}

\subsection*{Bernoulli's Equation}
\index{Bernoulli's equation}

Let $r,s:\R\rightarrow\R$ be continuous functions, and let $p\in\R$
be a real number.  Then an equation of the form
\begin{equation}
\label{eq:bern1}
\frac{dx}{dt} = r(t)x + s(t)x^p
\end{equation}
is called a {\em Bernoulli equation}.  For $p=0$ or $p=1$ the
equation is linear and we can, in principle, find all the
solutions by variation of parameters (see Chapter~\ref{C:LDE}).  
Hence we assume that
\[
p\not= 0,1.
\]
The idea is to substitute $x(t)$ in such a way that also for these
values of $p$ \Ref{eq:bern1} becomes linear in the new function $v(t)$.  
We try the guess
\[
v(t) = x(t)^{1/\alpha},
\]
where we choose the real constant $\alpha\not= 0$ later to simplify
the transformed equation.  Using the chain rule on $x(t) =
v(t)^\alpha$, we compute
\[
\frac{dx}{dt} = \alpha v^{\alpha-1} \frac{dv}{dt}.
\]
Substitution into \Ref{eq:bern1} yields
\[
\alpha v^{\alpha-1} \frac{dv}{dt} = r(t)v^\alpha + s(t)v^{p\alpha}.
\]
Thus
\[
\frac{dv}{dt}  = \frac{1}{\alpha} r(t) v + 
\frac{1}{\alpha} s(t) v^{p\alpha-\alpha+1}.
\]
To simplify this equation, choose the constant $\alpha$ so that
$p\alpha-\alpha+1=0$; that is, set
\[
\alpha = \frac{1}{1-p}.
\]
Then we arrive at the {\em linear\/} equation
\begin{equation}
\label{eq:bern2}
\frac{dv}{dt} = (1-p) r(t) v + (1-p) s(t),
\end{equation}
which we can, in principle, be solved by 
variation of parameters\index{variation of parameters}.
If this can be done, then we obtain a solution $x(t)$ of
\Ref{eq:bern1} by setting 
\[
x(t) = (v(t))^{\frac{1}{1-p}}.
\]
If an initial condition $x(t_0)=x_0$ is specified in
\Ref{eq:bern1},
then we have to solve \Ref{eq:bern2} with the transformed initial
condition
$v(t_0)=x_0^{1/\alpha}=x_0^{1-p}$.

\subsubsection*{An Example}
Consider the initial value problem
\[
\begin{array}{rcl}
\dps \frac{dx}{dt} & = &  3t^2 x + 3t^2x^{\frac{2}{3}}\\
x(0) & = & 27.
\end{array}
\]
{\rm The differential equation is a Bernoulli equation with
\[
r(t) = 3t^2,\quad s(t)=3t^2, \AND p=\frac{2}{3}.
\]
Hence we have to find a solution of the linear initial value
problem (see \Ref{eq:bern2})
\[
\begin{array}{rcl}
\dps \frac{dv}{dt} & = &  t^2 v + t^2 \\
v(0) & = & 27^{1/3} = 3.
\end{array}
\]
The solution to this differential equation can be obtained by variation of 
parameters and is 
\[
v(t) = -1 +4 e^{t^3/3}.
\]
Therefore, a solution of the Bernoulli equation is given by
\[
x(t) = v(t)^{\frac{1}{1-p}} = \left(-1 +4 e^{t^3/3}\right)^3.
\]}

\EXER

\TEXER

\noindent In Exercises~\ref{c14.5.1} -- \ref{c14.5.5} decide whether the 
given differential equation has homogeneous coefficients or is a Bernoulli 
equation.
\begin{exercise} \label{c14.5.1}
$\dps \frac{dx}{dt} = \frac{x^2}{t^3}$.
\end{exercise}
\begin{exercise} \label{c14.5.2}
$\dps \frac{dx}{dt} = t$.
\end{exercise}
\begin{exercise} \label{c14.5.4}
$\dps \frac{dx}{dt} = \frac{\cos t}{x^4}\sqrt{t}+x^2$.
\end{exercise}
\begin{exercise} \label{c14.5.3}
$\dps \frac{dx}{dt} = \frac{t^2}{\sqrt{x}}+x\sin t$.
\end{exercise}
\begin{exercise} \label{c14.5.5}
$\dps \frac{dx}{dt} = \frac{x}{t} +\frac{\sqrt{t}}{\sqrt{x}}$.
\end{exercise}

\noindent In Exercises~\ref{c14.5.6} -- \ref{c14.5.9} solve the given initial 
value problem by an appropriate solution technique.
\begin{exercise} \label{c14.5.6}
$\dps \frac{dx}{dt} = \sec\left(\frac{x}{t}\right)+\frac{x}{t}$ where
$x(1)=\pi$.
\end{exercise}
\begin{exercise} \label{c14.5.7}
$\dps \frac{dx}{dt} = x+x^2$ where $x(2)=1$.
\end{exercise}
\begin{exercise} \label{c14.5.8}
$\dps \frac{dx}{dt} = -\frac{x}{t}-t^3 x^3$ where $x(1)=1$.
\end{exercise}
\begin{exercise} \label{c14.5.9}
$\dps \frac{dx}{dt} = \frac{x(x+t)}{t^2}$ where $x(1)=1$.
\end{exercise}

\begin{exercise} \label{c14.5.10}
We set $v(t) = x(t)^{1-p}$ to transform the Bernoulli equation 
\Ref{eq:bern1} into the equation
\[
\frac{dv}{dt} = (1-p) r(t) v + (1-p) s(t).
\]
Now set $w(t) = v(\beta t)$ and show that $\beta$ can be chosen so that 
$w(t)$ is a solution to the linear differential equation
\[
\frac{dw}{dt} = r\left(\frac{t}{1-p}\right) w + s\left(\frac{t}{1-p}\right).
\]
\end{exercise}


\end{document}
