\documentclass{ximera}
 

\usepackage{epsfig}

\graphicspath{
  {./}
  {figures/}
}

\usepackage{morewrites}
\makeatletter
\newcommand\subfile[1]{%
\renewcommand{\input}[1]{}%
\begingroup\skip@preamble\otherinput{#1}\endgroup\par\vspace{\topsep}
\let\input\otherinput}
\makeatother

\newcommand{\includeexercises}{\directlua{dofile("/home/jim/linearAlgebra/laode/exercises.lua")}}

%\newcounter{ccounter}
%\setcounter{ccounter}{1}
%\newcommand{\Chapter}[1]{\setcounter{chapter}{\arabic{ccounter}}\chapter{#1}\addtocounter{ccounter}{1}}

%\newcommand{\section}[1]{\section{#1}\setcounter{thm}{0}\setcounter{equation}{0}}

%\renewcommand{\theequation}{\arabic{chapter}.\arabic{section}.\arabic{equation}}
%\renewcommand{\thefigure}{\arabic{chapter}.\arabic{figure}}
%\renewcommand{\thetable}{\arabic{chapter}.\arabic{table}}

%\newcommand{\Sec}[2]{\section{#1}\markright{\arabic{ccounter}.\arabic{section}.#2}\setcounter{equation}{0}\setcounter{thm}{0}\setcounter{figure}{0}}

\newcommand{\Sec}[2]{\section{#1}}

\setcounter{secnumdepth}{2}
%\setcounter{secnumdepth}{1} 

%\newcounter{THM}
%\renewcommand{\theTHM}{\arabic{chapter}.\arabic{section}}

\newcommand{\trademark}{{R\!\!\!\!\!\bigcirc}}
%\newtheorem{exercise}{}

\newcommand{\dfield}{{\sf dfield9}}
\newcommand{\pplane}{{\sf pplane9}}

\newcommand{\EXER}{\section*{Exercises}}%\vspace*{0.2in}\hrule\small\setcounter{exercise}{0}}
\newcommand{\CEXER}{}%\vspace{0.08in}\begin{center}Computer Exercises\end{center}}
\newcommand{\TEXER}{} %\vspace{0.08in}\begin{center}Hand Exercises\end{center}}
\newcommand{\AEXER}{} %\vspace{0.08in}\begin{center}Hand Exercises\end{center}}

% BADBAD: \newcommand{\Bbb}{\bf}

\newcommand{\R}{\mbox{$\Bbb{R}$}}
\newcommand{\C}{\mbox{$\Bbb{C}$}}
\newcommand{\Z}{\mbox{$\Bbb{Z}$}}
\newcommand{\N}{\mbox{$\Bbb{N}$}}
\newcommand{\D}{\mbox{{\bf D}}}
\usepackage{amssymb}
%\newcommand{\qed}{\hfill\mbox{\raggedright$\square$} \vspace{1ex}}
%\newcommand{\proof}{\noindent {\bf Proof:} \hspace{0.1in}}

\newcommand{\setmin}{\;\mbox{--}\;}
\newcommand{\Matlab}{{M\small{AT\-LAB}} }
\newcommand{\Matlabp}{{M\small{AT\-LAB}}}
\newcommand{\computer}{\Matlab Instructions}
\newcommand{\half}{\mbox{$\frac{1}{2}$}}
\newcommand{\compose}{\raisebox{.15ex}{\mbox{{\scriptsize$\circ$}}}}
\newcommand{\AND}{\quad\mbox{and}\quad}
\newcommand{\vect}[2]{\left(\begin{array}{c} #1_1 \\ \vdots \\
 #1_{#2}\end{array}\right)}
\newcommand{\mattwo}[4]{\left(\begin{array}{rr} #1 & #2\\ #3
&#4\end{array}\right)}
\newcommand{\mattwoc}[4]{\left(\begin{array}{cc} #1 & #2\\ #3
&#4\end{array}\right)}
\newcommand{\vectwo}[2]{\left(\begin{array}{r} #1 \\ #2\end{array}\right)}
\newcommand{\vectwoc}[2]{\left(\begin{array}{c} #1 \\ #2\end{array}\right)}

\newcommand{\ignore}[1]{}


\newcommand{\inv}{^{-1}}
\newcommand{\CC}{{\cal C}}
\newcommand{\CCone}{\CC^1}
\newcommand{\Span}{{\rm span}}
\newcommand{\rank}{{\rm rank}}
\newcommand{\trace}{{\rm tr}}
\newcommand{\RE}{{\rm Re}}
\newcommand{\IM}{{\rm Im}}
\newcommand{\nulls}{{\rm null\;space}}

\newcommand{\dps}{\displaystyle}
\newcommand{\arraystart}{\renewcommand{\arraystretch}{1.8}}
\newcommand{\arrayfinish}{\renewcommand{\arraystretch}{1.2}}
\newcommand{\Start}[1]{\vspace{0.08in}\noindent {\bf Section~\ref{#1}}}
\newcommand{\exer}[1]{\noindent {\bf \ref{#1}}}
\newcommand{\ans}{}
\newcommand{\matthree}[9]{\left(\begin{array}{rrr} #1 & #2 & #3 \\ #4 & #5 & #6
\\ #7 & #8 & #9\end{array}\right)}
\newcommand{\cvectwo}[2]{\left(\begin{array}{c} #1 \\ #2\end{array}\right)}
\newcommand{\cmatthree}[9]{\left(\begin{array}{ccc} #1 & #2 & #3 \\ #4 & #5 &
#6 \\ #7 & #8 & #9\end{array}\right)}
\newcommand{\vecthree}[3]{\left(\begin{array}{r} #1 \\ #2 \\
#3\end{array}\right)}
\newcommand{\cvecthree}[3]{\left(\begin{array}{c} #1 \\ #2 \\
#3\end{array}\right)}
\newcommand{\cmattwo}[4]{\left(\begin{array}{cc} #1 & #2\\ #3
&#4\end{array}\right)}

\newcommand{\Matrix}[1]{\ensuremath{\left(\begin{array}{rrrrrrrrrrrrrrrrrr} #1 \end{array}\right)}}

\newcommand{\Matrixc}[1]{\ensuremath{\left(\begin{array}{cccccccccccc} #1 \end{array}\right)}}



\renewcommand{\labelenumi}{\theenumi)}
\newenvironment{enumeratea}%
{\begingroup
 \renewcommand{\theenumi}{\alph{enumi}}
 \renewcommand{\labelenumi}{(\theenumi)}
 \begin{enumerate}}
 {\end{enumerate}\endgroup}



\newcounter{help}
\renewcommand{\thehelp}{\thesection.\arabic{equation}}

%\newenvironment{equation*}%
%{\renewcommand\endequation{\eqno (\theequation)* $$}%
%   \begin{equation}}%
%   {\end{equation}\renewcommand\endequation{\eqno \@eqnnum
%$$\global\@ignoretrue}}

%\input{psfig.tex}

\author{Martin Golubitsky and Michael Dellnitz}

%\newenvironment{matlabEquation}%
%{\renewcommand\endequation{\eqno (\theequation*) $$}%
%   \begin{equation}}%
%   {\end{equation}\renewcommand\endequation{\eqno \@eqnnum
% $$\global\@ignoretrue}}

\newcommand{\soln}{\textbf{Solution:} }
\newcommand{\exercap}[1]{\centerline{Figure~\ref{#1}}}
\newcommand{\exercaptwo}[1]{\centerline{Figure~\ref{#1}a\hspace{2.1in}
Figure~\ref{#1}b}}
\newcommand{\exercapthree}[1]{\centerline{Figure~\ref{#1}a\hspace{1.2in}
Figure~\ref{#1}b\hspace{1.2in}Figure~\ref{#1}c}}
\newcommand{\para}{\hspace{0.4in}}

\renewenvironment{solution}{\suppress}{\endsuppress}

\ifxake
\newenvironment{matlabEquation}{\begin{equation}}{\end{equation}}
\else
\newenvironment{matlabEquation}%
{\let\oldtheequation\theequation\renewcommand{\theequation}{\oldtheequation*}\begin{equation}}%
  {\end{equation}\let\theequation\oldtheequation}
\fi

\makeatother

\begin{document}

\TEXER

\noindent In Exercises~\ref{c14.3.1a} -- \ref{c14.3.1b} solve the given 
initial value problems by variation of parameters.

\begin{exercise}  \label{c14.3.1a}
\[
\frac{dX}{dt}=\frac{1}{2}\mattwo{1}{1}{1}{1} X + \vectwo{e^t}{0},\quad
X(0) = \vectwo{1}{-1}.
\]

\begin{solution}
\ans The solution to the initial value problem is
\[
X(t) = \frac{1}{2}\cvectwo{te^t + e^t + 1}{te^t - e^t - 1}.
\]

\soln This system can be written as $\dot{X} = AX + G(t)$, where
\[
A = \frac{1}{2}\mattwo{1}{1}{1}{1} \AND G(t) = \vectwo{e^t}{0}.
\]
We can then solve the system by variation of parameters, as described in
Theorem~\ref{thm:varparsys}:

\paragraph{Step 1.} The eigenvalues of $A$ are $\lambda_1 = 0$ and
$\lambda_2 = 1$, with associated eigenvectors $v_1 = (1,-1)^t$ and
$v_2 = (1,1)^t$.  Thus, a basis of solutions to the homogeneous system
$\dot{X} = AX$ is
\[
X_1(t) = \vectwo{1}{-1} \AND X_2(t) = e^t\vectwo{1}{1}.
\]
\paragraph{Step 2.} To compute the vector $D(t)$, first compute
\[
Y(t)^{-1} = (X_1(t)|X_2(t))^{-1} = \cmattwo{1}{e^t}{-1}{e^t}^{-1} =
\frac{1}{2e^t}\cmattwo{e^t}{-e^t}{1}{1} =
\frac{1}{2}\cmattwo{1}{-1}{e^{-t}}{e^{-t}}.
\]
Then,
\[
D(t) = Y(t)^{-1}G(t) =
\frac{1}{2}\cmattwo{1}{-1}{e^{-t}}{e^{-t}}\vectwo{e^t}{0} =
\frac{1}{2}\vectwo{e^{t}}{1}.
\]
\paragraph{Step 3.} We find $c_1(0)$ and $c_2(0)$ using the initial
condition
\[
X_0 = c_1(0)X_1(0) + c_2(0)X_2(0).
\]
Therefore,
\[
\vectwo{1}{-1} = c_1(0)\vectwo{1}{-1} + c_2(0)\vectwo{1}{1}.
\]
Solve this system to obtain $c_1(0) = 1$ and $c_2(0) = 0$.  By definition,
$\dot{c_j} = d_j$.  Thus,
\[
c_1(t) = \int_{t_0}^td_1(\tau)d\tau + c_1(0)
= \frac{1}{2}\int_0^te^\tau d\tau + 1 = \frac{1}{2}(e^t + 1).
\]
\[
c_2(t) = \int_{t_0}^td_2(\tau)d\tau + c_2(0)
= \frac{1}{2}\int_0^td\tau = \frac{t}{2}.
\]
\paragraph{Step 4.} So, the solution to the initial value problem is
\[
X(t) = c_1(t)X_1(t) + c_2(t)X_2(t)
= \frac{1}{2}\left((e^t + 1)\vectwo{1}{-1} + \frac{t}{2}e^t\vectwo{1}{1}
\right).
\]

\end{solution}
\end{exercise}
\begin{exercise}  \label{c14.3.1b}
\[
\frac{dX}{dt}=\mattwo{1}{3}{3}{1} X + \vectwo{2t}{t},\quad
X(0) = \vectwo{0}{1}.
\]

\begin{solution}
\ans The solution to the given initial value problem is
\[
X(t) = \frac{1}{32}\cvectwo{19e^{4t} - 12e^{-2t} - 4t - 7}
{19e^{4t} + 12e^{-2t} - 20t + 1}.
\]
\soln This system can be written as $\dot{X} = AX + G(t)$, where
\[
A = \mattwo{1}{3}{3}{1} \AND G(t) = \vectwo{2t}{t}.
\]
We can then solve the system by variation of parameters, as described in
Theorem~\ref{thm:varparsys}:

\paragraph{Step 1.} The eigenvalues of $A$ are $\lambda_1 = -2$ and
$\lambda_2 = 4$, with associated eigenvectors $v_1 = (1,-1)^t$ and
$v_2 = (1,1)^t$.  Thus, a basis of solutions to the homogeneous system
$\dot{X} = AX$ is
\[
X_1(t) = e^{-2t}\vectwo{1}{-1} \AND X_2(t) = e^{4t}\vectwo{1}{1}.
\]
\paragraph{Step 2.} To compute the vector $D(t)$, first compute
\[
Y(t)^{-1} = \cmattwo{e^{-2t}}{e^{4t}}{-e^{-2t}}{e^{4t}}^{-1}
= \frac{1}{2e^{2t}}\cmattwo{e^{4t}}{-e^{4t}}{e^{-2t}}{e^{-2t}}
= \frac{1}{2}\cmattwo{e^{2t}}{-e^{2t}}{e^{-4t}}{e^{-4t}}.
\]
Then,
\[
D(t) = Y(t)^{-1}G(t) =
\frac{1}{2}\cmattwo{e^{2t}}{-e^{2t}}{e^{-4t}}{e^{-4t}}\vectwo{2t}{t} =
\frac{1}{2}\vectwo{te^{2t}}{3te^{-4t}}.
\]
\paragraph{Step 3.} We find $c_1(0)$ and $c_2(0)$ using the initial
condition
\[
X_0 = c_1(0)X_1(0) + c_2(0)X_2(0).
\]
Therefore,
\[
\vectwo{0}{1} = c_1(0)\vectwo{1}{-1} + c_2(0)\vectwo{1}{1}.
\]
Solve this system to obtain $c_1(0) = -\frac{1}{2}$ and $c_2(0) =
\frac{1}{2}$.  By definition, $\dot{c_j} = d_j$.  Thus,
\[
c_1(t) = \int_{t_0}^td_1(\tau)d\tau + c_1(0)
= \frac{1}{2}\int_0^t \tau e^{2\tau}d\tau - \frac{1}{2}
= \frac{1}{8}(2te^{2t} - e^{2t} - 3).
\]
\[
c_2(t) = \int_{t_0}^td_2(\tau)d\tau + c_2(0)
= \frac{3}{2}\int_0^t \tau e^{-4\tau}d\tau + \frac{1}{2}
= \frac{1}{32}(-4te^{-4t} - e^{-4t} + 19).
\]

\paragraph{Step 4.} So, the solution to the initial value problem is
\[
X(t) = c_1(t)X_1(t) + c_2(t)X_2(t)
= \frac{1}{8}(2t - 1 - 3e^{-2t})\vectwo{1}{-1} +
\frac{1}{32}(-4t - 1 + 19e^{4t})\vectwo{1}{1}.
\]


\end{solution}
\end{exercise}
\end{document}
