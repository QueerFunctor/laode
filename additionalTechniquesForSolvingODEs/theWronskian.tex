\documentclass{ximera}

 

\usepackage{epsfig}

\graphicspath{
  {./}
  {figures/}
}

\usepackage{morewrites}
\makeatletter
\newcommand\subfile[1]{%
\renewcommand{\input}[1]{}%
\begingroup\skip@preamble\otherinput{#1}\endgroup\par\vspace{\topsep}
\let\input\otherinput}
\makeatother

\newcommand{\includeexercises}{\directlua{dofile("/home/jim/linearAlgebra/laode/exercises.lua")}}

%\newcounter{ccounter}
%\setcounter{ccounter}{1}
%\newcommand{\Chapter}[1]{\setcounter{chapter}{\arabic{ccounter}}\chapter{#1}\addtocounter{ccounter}{1}}

%\newcommand{\section}[1]{\section{#1}\setcounter{thm}{0}\setcounter{equation}{0}}

%\renewcommand{\theequation}{\arabic{chapter}.\arabic{section}.\arabic{equation}}
%\renewcommand{\thefigure}{\arabic{chapter}.\arabic{figure}}
%\renewcommand{\thetable}{\arabic{chapter}.\arabic{table}}

%\newcommand{\Sec}[2]{\section{#1}\markright{\arabic{ccounter}.\arabic{section}.#2}\setcounter{equation}{0}\setcounter{thm}{0}\setcounter{figure}{0}}

\newcommand{\Sec}[2]{\section{#1}}

\setcounter{secnumdepth}{2}
%\setcounter{secnumdepth}{1} 

%\newcounter{THM}
%\renewcommand{\theTHM}{\arabic{chapter}.\arabic{section}}

\newcommand{\trademark}{{R\!\!\!\!\!\bigcirc}}
%\newtheorem{exercise}{}

\newcommand{\dfield}{{\sf dfield9}}
\newcommand{\pplane}{{\sf pplane9}}

\newcommand{\EXER}{\section*{Exercises}}%\vspace*{0.2in}\hrule\small\setcounter{exercise}{0}}
\newcommand{\CEXER}{}%\vspace{0.08in}\begin{center}Computer Exercises\end{center}}
\newcommand{\TEXER}{} %\vspace{0.08in}\begin{center}Hand Exercises\end{center}}
\newcommand{\AEXER}{} %\vspace{0.08in}\begin{center}Hand Exercises\end{center}}

% BADBAD: \newcommand{\Bbb}{\bf}

\newcommand{\R}{\mbox{$\Bbb{R}$}}
\newcommand{\C}{\mbox{$\Bbb{C}$}}
\newcommand{\Z}{\mbox{$\Bbb{Z}$}}
\newcommand{\N}{\mbox{$\Bbb{N}$}}
\newcommand{\D}{\mbox{{\bf D}}}
\usepackage{amssymb}
%\newcommand{\qed}{\hfill\mbox{\raggedright$\square$} \vspace{1ex}}
%\newcommand{\proof}{\noindent {\bf Proof:} \hspace{0.1in}}

\newcommand{\setmin}{\;\mbox{--}\;}
\newcommand{\Matlab}{{M\small{AT\-LAB}} }
\newcommand{\Matlabp}{{M\small{AT\-LAB}}}
\newcommand{\computer}{\Matlab Instructions}
\newcommand{\half}{\mbox{$\frac{1}{2}$}}
\newcommand{\compose}{\raisebox{.15ex}{\mbox{{\scriptsize$\circ$}}}}
\newcommand{\AND}{\quad\mbox{and}\quad}
\newcommand{\vect}[2]{\left(\begin{array}{c} #1_1 \\ \vdots \\
 #1_{#2}\end{array}\right)}
\newcommand{\mattwo}[4]{\left(\begin{array}{rr} #1 & #2\\ #3
&#4\end{array}\right)}
\newcommand{\mattwoc}[4]{\left(\begin{array}{cc} #1 & #2\\ #3
&#4\end{array}\right)}
\newcommand{\vectwo}[2]{\left(\begin{array}{r} #1 \\ #2\end{array}\right)}
\newcommand{\vectwoc}[2]{\left(\begin{array}{c} #1 \\ #2\end{array}\right)}

\newcommand{\ignore}[1]{}


\newcommand{\inv}{^{-1}}
\newcommand{\CC}{{\cal C}}
\newcommand{\CCone}{\CC^1}
\newcommand{\Span}{{\rm span}}
\newcommand{\rank}{{\rm rank}}
\newcommand{\trace}{{\rm tr}}
\newcommand{\RE}{{\rm Re}}
\newcommand{\IM}{{\rm Im}}
\newcommand{\nulls}{{\rm null\;space}}

\newcommand{\dps}{\displaystyle}
\newcommand{\arraystart}{\renewcommand{\arraystretch}{1.8}}
\newcommand{\arrayfinish}{\renewcommand{\arraystretch}{1.2}}
\newcommand{\Start}[1]{\vspace{0.08in}\noindent {\bf Section~\ref{#1}}}
\newcommand{\exer}[1]{\noindent {\bf \ref{#1}}}
\newcommand{\ans}{}
\newcommand{\matthree}[9]{\left(\begin{array}{rrr} #1 & #2 & #3 \\ #4 & #5 & #6
\\ #7 & #8 & #9\end{array}\right)}
\newcommand{\cvectwo}[2]{\left(\begin{array}{c} #1 \\ #2\end{array}\right)}
\newcommand{\cmatthree}[9]{\left(\begin{array}{ccc} #1 & #2 & #3 \\ #4 & #5 &
#6 \\ #7 & #8 & #9\end{array}\right)}
\newcommand{\vecthree}[3]{\left(\begin{array}{r} #1 \\ #2 \\
#3\end{array}\right)}
\newcommand{\cvecthree}[3]{\left(\begin{array}{c} #1 \\ #2 \\
#3\end{array}\right)}
\newcommand{\cmattwo}[4]{\left(\begin{array}{cc} #1 & #2\\ #3
&#4\end{array}\right)}

\newcommand{\Matrix}[1]{\ensuremath{\left(\begin{array}{rrrrrrrrrrrrrrrrrr} #1 \end{array}\right)}}

\newcommand{\Matrixc}[1]{\ensuremath{\left(\begin{array}{cccccccccccc} #1 \end{array}\right)}}



\renewcommand{\labelenumi}{\theenumi)}
\newenvironment{enumeratea}%
{\begingroup
 \renewcommand{\theenumi}{\alph{enumi}}
 \renewcommand{\labelenumi}{(\theenumi)}
 \begin{enumerate}}
 {\end{enumerate}\endgroup}



\newcounter{help}
\renewcommand{\thehelp}{\thesection.\arabic{equation}}

%\newenvironment{equation*}%
%{\renewcommand\endequation{\eqno (\theequation)* $$}%
%   \begin{equation}}%
%   {\end{equation}\renewcommand\endequation{\eqno \@eqnnum
%$$\global\@ignoretrue}}

%\input{psfig.tex}

\author{Martin Golubitsky and Michael Dellnitz}

%\newenvironment{matlabEquation}%
%{\renewcommand\endequation{\eqno (\theequation*) $$}%
%   \begin{equation}}%
%   {\end{equation}\renewcommand\endequation{\eqno \@eqnnum
% $$\global\@ignoretrue}}

\newcommand{\soln}{\textbf{Solution:} }
\newcommand{\exercap}[1]{\centerline{Figure~\ref{#1}}}
\newcommand{\exercaptwo}[1]{\centerline{Figure~\ref{#1}a\hspace{2.1in}
Figure~\ref{#1}b}}
\newcommand{\exercapthree}[1]{\centerline{Figure~\ref{#1}a\hspace{1.2in}
Figure~\ref{#1}b\hspace{1.2in}Figure~\ref{#1}c}}
\newcommand{\para}{\hspace{0.4in}}

\renewenvironment{solution}{\suppress}{\endsuppress}

\ifxake
\newenvironment{matlabEquation}{\begin{equation}}{\end{equation}}
\else
\newenvironment{matlabEquation}%
{\let\oldtheequation\theequation\renewcommand{\theequation}{\oldtheequation*}\begin{equation}}%
  {\end{equation}\let\theequation\oldtheequation}
\fi

\makeatother


\title{The Wronskian}

\begin{document}
\begin{abstract}
\end{abstract}
\maketitle

  \label{S:wronskian}
\index{Wronskian} 

We return to the homogeneous system of linear differential equations
\arraystart
\begin{equation}  \label{eq:linihsys2}
\frac{dX}{dt}  =  A(t)X
\end{equation}
\arrayfinish
where $A(t)=(a_{ij}(t))$ is an $n\times n$ matrix.  Let $X_1,\ldots,X_n$ be a 
linearly independent set of solutions\index{basis!of solutions} 
to \eqref{eq:linihsys2} and let 
\begin{equation}   \label{E:Y(t)2}
Y(t) = \left(X_1(t)|\cdots |X_n(t)\right) 
\end{equation}
Using existence and uniqueness of solutions to the initial problem for
\eqref{eq:linihsys2}, we showed in Lemma~\ref{L:DEspan} that $Y(t)$ is an 
invertible matrix for every time $t$.  In this section we prove this same 
result by explicitly showing that the determinant\index{determinant} 
of $Y(t)$ is always nonzero.

Define the {\em Wronskian\/}\index{Wronskian} to be
\[
W(t) = \det Y(t).
\]

\begin{theorem}  \label{T:Wronskian}
Let $X_1,\ldots,X_n$ be solutions to \eqref{E:NCCH} such that the vectors
$X_j(0)$ form a basis of $\R^n$.  Let $Y(t)$ be the matrix defined in 
\eqref{E:Y(t)}.  Then
\begin{equation}  \label{L:Wronskian}
W(t) = e^{\int_0^t\trace(A(\tau))d\tau}W(0).
\end{equation}
\end{theorem}

It follows directly from \eqref{L:Wronskian} that the determinant of $Y(t)$ is 
nonzero when the determinant of $Y(0)$ is nonzero.  But $\det Y(0)\neq 0$ 
since the vectors $X_j(0)$ form a basis of $\R^n$. 

We prove Theorem~\ref{T:Wronskian} in two important special cases: linear 
constant coefficient systems and linear nonconstant $2\times 2$ systems.  
The proof for constant coefficient systems is based on Jordan normal forms, 
while the proof for $2\times 2$ systems is based on solving a separable 
differential equation for the Wronskian itself.  It is this latter proof 
that generalizes to a proof of the theorem.

\subsubsection*{Wronskians for Constant Coefficient Systems}
\index{Wronskian!constant coefficient system}

First, we interpret the Wronskian directly in terms of the constant coefficient 
matrix $A$.  Note that $X_j(t)=e^{tA}e_j$ is just the $j^{th}$ column of the 
matrix $e^{tA}$.   It follows that 
\[
W(t) = \det e^{tA}.
\]

\begin{lemma} 
Let $A$ be an $n\times n$ matrix.  Then
\[
\det e^A = e^{\trace(A)}.
\]
\end{lemma}\index{determinant}\index{matrix!exponential}
\index{trace}

\begin{proof}  This result is proved using 
Jordan normal forms\index{Jordan normal form}.  To see why normal
form theory is relevant, suppose that $A$ and $B$ are similar matrices.  Then 
$e^A$ and $e^B$ are similar matrices, and $\trace(A)=\trace(B)$ and 
$\det e^A = \det e^B$.  So if we can show that the lemma is valid for matrices 
in Jordan normal form, then the lemma is valid for all matrices.

Suppose that the matrix $J$ is  a $k\times k$ 
Jordan block\index{Jordan block} matrix associated
to the eigenvalue $\lambda$.  Then $J$ is upper triangular and the 
diagonal entries of $J$ all equal $\lambda$.  
It follows that $\trace(J)=k\lambda$.
It also follows that $e^J$ is an upper triangular matrix whose diagonal 
entries all equal $e^\lambda$.  Hence
\[
\det e^J = \left(e^\lambda\right)^k = e^{k\lambda} = e^{\trace(J)}.
\]
So the lemma is valid for Jordan block matrices.

Next suppose that $A$ is in block diagonal\index{matrix!block diagonal}, 
that is
\[
A=\mattwo{B}{0}{0}{C}.
\]
We claim that if the lemma is valid for matrices $B$ and $C$, then it 
is valid for the matrix $A$.  To see this observe that 
\[
\trace(A) = \trace(B) + \trace(C),
\]
and that 
\[
e^A = \mattwo{e^B}{0}{0}{e^C}.
\]
Hence 
\[
\det e^A = \det e^B \det e^C = e^{\trace(B)}e^{\trace(C)}, 
\]
by assumption.  It follows that 
\[
\det e^A = e^{\trace(B)+\trace(C)} = e^{\trace(A)},
\]
as desired.  By induction,  the lemma is valid for Jordan normal form 
matrices and hence for all matrices.  \end{proof}

\subsubsection*{Wronskians for Planar Systems}
\index{Wronskian!planar system}

In the time dependent case we verify Theorem~\ref{T:Wronskian} only 
for $2\times 2$ systems, as this 
substantially simplifies the discussion.   Let 
\[
A(t) = \mattwo{a_{11}(t)}{a_{12}(t)}{a_{21}(t)}{a_{22}(t)},
\]
and let 
\[
X_1(t) = \vectwo{x_1(t)}{y_1(t)} \AND  X_2(t) = \vectwo{x_2(t)}{y_2(t)}
\]
be solutions of \eqref{E:NCCH}.  It follows that 
\begin{equation}   \label{E:xyderiv}
\begin{array}{rcl}
\dot{x}_1 & = & a_{11}x_1 + a_{12}y_1 \\
\dot{y}_1 & = & a_{21}x_1 + a_{22}y_1 \\
\dot{x}_2 & = & a_{11}x_2 + a_{12}y_2 \\
\dot{y}_2 & = & a_{21}x_2 + a_{22}y_2.
\end{array}
\end{equation}

In this notation 
\[
Y(t) = \mattwo{x_1(t)}{x_2(t)}{y_1(t)}{y_2(t)},
\]
and
\[
W(t) = x_1(t)y_2(t) - x_2(t)y_1(t).
\]
We claim that 
\[
\dot{W} = \trace(A) W.
\]
If so, we can use separation of variables to solve this differential equation 
obtaining
\[
\ln |W| = \int \trace(A(t))dt 
\]
from which the proof of Theorem~\ref{T:Wronskian} follows.

Use the product rule to compute
\[
\dot{W}  =  \dot{x}_1y_2 + x_1\dot{y}_2 - \dot{x}_2y_1 - x_2\dot{y}_1.
\]
Now substitute \eqref{E:xyderiv} to see that 
\[
\dot{W} = (a_{11}x_1 + a_{12}y_1)y_2 + x_1(a_{21}x_2 + a_{22}y_2)
- (a_{11}x_2 + a_{12}y_2)y_1 - x_2(a_{21}x_1 + a_{22}y_1)
\]
from which it follows that 
\[
\dot{W} = (a_{11}+a_{22})W = \trace(A)W,
\]
as claimed.




\includeexercises

 


\end{document}
