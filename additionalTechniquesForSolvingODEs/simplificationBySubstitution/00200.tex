\documentclass{ximera}
 

\usepackage{epsfig}

\graphicspath{
  {./}
  {figures/}
}

\usepackage{morewrites}
\makeatletter
\newcommand\subfile[1]{%
\renewcommand{\input}[1]{}%
\begingroup\skip@preamble\otherinput{#1}\endgroup\par\vspace{\topsep}
\let\input\otherinput}
\makeatother

\newcommand{\includeexercises}{\directlua{dofile("/home/jim/linearAlgebra/laode/exercises.lua")}}

%\newcounter{ccounter}
%\setcounter{ccounter}{1}
%\newcommand{\Chapter}[1]{\setcounter{chapter}{\arabic{ccounter}}\chapter{#1}\addtocounter{ccounter}{1}}

%\newcommand{\section}[1]{\section{#1}\setcounter{thm}{0}\setcounter{equation}{0}}

%\renewcommand{\theequation}{\arabic{chapter}.\arabic{section}.\arabic{equation}}
%\renewcommand{\thefigure}{\arabic{chapter}.\arabic{figure}}
%\renewcommand{\thetable}{\arabic{chapter}.\arabic{table}}

%\newcommand{\Sec}[2]{\section{#1}\markright{\arabic{ccounter}.\arabic{section}.#2}\setcounter{equation}{0}\setcounter{thm}{0}\setcounter{figure}{0}}

\newcommand{\Sec}[2]{\section{#1}}

\setcounter{secnumdepth}{2}
%\setcounter{secnumdepth}{1} 

%\newcounter{THM}
%\renewcommand{\theTHM}{\arabic{chapter}.\arabic{section}}

\newcommand{\trademark}{{R\!\!\!\!\!\bigcirc}}
%\newtheorem{exercise}{}

\newcommand{\dfield}{{\sf dfield9}}
\newcommand{\pplane}{{\sf pplane9}}

\newcommand{\EXER}{\section*{Exercises}}%\vspace*{0.2in}\hrule\small\setcounter{exercise}{0}}
\newcommand{\CEXER}{}%\vspace{0.08in}\begin{center}Computer Exercises\end{center}}
\newcommand{\TEXER}{} %\vspace{0.08in}\begin{center}Hand Exercises\end{center}}
\newcommand{\AEXER}{} %\vspace{0.08in}\begin{center}Hand Exercises\end{center}}

% BADBAD: \newcommand{\Bbb}{\bf}

\newcommand{\R}{\mbox{$\Bbb{R}$}}
\newcommand{\C}{\mbox{$\Bbb{C}$}}
\newcommand{\Z}{\mbox{$\Bbb{Z}$}}
\newcommand{\N}{\mbox{$\Bbb{N}$}}
\newcommand{\D}{\mbox{{\bf D}}}
\usepackage{amssymb}
%\newcommand{\qed}{\hfill\mbox{\raggedright$\square$} \vspace{1ex}}
%\newcommand{\proof}{\noindent {\bf Proof:} \hspace{0.1in}}

\newcommand{\setmin}{\;\mbox{--}\;}
\newcommand{\Matlab}{{M\small{AT\-LAB}} }
\newcommand{\Matlabp}{{M\small{AT\-LAB}}}
\newcommand{\computer}{\Matlab Instructions}
\newcommand{\half}{\mbox{$\frac{1}{2}$}}
\newcommand{\compose}{\raisebox{.15ex}{\mbox{{\scriptsize$\circ$}}}}
\newcommand{\AND}{\quad\mbox{and}\quad}
\newcommand{\vect}[2]{\left(\begin{array}{c} #1_1 \\ \vdots \\
 #1_{#2}\end{array}\right)}
\newcommand{\mattwo}[4]{\left(\begin{array}{rr} #1 & #2\\ #3
&#4\end{array}\right)}
\newcommand{\mattwoc}[4]{\left(\begin{array}{cc} #1 & #2\\ #3
&#4\end{array}\right)}
\newcommand{\vectwo}[2]{\left(\begin{array}{r} #1 \\ #2\end{array}\right)}
\newcommand{\vectwoc}[2]{\left(\begin{array}{c} #1 \\ #2\end{array}\right)}

\newcommand{\ignore}[1]{}


\newcommand{\inv}{^{-1}}
\newcommand{\CC}{{\cal C}}
\newcommand{\CCone}{\CC^1}
\newcommand{\Span}{{\rm span}}
\newcommand{\rank}{{\rm rank}}
\newcommand{\trace}{{\rm tr}}
\newcommand{\RE}{{\rm Re}}
\newcommand{\IM}{{\rm Im}}
\newcommand{\nulls}{{\rm null\;space}}

\newcommand{\dps}{\displaystyle}
\newcommand{\arraystart}{\renewcommand{\arraystretch}{1.8}}
\newcommand{\arrayfinish}{\renewcommand{\arraystretch}{1.2}}
\newcommand{\Start}[1]{\vspace{0.08in}\noindent {\bf Section~\ref{#1}}}
\newcommand{\exer}[1]{\noindent {\bf \ref{#1}}}
\newcommand{\ans}{}
\newcommand{\matthree}[9]{\left(\begin{array}{rrr} #1 & #2 & #3 \\ #4 & #5 & #6
\\ #7 & #8 & #9\end{array}\right)}
\newcommand{\cvectwo}[2]{\left(\begin{array}{c} #1 \\ #2\end{array}\right)}
\newcommand{\cmatthree}[9]{\left(\begin{array}{ccc} #1 & #2 & #3 \\ #4 & #5 &
#6 \\ #7 & #8 & #9\end{array}\right)}
\newcommand{\vecthree}[3]{\left(\begin{array}{r} #1 \\ #2 \\
#3\end{array}\right)}
\newcommand{\cvecthree}[3]{\left(\begin{array}{c} #1 \\ #2 \\
#3\end{array}\right)}
\newcommand{\cmattwo}[4]{\left(\begin{array}{cc} #1 & #2\\ #3
&#4\end{array}\right)}

\newcommand{\Matrix}[1]{\ensuremath{\left(\begin{array}{rrrrrrrrrrrrrrrrrr} #1 \end{array}\right)}}

\newcommand{\Matrixc}[1]{\ensuremath{\left(\begin{array}{cccccccccccc} #1 \end{array}\right)}}



\renewcommand{\labelenumi}{\theenumi)}
\newenvironment{enumeratea}%
{\begingroup
 \renewcommand{\theenumi}{\alph{enumi}}
 \renewcommand{\labelenumi}{(\theenumi)}
 \begin{enumerate}}
 {\end{enumerate}\endgroup}



\newcounter{help}
\renewcommand{\thehelp}{\thesection.\arabic{equation}}

%\newenvironment{equation*}%
%{\renewcommand\endequation{\eqno (\theequation)* $$}%
%   \begin{equation}}%
%   {\end{equation}\renewcommand\endequation{\eqno \@eqnnum
%$$\global\@ignoretrue}}

%\input{psfig.tex}

\author{Martin Golubitsky and Michael Dellnitz}

%\newenvironment{matlabEquation}%
%{\renewcommand\endequation{\eqno (\theequation*) $$}%
%   \begin{equation}}%
%   {\end{equation}\renewcommand\endequation{\eqno \@eqnnum
% $$\global\@ignoretrue}}

\newcommand{\soln}{\textbf{Solution:} }
\newcommand{\exercap}[1]{\centerline{Figure~\ref{#1}}}
\newcommand{\exercaptwo}[1]{\centerline{Figure~\ref{#1}a\hspace{2.1in}
Figure~\ref{#1}b}}
\newcommand{\exercapthree}[1]{\centerline{Figure~\ref{#1}a\hspace{1.2in}
Figure~\ref{#1}b\hspace{1.2in}Figure~\ref{#1}c}}
\newcommand{\para}{\hspace{0.4in}}

\renewenvironment{solution}{\suppress}{\endsuppress}

\ifxake
\newenvironment{matlabEquation}{\begin{equation}}{\end{equation}}
\else
\newenvironment{matlabEquation}%
{\let\oldtheequation\theequation\renewcommand{\theequation}{\oldtheequation*}\begin{equation}}%
  {\end{equation}\let\theequation\oldtheequation}
\fi

\makeatother

\begin{document}

\noindent In Exercises~\ref{c14.5.6} -- \ref{c14.5.9} solve the given initial 
value problem by an appropriate solution technique.
\begin{exercise} \label{c14.5.6}
$\dps \frac{dx}{dt} = \sec\left(\frac{x}{t}\right)+\frac{x}{t}$ where
$x(1)=\pi$.

\begin{solution}
\ans $x(t) = t\sin\inv(\ln|t|)$.

\soln  This differential equation is an equation with homogeneous
coefficients where $F(v) = v + \sec v$.  Using the substitution $v=x/t$, we
are led to the differential equation $t\dot{v}=\sec v$, which can be solved
by separation of variables.  That is, solve
\[
\cos v \frac{dv}{dt} = \frac{1}{t}
\]
by integration, obtaining
\[
\sin v = \ln|t| + c.
\]
We can solve for $c$ using the initial condition $x(1)=\pi$, which implies
that $v(1)=x(1)/1=\pi$.  Therefore, $c=0$.  Hence
\[
v = \sin\inv(\ln|t|).
\]
Using the identity $v=x/t$ leads to the answer. 
 

\end{solution}
\end{exercise}
\begin{exercise} \label{c14.5.7}
$\dps \frac{dx}{dt} = x+x^2$ where $x(2)=1$.

\begin{solution}
\ans $x(t) = \dps\frac{t}{4-t}$.

\soln  Although this equation is a Bernoulli equation, it is most directly
solved by separation of variables.  That is, solve
\[
\frac{1}{x(x+1)}\frac{dx}{dt} = 1.
\]
Using partial fractions, we obtain
\[
\left(\frac{1}{x}-\frac{1}{x+1}\right)\frac{dx}{dt} = 1
\]
On integration, we find
\[
\ln\left|\frac{x}{x+1}\right| = \ln|t| + c.
\]
Therefore,
\[
\frac{x}{x+1} = ct,
\]
for some constant $c$.  Using the initial condition $x(2)=1$ leads to
$c=\frac{1}{4}$.  After cross-multiplying we are led to the answer.

\end{solution}
\end{exercise}
\begin{exercise} \label{c14.5.8}
$\dps \frac{dx}{dt} = -\frac{x}{t}-t^3 x^3$ where $x(1)=1$.

\begin{solution}
\ans $x(t) = \dps\frac{1}{t^2}$.

\soln This differential equation is a Bernoulli equation with
\[
r(t) = -\frac{1}{t},\quad s(t) = -t^3 \AND p=3.
\]
After transformation the corresponding linear initial value problem
is given by
\[
\frac{dv}{dt} = \frac{2}{t} v + 2t^3,\quad v(1)=1.
\]
We solve this equation by variation of parameters.  The solution
of the homogeneous equation $\dot v = \frac{2}{t} v$ is
$v_h(t)=t^2$, and we use Theorem~\ref{thm:varpar} to find the solution
\[
v(t) = c(t) v_h(t) = t^2 v_h(t) = t^4.
\]
Back substitution now leads to the solution
\[
x(t) = (v(t))^{-\frac{1}{2}} = \frac{1}{t^2}.
\]


\end{solution}
\end{exercise}
\begin{exercise} \label{c14.5.9}
$\dps \frac{dx}{dt} = \frac{x(x+t)}{t^2}$ where $x(1)=1$.

\begin{solution}
\ans $x(t) = \dps\frac{t}{1-\ln|t|}$.

\soln  This differential equation is an equation with homogeneous
coefficients where $F(v) = v + v^2$.  Using the substitution $v=x/t$, we
are led to the differential equation $t\dot{v}=v^2$, which can be solved
by separation of variables.  That is, solve
\[
 \frac{1}{v^2} \frac{dv}{dt} = \frac{1}{t}
\]
by integration, obtaining
\[
-\frac{1}{v} = \ln|t| + c.
\]
We can solve for $c$ using the initial condition $x(1)=1$, which implies
that $v(1)=x(1)/1=1$.  Therefore, $c=-1$.  Hence
\[
v = \frac{1}{1-\ln|t|}.
\]
Using the identity $v=x/t$ leads to the answer.   Note that this equation is
also a Bernoulli equation and could be solved using the techniques for
Bernoulli equations, as well.

\end{solution}
\end{exercise}
\end{document}
