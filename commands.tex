\documentclass{ximera}

 

\usepackage{epsfig}

\graphicspath{
  {./}
  {figures/}
}

\usepackage{morewrites}
\makeatletter
\newcommand\subfile[1]{%
\renewcommand{\input}[1]{}%
\begingroup\skip@preamble\otherinput{#1}\endgroup\par\vspace{\topsep}
\let\input\otherinput}
\makeatother

\newcommand{\includeexercises}{\directlua{dofile("/home/jim/linearAlgebra/laode/exercises.lua")}}

%\newcounter{ccounter}
%\setcounter{ccounter}{1}
%\newcommand{\Chapter}[1]{\setcounter{chapter}{\arabic{ccounter}}\chapter{#1}\addtocounter{ccounter}{1}}

%\newcommand{\section}[1]{\section{#1}\setcounter{thm}{0}\setcounter{equation}{0}}

%\renewcommand{\theequation}{\arabic{chapter}.\arabic{section}.\arabic{equation}}
%\renewcommand{\thefigure}{\arabic{chapter}.\arabic{figure}}
%\renewcommand{\thetable}{\arabic{chapter}.\arabic{table}}

%\newcommand{\Sec}[2]{\section{#1}\markright{\arabic{ccounter}.\arabic{section}.#2}\setcounter{equation}{0}\setcounter{thm}{0}\setcounter{figure}{0}}

\newcommand{\Sec}[2]{\section{#1}}

\setcounter{secnumdepth}{2}
%\setcounter{secnumdepth}{1} 

%\newcounter{THM}
%\renewcommand{\theTHM}{\arabic{chapter}.\arabic{section}}

\newcommand{\trademark}{{R\!\!\!\!\!\bigcirc}}
%\newtheorem{exercise}{}

\newcommand{\dfield}{{\sf dfield9}}
\newcommand{\pplane}{{\sf pplane9}}

\newcommand{\EXER}{\section*{Exercises}}%\vspace*{0.2in}\hrule\small\setcounter{exercise}{0}}
\newcommand{\CEXER}{}%\vspace{0.08in}\begin{center}Computer Exercises\end{center}}
\newcommand{\TEXER}{} %\vspace{0.08in}\begin{center}Hand Exercises\end{center}}
\newcommand{\AEXER}{} %\vspace{0.08in}\begin{center}Hand Exercises\end{center}}

% BADBAD: \newcommand{\Bbb}{\bf}

\newcommand{\R}{\mbox{$\Bbb{R}$}}
\newcommand{\C}{\mbox{$\Bbb{C}$}}
\newcommand{\Z}{\mbox{$\Bbb{Z}$}}
\newcommand{\N}{\mbox{$\Bbb{N}$}}
\newcommand{\D}{\mbox{{\bf D}}}
\usepackage{amssymb}
%\newcommand{\qed}{\hfill\mbox{\raggedright$\square$} \vspace{1ex}}
%\newcommand{\proof}{\noindent {\bf Proof:} \hspace{0.1in}}

\newcommand{\setmin}{\;\mbox{--}\;}
\newcommand{\Matlab}{{M\small{AT\-LAB}} }
\newcommand{\Matlabp}{{M\small{AT\-LAB}}}
\newcommand{\computer}{\Matlab Instructions}
\newcommand{\half}{\mbox{$\frac{1}{2}$}}
\newcommand{\compose}{\raisebox{.15ex}{\mbox{{\scriptsize$\circ$}}}}
\newcommand{\AND}{\quad\mbox{and}\quad}
\newcommand{\vect}[2]{\left(\begin{array}{c} #1_1 \\ \vdots \\
 #1_{#2}\end{array}\right)}
\newcommand{\mattwo}[4]{\left(\begin{array}{rr} #1 & #2\\ #3
&#4\end{array}\right)}
\newcommand{\mattwoc}[4]{\left(\begin{array}{cc} #1 & #2\\ #3
&#4\end{array}\right)}
\newcommand{\vectwo}[2]{\left(\begin{array}{r} #1 \\ #2\end{array}\right)}
\newcommand{\vectwoc}[2]{\left(\begin{array}{c} #1 \\ #2\end{array}\right)}

\newcommand{\ignore}[1]{}


\newcommand{\inv}{^{-1}}
\newcommand{\CC}{{\cal C}}
\newcommand{\CCone}{\CC^1}
\newcommand{\Span}{{\rm span}}
\newcommand{\rank}{{\rm rank}}
\newcommand{\trace}{{\rm tr}}
\newcommand{\RE}{{\rm Re}}
\newcommand{\IM}{{\rm Im}}
\newcommand{\nulls}{{\rm null\;space}}

\newcommand{\dps}{\displaystyle}
\newcommand{\arraystart}{\renewcommand{\arraystretch}{1.8}}
\newcommand{\arrayfinish}{\renewcommand{\arraystretch}{1.2}}
\newcommand{\Start}[1]{\vspace{0.08in}\noindent {\bf Section~\ref{#1}}}
\newcommand{\exer}[1]{\noindent {\bf \ref{#1}}}
\newcommand{\ans}{}
\newcommand{\matthree}[9]{\left(\begin{array}{rrr} #1 & #2 & #3 \\ #4 & #5 & #6
\\ #7 & #8 & #9\end{array}\right)}
\newcommand{\cvectwo}[2]{\left(\begin{array}{c} #1 \\ #2\end{array}\right)}
\newcommand{\cmatthree}[9]{\left(\begin{array}{ccc} #1 & #2 & #3 \\ #4 & #5 &
#6 \\ #7 & #8 & #9\end{array}\right)}
\newcommand{\vecthree}[3]{\left(\begin{array}{r} #1 \\ #2 \\
#3\end{array}\right)}
\newcommand{\cvecthree}[3]{\left(\begin{array}{c} #1 \\ #2 \\
#3\end{array}\right)}
\newcommand{\cmattwo}[4]{\left(\begin{array}{cc} #1 & #2\\ #3
&#4\end{array}\right)}

\newcommand{\Matrix}[1]{\ensuremath{\left(\begin{array}{rrrrrrrrrrrrrrrrrr} #1 \end{array}\right)}}

\newcommand{\Matrixc}[1]{\ensuremath{\left(\begin{array}{cccccccccccc} #1 \end{array}\right)}}



\renewcommand{\labelenumi}{\theenumi)}
\newenvironment{enumeratea}%
{\begingroup
 \renewcommand{\theenumi}{\alph{enumi}}
 \renewcommand{\labelenumi}{(\theenumi)}
 \begin{enumerate}}
 {\end{enumerate}\endgroup}



\newcounter{help}
\renewcommand{\thehelp}{\thesection.\arabic{equation}}

%\newenvironment{equation*}%
%{\renewcommand\endequation{\eqno (\theequation)* $$}%
%   \begin{equation}}%
%   {\end{equation}\renewcommand\endequation{\eqno \@eqnnum
%$$\global\@ignoretrue}}

%\input{psfig.tex}

\author{Martin Golubitsky and Michael Dellnitz}

%\newenvironment{matlabEquation}%
%{\renewcommand\endequation{\eqno (\theequation*) $$}%
%   \begin{equation}}%
%   {\end{equation}\renewcommand\endequation{\eqno \@eqnnum
% $$\global\@ignoretrue}}

\newcommand{\soln}{\textbf{Solution:} }
\newcommand{\exercap}[1]{\centerline{Figure~\ref{#1}}}
\newcommand{\exercaptwo}[1]{\centerline{Figure~\ref{#1}a\hspace{2.1in}
Figure~\ref{#1}b}}
\newcommand{\exercapthree}[1]{\centerline{Figure~\ref{#1}a\hspace{1.2in}
Figure~\ref{#1}b\hspace{1.2in}Figure~\ref{#1}c}}
\newcommand{\para}{\hspace{0.4in}}

\renewenvironment{solution}{\suppress}{\endsuppress}

\ifxake
\newenvironment{matlabEquation}{\begin{equation}}{\end{equation}}
\else
\newenvironment{matlabEquation}%
{\let\oldtheequation\theequation\renewcommand{\theequation}{\oldtheequation*}\begin{equation}}%
  {\end{equation}\let\theequation\oldtheequation}
\fi

\makeatother


\title{Matlab Commands}

\begin{document}
\begin{abstract}
\end{abstract}
\maketitle

\makeatletter
\newcommand\iflabelexists[2]{%
  \@ifundefined{r@#1}{%
  }{%
    #2
  }%
}
\makeatother

{$\dagger$ indicates an {\tt laode} toolbox command not found in \Matlab.}

%%%%%%%%%%%%%%%%%%%%%%%%%%%%%%%%%%%%%%%%%%%%%%%%%%%%%%%%%%%%%%%%
\iflabelexists{chap:prelim}{
\subsection*{Chapter~\ref{chap:prelim}: Preliminaries}

\begin{center}
{\bf Editing and Number Commands}
\end{center}

\begin{tabbing}
 \hspace{1.1in} \= \\
{\tt quit} \index{\computer!quit}  \> Ends MATLAB session\\ 
{\tt ;} \index{\computer!;} \>
   (a) At end of line the semicolon suppresses echo printing\\ 
  \>      (b) When entering an array the semicolon indicates a new row\\
{\tt $\uparrow$} \index{\computer!$\uparrow$} \>
 Displays previous MATLAB command \\
{\tt []} \> Brackets indicating the beginning and the end of a vector or
	a matrix\\
{\tt x=y} \> Assigns {\tt x} the value of {\tt y}\\
{\tt x(j)} \> Recalls $j^{th}$ entry of vector $x$\\
{\tt A(i,j)}  \> Recalls $i^{th}$ row, $j^{th}$ column of matrix $A$\\
{\tt A(i,:)} \>  Recalls $i^{th}$ row of matrix $A$\index{\computer!:}\\
{\tt A(:,j)} \> Recalls $j^{th}$ column of matrix $A$ 
\end{tabbing}



\begin{center}
{\bf Vector Commands}
\end{center}
 
\begin{tabbing}
 \hspace{1.1in} \= \\
{\tt norm(x)}  \index{\computer!norm}\> The norm or length of a vector $x$ \\  
{\tt dot(x,y)}  \index{\computer!dot}\> Computes the dot product of vectors $x$ and $y$ \\
$\dagger${\tt addvec(x,y)} \index{\computer!addvec} \> Graphics display of vector addition in the plane \\
$\dagger${\tt addvec3(x,y)} \index{\computer!addvec3} \> Graphics display of vector addition in three dimensions 
\end{tabbing}
 


\begin{center}
{\bf Matrix Commands}
\end{center}
 
\begin{tabbing}
 \hspace{1.1in} \= \\
{\tt A$'$}\index{\computer!'}\> (Conjugate) transpose of matrix\\ 
{\tt zeros(m,n)}\index{\computer!zeros} \> 
Creates an $m\times n$ matrix all of whose entries equal $0$  \\ 
{\tt zeros(n)} \> Creates an $n\times n$ matrix all of whose entries equal $0$\\
{\tt diag(x)} \index{\computer!diag}\> Creates an 
$n\times n$ diagonal matrix whose diagonal entries
\\ \> are the components of the vector $x\in\R^n$\\
{\tt eye(n)}\index{\computer!eye}\>  Creates an $n\times n$ identity matrix 
\end{tabbing}



\begin{center}
{\bf Special Numbers in \Matlab}
\end{center}

\begin{tabbing}
 \hspace{1.1in} \= \\
 {\tt pi} \index{\computer!pi}\> The number $\pi=3.1415\ldots$ \\ 
{\tt acos(a)} \index{\computer!acos}\>
  The inverse cosine of the number $a$  
\end{tabbing}

}
%%%%%%%%%%%%%%%%%%%%%%%%%%%%%%%%%%%%%%%%%%%%%%%%%%%%%%%%%%%%%%%%
\iflabelexists{lineq}{
\subsection*{Chapter~\ref{lineq}: Solving Linear Equations}


\begin{center}
{\bf Editing and Number Commands}
\end{center}

\begin{tabbing}
 \hspace{1.1in} \=  \hspace{1.3in} \= \\
	{\tt format}  \index{\computer!format} \>
 Changes the numbers display format  
			to standard five digit format   \\
\> 	{\tt format long}  \index{\computer!format!long} \>
 Changes display format to $15$ digits \\
\>	{\tt format rational}  \index{\computer!format!rational} \>
 Changes display format to rational numbers \\
\> {\tt format short e} \index{\computer!format!e} \>
 Changes display to five digit floating point numbers  
\end{tabbing}

\begin{center}
{\bf Vector Commands}
\end{center}

\begin{tabbing}
 \hspace{1.3in} \= \\
{\tt x.*y} \index{\computer!{\tt .*}}\>
Componentwise multiplication of the vectors {\tt x} and {\tt y}\\
{\tt x./y} \index{\computer!{\tt ./}}\>
Componentwise division of the vectors {\tt x} and {\tt y}\\
{\tt x.\^{}y} \index{\computer!.\^{}}\>
Componentwise exponentiation of the vectors {\tt x} and {\tt y}
\end{tabbing}
 
\begin{center}
{\bf Matrix Commands}
\end{center}
\begin{tabbing}
 \hspace{1.3in} \= \\
	{\tt A([i j],:) = A([j i],:)}  \>  \\
\>  Swaps $i^{th}$ and $j^{th}$ rows of matrix $A$ \\
	{\tt A$\backslash$b} \index{\computer!$\backslash$} \>
 Solves the system of linear equations associated with\\
\> the augmented matrix $(A|b)$ \\
	{\tt x  = linspace(xmin,xmax,N)} \index{\computer!linspace}  \>
\\ \> Generates a vector {\tt x} whose entries are  
	$N$ equally spaced points \\ \> from {\tt xmin} to {\tt xmax} \\
	{\tt x = xmin:xstep:xmax} \>  \\ \>
 Generates a vector whose entries are  
	equally spaced points from {\tt xmin} to {\tt xmax} \\ 
	\> with stepsize {\tt xstep}\\
 {\tt [x,y] = meshgrid(XMIN:XSTEP:XMAX,YMIN:YSTEP:YMAX);} 
		\index{\computer!meshgrid} \> \\
\>  	Generates two vectors $x$ and $y$.  The entries of $x$ are values 
	from {\tt XMIN} to {\tt XMAX} \\
\> 	in steps of {\tt XSTEP}.  Similarly for $y$. \\
	{\tt rand(m,n)}  \index{\computer!rand} \>
 Generates an $m\times n$ matrix whose entries 
	are randomly and uniformly chosen \\
\> 	from the interval $[0,1]$ \\
	{\tt rref(A)}  \index{echelon form}\index{\computer!rref} \>
 Returns the reduced row echelon form of the $m\times n$ 
matrix $A$ \\
%{\tt rrefmovie(A)}  \index{echelon form} \>
% Puts the $m\times n$ matrix $A$ into reduced row echelon 
%form by showing \\
\>  the matrix after each step in the row reduction process \\
{\tt rank(A)} \index{\computer!rank} \>
 Returns the rank of the $m\times n$ matrix $A$ 
\end{tabbing}

\begin{center}
{\bf Graphics Commands}
\end{center}

\begin{tabbing}
 \hspace{1.4in} \= \\

	{\tt plot(x,y)}  \index{\computer!plot} \>
 Plots a graph connecting the points $(x(i),y(i))$ 
	in sequence \\

	{\tt xlabel('labelx')} \index{\computer!xlabel}  \>
 Prints {\tt labelx} along the $x$ axis   \\

	{\tt ylabel('labely')}  \index{\computer!ylabel} \>
 Prints {\tt labely} along the $y$ axis \\

	{\tt surf(x,y,z)}  \index{\computer!surf} \>
 Plots a three dimensional graph of $z(j)$ as 
	a function of $x(j)$ and $y(j)$  \\

	{\tt hold on}	\index{\computer!hold} \>
 Instructs MATLAB to {\em add\/} new graphics to the 
	previous figure \\

	{\tt hold off}  \>
 Instructs MATLAB to {\em clear\/} figure when new graphics 
	are generated \\

	{\tt grid} 	\index{\computer!grid} \>
 Toggles grid lines on a figure \\

	{\tt axis('equal')} \index{\computer!axis('equal')} \>
 Forces MATLAB to use equal $x$ and $y$ 
	dimensions \\

	{\tt view([a b c])} \index{\computer!view} \>
 Sets viewpoint from which an observer sees the current 3-D plot \\

	{\tt zoom} \index{\computer!zoom} \>
 Zoom in and out on 2-D plot.  On each mouse click, 
        axes change by a factor of 2 

\end{tabbing}

\begin{center}
{\bf Special Numbers and Functions in \Matlab}
\end{center}

\begin{tabbing}
 \hspace{1.1in} \= \\

	{\tt exp(x)}  \index{\computer!exp(1)} \>
 The number $e^x$ where $e={\tt exp(1)}=2.7182\ldots$ \\

	{\tt sqrt(x)} \index{\computer!sqrt} \>
 The number $\sqrt{x}$ \\

	{\tt i}  \index{\computer!i} \>
 The number $\sqrt{-1}$ 
\end{tabbing}

}
%%%%%%%%%%%%%%%%%%%%%%%%%%%%%%%%%%%%%%%%%%%%%%%%%%%%%%%%%%%%%%%%
\iflabelexists{chap:matrices}{
\subsection*{Chapter~\ref{chap:matrices}: Matrices and Linearity}



\begin{center}
{\bf Matrix Commands}
\end{center}
 
\begin{tabbing}
 \hspace{1.1in} \= \\

     {\tt A*x}  \index{\computer!*} \>
 Performs the matrix vector product of the matrix $A$
                        with the vector $x$   \\
 
     {\tt A*B}  \index{\computer!*} \>
 Performs the matrix product of the matrices 
			$A$ and $B$   \\
 
     {\tt size(A)}  \index{\computer!size} \>
 Determines the numbers of rows and columns of a matrix $A$   \\

     {\tt inv(A)}  \index{\computer!inv} \>
 Computes the inverse of a matrix $A$ 

\end{tabbing}

\begin{center}
{\bf Program for Matrix Mappings}
\end{center}
 
 \begin{tabbing}
 \hspace{1.1in} \= \\

    $\dagger${\tt map}  \index{\computer!map} \>
 Allows the graphic exploration of planar matrix mappings   

\end{tabbing}

}
%%%%%%%%%%%%%%%%%%%%%%%%%%%%%%%%%%%%%%%%%%%%%%%%%%%%%%%%%%%%%%%%
\iflabelexists{chap:SolveOdes}{
\subsection*{Chapter~\ref{chap:SolveOdes}: Solving Ordinary Differential Equations}

\begin{center}
{\bf Special Functions in \Matlab}
\end{center}

\begin{tabbing}
 \hspace{1.1in} \= \\

	{\tt sin(x)}  \index{\computer!sin} \>
 The number $\sin(x)$ \\

	{\tt cos(x)} \index{\computer!cos} \>
 The number $\cos(x)$
\end{tabbing}
 

\begin{center}
{\bf Matrix Commands}
\end{center}
 
\begin{tabbing}
 \hspace{1.1in} \= \\

     {\tt eig(A)}  \index{\computer!eig} \>
 Computes the eigenvalues of the matrix $A$ \\

     {\tt null(A)}  \index{\computer!null} \>
 Computes the solutions to the homogeneous equation $Ax=0$ \\
\end{tabbing}



\begin{center}
{\bf Programs for the Solution of ODEs}
\end{center}
 
\begin{tabbing}
 \hspace{1.1in} \= \\

     $\dagger${\tt dfield8}  \index{\computer!dfield5} \>
 Displays graphs of solutions to differential
                        equations   \\
 
       $\dagger${\tt pline}  \index{\computer!pline} \>
 Dynamic illustration of phase line plots for single\\
\>			autonomous differential equations \\
 
       $\dagger${\pplane}  \index{\computer!pplane5} \>
 Displays phase space and time series plots for systems of
			autonomous differential equations 
\end{tabbing}

}
%%%%%%%%%%%%%%%%%%%%%%%%%%%%%%%%%%%%%%%%%%%%%%%%%%%%%%%%%%%%%%%%
\iflabelexists{chap:Planar}{
\subsection*{Chapter~\ref{Chap:Planar}: Closed Form Solutions for Planar ODEs}

\begin{center}
{\bf Matrix Commands}
\end{center}
 
\begin{tabbing}
 \hspace{1.1in} \= \\

     {\tt expm(A)}  \index{\computer!expm} \>
 Computes the matrix exponential of the matrix $A$  

\end{tabbing}

\begin{center}
{\bf Functions in \Matlab}
\end{center}

\begin{tabbing}
 \hspace{1.1in} \= \\

        {\tt prod(1:n)}  \index{\computer!prod} \>
Computes the product of the integers $1,\ldots,n$
\end{tabbing}

}
%%%%%%%%%%%%%%%%%%%%%%%%%%%%%%%%%%%%%%%%%%%%%%%%%%%%%%%%%%%%%%%%
\iflabelexists{C:D&E}{
\subsection*{Chapter~\ref{C:D&E}: Determinants and Eigenvalues}

\begin{center}
{\bf Matrix Commands}
\end{center}
 
\begin{tabbing}
 \hspace{1.2in} \= \\

     {\tt det(A)}  \index{\computer!det} \>
 Computes the determinant of the matrix $A$   \\

     {\tt poly(A)}  \index{\computer!poly} \>
 Returns the characteristic polynomial 
			of the matrix $A$   \\
     {\tt sum(v)}  \index{\computer!sum} \>
 Computes the sum of the components of the vector $v$   \\

     {\tt trace(A)}  \index{\computer!trace} \>
 Computes the trace of the matrix $A$  \\

     {\tt [V,D] = eig(A)}  \index{\computer!eig} \>
 Computes eigenvectors and eigenvalues of the matrix $A$ 
\end{tabbing}


}

\iflabelexists{C:LMCC}{

\subsection*{Chapter~\ref{C:LMCC}: Linear Maps and Changes of Coordinates}


\begin{center}
{\bf Vector Commands}
\end{center}
 
\begin{tabbing}
 \hspace{1.1in} \= \\

     $\dagger${\tt bcoord}  \index{\computer!bcoord} \>
 Geometric illustration of planar coordinates
			by vector addition   \\
 
     $\dagger${\tt ccoord}  \index{\computer!ccoord} \>
 Geometric illustration of coordinates relative to two bases   

\end{tabbing}

}
%%%%%%%%%%%%%%%%%%%%%%%%%%%%%%%%%%%%%%%%%%%%%%%%%%%%%%%%%%%%%%%%
\iflabelexists{Chap:LinTrans}{
\subsection*{Chapter~\ref{Chap:LinTrans}: Orthogonality}


\begin{center}
{\bf Matrix Commands}
\end{center}
 
\begin{tabbing}
 \hspace{1.3in} \= \\

     {\tt orth(A)}  \index{\computer!orth} \>
 Computes an orthonormal basis for the column space
		of the matrix $A$ \\ 
     {\tt [Q,R] = qr(A,0)}  \index{\computer!qr} \>
 Computes the $QR$ decomposition of the matrix $A$

\end{tabbing}
 
\begin{center}
{\bf Graphics Commands}
\end{center}
 
\begin{tabbing}
 \hspace{1.2in} \= \\

     {\tt axis([xmin,xmax,ymin,ymax])} \index{\computer!axis} \> \\
 \> Forces MATLAB to use in a twodimensional plot the intervals \\
 \> {\tt [xmin,xmax]} resp.\ {\tt [ymin,ymax]}
    labeling the $x$- resp.\ $y$-axis \\

     {\tt plot(x,y,'o')}  \index{\computer!plot} \>
 Same as {\tt plot} but now the points $(x(i),y(i))$
	are marked by \\ \> circles and no longer connected in 
	sequence
\end{tabbing} 

}

%%%%%%%%%%%%%%%%%%%%%%%%%%%%%%%%%%%%%%%%%%%%%%%%%%%%%%%%%%%%%%%%
\iflabelexists{Chap:NPS}{

\ignore{

\subsection*{Chapter~\ref{C:NPS}: Autonomous Planar Nonlinear Systems}

\begin{center}
{\bf Matrix Commands}
\end{center}
 

\begin{tabbing}
 \hspace{1.2in} \= \\
     {\tt [V,D] = eig(A)}  \index{\computer!eig} \>
 Computes eigenvectors and eigenvalues of the 
			matrix $A$ 
\end{tabbing}

}
}

%%%%%%%%%%%%%%%%%%%%%%%%%%%%%%%%%%%%%%%%%%%%%%%%%%%%%%%%%%%%%%%%
\iflabelexists{C:HDeigenvalues}{

\subsection*{Chapter~\ref{C:HDeigenvalues}: Matrix Normal Forms}


\begin{center}
{\bf Vector Commands}
\end{center}
 
\begin{tabbing}
 \hspace{1.2in} \= \\

     {\tt real(v)}  \index{\computer!real} \>
 Returns the vector of the real parts of the components \\
\> of the vector $v$ \\

     {\tt imag(v)}  \index{\computer!imag} \>
 Returns the vector of the imaginary parts of the components \\
\> of the vector $v$ 


\end{tabbing}

}
%%%%%%%%%%%%%%%%%%%%%%%%%%%%%%%%%%%%%%%%%%%%%%%%%%%%%%%%%%%%%%%%
\iflabelexists{C:HDS}{
\subsection*{Chapter~\ref{C:HDS}: Higher Dimensional Systems}



\begin{center}
{\bf Commands for the Solution of Initial Value Problems}
\end{center}
 
\begin{tabbing}
 \hspace{1.1in} \= \\

     {\tt [t,x]=ode45('fun',[t0 te],x0)}  \index{\computer!ode45} \> \\
\>  Computes the solution to differential equation with
			right hand side {\tt fun} \\
\>			on interval {\tt [t0 te]} with the initial 
			condition {\tt x0} at time {\tt t0}   \\
 
     {\tt odeset}  \index{\computer!odeset} \> 
    Displays a list of options that can be used in {\tt ode45}\\

     {\tt lorenz}  \index{\computer!lorenz} \> 
    Displays a dynamic simulation of a solution to the Lorenz equations

\end{tabbing}


\begin{center}
{\bf Graphics Commands}
\end{center}
 
\begin{tabbing}
 \hspace{1.4in} \= \\

     {\tt subplot(m,n,p)}  \index{\computer!subplot} \>
 Activates the $p^{th}$ subfigure in a matrix of $m\times n$
	subfigures   \\
 
     {\tt plot3(x,y,z)}  \index{\computer!plot3} \>
 Plots curve in three dimensional space connecting \\ 
	\> the points $(x(i),y(i),z(i))$ in sequence   \\

	{\tt zlabel('labelz')} \index{\computer!zlabel}  \>
 Prints {\tt labelz} along the $z$ axis   \\

     {\tt clf}  \index{\computer!clf} \>
 Clears the previous graphics   

\end{tabbing}

\begin{center}
{\bf Special Functions in \Matlab}
\end{center}
 
\begin{tabbing}
 \hspace{1.2in} \= \\

     {\tt abs(v)}  \index{\computer!abs} \>
 Computes the absolute value of the components of the vector {\tt v}\\
	\> and returns the answer in a vector of the same length

\end{tabbing}

}
%%%%%%%%%%%%%%%%%%%%%%%%%%%%%%%%%%%%%%%%%%%%%%%%%%%%%%%%%%%%%%%%
\iflabelexists{C:LDE}{
\subsection*{Chapter~\ref{C:LDE}: Linear Differential Equations}



\begin{center}
{\bf Commands for Polynomials}
\end{center}
 
\begin{tabbing}
 \hspace{1.1in} \= \\

     {\tt roots(a)}  \index{\computer!roots} \>
 Computes the roots of the polynomial with coefficients
			specified in the vector {\tt a}  

\end{tabbing}


}
%%%%%%%%%%%%%%%%%%%%%%%%%%%%%%%%%%%%%%%%%%%%%%%%%%%%%%%%%%%%%%%%
\iflabelexists{C:LT}{
\subsection*{Chapter~\ref{C:LT}: Laplace Transforms}


\begin{center}
{\bf Commands for Polynomials}
\end{center}
 
\begin{tabbing}
 \hspace{1.1in} \= \\

     {\tt residue(p,q)}  \index{\computer!residue} \> 
 Determines partial fractions expansion of
		{\tt p/q} where {\tt p} and {\tt q} are polynomials 
 
\end{tabbing}


}
%%%%%%%%%%%%%%%%%%%%%%%%%%%%%%%%%%%%%%%%%%%%%%%%%%%%%%%%%%%%%%%%
\iflabelexists{chap:SingleOdes}{
\subsection*{Chapter~\ref{chap:SingleOdes}: Additional Techniques for Solving ODEs}



\begin{center}
{\bf Graphics Commands}
\end{center}
 
\begin{tabbing}
 \hspace{1.2in} \= \\

     {\tt contour(F)}  \index{\computer!contour} \>
 Plots contour lines of the function {\tt F} \\
 
     {\tt contour(x,y,F)}  \index{\computer!contour} \>
 Plots contour lines of the function {\tt F}
	where the axis scales are given by {\tt x} and {\tt y} \\

     {\tt clabel(c)}  \index{\computer!clabel} \>
 Labels contour lines obtained by {\tt contour}
	by their actual levels
 


\end{tabbing}

}
%%%%%%%%%%%%%%%%%%%%%%%%%%%%%%%%%%%%%%%%%%%%%%%%%%%%%%%%%%%%%%%%
\iflabelexists{ch:NumSolODE}{
\subsection*{Chapter~\ref{ch:NumSolODE}: Numerical Solutions of ODEs}
 
\begin{center}
{\bf Graphics Commands}
\end{center}
 
\begin{tabbing}
 \hspace{1.2in} \= \\
 
     {\tt plot(x,y,'--')}  \index{\computer!plot} \>
 Plots a graph connecting the points $(x(i),y(i))$
        in sequence and connects \\ \> subsequent points with a dashed line\\

     {\tt plot(x,y,'+')}  \index{\computer!plot} \>
 Plots a graph connecting the points $(x(i),y(i))$
        in sequence and \\ \>  marks each point with a `+'\\

     {\tt plot(x,y,'x')}  \index{\computer!plot} \>
 Plots a graph connecting the points $(x(i),y(i))$
        in sequence and \\ \>  marks each point with an `x'  

 
\end{tabbing}


\begin{center}
{\bf \Matlab Function}
\end{center}
\begin{tabbing}
 \hspace{1.1in} \= \\

     {\tt round(x)}  \index{\computer!round} \>
Rounds the number $x$ towards the nearest integer.

\end{tabbing}

\begin{center}
{\bf Vector Commands}
\end{center}
\begin{tabbing}
 \hspace{1.1in} \= \\
 
     {\tt diff(v)}  \index{\computer!diff} \>
Compute the differences of consecutive entries in the vector {\tt v}\\ 
     {\tt length(v)}  \index{\computer!length} \>
The length of the vector {\tt v}
 
\end{tabbing}
 

\begin{center}
{\bf Programming Commands}
\end{center}
\begin{tabbing}
 \hspace{1.1in} \= \\

     {\tt for k = 1:K} \index{\computer!for} \index{\computer!for\ldots end} \>\\
	\hspace{0.1in} MATLAB {\tt commands} \> \\
     {\tt end} \> \\
\> The MATLAB commands between {\tt for k = 1:K} and {\tt end}\\ \>
are done $K$ times where $k$ varies from $1,2,\ldots,K$.

\end{tabbing}
}

\end{document}
