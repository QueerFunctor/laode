\documentclass{ximera}

 

\usepackage{epsfig}

\graphicspath{
  {./}
  {figures/}
}

\usepackage{morewrites}
\makeatletter
\newcommand\subfile[1]{%
\renewcommand{\input}[1]{}%
\begingroup\skip@preamble\otherinput{#1}\endgroup\par\vspace{\topsep}
\let\input\otherinput}
\makeatother

\newcommand{\includeexercises}{\directlua{dofile("/home/jim/linearAlgebra/laode/exercises.lua")}}

%\newcounter{ccounter}
%\setcounter{ccounter}{1}
%\newcommand{\Chapter}[1]{\setcounter{chapter}{\arabic{ccounter}}\chapter{#1}\addtocounter{ccounter}{1}}

%\newcommand{\section}[1]{\section{#1}\setcounter{thm}{0}\setcounter{equation}{0}}

%\renewcommand{\theequation}{\arabic{chapter}.\arabic{section}.\arabic{equation}}
%\renewcommand{\thefigure}{\arabic{chapter}.\arabic{figure}}
%\renewcommand{\thetable}{\arabic{chapter}.\arabic{table}}

%\newcommand{\Sec}[2]{\section{#1}\markright{\arabic{ccounter}.\arabic{section}.#2}\setcounter{equation}{0}\setcounter{thm}{0}\setcounter{figure}{0}}

\newcommand{\Sec}[2]{\section{#1}}

\setcounter{secnumdepth}{2}
%\setcounter{secnumdepth}{1} 

%\newcounter{THM}
%\renewcommand{\theTHM}{\arabic{chapter}.\arabic{section}}

\newcommand{\trademark}{{R\!\!\!\!\!\bigcirc}}
%\newtheorem{exercise}{}

\newcommand{\dfield}{{\sf dfield9}}
\newcommand{\pplane}{{\sf pplane9}}

\newcommand{\EXER}{\section*{Exercises}}%\vspace*{0.2in}\hrule\small\setcounter{exercise}{0}}
\newcommand{\CEXER}{}%\vspace{0.08in}\begin{center}Computer Exercises\end{center}}
\newcommand{\TEXER}{} %\vspace{0.08in}\begin{center}Hand Exercises\end{center}}
\newcommand{\AEXER}{} %\vspace{0.08in}\begin{center}Hand Exercises\end{center}}

% BADBAD: \newcommand{\Bbb}{\bf}

\newcommand{\R}{\mbox{$\Bbb{R}$}}
\newcommand{\C}{\mbox{$\Bbb{C}$}}
\newcommand{\Z}{\mbox{$\Bbb{Z}$}}
\newcommand{\N}{\mbox{$\Bbb{N}$}}
\newcommand{\D}{\mbox{{\bf D}}}
\usepackage{amssymb}
%\newcommand{\qed}{\hfill\mbox{\raggedright$\square$} \vspace{1ex}}
%\newcommand{\proof}{\noindent {\bf Proof:} \hspace{0.1in}}

\newcommand{\setmin}{\;\mbox{--}\;}
\newcommand{\Matlab}{{M\small{AT\-LAB}} }
\newcommand{\Matlabp}{{M\small{AT\-LAB}}}
\newcommand{\computer}{\Matlab Instructions}
\newcommand{\half}{\mbox{$\frac{1}{2}$}}
\newcommand{\compose}{\raisebox{.15ex}{\mbox{{\scriptsize$\circ$}}}}
\newcommand{\AND}{\quad\mbox{and}\quad}
\newcommand{\vect}[2]{\left(\begin{array}{c} #1_1 \\ \vdots \\
 #1_{#2}\end{array}\right)}
\newcommand{\mattwo}[4]{\left(\begin{array}{rr} #1 & #2\\ #3
&#4\end{array}\right)}
\newcommand{\mattwoc}[4]{\left(\begin{array}{cc} #1 & #2\\ #3
&#4\end{array}\right)}
\newcommand{\vectwo}[2]{\left(\begin{array}{r} #1 \\ #2\end{array}\right)}
\newcommand{\vectwoc}[2]{\left(\begin{array}{c} #1 \\ #2\end{array}\right)}

\newcommand{\ignore}[1]{}


\newcommand{\inv}{^{-1}}
\newcommand{\CC}{{\cal C}}
\newcommand{\CCone}{\CC^1}
\newcommand{\Span}{{\rm span}}
\newcommand{\rank}{{\rm rank}}
\newcommand{\trace}{{\rm tr}}
\newcommand{\RE}{{\rm Re}}
\newcommand{\IM}{{\rm Im}}
\newcommand{\nulls}{{\rm null\;space}}

\newcommand{\dps}{\displaystyle}
\newcommand{\arraystart}{\renewcommand{\arraystretch}{1.8}}
\newcommand{\arrayfinish}{\renewcommand{\arraystretch}{1.2}}
\newcommand{\Start}[1]{\vspace{0.08in}\noindent {\bf Section~\ref{#1}}}
\newcommand{\exer}[1]{\noindent {\bf \ref{#1}}}
\newcommand{\ans}{}
\newcommand{\matthree}[9]{\left(\begin{array}{rrr} #1 & #2 & #3 \\ #4 & #5 & #6
\\ #7 & #8 & #9\end{array}\right)}
\newcommand{\cvectwo}[2]{\left(\begin{array}{c} #1 \\ #2\end{array}\right)}
\newcommand{\cmatthree}[9]{\left(\begin{array}{ccc} #1 & #2 & #3 \\ #4 & #5 &
#6 \\ #7 & #8 & #9\end{array}\right)}
\newcommand{\vecthree}[3]{\left(\begin{array}{r} #1 \\ #2 \\
#3\end{array}\right)}
\newcommand{\cvecthree}[3]{\left(\begin{array}{c} #1 \\ #2 \\
#3\end{array}\right)}
\newcommand{\cmattwo}[4]{\left(\begin{array}{cc} #1 & #2\\ #3
&#4\end{array}\right)}

\newcommand{\Matrix}[1]{\ensuremath{\left(\begin{array}{rrrrrrrrrrrrrrrrrr} #1 \end{array}\right)}}

\newcommand{\Matrixc}[1]{\ensuremath{\left(\begin{array}{cccccccccccc} #1 \end{array}\right)}}



\renewcommand{\labelenumi}{\theenumi)}
\newenvironment{enumeratea}%
{\begingroup
 \renewcommand{\theenumi}{\alph{enumi}}
 \renewcommand{\labelenumi}{(\theenumi)}
 \begin{enumerate}}
 {\end{enumerate}\endgroup}



\newcounter{help}
\renewcommand{\thehelp}{\thesection.\arabic{equation}}

%\newenvironment{equation*}%
%{\renewcommand\endequation{\eqno (\theequation)* $$}%
%   \begin{equation}}%
%   {\end{equation}\renewcommand\endequation{\eqno \@eqnnum
%$$\global\@ignoretrue}}

%\input{psfig.tex}

\author{Martin Golubitsky and Michael Dellnitz}

%\newenvironment{matlabEquation}%
%{\renewcommand\endequation{\eqno (\theequation*) $$}%
%   \begin{equation}}%
%   {\end{equation}\renewcommand\endequation{\eqno \@eqnnum
% $$\global\@ignoretrue}}

\newcommand{\soln}{\textbf{Solution:} }
\newcommand{\exercap}[1]{\centerline{Figure~\ref{#1}}}
\newcommand{\exercaptwo}[1]{\centerline{Figure~\ref{#1}a\hspace{2.1in}
Figure~\ref{#1}b}}
\newcommand{\exercapthree}[1]{\centerline{Figure~\ref{#1}a\hspace{1.2in}
Figure~\ref{#1}b\hspace{1.2in}Figure~\ref{#1}c}}
\newcommand{\para}{\hspace{0.4in}}

\renewenvironment{solution}{\suppress}{\endsuppress}

\ifxake
\newenvironment{matlabEquation}{\begin{equation}}{\end{equation}}
\else
\newenvironment{matlabEquation}%
{\let\oldtheequation\theequation\renewcommand{\theequation}{\oldtheequation*}\begin{equation}}%
  {\end{equation}\let\theequation\oldtheequation}
\fi

\makeatother


\title{Properties of Matrix Multiplication}

\begin{document}
\begin{abstract}
\end{abstract}
\maketitle

 \label{S:4.7}
\index{matrix!multiplication}

In this section we discuss the facts that matrix multiplication is
associative (but not commutative) and that certain distributive
properties hold.  We also discuss how matrix multiplication is
performed in \Matlab.

\subsubsection*{Matrix Multiplication is Associative}
\index{associative}

\begin{theorem} \label{assoc}
Matrix multiplication is associative.  That is, let $A$ be an
$m\times n$ matrix, let $B$ be a $n\times p$ matrix, and let $C$
be a $p\times q$ matrix.  Then
\[
(AB)C = A(BC).
\]
\end{theorem}

\begin{proof} Begin by observing that composition of mappings is always
associative.  In symbols, let $f:\R^n\to\R^m$, $g:\R^p\to\R^n$,
and $h:\R^q\to\R^p$.  Then
\begin{eqnarray*}
f\compose (g\compose h)(x) & = & f[(g\compose h)(x)] \\
  & = & f[g(h(x))] \\
  & = & (f\compose g)(h(x)) \\
  & = & [(f\compose g)\compose h](x).
\end{eqnarray*}
It follows that
\[
f\compose (g\compose h) = (f\compose g)\compose h.
\]

We can apply this result to linear mappings.  Thus
\[
L_A\compose (L_B\compose L_C) = (L_A\compose L_B)\compose L_C.
\]
Since
\[
L_{A(BC)} = L_A\compose L_{BC} = L_A\compose (L_B\compose L_C)
\]
and
\[
L_{(AB)C} = L_{AB}\compose L_C = (L_A\compose L_B)\compose L_C,
\]
it follows that
\[
L_{A(BC)} = L_{(AB)C},
\]
and
\[
A(BC) = (AB)C.
\]
\end{proof}

It is worth convincing yourself that Theorem~\ref{assoc} has
content by verifying by hand that matrix multiplication of
$2\times 2$ matrices is associative.

\subsubsection*{Matrix Multiplication is Not Commutative}
\index{commutative}

Although matrix multiplication is associative, it is {\em not\/}
commutative. This statement is trivially true when the matrix
$AB$ is defined while that matrix $BA$ is not.  Suppose, for
example, that $A$ is a $2\times 3$ matrix and that $B$ is a
$3\times 4$ matrix.  Then $AB$ is a $2\times 4$ matrix, while
the multiplication $BA$ makes no sense whatsoever.

More importantly, suppose that $A$ and $B$ are both $n\times n$
square matrices.  Then $AB=BA$ is generally not valid.  For
example, let
\[
A=\left(\begin{array}{cc} 1 & 0 \\ 0 & 0 \end{array}\right) \AND
B =\left(\begin{array}{cc} 0 & 1 \\ 0 & 0 \end{array}\right).
\]
Then
\[
AB =\left(\begin{array}{cc} 0 & 1 \\ 0 & 0 \end{array}\right)
\AND BA=\left(\begin{array}{cc} 0 & 0 \\ 0 & 0 \end{array}\right).
\]
So $AB\neq BA$.  In certain cases it does happen that $AB=BA$.
For example, when $B=I_n$,
\[
AI_n = A = I_nA.
\]
But these cases are rare.

\subsubsection*{Additional Properties of Matrix Multiplication}
\index{distributive}

Recall that if $A=(a_{ij})$ and $B=(b_{ij})$ are both $m\times n$ 
matrices, then $A+B$ is the $m\times n$ matrix $(a_{ij}+b_{ij})$. 
We now enumerate several properties of matrix multiplication.

\begin{itemize}

\item	Let $A$ and $B$ be $m\times n$ matrices and let $C$ be an $n\times p$
matrix.  Then
\[
(A+B)C = AC + BC.
\]
Similarly, if $D$ is a $q\times m$ matrix, then
\[
D(A+B) = DA + DB.
\]
So matrix multiplication distributes across matrix addition.

\item	If $\alpha$ and $\beta$ are scalars, then
\[
(\alpha+\beta)A = \alpha A + \beta A.
\]
So addition distributes with scalar multiplication.

\item	Scalar multiplication and matrix multiplication satisfy:
\[
(\alpha A)C = \alpha (AC).
\]
\end{itemize}

\subsubsection*{Matrix Multiplication and Transposes}
\index{matrix!transpose}

Let $A$ be an $m\times n$ matrix and let $B$ be an $n\times p$
matrix, so that the matrix product $AB$ is defined and $AB$ is an
$m\times p$ matrix.  Note that $A^t$
is an $n\times m$ matrix and that $B^t$ is a $p\times n$ matrix, so
that in general the product $A^tB^t$ is {\em not\/} defined.  However,
the product $B^tA^t$ is defined and is an $p\times m$ matrix, as is
the matrix $(AB)^t$.  We claim that
\begin{equation}  \label{e:transposeprod}
(AB)^t = B^tA^t.
\end{equation}
We verify this claim by direct computation.  The $(i,k)^{th}$ entry
in $(AB)^t$ is the $(k,i)^{th}$ entry in $AB$.   That entry is:
\[
\sum_{j=1}^{n} a_{kj}b_{ji}.
\]
The $(i,k)^{th}$ entry in $B^tA^t$ is:
\[
\sum_{j=1}^n b^t_{ij}a^t_{jk},
\]
where $a^t_{jk}$ is the $(j,k)^{th}$ entry in $A^t$ and $b^t_{ij}$
is the $(i,j)^{th}$ entry in $B^t$.  It follows from the definition
of transpose that the $(i,k)^{th}$ entry in $B^tA^t$ is:
\[
\sum_{j=1}^n b_{ji}a_{kj} = \sum_{j=1}^n a_{kj}b_{ji},
\]
which verifies the claim.


\subsubsection*{Matrix Multiplication in \Matlab}
\index{matrix!multiplication!in \protect\Matlab}

Let us now explain how matrix multiplication works in \Matlabp.
We load the matrices
\begin{equation*}  \label{examp_AB}
A=\left(\begin{array}{rrr} -5  &  2  &  0\\
               -1  &  1  & -4\\
               -4  &  4  &  2\\
               -1  &  3  & -1 \end{array}\right) \AND
B =\left(\begin{array}{rrrrr}      2  & -2  & -2  &  5  &  5\\
                    4  & -5  &  1  & -1  &  2\\
                    3  &  2  &  3  & -3  &  3
 \end{array}\right)
\end{equation*}%
by typing
\begin{verbatim}
e3_6_2
\end{verbatim}
Now the command {\tt C = A*B} \index{\computer!*} asks \Matlab to compute
the matrix $C$ as the product of $A$ and $B$.  We obtain
\begin{verbatim}
C =
    -2     0    12   -27   -21
   -10   -11    -9     6   -15
    14    -8    18   -30    -6
     7   -15     2    -5    -2
\end{verbatim}
Let us confirm this result by another computation.  As we have
seen above the $4^{th}$ column of $C$ should be given by the
product of $A$ with the $4^{th}$ column of $B$.  Indeed, if we
perform this computation and type
\begin{verbatim}
A*B(:,4)
\end{verbatim}
the result is
\begin{verbatim}
ans =
   -27
     6
   -30
    -5
\end{verbatim}
which is precisely the $4^{th}$ column of $C$.

\Matlab also recognizes when a matrix multiplication of two
matrices is not defined.  For example, the product of
the $3\times 5$ matrix $B$ with the $4\times 3$ matrix $A$
is not defined, and if we type {\tt B*A} then we obtain the
error message
\begin{verbatim}
??? Error using ==> *
Inner matrix dimensions must agree.
\end{verbatim}
We remark that the size of a matrix $A$ can be seen using
the \Matlab command {\tt size}\index{\computer!size}.
For example, the command {\tt size(A)} leads to
\begin{verbatim}
ans =
     4     3
\end{verbatim}
reflecting the fact that $A$ is a matrix with four
rows and three columns.



\EXER


\TEXER


\begin{exercise} \label{c4.7.2.2}
Let $A$ be an $m\times n$ matrix.  Show that the matrices $A A^t$ and
$A^t A$ are symmetric.
\end{exercise}


\begin{exercise} \label{c4.7.3}
Let
\[
A=\mattwo{1}{2}{-1}{-1} \AND B =\mattwo{2}{3}{1}{4}.
\]
Compute $AB$ and $B^tA^t$.  Verify that $(AB)^t=B^tA^t$ for these
matrices $A$ and $B$.
\end{exercise}

\begin{exercise} \label{c4.7.4}
Let
\[
A = \left(\begin{array}{ccc} 0 & 1 & 0\\ 0 & 0 & 1 \\ 0 & 0 & 0 \end{array}
\right).
\]
Compute $B=I+A+\frac{1}{2}A^2$ and $C=I+tA+\frac{1}{2}(tA)^2$.
\end{exercise}

\begin{exercise} \label{c4.7.5}
Let
\[
I = \mattwo{1}{0}{0}{1} \AND J =\mattwo{0}{-1}{1}{0}.
\]
\begin{itemize}
\item[(a)] Show that $J^2=-I$.
\item[(b)] Evaluate $(aI+bJ)(cI+dJ)$ in terms of $I$ and $J$.
\end{itemize}
\end{exercise}

\begin{exercise} \label{c4.7.8}
Recall that a square matrix $C$ is {\em upper triangular\/} if $c_{ij}=0$
when $i>j$.  Show that the matrix product of two upper triangular $n\times n$
matrices is also upper triangular.
\end{exercise}


\CEXER

\noindent In Exercises~\ref{c4.7.0a} -- \ref{c4.7.0c} use \Matlab to
verify that $(A+B)C = AC+BC$ for the given matrices.
\begin{exercise}  \label{c4.7.0a}
$A=\mattwo{0}{2}{2}{1}$, $B=\mattwo{-2}{1}{3}{0}$ and $C=\mattwo{2}{-1}{1}{5}$
\end{exercise}
\begin{exercise}  \label{c4.7.0b}
$A=\mattwo{12}{-2}{3}{1}$,
$B=\mattwo{8}{-20}{3}{10}$ and
$C=\left(\begin{array}{rrr} 10 & 2 & 4\\ 2 & 13 & -4 \end{array}\right)$
\end{exercise}
\begin{exercise}  \label{c4.7.0c}
$A=\left(\begin{array}{rr} 6 & 1 \\ 3 & 20 \\ -5 & 3\end{array}\right)$,
$B=\left(\begin{array}{rr} 2 & -10 \\ 5 & 0 \\ 3 & 1\end{array}\right)$ and
$C=\mattwo{-2}{10}{12}{10}$
\end{exercise}

\begin{exercise} \label{c4.7.2}
Use the {\tt rand(3,3)} command in \Matlab to choose five pairs of
$3\times 3$ matrices $A$ and $B$ at random.  Compute $AB$ and $BA$
using \Matlab to see that in general these matrix products are unequal.
\end{exercise}

\begin{exercise} \label{c4.7.2.1}
Experimentally, find two symmetric $2\times 2$ matrices $A$ and $B$ for
which the matrix product $AB$ is {\em not\/} symmetric.
\end{exercise}



\end{document}
