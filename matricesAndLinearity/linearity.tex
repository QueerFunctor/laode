\documentclass{ximera}

 

\usepackage{epsfig}

\graphicspath{
  {./}
  {figures/}
}

\usepackage{morewrites}
\makeatletter
\newcommand\subfile[1]{%
\renewcommand{\input}[1]{}%
\begingroup\skip@preamble\otherinput{#1}\endgroup\par\vspace{\topsep}
\let\input\otherinput}
\makeatother

\newcommand{\includeexercises}{\directlua{dofile("/home/jim/linearAlgebra/laode/exercises.lua")}}

%\newcounter{ccounter}
%\setcounter{ccounter}{1}
%\newcommand{\Chapter}[1]{\setcounter{chapter}{\arabic{ccounter}}\chapter{#1}\addtocounter{ccounter}{1}}

%\newcommand{\section}[1]{\section{#1}\setcounter{thm}{0}\setcounter{equation}{0}}

%\renewcommand{\theequation}{\arabic{chapter}.\arabic{section}.\arabic{equation}}
%\renewcommand{\thefigure}{\arabic{chapter}.\arabic{figure}}
%\renewcommand{\thetable}{\arabic{chapter}.\arabic{table}}

%\newcommand{\Sec}[2]{\section{#1}\markright{\arabic{ccounter}.\arabic{section}.#2}\setcounter{equation}{0}\setcounter{thm}{0}\setcounter{figure}{0}}

\newcommand{\Sec}[2]{\section{#1}}

\setcounter{secnumdepth}{2}
%\setcounter{secnumdepth}{1} 

%\newcounter{THM}
%\renewcommand{\theTHM}{\arabic{chapter}.\arabic{section}}

\newcommand{\trademark}{{R\!\!\!\!\!\bigcirc}}
%\newtheorem{exercise}{}

\newcommand{\dfield}{{\sf dfield9}}
\newcommand{\pplane}{{\sf pplane9}}

\newcommand{\EXER}{\section*{Exercises}}%\vspace*{0.2in}\hrule\small\setcounter{exercise}{0}}
\newcommand{\CEXER}{}%\vspace{0.08in}\begin{center}Computer Exercises\end{center}}
\newcommand{\TEXER}{} %\vspace{0.08in}\begin{center}Hand Exercises\end{center}}
\newcommand{\AEXER}{} %\vspace{0.08in}\begin{center}Hand Exercises\end{center}}

% BADBAD: \newcommand{\Bbb}{\bf}

\newcommand{\R}{\mbox{$\Bbb{R}$}}
\newcommand{\C}{\mbox{$\Bbb{C}$}}
\newcommand{\Z}{\mbox{$\Bbb{Z}$}}
\newcommand{\N}{\mbox{$\Bbb{N}$}}
\newcommand{\D}{\mbox{{\bf D}}}
\usepackage{amssymb}
%\newcommand{\qed}{\hfill\mbox{\raggedright$\square$} \vspace{1ex}}
%\newcommand{\proof}{\noindent {\bf Proof:} \hspace{0.1in}}

\newcommand{\setmin}{\;\mbox{--}\;}
\newcommand{\Matlab}{{M\small{AT\-LAB}} }
\newcommand{\Matlabp}{{M\small{AT\-LAB}}}
\newcommand{\computer}{\Matlab Instructions}
\newcommand{\half}{\mbox{$\frac{1}{2}$}}
\newcommand{\compose}{\raisebox{.15ex}{\mbox{{\scriptsize$\circ$}}}}
\newcommand{\AND}{\quad\mbox{and}\quad}
\newcommand{\vect}[2]{\left(\begin{array}{c} #1_1 \\ \vdots \\
 #1_{#2}\end{array}\right)}
\newcommand{\mattwo}[4]{\left(\begin{array}{rr} #1 & #2\\ #3
&#4\end{array}\right)}
\newcommand{\mattwoc}[4]{\left(\begin{array}{cc} #1 & #2\\ #3
&#4\end{array}\right)}
\newcommand{\vectwo}[2]{\left(\begin{array}{r} #1 \\ #2\end{array}\right)}
\newcommand{\vectwoc}[2]{\left(\begin{array}{c} #1 \\ #2\end{array}\right)}

\newcommand{\ignore}[1]{}


\newcommand{\inv}{^{-1}}
\newcommand{\CC}{{\cal C}}
\newcommand{\CCone}{\CC^1}
\newcommand{\Span}{{\rm span}}
\newcommand{\rank}{{\rm rank}}
\newcommand{\trace}{{\rm tr}}
\newcommand{\RE}{{\rm Re}}
\newcommand{\IM}{{\rm Im}}
\newcommand{\nulls}{{\rm null\;space}}

\newcommand{\dps}{\displaystyle}
\newcommand{\arraystart}{\renewcommand{\arraystretch}{1.8}}
\newcommand{\arrayfinish}{\renewcommand{\arraystretch}{1.2}}
\newcommand{\Start}[1]{\vspace{0.08in}\noindent {\bf Section~\ref{#1}}}
\newcommand{\exer}[1]{\noindent {\bf \ref{#1}}}
\newcommand{\ans}{}
\newcommand{\matthree}[9]{\left(\begin{array}{rrr} #1 & #2 & #3 \\ #4 & #5 & #6
\\ #7 & #8 & #9\end{array}\right)}
\newcommand{\cvectwo}[2]{\left(\begin{array}{c} #1 \\ #2\end{array}\right)}
\newcommand{\cmatthree}[9]{\left(\begin{array}{ccc} #1 & #2 & #3 \\ #4 & #5 &
#6 \\ #7 & #8 & #9\end{array}\right)}
\newcommand{\vecthree}[3]{\left(\begin{array}{r} #1 \\ #2 \\
#3\end{array}\right)}
\newcommand{\cvecthree}[3]{\left(\begin{array}{c} #1 \\ #2 \\
#3\end{array}\right)}
\newcommand{\cmattwo}[4]{\left(\begin{array}{cc} #1 & #2\\ #3
&#4\end{array}\right)}

\newcommand{\Matrix}[1]{\ensuremath{\left(\begin{array}{rrrrrrrrrrrrrrrrrr} #1 \end{array}\right)}}

\newcommand{\Matrixc}[1]{\ensuremath{\left(\begin{array}{cccccccccccc} #1 \end{array}\right)}}



\renewcommand{\labelenumi}{\theenumi)}
\newenvironment{enumeratea}%
{\begingroup
 \renewcommand{\theenumi}{\alph{enumi}}
 \renewcommand{\labelenumi}{(\theenumi)}
 \begin{enumerate}}
 {\end{enumerate}\endgroup}



\newcounter{help}
\renewcommand{\thehelp}{\thesection.\arabic{equation}}

%\newenvironment{equation*}%
%{\renewcommand\endequation{\eqno (\theequation)* $$}%
%   \begin{equation}}%
%   {\end{equation}\renewcommand\endequation{\eqno \@eqnnum
%$$\global\@ignoretrue}}

%\input{psfig.tex}

\author{Martin Golubitsky and Michael Dellnitz}

%\newenvironment{matlabEquation}%
%{\renewcommand\endequation{\eqno (\theequation*) $$}%
%   \begin{equation}}%
%   {\end{equation}\renewcommand\endequation{\eqno \@eqnnum
% $$\global\@ignoretrue}}

\newcommand{\soln}{\textbf{Solution:} }
\newcommand{\exercap}[1]{\centerline{Figure~\ref{#1}}}
\newcommand{\exercaptwo}[1]{\centerline{Figure~\ref{#1}a\hspace{2.1in}
Figure~\ref{#1}b}}
\newcommand{\exercapthree}[1]{\centerline{Figure~\ref{#1}a\hspace{1.2in}
Figure~\ref{#1}b\hspace{1.2in}Figure~\ref{#1}c}}
\newcommand{\para}{\hspace{0.4in}}

\renewenvironment{solution}{\suppress}{\endsuppress}

\ifxake
\newenvironment{matlabEquation}{\begin{equation}}{\end{equation}}
\else
\newenvironment{matlabEquation}%
{\let\oldtheequation\theequation\renewcommand{\theequation}{\oldtheequation*}\begin{equation}}%
  {\end{equation}\let\theequation\oldtheequation}
\fi

\makeatother


\title{Linearity}

\begin{document}
\begin{abstract}
\end{abstract}
\maketitle

  \label{S:linearity}

We begin by recalling the vector operations of addition and
scalar multiplication.  Given two $n$ vectors, vector addition
\index{vector!addition} is defined by
\[
\vect{x}{n}+\vect{y}{n}=\left(\begin{array}{c} x_1+y_1 \\ \vdots \\
x_n+y_n\end{array}\right).
\]
Multiplication of a scalar \index{scalar multiplication} times a vector
is defined by
\[
c\vect{x}{n} = \vect{cx}{n}.
\]
Using \Ref{Atimesx} we can check that matrix multiplication
satisfies
\begin{eqnarray}
A(x+y) & = & Ax + Ay \label{sum} \\
A(cx) & = & c(Ax). \label{product}
\end{eqnarray}
Using \Matlab we can also verify that the identities \Ref{sum}
and \Ref{product} are valid for some particular choices of $x$,
$y$, $c$ and $A$.  For example, let
\begin{equation*}
A = \left(\begin{array}{cccc} 2 & 3 & 4 & 1\\ 1 & 1 & 2 & 3
\end{array}\right), \quad x = \left(\begin{array}{r} 1 \\ 5 \\ 4 \\
3 \end{array}\right), \quad y = \left(\begin{array}{r} 1 \\ -1 \\ -1 \\
4 \end{array}\right), \AND c=5.
\end{equation*}
Typing {\tt e3\_3\_3} enters this information into \Matlabp.  Now
type
\begin{verbatim}
z1 = A*(x+y)
z2 = A*x + A*y
\end{verbatim}
and compare {\tt z1} and {\tt z2}.  The fact that they are both
equal to
\[
\vectwo{35}{33}
\]
verifies \Ref{sum} in this case.  Similarly, type
\begin{verbatim}
w1 = A*(c*x)
w2 = c*(A*x)
\end{verbatim}
and compare {\tt w1} and {\tt w2} to verify \Ref{product}.


The central idea in linear algebra is the notion of
{\em linearity\/}. \index{linear}
\begin{Def} \label{linearity}
A mapping $L:\R^n\to\R^m$ is {\em linear}\index{linear!mapping}
if
\begin{itemize}
\item[(a)]  $L(x+y) = L(x) + L(y)$ for all $x,y\in\R^n$.
\item[(b)]  $L(cx) = cL(x)$ for all $x\in\R^n$ and all scalars
$c\in\R$.
\end{itemize}
\end{Def}

To better understand the meaning of Definition~\ref{linearity}(a,b),
we verify these conditions for the mapping $L:\R^2\to\R^2$ defined by
\begin{equation}  \label{E:mme}
L(x) = (x_1+3x_2,2x_1-x_2),
\end{equation}
where $x=(x_1,x_2)\in\R^2$.  To verify Definition~\ref{linearity}(a), let
$y=(y_1,y_2)\in\R^2$.  Then
\begin{eqnarray*}
L(x+y) & = & L(x_1+y_1,x_2+y_2)\\
& = & ((x_1+y_1)+3(x_2+y_2), 2(x_1+y_1)-(x_2+y_2)) \\
 & = & (x_1+y_1+3x_2+3y_2, 2x_1+2y_1-x_2-y_2).
\end{eqnarray*}
On the other hand,
\begin{eqnarray*}
L(x)+L(y) & = & (x_1+3x_2,2x_1-x_2) + (y_1+3y_2,2y_1-y_2) \\
 & = & (x_1+3x_2+y_1+3y_2, 2x_1-x_2+2y_1-y_2).
\end{eqnarray*}
Hence
\[
L(x+y) = L(x) + L(y)
\]
for every pair of vectors $x$ and $y $ in $\R^2$.

Similarly, to verify Definition~\ref{linearity}(b), let $c\in\R$ be a scalar
and compute
\[
L(cx) = L(cx_1,cx_2)=((cx_1)+3(cx_2),2(cx_1)-(cx_2)).
\]
Then compute
\[
cL(x) = c(x_1+3x_2,2x_1-x_2)= (c(x_1+3x_2), c(2x_1-x_2)),
\]
from which it follows that
\[
L(cx) = cL(x)
\]
for every vector $x\in\R^2$ and every scalar $c\in\R$.  Thus $L$ is a linear 
mapping.

In fact, the mapping \Ref{E:mme} is a matrix mapping and could
have been written in the form
\[
L(x) = \mattwo{1}{3}{2}{-1}x.
\]
Hence the linearity of $L$ could have been checked using identities
\Ref{sum} and \Ref{product}.  Indeed, matrix mappings are always linear
mappings, as we now discuss.

\subsubsection*{Matrix Mappings are Linear Mappings}

Let $A$ be an $m\times n$ matrix and recall that the matrix mapping
$L_A:\R^n\to\R^m$ is defined by $L_A(x)=Ax$.  We may rewrite \Ref{sum} and
\Ref{product} using this notation as
\begin{eqnarray*}
L_A(x+y) & = & L_A(x) + L_A(y) \\
L_A(cx) & = & cL_A(x).
\end{eqnarray*}
Thus all matrix mappings\index{matrix!mappings} are linear
mappings\index{linear!mapping}.  We will show that all linear
mappings are matrix mappings (see Theorem~\ref{lin-matrices}).  But first
we discuss linearity in the simplest context of mappings from $\R\to\R$.

\subsection*{Linear and Nonlinear Mappings of $\R\to\R$}

Note that $1\times 1$ matrices are just scalars $A=(a)$.  It follows from
\Ref{sum} and \Ref{product} that we have shown that the matrix mappings
$L_A(x)=ax$ are all linear, though this point could have been verified
directly.  Before showing that these are all the linear mappings of
$\R\to\R$, we focus on examples of functions of $\R\to\R$ that are
{\em not\/} linear.

\subsubsection*{Examples of Mappings that are Not Linear}

\begin{itemize}
\item   $f(x)=x^2$.  Calculate
\[
f(x+y) = (x+y)^2 = x^2+2xy+y^2
\]
while
\[
f(x)+f(y) = x^2 + y^2.
\]
The two expressions are not equal and $f(x)=x^2$ is not linear.
\item   $f(x)=e^x$.  Calculate
\[
f(x+y) = e^{x+y} = e^x e^y
\]
while
\[
f(x)+f(y) = e^x + e^y.
\]
The two expressions are not equal and $f(x)=e^x$ is not linear.
\item   $f(x) = \sin x$.  Recall that
\[
f(x+y) =\sin(x+y) = \sin x \cos y +\cos x \sin y
\]
while
\[
f(x)+f(y) = \sin x + \sin y.
\]
The two expressions are not equal and $f(x)=\sin x $ is not
linear.
\end{itemize}

\subsubsection*{Linear Functions of One Variable}

Suppose we take the opposite approach and ask what functions of
$\R\to\R$ are linear.  Observe that if $L:\R\to\R$ is linear,
then
\[
L(x) = L(x\cdot 1).
\]
Since we are looking at the special case of linear mappings on
$\R$, we note that $x$ is a real number as well as a vector.
Thus we can use Definition~\ref{linearity}(b) to observe that
\[
L(x\cdot 1)=xL(1).
\]
So if we let $a=L(1)$, then we see that
\[
L(x)=ax.
\]
Thus linear mappings of $\R$ into $\R$ are very special mappings
indeed; they are all scalar multiples of the identity mapping.

\subsection*{All Linear Mappings are Matrix Mappings}

We end this section by proving that every linear mapping is
given by matrix multiplication. But first we state and prove two
lemmas.  There is a standard set of vectors that is used over
and over again in linear algebra, which we now define.

\begin{Def}  \label{D:canonicalbasis}
Let $j$ be an integer between $1$ and $n$.  The $n$-vector $e_j$ is
the vector that has a $1$ in the $j^{th}$ entry and zeros in all
other entries.
\end{Def} \index{$e_j$}

\begin{lemma}  \label{linequal}
Let $L_1:\R^n\to\R^m$ and $L_2:\R^n\to\R^m$ be linear mappings.
Suppose that $L_1(e_j)=L_2(e_j)$ for every $j=1,\ldots,n$.  Then
$L_1=L_2$.
\end{lemma}

\begin{proof}  Let $x=(x_1,\ldots,x_n)$ be a vector in $\R^n$.  Then
\[
x = x_1e_1 + \cdots + x_ne_n.
\]
Linearity of $L_1$ and $L_2$ implies that
\begin{eqnarray*}
L_1(x) & = & x_1L_1(e_1) + \cdots + x_nL_1(e_n) \\
  & = & x_1L_2(e_1) + \cdots + x_nL_2(e_n) \\
  & = & L_2(x).
\end{eqnarray*}
Since $L_1(x)=L_2(x)$ for all $x\in\R^n$, it follows that
$L_1=L_2$.  \end{proof}

\begin{lemma}  \label{columnsA}
Let $A$ be an $m\times n$ matrix.  Then $Ae_j$ is the $j^{th}$
column of $A$.
\end{lemma}

\begin{proof}  Recall the definition of matrix multiplication given in \Ref{Atimesx}.
In that formula, just set $x_i$ equal to zero for all $i\neq j$ and set
$x_j=1$. \end{proof}

\begin{thm}  \label{lin-matrices}
Let $L:\R^n\to\R^m$ be a linear mapping\index{linear!mapping}.
Then there exists an $m\times n$ matrix $A$ such that $L=L_A$.
\end{thm}

\begin{proof}
There are two steps to the proof: determine the matrix $A$ and
verify that $L_A=L$.

Let $A$ be the matrix whose $j^{th}$ column is $L(e_j)$.  By
Lemma~\ref{columnsA} $L(e_j) = Ae_j$; that is, $L(e_j) = L_A(e_j)$.
Lemma~\ref{linequal} implies that $L=L_A$.  \end{proof}

Theorem~\ref{lin-matrices} provides a simple way of showing that
\[
L(0) = 0
\]
for any linear map $L$.  Indeed, $L(0)=L_A(0)=A0=0$ for some matrix $A$.  
(This fact can also be proved directly from the definition of linear mapping.)

\subsubsection*{Using Theorem~\protect\ref{lin-matrices} to Find Matrices
Associated to Linear Maps}

The proof of Theorem~\ref{lin-matrices} shows that the $j^{th}$ column of the
matrix $A$ associated to a linear mapping $L$ is $L(e_j)$ viewed as a column
vector.  As an example, let $L:\R^2\to\R^2$ be rotation clockwise through
$90^\circ$.  Geometrically, it is easy to see that
\[
L(e_1) = L\left(\vectwo{1}{0}\right) = \vectwo{0}{-1} \AND
L(e_2) = L\left(\vectwo{0}{1}\right) = \vectwo{1}{0}.
\]
Since we know that rotations are linear maps, it follows that the matrix
$A$ associated to the linear map $L$ is:
\[
A = \mattwo{0}{1}{-1}{0}.
\]
Additional examples of linear mappings whose associated matrices can be found
using Theorem~\ref{lin-matrices} are given in Exercises \ref{c4.3.7} --
\ref{c4.3.10}.



\EXER

\TEXER

\begin{exercise} \label{c4.3.1}
Compute $ax+by$ for each of the following:
\begin{itemize}
\item[(a)] $a=2$, $b=-3$, $x=(2,4)$ and $y=(3,-1)$.
\item[(b)] $a=10$, $b=-2$, $x=(1,0,-1)$ and $y=(2,-4,3)$.
\item[(c)] $a=5$, $b=-1$, $x=(4,2,-1,1)$ and $y=(-1,3,5,7)$.
\end{itemize}
\end{exercise}

\begin{exercise} \label{c4.3.2}
Let $x=(4,7)$ and $y=(2,-1)$.  Write the vector $\alpha x+\beta
y$ as a vector in coordinates.
\end{exercise}

\begin{exercise} \label{c4.3.3}
Let $x=(1,2)$, $y=(1,-3)$, and $z=(-2,-1)$.  Show that you can
write
\[
z=\alpha x+ \beta y
\]
for some $\alpha,\beta\in\R$.

\noindent {\bf Hint:} Set up a system of two linear equations in
the unknowns $\alpha$ and $\beta$, and then solve this linear
system.
\end{exercise}

\begin{exercise} \label{c4.3.4}
Can the vector $z=(2,3,-1)$ be written as
\[
z=\alpha x+ \beta y
\]
where $x=(2,3,0)$ and $y=(1,-1,1)$?
\end{exercise}

\begin{exercise} \label{c4.3.5}
Let $x=(3,-2)$, $y=(2,3)$, and $z=(1,4)$.  For which real
numbers $\alpha,\beta,\gamma$ does
\[
\alpha x + \beta y + \gamma z = (1,-2)?
\]
\end{exercise}

\noindent In Exercises~\ref{c4.3.6a} -- \ref{c4.3.6d} determine
whether the given transformation is linear.
\begin{exercise} \label{c4.3.6a}
  $T:\R^3\to\R^2$ defined by $T(x_1,x_2,x_3)=(x_1+2x_2-x_3,x_1-4x_3)$.
\end{exercise}
\begin{exercise} \label{c4.3.6b}
  $T:\R^2\to\R^2$ defined by $T(x_1,x_2)=(x_1+x_1x_2,2x_2)$.
\end{exercise}
\begin{exercise} \label{c4.3.6c}
  $T:\R^2\to\R^2$ defined by $T(x_1,x_2)=(x_1+x_2,x_1-x_2-1)$.
\end{exercise}
\begin{exercise} \label{c4.3.6d}
  $T:\R^2\to\R^3$ defined by $T(x_1,x_2)=(1,x_1+x_2,2x_2)$
\end{exercise}

\begin{exercise} \label{c4.3.7}
Find the $2\times 3$ matrix $A$ that satisfies
\[
Ae_1  =  \vectwo{2}{3},\qquad
Ae_2  =  \vectwo{1}{-1}, \AND
Ae_3  = \vectwo{0}{1}.
\]
\end{exercise}

\begin{exercise} \label{c4.3.8}
The {\em cross product\/} of two $3$-vectors $x=(x_1,x_2,x_3)$
and $y=(y_1,y_2,y_3)$ is the $3$-vector
\[
x\times y = (x_2y_3-x_3y_2,-(x_1y_3-x_3y_1),x_1y_2-x_2y_1).
\]
Let $K=(2,1,-1)$.  Show that the mapping $L:\R^3\to\R^3$ defined by
\[
L(x) = x\times K
\]
is a linear mapping.  Find the $3\times 3$ matrix $A$ such that
\[
L(x) = Ax,
\]
that is, $L=L_A$.
\end{exercise}

\begin{exercise} \label{c4.3.9}
Argue geometrically that rotation of the plane counterclockwise
through an angle of $45^\circ$ is a linear mapping.  Find a
$2\times 2$ matrix $A$ such that $L_A$ rotates the plane
counterclockwise by $45^\circ$.
\end{exercise}

\begin{exercise} \label{c4.3.10}
Let $\sigma$ permute coordinates cyclically in $\R^3$; that is,
\[
\sigma(x_1,x_2,x_3) = (x_2,x_3,x_1).
\]
Find a $3\times 3$ matrix $A$ such that $\sigma = L_A$.
\end{exercise}

\begin{exercise} \label{c4.3.11}
Let $L$ be a linear map.  Using the definition of linearity,
prove that $L(0)=0$.
\end{exercise}

\begin{exercise}  \label{c4.3.12}
Let $L_1:\R^n\to\R^m$ and $L_2:\R^n\to\R^m$ be linear mappings. Prove
that $L:\R^n\to\R^m$ defined by
\[
L(x) = L_1(x) + L_2(x)
\]
is also a linear mapping.  Theorem~\ref{lin-matrices} states that there
are matrices $A$, $A_1$ and $A_2$ such that
\[
L = L_A \AND L_j = L_{A_j}
\]
for $j=1,2$.  What is the relationship between the matrices $A$, $A_1$
and $A_2$?
\end{exercise}

\CEXER

\begin{exercise} \label{c4.3.13}
Let
\[
A = \mattwo{0.5}{0}{0}{2}.
\]
Use {\sf map} to verify that the linear mapping $L_A$ halves
the $x$-component of a point while it doubles the $y$-component.
\end{exercise}

\begin{exercise} \label{c4.3.14}
Let
\[
A = \mattwo{0}{0.5}{-0.5}{0}.
\]
Use {\sf map} to determine how the mapping $L_A$ acts on $2$-vectors.
Describe this action in words.
\end{exercise}

\noindent In Exercises~\ref{c4.3.15A} -- \ref{c4.3.15B} use \Matlab to
verify \Ref{sum} and \Ref{product}.
\begin{exercise} \label{c4.3.15A}
\begin{equation*} \label{eq4.3.15a}
A = \left(
\begin{array}{rrr}
 1 & 2 & 3  \\
 0 & 1 & -2  \\
 4 & 0 & 1
\end{array}
\right),\quad
x=\left(
\begin{array}{r}
 3   \\
 2   \\
 -1
\end{array}
\right),\quad
y=\left(
\begin{array}{r}
 0   \\
 -5   \\
 10
\end{array}
\right),\quad c=21;
\end{equation*}
\end{exercise}
\begin{exercise} \label{c4.3.15B}
\begin{equation*} \label{eq4.3.15b}
A = \left(
\begin{array}{rrrrr}
 4 & 0 & -3 & 2 & 4 \\
 2 & 8 & -4 & -1 & 3 \\
 -1 & 2 & 1 & 10 & -2 \\
 4 & 4 & -2 & 1 & 2 \\
 -2 & 3 & 1 & 1 & -1
\end{array}
\right),\quad
x=\left(
\begin{array}{r}
 1   \\
 3   \\
 -2   \\
 3   \\
 -1
\end{array}
\right),\quad
y=\left(
\begin{array}{r}
 2   \\
 0   \\
 13   \\
 -2   \\
 1
\end{array}
\right),\quad c=-13.
\end{equation*}
\end{exercise}




\end{document}
