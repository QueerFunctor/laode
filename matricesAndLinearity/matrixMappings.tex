\documentclass{ximera}

 

\usepackage{epsfig}

\graphicspath{
  {./}
  {figures/}
}

\usepackage{morewrites}
\makeatletter
\newcommand\subfile[1]{%
\renewcommand{\input}[1]{}%
\begingroup\skip@preamble\otherinput{#1}\endgroup\par\vspace{\topsep}
\let\input\otherinput}
\makeatother

\newcommand{\includeexercises}{\directlua{dofile("/home/jim/linearAlgebra/laode/exercises.lua")}}

%\newcounter{ccounter}
%\setcounter{ccounter}{1}
%\newcommand{\Chapter}[1]{\setcounter{chapter}{\arabic{ccounter}}\chapter{#1}\addtocounter{ccounter}{1}}

%\newcommand{\section}[1]{\section{#1}\setcounter{thm}{0}\setcounter{equation}{0}}

%\renewcommand{\theequation}{\arabic{chapter}.\arabic{section}.\arabic{equation}}
%\renewcommand{\thefigure}{\arabic{chapter}.\arabic{figure}}
%\renewcommand{\thetable}{\arabic{chapter}.\arabic{table}}

%\newcommand{\Sec}[2]{\section{#1}\markright{\arabic{ccounter}.\arabic{section}.#2}\setcounter{equation}{0}\setcounter{thm}{0}\setcounter{figure}{0}}

\newcommand{\Sec}[2]{\section{#1}}

\setcounter{secnumdepth}{2}
%\setcounter{secnumdepth}{1} 

%\newcounter{THM}
%\renewcommand{\theTHM}{\arabic{chapter}.\arabic{section}}

\newcommand{\trademark}{{R\!\!\!\!\!\bigcirc}}
%\newtheorem{exercise}{}

\newcommand{\dfield}{{\sf dfield9}}
\newcommand{\pplane}{{\sf pplane9}}

\newcommand{\EXER}{\section*{Exercises}}%\vspace*{0.2in}\hrule\small\setcounter{exercise}{0}}
\newcommand{\CEXER}{}%\vspace{0.08in}\begin{center}Computer Exercises\end{center}}
\newcommand{\TEXER}{} %\vspace{0.08in}\begin{center}Hand Exercises\end{center}}
\newcommand{\AEXER}{} %\vspace{0.08in}\begin{center}Hand Exercises\end{center}}

% BADBAD: \newcommand{\Bbb}{\bf}

\newcommand{\R}{\mbox{$\Bbb{R}$}}
\newcommand{\C}{\mbox{$\Bbb{C}$}}
\newcommand{\Z}{\mbox{$\Bbb{Z}$}}
\newcommand{\N}{\mbox{$\Bbb{N}$}}
\newcommand{\D}{\mbox{{\bf D}}}
\usepackage{amssymb}
%\newcommand{\qed}{\hfill\mbox{\raggedright$\square$} \vspace{1ex}}
%\newcommand{\proof}{\noindent {\bf Proof:} \hspace{0.1in}}

\newcommand{\setmin}{\;\mbox{--}\;}
\newcommand{\Matlab}{{M\small{AT\-LAB}} }
\newcommand{\Matlabp}{{M\small{AT\-LAB}}}
\newcommand{\computer}{\Matlab Instructions}
\newcommand{\half}{\mbox{$\frac{1}{2}$}}
\newcommand{\compose}{\raisebox{.15ex}{\mbox{{\scriptsize$\circ$}}}}
\newcommand{\AND}{\quad\mbox{and}\quad}
\newcommand{\vect}[2]{\left(\begin{array}{c} #1_1 \\ \vdots \\
 #1_{#2}\end{array}\right)}
\newcommand{\mattwo}[4]{\left(\begin{array}{rr} #1 & #2\\ #3
&#4\end{array}\right)}
\newcommand{\mattwoc}[4]{\left(\begin{array}{cc} #1 & #2\\ #3
&#4\end{array}\right)}
\newcommand{\vectwo}[2]{\left(\begin{array}{r} #1 \\ #2\end{array}\right)}
\newcommand{\vectwoc}[2]{\left(\begin{array}{c} #1 \\ #2\end{array}\right)}

\newcommand{\ignore}[1]{}


\newcommand{\inv}{^{-1}}
\newcommand{\CC}{{\cal C}}
\newcommand{\CCone}{\CC^1}
\newcommand{\Span}{{\rm span}}
\newcommand{\rank}{{\rm rank}}
\newcommand{\trace}{{\rm tr}}
\newcommand{\RE}{{\rm Re}}
\newcommand{\IM}{{\rm Im}}
\newcommand{\nulls}{{\rm null\;space}}

\newcommand{\dps}{\displaystyle}
\newcommand{\arraystart}{\renewcommand{\arraystretch}{1.8}}
\newcommand{\arrayfinish}{\renewcommand{\arraystretch}{1.2}}
\newcommand{\Start}[1]{\vspace{0.08in}\noindent {\bf Section~\ref{#1}}}
\newcommand{\exer}[1]{\noindent {\bf \ref{#1}}}
\newcommand{\ans}{}
\newcommand{\matthree}[9]{\left(\begin{array}{rrr} #1 & #2 & #3 \\ #4 & #5 & #6
\\ #7 & #8 & #9\end{array}\right)}
\newcommand{\cvectwo}[2]{\left(\begin{array}{c} #1 \\ #2\end{array}\right)}
\newcommand{\cmatthree}[9]{\left(\begin{array}{ccc} #1 & #2 & #3 \\ #4 & #5 &
#6 \\ #7 & #8 & #9\end{array}\right)}
\newcommand{\vecthree}[3]{\left(\begin{array}{r} #1 \\ #2 \\
#3\end{array}\right)}
\newcommand{\cvecthree}[3]{\left(\begin{array}{c} #1 \\ #2 \\
#3\end{array}\right)}
\newcommand{\cmattwo}[4]{\left(\begin{array}{cc} #1 & #2\\ #3
&#4\end{array}\right)}

\newcommand{\Matrix}[1]{\ensuremath{\left(\begin{array}{rrrrrrrrrrrrrrrrrr} #1 \end{array}\right)}}

\newcommand{\Matrixc}[1]{\ensuremath{\left(\begin{array}{cccccccccccc} #1 \end{array}\right)}}



\renewcommand{\labelenumi}{\theenumi)}
\newenvironment{enumeratea}%
{\begingroup
 \renewcommand{\theenumi}{\alph{enumi}}
 \renewcommand{\labelenumi}{(\theenumi)}
 \begin{enumerate}}
 {\end{enumerate}\endgroup}



\newcounter{help}
\renewcommand{\thehelp}{\thesection.\arabic{equation}}

%\newenvironment{equation*}%
%{\renewcommand\endequation{\eqno (\theequation)* $$}%
%   \begin{equation}}%
%   {\end{equation}\renewcommand\endequation{\eqno \@eqnnum
%$$\global\@ignoretrue}}

%\input{psfig.tex}

\author{Martin Golubitsky and Michael Dellnitz}

%\newenvironment{matlabEquation}%
%{\renewcommand\endequation{\eqno (\theequation*) $$}%
%   \begin{equation}}%
%   {\end{equation}\renewcommand\endequation{\eqno \@eqnnum
% $$\global\@ignoretrue}}

\newcommand{\soln}{\textbf{Solution:} }
\newcommand{\exercap}[1]{\centerline{Figure~\ref{#1}}}
\newcommand{\exercaptwo}[1]{\centerline{Figure~\ref{#1}a\hspace{2.1in}
Figure~\ref{#1}b}}
\newcommand{\exercapthree}[1]{\centerline{Figure~\ref{#1}a\hspace{1.2in}
Figure~\ref{#1}b\hspace{1.2in}Figure~\ref{#1}c}}
\newcommand{\para}{\hspace{0.4in}}

\renewenvironment{solution}{\suppress}{\endsuppress}

\ifxake
\newenvironment{matlabEquation}{\begin{equation}}{\end{equation}}
\else
\newenvironment{matlabEquation}%
{\let\oldtheequation\theequation\renewcommand{\theequation}{\oldtheequation*}\begin{equation}}%
  {\end{equation}\let\theequation\oldtheequation}
\fi

\makeatother


\title{Matrix Mappings}

\begin{document}
\begin{abstract}
\end{abstract}
\maketitle

  \index{matrix!mappings} \label{s:4.2}

Having illustrated the notational advantage of using matrices
and matrix multiplication, we now begin to discuss why there
is also a {\em conceptual advantage\/} to matrix
multiplication, a conceptual advantage that will help
us to understand how systems of linear equations and linear
differential equations may be solved.

Matrix multiplication allows us to view $m\times n$ matrices as
mappings from $\R^n$ to $\R^m$.  Let $A$ be an $m\times n$
matrix and let $x$ be an $n$ vector.  Then
\[
x \mapsto Ax
\]
defines a mapping from $\R^n$ to $\R^m$.

The simplest example of a matrix mapping is given by $1\times 1$
matrices.  Matrix mappings defined from $\R\to\R$ are
\[
x \mapsto ax
\]
where $a$ is a real number.  Note that the graph of this
function is just a straight line through the origin (with slope
$a$).  From this example we see that matrix mappings are very
special mappings indeed. In higher dimensions, matrix mappings
provide a richer set of mappings; we explore here {\em planar\/}
mappings\index{planar mappings} --- mappings of the plane into
itself --- using \Matlab graphics and the program {\sf map}.

The simplest planar matrix mappings are the
{\em dilatations\/}\index{dilatation}.
Let $A=cI_2$ where $c>0$ is a scalar.  When $c<1$ vectors are
contracted by a factor of $c$ and and these mappings are
examples of {\em contractions}\index{contraction}.
When $c>1$ vectors are
stretched or expanded by a factor of $c$ and these dilatations
are examples of {\em expansions}\index{expansion}.
We now explore some more complicated planar matrix mappings.

The next planar motions that we study are those given by the
matrices
\[
A=\mattwo{\lambda}{0}{0}{\mu}.
\]
Here the matrix mapping is given by $(x,y)\mapsto(\lambda x,\mu y)$;
that is, a mapping that independently stretches and/or contracts the
$x$ and $y$ coordinates.  Even these simple looking mappings can move 
objects in the plane in a somewhat complicated fashion.

\subsection*{The Program {\sf map}}  \index{\computer!map}

We use \Matlab to explore planar matrix mappings 
using the program {\tt map}.  In \Matlab type the command
\begin{verbatim}
map
\end{verbatim}
and a window appears labeled {\sf Map}.  The $2\times 2$ matrix
\begin{equation}. \label{e:map_A}
\mattwo{0}{-1}{1}{0}
\end{equation}
has been pre-entered.  Click on the {\sf Custom} button. In 
the {\sf Icons} menu click on an icon --- say {\sf Dog} --- and 
a blue `{\sf Dog}' will appear in the graphing window. \index{\computer!dog}
Next click on the {\sf Iterate} button and a new version of the {\sf Dog} will
appear in yellow ---the yellow {\sf Dog} is just rotated about
the origin counterclockwise by $90^\circ$ from the blue dog.
Indeed, the matrix \eqref{e:map_A} rotates the plane counterclockwise
by $90^\circ$.  To verify this statement click on {\sf Iterate}
again and see that the yellow dog rotates $90^\circ$
counterclockwise into the magenta dog.  Of course, the magenta
dog is rotated $180^\circ$ from the original blue dog.
Clicking on {\sf Iterate} once more produces a fourth dog --- this one
in cyan.  Finally, one more click on the {\sf Iterate} button will
rotate the cyan dog into a red dog that exactly covers the
original blue dog.

Other matrices will produce different motions of the plane.  Click 
on the {\sf Reset} button.  Then either push the {\sf Custom} button,  
type the entries in the matrix, and click on the {\sf Iterate} button; 
or choose one of the pre-assigned matrices listed in the {\sf Gallery} menu
and click on the {\sf Iterate} button.  
For example, clicking on the {\sf Contracting rotation} button recalls the matrix
\[
\mattwo{0.3}{-0.8}{0.8}{0.3}
\]
This matrix rotates the plane through an angle of approximately
$69.4^\circ$ counterclockwise and contracts the plane by a
factor of approximately $0.85$.  Now click on {\sf Dog} in the
{\sf Icons} menu to bring up the blue dog again.  Repeated
clicking on {\sf Iterate} rotates and contracts the dog so that dogs
in a cycling set of colors slowly converge towards the origin in
a spiral of dogs.\footnote{When using the program {\sf map} first 
choose an Icon (or Vector), second choose a Matrix from the Gallery 
(or a Custom matrix), and finally click on {\sf Iterate}. Then {\sf Iterate} 
again or {\sf Reset} to start over.}

\subsubsection*{Rotations}\index{rotation}

Rotating the plane counterclockwise through an angle $\theta$ is
a motion given by a matrix mapping.  We show that the matrix that
performs this rotation is:
\begin{equation} \label{e:rotmat}
R_\theta = \mattwo{\cos\theta}{-\sin\theta}{\sin\theta}{\cos\theta}.
\end{equation}
To verify that $R_\theta$ rotates the plane counterclockwise
through angle $\theta$, let $v_\varphi$ be the unit vector whose
angle from the horizontal is $\varphi$; that is,
$v_\varphi=(\cos\varphi,\sin\varphi)$.  We can write every vector in
$\R^2$ as $rv_\varphi$ for some number $r\ge 0$.   Using the 
trigonometric identities for the cosine and sine of the sum of two angles, we
have:
\begin{eqnarray*}
R_\theta (rv_\varphi) & = &
\mattwo{\cos\theta}{-\sin\theta}{\sin\theta}{\cos\theta}
\vectwo{r\cos\varphi}{r\sin\varphi} \\
& = & \vectwo{r\cos\theta\cos\varphi -r\sin\theta\sin\varphi}
{r\sin\theta\cos\varphi + r\cos\theta\sin\varphi}\\
& = & r\vectwo{\cos(\theta+\varphi)}{\sin(\theta+\varphi)}  \\
& = & rv_{\varphi+\theta}.
\end{eqnarray*}
This calculation shows that $R_\theta$ rotates every vector in the plane
counterclockwise through angle $\theta$.

It follows from \eqref{e:rotmat} that $R_{180^\circ} = -I_2$.  So rotating a
vector in the plane by $180^\circ$ is the same as reflecting the vector
through the origin.  It also follows that the movement associated with the
linear map $x\mapsto -cx$ where $x\in\R^2$ and $c>0$ may be thought of as a dilatation
($x\mapsto cx$) followed by rotation through $180^\circ$ ($x\mapsto -x$).

We claim that combining dilatations with general rotations produces spirals.  
Consider the matrix
\[
S = \mattwo{c\cos\theta}{-c\sin\theta}{c\sin\theta}{c\cos\theta} = cR_\theta
\]
where $c<1$.  Then a calculation similar to the previous one shows that
\[
S(rv_\varphi) = c(rv_{\varphi+\theta}).
\]
So $S$ rotates vectors in the plane while contracting them by
the factor $c$.  Thus, multiplying a vector repeatedly by $S$ 
spirals that vector into the origin.  The example that we just 
considered while using {\sf map} is
\[
\mattwo{0.3}{-0.8}{0.8}{0.3}\cong\mattwo{0.85\cos(69.4^\circ)}
{-0.85\sin(69.4^\circ)}{0.85\sin(69.4^\circ)}{0.85\cos(69.4^\circ)},
\]
which is an example of $S$ with $c = 0.85$ and $\theta = 69.4^\circ$.

\subsection*{A Notation for Matrix Mappings}
\index{matrix!mappings}

We reinforce the idea that matrices are mappings by introducing a notation 
for the mapping associated with an $m\times n$ matrix $A$.  Define
\[
L_A:\R^n\to\R^m
\]
by
\[
L_A(x) = Ax,
\]
for every $x\in\R^n$.

There are two special matrices:  the $m\times n$ zero matrix 
\index{matrix!zero} $O$ all of whose entries are $0$ and the 
$n\times n$ identity matrix \index{matrix!identity} $I_n$ whose diagonal 
entries are $1$ and whose off diagonal entries are $0$.  For instance,
\[
	I_3 = \left(
\begin{array}{rrr}
 1 & 0 & 0  \\
 0 & 1 & 0  \\
 0 & 0 & 1
\end{array}
\right).
\]

The mappings associated with these special matrices are also special.  
Let $x$ be an $n$ vector.  Then
\begin{equation} \label{multby0}
Ox=0,
\end{equation}
where the $0$ on the right hand side of \eqref{multby0} is the $m$
vector all of whose entries are $0$.  The mapping $L_O$ is the 
{\em zero mapping\/} \index{zero mapping} --- the mapping 
that maps every vector $x$ to $0$.

Similarly,
\[
I_nx=x
\]
for every vector $x$.  It follows that
\[
L_{I_n}(x) = x
\]
is the {\em identity mapping\/}, \index{identity mapping} since
it maps every vector to itself.  It is for this reason that the
matrix $I_n$ is called the $n\times n$ {\em identity matrix\/}.



\includeexercises


\end{document}
