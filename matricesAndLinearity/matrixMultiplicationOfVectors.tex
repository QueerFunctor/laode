\documentclass{ximera}

 

\usepackage{epsfig}

\graphicspath{
  {./}
  {figures/}
}

\usepackage{morewrites}
\makeatletter
\newcommand\subfile[1]{%
\renewcommand{\input}[1]{}%
\begingroup\skip@preamble\otherinput{#1}\endgroup\par\vspace{\topsep}
\let\input\otherinput}
\makeatother

\newcommand{\includeexercises}{\directlua{dofile("/home/jim/linearAlgebra/laode/exercises.lua")}}

%\newcounter{ccounter}
%\setcounter{ccounter}{1}
%\newcommand{\Chapter}[1]{\setcounter{chapter}{\arabic{ccounter}}\chapter{#1}\addtocounter{ccounter}{1}}

%\newcommand{\section}[1]{\section{#1}\setcounter{thm}{0}\setcounter{equation}{0}}

%\renewcommand{\theequation}{\arabic{chapter}.\arabic{section}.\arabic{equation}}
%\renewcommand{\thefigure}{\arabic{chapter}.\arabic{figure}}
%\renewcommand{\thetable}{\arabic{chapter}.\arabic{table}}

%\newcommand{\Sec}[2]{\section{#1}\markright{\arabic{ccounter}.\arabic{section}.#2}\setcounter{equation}{0}\setcounter{thm}{0}\setcounter{figure}{0}}

\newcommand{\Sec}[2]{\section{#1}}

\setcounter{secnumdepth}{2}
%\setcounter{secnumdepth}{1} 

%\newcounter{THM}
%\renewcommand{\theTHM}{\arabic{chapter}.\arabic{section}}

\newcommand{\trademark}{{R\!\!\!\!\!\bigcirc}}
%\newtheorem{exercise}{}

\newcommand{\dfield}{{\sf dfield9}}
\newcommand{\pplane}{{\sf pplane9}}

\newcommand{\EXER}{\section*{Exercises}}%\vspace*{0.2in}\hrule\small\setcounter{exercise}{0}}
\newcommand{\CEXER}{}%\vspace{0.08in}\begin{center}Computer Exercises\end{center}}
\newcommand{\TEXER}{} %\vspace{0.08in}\begin{center}Hand Exercises\end{center}}
\newcommand{\AEXER}{} %\vspace{0.08in}\begin{center}Hand Exercises\end{center}}

% BADBAD: \newcommand{\Bbb}{\bf}

\newcommand{\R}{\mbox{$\Bbb{R}$}}
\newcommand{\C}{\mbox{$\Bbb{C}$}}
\newcommand{\Z}{\mbox{$\Bbb{Z}$}}
\newcommand{\N}{\mbox{$\Bbb{N}$}}
\newcommand{\D}{\mbox{{\bf D}}}
\usepackage{amssymb}
%\newcommand{\qed}{\hfill\mbox{\raggedright$\square$} \vspace{1ex}}
%\newcommand{\proof}{\noindent {\bf Proof:} \hspace{0.1in}}

\newcommand{\setmin}{\;\mbox{--}\;}
\newcommand{\Matlab}{{M\small{AT\-LAB}} }
\newcommand{\Matlabp}{{M\small{AT\-LAB}}}
\newcommand{\computer}{\Matlab Instructions}
\newcommand{\half}{\mbox{$\frac{1}{2}$}}
\newcommand{\compose}{\raisebox{.15ex}{\mbox{{\scriptsize$\circ$}}}}
\newcommand{\AND}{\quad\mbox{and}\quad}
\newcommand{\vect}[2]{\left(\begin{array}{c} #1_1 \\ \vdots \\
 #1_{#2}\end{array}\right)}
\newcommand{\mattwo}[4]{\left(\begin{array}{rr} #1 & #2\\ #3
&#4\end{array}\right)}
\newcommand{\mattwoc}[4]{\left(\begin{array}{cc} #1 & #2\\ #3
&#4\end{array}\right)}
\newcommand{\vectwo}[2]{\left(\begin{array}{r} #1 \\ #2\end{array}\right)}
\newcommand{\vectwoc}[2]{\left(\begin{array}{c} #1 \\ #2\end{array}\right)}

\newcommand{\ignore}[1]{}


\newcommand{\inv}{^{-1}}
\newcommand{\CC}{{\cal C}}
\newcommand{\CCone}{\CC^1}
\newcommand{\Span}{{\rm span}}
\newcommand{\rank}{{\rm rank}}
\newcommand{\trace}{{\rm tr}}
\newcommand{\RE}{{\rm Re}}
\newcommand{\IM}{{\rm Im}}
\newcommand{\nulls}{{\rm null\;space}}

\newcommand{\dps}{\displaystyle}
\newcommand{\arraystart}{\renewcommand{\arraystretch}{1.8}}
\newcommand{\arrayfinish}{\renewcommand{\arraystretch}{1.2}}
\newcommand{\Start}[1]{\vspace{0.08in}\noindent {\bf Section~\ref{#1}}}
\newcommand{\exer}[1]{\noindent {\bf \ref{#1}}}
\newcommand{\ans}{}
\newcommand{\matthree}[9]{\left(\begin{array}{rrr} #1 & #2 & #3 \\ #4 & #5 & #6
\\ #7 & #8 & #9\end{array}\right)}
\newcommand{\cvectwo}[2]{\left(\begin{array}{c} #1 \\ #2\end{array}\right)}
\newcommand{\cmatthree}[9]{\left(\begin{array}{ccc} #1 & #2 & #3 \\ #4 & #5 &
#6 \\ #7 & #8 & #9\end{array}\right)}
\newcommand{\vecthree}[3]{\left(\begin{array}{r} #1 \\ #2 \\
#3\end{array}\right)}
\newcommand{\cvecthree}[3]{\left(\begin{array}{c} #1 \\ #2 \\
#3\end{array}\right)}
\newcommand{\cmattwo}[4]{\left(\begin{array}{cc} #1 & #2\\ #3
&#4\end{array}\right)}

\newcommand{\Matrix}[1]{\ensuremath{\left(\begin{array}{rrrrrrrrrrrrrrrrrr} #1 \end{array}\right)}}

\newcommand{\Matrixc}[1]{\ensuremath{\left(\begin{array}{cccccccccccc} #1 \end{array}\right)}}



\renewcommand{\labelenumi}{\theenumi)}
\newenvironment{enumeratea}%
{\begingroup
 \renewcommand{\theenumi}{\alph{enumi}}
 \renewcommand{\labelenumi}{(\theenumi)}
 \begin{enumerate}}
 {\end{enumerate}\endgroup}



\newcounter{help}
\renewcommand{\thehelp}{\thesection.\arabic{equation}}

%\newenvironment{equation*}%
%{\renewcommand\endequation{\eqno (\theequation)* $$}%
%   \begin{equation}}%
%   {\end{equation}\renewcommand\endequation{\eqno \@eqnnum
%$$\global\@ignoretrue}}

%\input{psfig.tex}

\author{Martin Golubitsky and Michael Dellnitz}

%\newenvironment{matlabEquation}%
%{\renewcommand\endequation{\eqno (\theequation*) $$}%
%   \begin{equation}}%
%   {\end{equation}\renewcommand\endequation{\eqno \@eqnnum
% $$\global\@ignoretrue}}

\newcommand{\soln}{\textbf{Solution:} }
\newcommand{\exercap}[1]{\centerline{Figure~\ref{#1}}}
\newcommand{\exercaptwo}[1]{\centerline{Figure~\ref{#1}a\hspace{2.1in}
Figure~\ref{#1}b}}
\newcommand{\exercapthree}[1]{\centerline{Figure~\ref{#1}a\hspace{1.2in}
Figure~\ref{#1}b\hspace{1.2in}Figure~\ref{#1}c}}
\newcommand{\para}{\hspace{0.4in}}

\renewenvironment{solution}{\suppress}{\endsuppress}

\ifxake
\newenvironment{matlabEquation}{\begin{equation}}{\end{equation}}
\else
\newenvironment{matlabEquation}%
{\let\oldtheequation\theequation\renewcommand{\theequation}{\oldtheequation*}\begin{equation}}%
  {\end{equation}\let\theequation\oldtheequation}
\fi

\makeatother


\title{Matrix Multiplication of Vectors}

\begin{document}
\begin{abstract}
\end{abstract}
\maketitle

 \label{S:4.1}
\index{matrix!multiplication}

In Chapter~\ref{lineq} we discussed how matrices appear when solving systems 
of $m$ linear equations in $n$ unknowns.  Given the system
\begin{equation}   \label{general2}
\arraycolsep 2pt
\begin{array}{rcrcrcrcrl}
a_{11}x_1 & + & a_{12}x_2 & + & \cdots & + & a_{1n}x_n & = & b_1 & \\
a_{21}x_1 & + & a_{22}x_2 & + & \cdots & + & a_{2n}x_n & = & b_2 & \\
    & \vdots &      & \vdots &    & \vdots &     & \vdots &   & \\
a_{m1}x_1 & + & a_{m2}x_2 & + & \cdots & + & a_{mn}x_n & = & b_m & ,
\end{array}
\end{equation}
we saw that all relevant information is contained in the $m\times n$ matrix 
of coefficients
\[
A=\left(
\begin{array}{rrrr}
 a_{11} & a_{12} & \cdots & a_{1n} \\
 a_{21} & a_{22} & \cdots & a_{2n}  \\
 \vdots & \vdots &        & \vdots  \\
 a_{m1} & a_{m2} & \cdots & a_{mn}
\end{array}
\right)
\]
and the $n$ vector
\[
b=\left(
\begin{array}{c}
b_1 \\ \vdots \\ b_n
\end{array}
\right).
\]



\subsection*{Matrices Times Vectors}

We motivate multiplication of a matrix times a vector just as a notational 
advance that simplifies the
presentation of the linear systems.  It is, however, much more
than that.  This concept of multiplication allows us to think of
matrices as mappings and these mappings tell us much about the
structure of solutions to linear systems.  But first we discuss
the notational advantage.

Multiplying an $m\times n$ matrix $A$ times an $n$ vector $x$
\index{matrix vector product}
produces an $m$ vector, as follows:
\begin{equation}  \label{Atimesx}
Ax=\left(
\begin{array}{ccc}
 a_{11}  & \cdots & a_{1n} \\
 \vdots  &        & \vdots  \\
 a_{m1}  & \cdots & a_{mn}
\end{array}
\right)
\left(
\begin{array}{c}
x_1 \\ \vdots \\ x_n
\end{array}
\right)
=
\left(
\begin{array}{c}
a_{11}x_1 + \cdots + a_{1n}x_n \\ \vdots \\
a_{m1}x_1 + \cdots + a_{mn}x_n
\end{array}
\right).
\end{equation}
For example, when $m=2$ and $n=3$, then the product is a $2$-vector
\begin{equation} \label{Atimesx231}
\left(
\begin{array}{ccc}
 a_{11}  & a_{12} & a_{13} \\
 a_{21}  & a_{22} & a_{23}
\end{array}
\right)
\left(
\begin{array}{c}
x_1 \\ x_2 \\ x_3
\end{array}
\right)
=
\left(
\begin{array}{c}
a_{11}x_1 + a_{12}x_2 + a_{13}x_3 \\
a_{21}x_1 + a_{22}x_2 + a_{23}x_3
\end{array}
\right).
\end{equation}
As a specific example, compute
\begin{align*}
\left(
\begin{matrix}
 2  & 3 & -1 \\
 4  & 1 &  5
\end{matrix}
\right)
\left(
\begin{matrix}
2 \\ -3 \\ 4
\end{matrix}
\right)
&=
\left(
\begin{matrix}
2\cdot 2 & + & 3\cdot(-3) & + & (-1)\cdot 4 \\
4\cdot 2 & + & 1\cdot(-3) & + & 5\cdot 4
\end{matrix}
\right) \\
&=
\left(
\begin{matrix}
-9 \\ 25
\end{matrix}
\right).
\end{align*}


Using \eqref{Atimesx} we have a compact notation for writing
systems of linear equations.  For example, using a special
instance of \eqref{Atimesx231},
\[
\left(
\begin{array}{rrr}
 2  & 3 & -1 \\
 4  & 1 &  5
\end{array}
\right)
\left(
\begin{array}{r}
x_1 \\ x_2 \\ x_3
\end{array}
\right)
=
\left(
\begin{array}{c}
2x_1 + 3x_2 - x_3 \\
4x_1 + x_2 + 5x_3
\end{array}
\right).
\]
In this notation we can write the system of two linear equations
in three unknowns
\[
\begin{array}{ccccccr}
2x_1 & + & 3x_2 & - &  x_3 & = &   2 \\
4x_1 & + &  x_2 & + & 5x_3 & = &  -1
\end{array}
\]
as the matrix equation
\[
\left(
\begin{array}{rrr}
 2  & 3 & -1 \\
 4  & 1 &  5
\end{array}
\right)
\left(
\begin{array}{r}
x_1 \\ x_2 \\ x_3
\end{array}
\right)
=
\left(
\begin{array}{r}
2 \\ -1
\end{array}
\right).
\]

Indeed, the general system of linear equations \eqref{general2}
can be written in matrix form using matrix multiplication as
\[
Ax=b
\]
where $A$ is the $m\times n$ matrix of coefficients, $x$ is the
$n$ vector of unknowns, and $b$ is the $m$ vector of constants
on the right hand side of \eqref{general2}.


\subsection*{Matrices Times Vectors in \Matlab}

We have already seen how to define matrices and vectors in
\Matlabp.  Now we show how to multiply a matrix times a vector
using \Matlabp\index{matrix vector product!in \protect\Matlab}.

Load the matrix $A$
\begin{matlabEquation}  \label{eq:5matrix}
A=\left(
\begin{array}{rrrrr}
 5 & -4 &  3 & -6 &  2 \\
 2 & -4 & -2 & -1 &  1 \\
 1 &  2 &  1 & -5 &  3 \\
-2 & -1 & -2 &  1 & -1 \\
 1 & -6 &  1 &  1 &  4
\end{array}
\right)
\end{matlabEquation}
and the vector $x$
\begin{matlabEquation} \label{eq:5rhs}
x=\left(
\begin{array}{r}
 -1 \\
  2 \\
  1 \\
 -1 \\
  3
\end{array}
\right)
\end{matlabEquation}
into \Matlab by typing
\begin{verbatim}
e3_1_4
e3_1_5
\end{verbatim}
The multiplication $Ax$ can be performed by typing
\begin{verbatim}
b = A*x
\end{verbatim}  \index{\computer!*}
and the result should be
\begin{verbatim}
b =
     2
    -8
    18
    -6
    -1
\end{verbatim}
We may verify this result by solving the system of linear
equations $Ax=b$.  Indeed if we type
\begin{verbatim}
A\b
\end{verbatim}  \index{\computer!$\backslash$}
then we get the vector $x$ back as the answer.

\EXER

\TEXER


\begin{exercise} \label{c4.1.1}
Let
\[
A = \mattwo{2}{1}{-1}{4} \AND x = \vectwo{3}{-2}.
\]
Compute $Ax\begin{prompt}=\left(\begin{array}{r} \answer{4} \\ \answer{-11}\end{array}\right)\end{prompt}$.
\begin{hint}
  \[
Ax =
\left(\begin{array}{rr} 2 & 1 \\ -1 & 4\end{array}\right)
\left(\begin{array}{r} 3 \\ -2\end{array}\right) =
\left(\begin{array}{r} 6 - 2 \\ -3 - 8\end{array}\right) =
\left(\begin{array}{r} 4 \\ -11\end{array}\right)
\]
\end{hint}

\begin{solution}

\[
Ax =
\left(\begin{array}{rr} 2 & 1 \\ -1 & 4\end{array}\right)
\left(\begin{array}{r} 3 \\ -2\end{array}\right) =
\left(\begin{array}{r} 6 - 2 \\ -3 - 8\end{array}\right) =
\left(\begin{array}{r} 4 \\ -11\end{array}\right)
\]

\end{solution}
\end{exercise}

\begin{exercise} \label{c4.1.2}
Let
\[
B=\left(\begin{array}{ccc} 3 & 4 & 1\\ 1& 2 & 3 \end{array}\right)
\AND y=\left(\begin{array}{r} 2 \\ 5 \\ -2 \end{array}\right).
\]
Compute $By\begin{prompt}=\left(\begin{array}{r} \answer{24} \\ \answer{6}\end{array}\right)\end{prompt}$.
\begin{hint}
  \[
By =
\left(\begin{array}{rrr} 3 & 4 & 1 \\ 1 & 2 & 3\end{array}\right)
\left(\begin{array}{r} 2 \\ 5 \\ -2\end{array}\right) = 
\left(\begin{array}{r} 6 + 20 - 2 \\ 2 + 10 - 6\end{array}\right) =
\left(\begin{array}{r} 24 \\ 6\end{array}\right)
\]
\end{hint}

\begin{solution}

\[
By =
\left(\begin{array}{rrr} 3 & 4 & 1 \\ 1 & 2 & 3\end{array}\right)
\left(\begin{array}{r} 2 \\ 5 \\ -2\end{array}\right) = 
\left(\begin{array}{r} 6 + 20 - 2 \\ 2 + 10 - 6\end{array}\right) =
\left(\begin{array}{r} 24 \\ 6\end{array}\right)
\]

\end{solution}
\end{exercise}

\noindent In Exercises~\ref{c4.1.a3a} -- \ref{c4.1.a3d} decide whether or
not the matrix vector product $Ax$ can be computed; if it can, compute the
product.
\begin{exercise} \label{c4.1.a3a}
  $A=\mattwo{1}{2}{0}{-5} \AND x = \vectwo{2}{2}$.
  \begin{multipleChoice}
    \choice[correct]{The matrix vector product can be computed.}
    \choice{The matrix vector product can not be computed.}    
  \end{multipleChoice}
  \begin{exercise}
    $Ax = \vectwo{\answer{6}}{\answer{-10}}$.

\begin{solution}
$Ax = \vectwo{6}{-10}$.

\end{solution}
  \end{exercise}

\begin{solution}


\end{solution}
\end{exercise}

\begin{exercise} \label{c4.1.a3b}
$A=\mattwo{1}{2}{0}{-5} \AND
x = \left(\begin{array}{r} 2\\ 2 \\4\end{array}\right)$.
  \begin{multipleChoice}
    \choice{The matrix vector product can be computed.}
    \choice[correct]{The matrix vector product can not be computed.}    
  \end{multipleChoice}

\begin{solution}
The product $Ax$ cannot be computed.

\end{solution}
\end{exercise}

\begin{exercise} \label{c4.1.a3c}
$A=\left(\begin{array}{rrr} 1 & 2 & 4 \end{array}\right) \AND
x = \left(\begin{array}{r} -1\\ 1 \\3\end{array}\right)$.
  \begin{multipleChoice}
    \choice[correct]{The matrix vector product can be computed.}
    \choice{The matrix vector product can not be computed.}    
  \end{multipleChoice}
  \begin{exercise}
    $Ax = \left(\answer{13}\right)$.

\begin{solution}
$Ax = \left(13\right)$.

\end{solution}
  \end{exercise}

\begin{solution}


\end{solution}
\end{exercise}

\begin{exercise} \label{c4.1.a3d}
  $A=\left(5\right) \AND x = \vectwo{1}{0}$.
  \begin{multipleChoice}
    \choice{The matrix vector product can be computed.}
    \choice[correct]{The matrix vector product can not be computed.}    
  \end{multipleChoice}  

\begin{solution}
The product $Ax$ cannot be computed.

\end{solution}
\end{exercise}

\begin{exercise} \label{c4.1.b3}
Let
\[
A=\left(
\begin{array}{rrrr}
 a_{11} & a_{12} & \cdots & a_{1n} \\
 a_{21} & a_{22} & \cdots & a_{2n}  \\
 \vdots & \vdots &        & \vdots  \\
 a_{m1} & a_{m2} & \cdots & a_{mn}
\end{array}
\right)\quad\mbox{and}\quad
x =
\left( \begin{array}{r} x_1\\ x_2\\ \vdots\\ x_n\end{array}\right).
\]
Denote the columns of the matrix $A$ by
\[
A_1 =
\left(\begin{array}{c} a_{11}\\ a_{21}\\ \vdots\\
a_{m1}\end{array}\right),\quad
A_2 =
\left(\begin{array}{c} a_{12}\\ a_{22}\\ \vdots\\
a_{m2}\end{array}\right),\quad
\cdots\quad
A_n =
\left(\begin{array}{c} a_{1n}\\ a_{2n}\\ \vdots\\ a_{mn}\end{array}\right).
\]
Show that the matrix vector product $Ax$ can be written as
\[
Ax = x_1 A_1 + x_2 A_2 + \cdots + x_n A_n,
\]
where $x_j A_j$ denotes scalar multiplication (see Chapter~\ref{chap:prelim}).

\begin{solution}

Compute $Ax$ directly:
\[ Ax = \left(\begin{array}{c} x_1a_{11} + x_2a_{12} + \cdots +
x_na_{1n} \\  x_1a_{21} + x_2a_{22} + \cdots + x_na_{2n} \\
\\ \vdots \\ x_1a_{m1} + x_2a_{m2} + \cdots + x_na_{mn}
\end{array}\right) = x_1\left(\begin{array}{r} a_{11} \\ a_{21} \\
\vdots \\ a_{m1} \end{array}\right) + x_2\left(\begin{array}{r}
a_{12} \\ a_{22} \\ \vdots \\ a_{m2} \end{array}\right) + \cdots
+ x_n\left(\begin{array}{r} a_{1n} \\ a_{2n} \\
\vdots \\ a_{mn} \end{array}\right). \]
So, it is indeed true that $Ax = x_1A_1 + x_2A_2 + \cdots
+ x_nA_n$.


\end{solution}
\end{exercise}


\begin{exercise} \label{c4.1.3}
Let
\[
C = \mattwo{1}{1}{2}{-1} \AND b = \vectwo{1}{1}.
\]
Find a $2$-vector $z$ such that $Cz=b$.

\begin{solution}

\ans The equation $Cz = b$ is valid for $z = (\frac{2}{3},\frac{1}{3})^t$.

\soln Let $z = (z_1,z_2)^t$.  Then $Cz = b$ implies
\[
\left(\begin{array}{rr} 1 & 1 \\ 2 & -1\end{array}\right)
\left(\begin{array}{r} z_1 \\ z_2\end{array}\right) =
\left(\begin{array}{r} 1 \\ 1\end{array}\right)
\]
which can be multiplied out, yielding the linear system
\[\begin{array}{rrrrl}
z_1 & + & z_2 & = & 1 \\
2z_1 & - & z_2 & = & 1.\end{array} \]
This system can be solved by substitution to obtain $z_1 =
\frac{2}{3}$ and $z_2 = \frac{1}{3}$.

\end{solution}
\end{exercise}

\begin{exercise} \label{c4.1.4}
Write the system of linear equations
\begin{eqnarray*}
2x_1 + 3x_2 - 2x_3 & = & 4\\
6x_1 -5x_3 & = & 1
\end{eqnarray*}
in the matrix form $Ax=b$.

\begin{solution}

\[
\left(\begin{array}{rrr} 2 & 3 & -2 \\ 6 & 0 & -5\end{array}\right) 
\left(\begin{array}{r} x_1 \\ x_2 \\ x_3\end{array}\right) = 
\left(\begin{array}{r} 4 \\ 1\end{array}\right)
\]


\end{solution}
\end{exercise}


\begin{exercise} \label{c4.1.6}
Find all solutions to
\[
\left(\begin{array}{rrrr} 1 & 3 & -1 & 4 \\ 2 & 1 & 5 & 7 \\
3 & 4 & 4 & 11 \end{array} \right)
\left(\begin{array}{c} x_1 \\ x_2 \\ x_3 \\ x_4\end{array}\right) =
\left(\begin{array}{c} 14 \\ 17 \\31 \end{array}\right).
\]

\begin{solution}

\ans All solutions are of the form
\[ \left(\begin{array}{r} x_1 \\ x_2 \\ x_3 \\ x_4\end{array}\right) =
\left(\begin{array}{c} \frac{37}{5} - \frac{16}{5}x_3 - \frac{17}{5}x_4 \\
\frac{11}{5} + \frac{7}{5}x_3 - \frac{1}{5}x_4 \\ x_3 \\ x_4\end{array}\right)
\]
where $x_3$ and $x_4$ are free parameters.

\soln Create the augmented matrix
\[ \left(\begin{array}{rrrr|r}
1 & 3 & -1 & 4 & 14 \\
2 & 1 & 5 & 7 & 17 \\
3 & 4 & 4 & 11 & 31\end{array}\right) \]
which can be row reduced to
\[ \left(\begin{array}{rrrr|r}
1 & 0 & \frac{16}{5} & \frac{17}{5} & \frac{37}{5} \\
0 & 1 & -\frac{7}{5} & \frac{1}{5} & \frac{11}{5} \\
0 & 0 & 0 & 0 & 0\end{array}\right), \]
yielding the desired solution.

\end{solution}
\end{exercise}

\begin{exercise} \label{c4.1.7}
Let $A$ be a $2\times 2$ matrix.  Find $A$ so that
\begin{eqnarray*}
A\left(\begin{array}{c} 1 \\ 0 \end{array}\right) =
\left(\begin{array}{r} 3 \\ -5 \end{array}\right) \\
A\left(\begin{array}{c} 0 \\ 1 \end{array}\right) =
\left(\begin{array}{r} 1 \\ 4 \end{array}\right).
\end{eqnarray*}

\begin{solution}

\ans The equations are valid when
\[ A = \left(\begin{array}{rr} 3 & 1 \\ -5 & 4\end{array}\right). \]
\soln Let
\[ A = \left(\begin{array}{rr} a_{11} & a_{12} \\ a_{21} & 
a_{22}\end{array}\right). \]
So
\[ \left(\begin{array}{rr} a_{11} & a_{12} \\ a_{21} & a_{22}\end{array}\right)
\left(\begin{array}{r} 1 \\ 0\end{array}\right) =
\left(\begin{array}{r} 3 \\ -5\end{array}\right) \AND
\left(\begin{array}{rr} a_{11} & a_{12} \\ a_{21} & a_{22}\end{array}\right)
\left(\begin{array}{r} 0 \\ 1\end{array}\right) =
\left(\begin{array}{r} 1 \\ 4.\end{array}\right) \]
These matrix equations are equivalent to the linear equations
\[ \begin{array}{rcl}
a_{11} & = & 3 \\
a_{21} & = & -5 \\
a_{12} & = & 1 \\
a_{22} & = & 4.\end{array} \]


\end{solution}
\end{exercise}

\begin{exercise} \label{c4.1.8}
Let $A$ be a $2\times 2$ matrix.  Find $A$ so that
\begin{eqnarray*}
A\left(\begin{array}{r} 1 \\ 1 \end{array}\right) =
\left(\begin{array}{r} 2 \\ -1 \end{array}\right) \\
A\left(\begin{array}{r} 1 \\ -1 \end{array}\right) =
\left(\begin{array}{r} 4 \\ 3 \end{array}\right).
\end{eqnarray*}

\begin{solution}

\ans The equations are valid when
\[ A = \left(\begin{array}{rr} 3 & -1 \\ 1 & -2\end{array}\right). \]
\soln Let
\[ A = \left(\begin{array}{rr} a_{11} & a_{12} \\ a_{21} & 
a_{22}\end{array}\right) \]
Then
\[ \left(\begin{array}{rr} a_{11} & a_{12} \\ a_{21} & a_{22}\end{array}\right)
\left(\begin{array}{r} 1 \\ 1\end{array}\right) =
\left(\begin{array}{r} 2 \\ -1\end{array}\right) \AND
\left(\begin{array}{rr} a_{11} & a_{12} \\ a_{21} & a_{22}\end{array}\right)
\left(\begin{array}{r} 1 \\ -1\end{array}\right) =
\left(\begin{array}{r} 4 \\ 3.\end{array}\right) \]
These matrix equations yield the linear system
\[ \begin{array}{rrrrrrrrr}
a_{11} & + & a_{12} & & & & & = & 2 \\
& & & & a_{21} & + & a_{22} & = & -1 \\
a_{11} & - & a_{12} & & & & & = & 4 \\
& & & & a_{21} & - & a_{22} & = & 3, \end{array} \]
which can be written as an augmented matrix and row-reduced to yield
the values $a_{ij}$:
\[ \left(\begin{array}{rrrr|r}
1 & 1 & 0 & 0 & 2 \\
0 & 0 & 1 & 1 & -1 \\
1 & -1 & 0 & 0 & 4 \\
0 & 0 & 1 & -1 & 3\end{array}\right)
\longrightarrow
\left(\begin{array}{rrrr|r}
1 & 0 & 0 & 0 & 3 \\
0 & 1 & 0 & 0 & -1 \\
0 & 0 & 1 & 0 & 1 \\
0 & 0 & 0 & 1 & 2\end{array}\right). \]



\end{solution}
\end{exercise}

\begin{exercise} \label{c4.1.9}
Is there an upper triangular $2\times 2$ matrix $A$ such that
\begin{equation}  \label{eq:avect}
A\vectwo{1}{0} = \vectwo{1}{2}?
\end{equation}
Is there a symmetric $2\times 2$ matrix $A$ satisfying \eqref{eq:avect}?

\begin{solution}

\ans There is no $2 \times 2$ upper triangular matrix $A$ that
satisfies equation \eqref{eq:avect}, but any symmetric matrix $A$ of the form
\[ A = \mattwo{1}{2}{2}{a_{22}}, \]
where $a_{22}$ is a real number, satisfies \eqref{eq:avect}.

\soln Let $A$ be the upper triangular matrix
\[ \mattwo{a_{11}}{a_{12}}{0}{a_{22}}. \]
The resulting matrix equation
\[ \mattwo{a_{11}}{a_{12}}{0}{a_{22}}
\vectwo{1}{0} = \vectwo{1}{2} \]
yields the linear equations
\[ \begin{array}{rcl}
a_{11} & = & 1 \\
0 & = & 2.\end{array} \]
The second equation is inconsistent, so there is no solution.

\para Then let $A$ be the symmetric matrix
\[ \mattwo{a_{11}}{a_{12}}{a_{12}}{a_{22}}. \]
Write the matrix equation
\[ \mattwo{a_{11}}{a_{12}}{a_{12}}{a_{22}}
\vectwo{1}{0} = \vectwo{1}{2}, \]
from which we obtain the consistent linear system
\[ \begin{array}{rcl}
a_{11} & = & 1 \\
a_{12} & = & 2.\end{array} \]

\end{solution}
\end{exercise}

\CEXER

\noindent In Exercises~\ref{c4.1.a10a} -- \ref{c4.1.a10b} use \Matlab to
compute $b=Ax$ for the given $A$ and $x$.
\begin{exercise} \label{c4.1.a10a}
\begin{matlabEquation}\label{multiplication-exercise}
A=\left(
\begin{array}{rrrrr}
   -0.2 &   -1.8 &    3.9 &    -6 &   -1.6\\
    6.3 &    8   &    3   &    2.5 &    5.1\\
   -0.8 &   -9.9 &    9.7 &    4.7 &    5.9\\
   -0.9 &   -4.1 &    1.1 &   -2.5 &    8.4\\
   -1 &   -9 &   -2 &  -9.8 &    6.9
\end{array}\right)
\AND
x = \left( \begin{array}{r} -2.6\\  2.4\\  4.6\\   -6.1\\    8.1
\end{array}\right).
\end{matlabEquation}

\begin{solution}
After loading the system into \Matlabp, typing
{\tt b = A*x} yields
\begin{verbatim}
b =
   37.7800
   42.6800
   42.0600
   77.6100
   87.4700
\end{verbatim}

\end{solution}
\end{exercise}
\begin{exercise} \label{c4.1.a10b}
\begin{matlabEquation}\label{multiplication-exercise-2}
A=\left(
\begin{array}{rrrrrrr}
    14 & -22  &-26 &  -2 & -77 & 100 & -90\\
    26 &  25  &-15 & -63 &  33 &  92 &  14\\
   -53 &  40  & 19 &  40 & -27 & -88 &  40\\
    10 & -21  & 13 &  97 & -72 & -28 &  92\\
    86 & -17  & 43 &  61 &  13 &  10 &  50\\
   -33 &  31  &  2 &  41 &  65 & -48 &  48\\
    31 &  68  & 55 &  -3 &  35 &  19 & -14
\end{array}
\right)\end{matlabEquation}
and
\begin{equation*}
x = \left( \begin{array}{r} 2.7\\   6.1\\   -8.3\\    8.9\\    8.3\\    2\\
  -4.9
\end{array}\right).
\end{equation*}

\begin{solution}
Load the system into \Matlabp, then type {\tt b = A*x}
to obtain:
\begin{verbatim}
b =
  103.5000
  175.8000
 -296.9000
 -450.1000
  197.4000
  656.6000
  412.4000
\end{verbatim}

\end{solution}
\end{exercise}



\begin{exercise} \label{c4.1.10}
Let
\begin{matlabEquation}\label{inverse-exercise}
A = \left(\begin{array}{rrr} 2 & 4 & -1 \\ 1 & 3 & 2\\
-1 & -2 & 5 \end{array}\right) \AND
b = \left(\begin{array}{c} 2 \\ 1 \\ 4 \end{array}\right).
\end{matlabEquation}
Find a $3$-vector $x$ such that $Ax=b$.

\begin{solution}
Using \Matlabp:
\begin{verbatim}
A\b =
    7.1111
   -2.7778
    1.1111
\end{verbatim}

\end{solution}
\end{exercise}

\begin{exercise} \label{c4.1.11}
Let
\begin{matlabEquation}\label{MATLAB:24}
A = \left(\begin{array}{rrr} 1.3 & -4.15 & -1.2 \\ 1.6 & -1.2 & 2.4\\
-2.5 & 2.35 & 5.09 \end{array}\right) \AND
b = \left(\begin{array}{c} 1.12 \\ -2.1 \\ 4.36 \end{array}\right).
\end{matlabEquation}
Find a $3$-vector $x$ such that $Ax=b$.

\begin{solution}
Using \Matlabp:
\begin{verbatim}
A\b =
   -2.3828
   -1.0682
    0.1794
\end{verbatim}

\end{solution}
\end{exercise}

\begin{exercise} \label{c4.1.12}
Let $A$ be a $3\times 3$ matrix.  Find $A$ so that
\begin{eqnarray*}
A\left(\begin{array}{r} 2 \\ -1 \\ 1 \end{array}\right) =
\left(\begin{array}{r} 1 \\ 1  \\ -1 \end{array}\right) \\
A\left(\begin{array}{r} 1 \\ -1 \\ 0 \end{array}\right) =
\left(\begin{array}{r} -1 \\ -2 \\ 1 \end{array}\right)\\
A\left(\begin{array}{r} 0 \\ 2 \\ 4 \end{array}\right) =
\left(\begin{array}{r} 5 \\ 1 \\ 1 \end{array}\right).
\end{eqnarray*}
{\bf Hint:} Rewrite these three conditions as a system of linear
equations in the nine entries of $A$.  Then solve this system
using \Matlabp.  (Then pray that there is an easier way.)

\begin{solution}

\ans The conditions on $A$ are met when
\[ A = \matthree{2.5}{3.5}{-0.5}{7.5}{9.5}{-4.5}{-5.5}{-6.5}{3.5}. \]

\soln Let
\[
A = \matthree{a_{11}}{a_{12}}{a_{13}}{a_{21}}{a_{22}}{a_{23}}{a_{31}}{a_{32}} 
{a_{33}}
\]
Substituting for $A$ in each of the three equations yields:
\[ \begin{array}{c}
\matthree{a_{11}}{a_{12}}{a_{13}}{a_{21}}{a_{22}}{a_{23}}{a_{31}}
{a_{32}} {a_{33}} \vecthree{2}{-1}{1} = \vecthree{1}{1}{-1} \\
\matthree{a_{11}}{a_{12}}{a_{13}}{a_{21}}{a_{22}}{a_{23}}{a_{31}}
{a_{32}} {a_{33}} \vecthree{1}{-1}{0} = \vecthree{-1}{-2}{1} \\
\matthree{a_{11}}{a_{12}}{a_{13}}{a_{21}}{a_{22}}{a_{23}}{a_{31}}
{a_{32}} {a_{33}} \vecthree{0}{2}{4} = \vecthree{5}{1}{1}.\end{array} \]
These equations can be rewritten as the linear system:
\[ \begin{array}{rrrrrrr}
2a_{11} & - & a_{12} & + & a_{13} & = & 1 \\
2a_{21} & - & a_{22} & + & a_{23} & = & 1 \\
2a_{31} & - & a_{32} & + & a_{33} & = & -1 \\
a_{11} & - & a_{12} & & & = & -1 \\
a_{21} & - & a_{22} & & & = & -2 \\
a_{31} & - & a_{32} & & & = & 1 \\
& & 2a_{12} & + & 4a_{13} & = & 5 \\
& & 2a_{22} & + & 4a_{23} & = & 1 \\
& & 2a_{32} & + & 4a_{33} & = & 1 \end{array} \]
We can enter the left-hand side of the system into \Matlab as
matrix {\tt C} and the right-hand side as vector {\tt b}. 
\[ C = \left(\begin{array}{rrrrrrrrr}
2 & -1 & 1 & 0 & 0 & 0 & 0 & 0 & 0 \\
0 & 0 & 0 & 2 & -1 & 1 & 0 & 0 & 0 \\
0 & 0 & 0 & 0 & 0 & 0 & 2 & -1 & 1 \\
1 & -1 & 0 & 0 & 0 & 0 & 0 & 0 & 0 \\
0 & 0 & 0 & 1 & -1 & 0 & 0 & 0 & 0 \\
0 & 0 & 0 & 0 & 0 & 0 & 1 & -1 & 0 \\
0 & 2 & 4 & 0 & 0 & 0 & 0 & 0 & 0 \\
0 & 0 & 0 & 0 & 2 & 4 & 0 & 0 & 0 \\
0 & 0 & 0 & 0 & 0 & 0 & 0 & 2 & 4 \end{array}\right)
\mbox{\hspace{1.0in}}
b = \left(\begin{array}{r}
1 \\ 1 \\ - 1 \\ -1 \\ -2 \\ 1 \\ 5 \\ 1 \\ 1 \end{array}\right). \]
Then solve to obtain:

\begin{verbatim}
C\b =
    2.5000
    3.5000
   -0.5000
    7.5000
    9.5000
   -4.5000
   -5.5000
   -6.5000
    3.5000
\end{verbatim}
Substitute these values back into $A$ to obtain the appropriate matrix.



\end{solution}
\end{exercise}


\end{document}

%%% Local Variables:
%%% mode: latex
%%% TeX-master: t
%%% End:
