\documentclass{ximera}

 

\usepackage{epsfig}

\graphicspath{
  {./}
  {figures/}
}

\usepackage{morewrites}
\makeatletter
\newcommand\subfile[1]{%
\renewcommand{\input}[1]{}%
\begingroup\skip@preamble\otherinput{#1}\endgroup\par\vspace{\topsep}
\let\input\otherinput}
\makeatother

\newcommand{\includeexercises}{\directlua{dofile("/home/jim/linearAlgebra/laode/exercises.lua")}}

%\newcounter{ccounter}
%\setcounter{ccounter}{1}
%\newcommand{\Chapter}[1]{\setcounter{chapter}{\arabic{ccounter}}\chapter{#1}\addtocounter{ccounter}{1}}

%\newcommand{\section}[1]{\section{#1}\setcounter{thm}{0}\setcounter{equation}{0}}

%\renewcommand{\theequation}{\arabic{chapter}.\arabic{section}.\arabic{equation}}
%\renewcommand{\thefigure}{\arabic{chapter}.\arabic{figure}}
%\renewcommand{\thetable}{\arabic{chapter}.\arabic{table}}

%\newcommand{\Sec}[2]{\section{#1}\markright{\arabic{ccounter}.\arabic{section}.#2}\setcounter{equation}{0}\setcounter{thm}{0}\setcounter{figure}{0}}

\newcommand{\Sec}[2]{\section{#1}}

\setcounter{secnumdepth}{2}
%\setcounter{secnumdepth}{1} 

%\newcounter{THM}
%\renewcommand{\theTHM}{\arabic{chapter}.\arabic{section}}

\newcommand{\trademark}{{R\!\!\!\!\!\bigcirc}}
%\newtheorem{exercise}{}

\newcommand{\dfield}{{\sf dfield9}}
\newcommand{\pplane}{{\sf pplane9}}

\newcommand{\EXER}{\section*{Exercises}}%\vspace*{0.2in}\hrule\small\setcounter{exercise}{0}}
\newcommand{\CEXER}{}%\vspace{0.08in}\begin{center}Computer Exercises\end{center}}
\newcommand{\TEXER}{} %\vspace{0.08in}\begin{center}Hand Exercises\end{center}}
\newcommand{\AEXER}{} %\vspace{0.08in}\begin{center}Hand Exercises\end{center}}

% BADBAD: \newcommand{\Bbb}{\bf}

\newcommand{\R}{\mbox{$\Bbb{R}$}}
\newcommand{\C}{\mbox{$\Bbb{C}$}}
\newcommand{\Z}{\mbox{$\Bbb{Z}$}}
\newcommand{\N}{\mbox{$\Bbb{N}$}}
\newcommand{\D}{\mbox{{\bf D}}}
\usepackage{amssymb}
%\newcommand{\qed}{\hfill\mbox{\raggedright$\square$} \vspace{1ex}}
%\newcommand{\proof}{\noindent {\bf Proof:} \hspace{0.1in}}

\newcommand{\setmin}{\;\mbox{--}\;}
\newcommand{\Matlab}{{M\small{AT\-LAB}} }
\newcommand{\Matlabp}{{M\small{AT\-LAB}}}
\newcommand{\computer}{\Matlab Instructions}
\newcommand{\half}{\mbox{$\frac{1}{2}$}}
\newcommand{\compose}{\raisebox{.15ex}{\mbox{{\scriptsize$\circ$}}}}
\newcommand{\AND}{\quad\mbox{and}\quad}
\newcommand{\vect}[2]{\left(\begin{array}{c} #1_1 \\ \vdots \\
 #1_{#2}\end{array}\right)}
\newcommand{\mattwo}[4]{\left(\begin{array}{rr} #1 & #2\\ #3
&#4\end{array}\right)}
\newcommand{\mattwoc}[4]{\left(\begin{array}{cc} #1 & #2\\ #3
&#4\end{array}\right)}
\newcommand{\vectwo}[2]{\left(\begin{array}{r} #1 \\ #2\end{array}\right)}
\newcommand{\vectwoc}[2]{\left(\begin{array}{c} #1 \\ #2\end{array}\right)}

\newcommand{\ignore}[1]{}


\newcommand{\inv}{^{-1}}
\newcommand{\CC}{{\cal C}}
\newcommand{\CCone}{\CC^1}
\newcommand{\Span}{{\rm span}}
\newcommand{\rank}{{\rm rank}}
\newcommand{\trace}{{\rm tr}}
\newcommand{\RE}{{\rm Re}}
\newcommand{\IM}{{\rm Im}}
\newcommand{\nulls}{{\rm null\;space}}

\newcommand{\dps}{\displaystyle}
\newcommand{\arraystart}{\renewcommand{\arraystretch}{1.8}}
\newcommand{\arrayfinish}{\renewcommand{\arraystretch}{1.2}}
\newcommand{\Start}[1]{\vspace{0.08in}\noindent {\bf Section~\ref{#1}}}
\newcommand{\exer}[1]{\noindent {\bf \ref{#1}}}
\newcommand{\ans}{}
\newcommand{\matthree}[9]{\left(\begin{array}{rrr} #1 & #2 & #3 \\ #4 & #5 & #6
\\ #7 & #8 & #9\end{array}\right)}
\newcommand{\cvectwo}[2]{\left(\begin{array}{c} #1 \\ #2\end{array}\right)}
\newcommand{\cmatthree}[9]{\left(\begin{array}{ccc} #1 & #2 & #3 \\ #4 & #5 &
#6 \\ #7 & #8 & #9\end{array}\right)}
\newcommand{\vecthree}[3]{\left(\begin{array}{r} #1 \\ #2 \\
#3\end{array}\right)}
\newcommand{\cvecthree}[3]{\left(\begin{array}{c} #1 \\ #2 \\
#3\end{array}\right)}
\newcommand{\cmattwo}[4]{\left(\begin{array}{cc} #1 & #2\\ #3
&#4\end{array}\right)}

\newcommand{\Matrix}[1]{\ensuremath{\left(\begin{array}{rrrrrrrrrrrrrrrrrr} #1 \end{array}\right)}}

\newcommand{\Matrixc}[1]{\ensuremath{\left(\begin{array}{cccccccccccc} #1 \end{array}\right)}}



\renewcommand{\labelenumi}{\theenumi)}
\newenvironment{enumeratea}%
{\begingroup
 \renewcommand{\theenumi}{\alph{enumi}}
 \renewcommand{\labelenumi}{(\theenumi)}
 \begin{enumerate}}
 {\end{enumerate}\endgroup}



\newcounter{help}
\renewcommand{\thehelp}{\thesection.\arabic{equation}}

%\newenvironment{equation*}%
%{\renewcommand\endequation{\eqno (\theequation)* $$}%
%   \begin{equation}}%
%   {\end{equation}\renewcommand\endequation{\eqno \@eqnnum
%$$\global\@ignoretrue}}

%\input{psfig.tex}

\author{Martin Golubitsky and Michael Dellnitz}

%\newenvironment{matlabEquation}%
%{\renewcommand\endequation{\eqno (\theequation*) $$}%
%   \begin{equation}}%
%   {\end{equation}\renewcommand\endequation{\eqno \@eqnnum
% $$\global\@ignoretrue}}

\newcommand{\soln}{\textbf{Solution:} }
\newcommand{\exercap}[1]{\centerline{Figure~\ref{#1}}}
\newcommand{\exercaptwo}[1]{\centerline{Figure~\ref{#1}a\hspace{2.1in}
Figure~\ref{#1}b}}
\newcommand{\exercapthree}[1]{\centerline{Figure~\ref{#1}a\hspace{1.2in}
Figure~\ref{#1}b\hspace{1.2in}Figure~\ref{#1}c}}
\newcommand{\para}{\hspace{0.4in}}

\renewenvironment{solution}{\suppress}{\endsuppress}

\ifxake
\newenvironment{matlabEquation}{\begin{equation}}{\end{equation}}
\else
\newenvironment{matlabEquation}%
{\let\oldtheequation\theequation\renewcommand{\theequation}{\oldtheequation*}\begin{equation}}%
  {\end{equation}\let\theequation\oldtheequation}
\fi

\makeatother


\title{*The Cayley Hamilton Theorem}

\begin{document}
\begin{abstract}
\end{abstract}
\maketitle

\label{S:6.6}
%\index{closed form solution}\index{matrix!exponential!computation}

The Jordan normal form theorem (Theorem~\ref{T:putinform}) for real $2\times 2$ matrices states that 
every $2\times 2$ matrix is similar to one of the matrices in Table~\ref{T:3sys}.
We use this theorem to prove the Cayley Hamilton theorem 
\index{Cayley Hamilton theorem} for $2\times 2$ 
matrices and then use the Cayley Hamilton theorem to present another method 
for computing solutions to planar linear systems of differential equations in the 
case of real equal eigenvalues.


The Cayley Hamilton theorem states that a matrix satisfies its own
characteristic polynomial.  More precisely:
\begin{theorem}[Cayley Hamilton Theorem] \label{T:CH2}
Let $A$ be a $2\times 2$ matrix and let
\[
p_A(\lambda) = \lambda^2 + a\lambda + b
\]
be the characteristic polynomial\index{characteristic polynomial} of $A$.  Then
\[
p_A(A) = A^2 + aA + bI_2 = 0.
\]
\end{theorem}

\begin{proof}  Suppose $B=P\inv AP$ and $A$ are similar matrices.  We claim that
if $p_A(A)=0$, then $p_B(B)=0$.  To verify this claim, recall from
Lemma~\ref{L:simdettr} that $p_A=p_B$ and calculate
\begin{align*}
  p_B(B) &= p_A(P\inv AP) = (P\inv AP)^2 + aP\inv AP + bI_2 \\
  &= P\inv p_A(A)P= 0.
\end{align*}
Theorem~\ref{T:putinform} classifies $2\times 2$ matrices up to similarity.
Thus, we need only verify this theorem for the matrices
\begin{align*}
C &=  \mattwoc{\lambda_1}{0}{0}{\lambda_2},
D &=  \mattwo{\sigma}{-\tau}{\tau}{\sigma},
E &=  \mattwoc{\lambda_1}{1}{0}{\lambda_1},
\end{align*}
that is, we need to verify that
\[
p_C(C) = 0 \qquad p_D(D)=0 \qquad p_E(E)=0.
\]

Using the fact that $p_A(\lambda)=\lambda^2-\trace(A)\lambda+\det(A)$, we see
that
\begin{eqnarray*}
p_C(\lambda) & = & (\lambda-\lambda_1)(\lambda-\lambda_2) \\
p_D(\lambda) & = & \lambda^2 - 2\sigma \lambda + (\sigma^2+\tau^2) \\
p_E(\lambda) & = & (\lambda-\lambda_1)^2.
\end{eqnarray*}
It now follows that
\begin{align*}
  p_C(C) &= (C-\lambda_1I_2)(C-\lambda_2I_2) 
           \\
&=\mattwoc{0}{0}{0}{\lambda_2-\lambda_1}\mattwoc{\lambda_1-\lambda_2}{0}{0}{0} \\
&=0,
\end{align*}
and
\begin{align*}
p_D(D) &=
\mattwoc{\sigma^2-\tau^2}{-2\sigma\tau}{2\sigma\tau}{\sigma^2-\tau^2} -\\
       &\quad 2\sigma \mattwo{\sigma}{-\tau}{\tau}{\sigma} + \\
  &\quad (\sigma^2+\tau^2)\mattwo{1}{0}{0}{1} = 0,
\end{align*}
and
\[
p_E(E) = (E-\lambda_1I_2)^2 = \mattwo{0}{1}{0}{0}^2 = 0.
\]
\end{proof}


\subsection*{The Example with Equal Eigenvalues Revisited}

When the eigenvalues $\lambda_1=\lambda_2$, the closed form solution of 
$\dot{X} = CX$ is a straightforward formula
\begin{equation}  \label{E:exeq}
X(t) = e^{\lambda_1 t}(I_2 + tN)
\end{equation}
where $N = C - \lambda_1I_2$.

Note that when using \eqref{E:exeq}, it is not necessary to 
compute the eigenvector or generalized eigenvector of $C$, 
and this is a substantial simplification.  

\subsubsection*{Verification of \protect\eqref{E:exeq}}

We use the Cayley-Hamilton theorem to verify \eqref{E:exeq} as follows.
Specifically, since $C$ is assumed to have a double eigenvalue $\lambda_1$, it follows that
\[
N = C - \lambda_1 I_2
\]
has zero as a double eigenvalue.  Hence, the characteristic polynomial
$p_N(\lambda) = \lambda^2$ and the Cayley Hamilton theorem implies that
$N^2=0$.  Therefore,
\[
CX(t) = e^{\lambda_1t}C(I_2+tN)X_0 = e^{\lambda_1t}(\lambda_1 I_2+N)(I_2+tN)X_0 
\]
Thus, using $N^2=0$, we see that 
\[
CX(t) = e^{\lambda_1t} (\lambda_1 I_2+t\lambda_1 N + N)X_0  =\dot{X}(t),
\]
as desired

%As observed in Section~\ref{S:Matrixexp} Exercise~\ref{c6.3.14} there is a similar formula in the case that the eigenvales are real and unequal; namely, \begin{equation}  \label{E:exdist2} e^{tC} = \frac{1}{\lambda_2-\lambda_1}\left(e^{\lambda_1 t}(C-\lambda_2I_2) - e^{\lambda_2 t}(C-\lambda_1I_2)\right).\end{equation}
%Note that the computation of the eigenvectors of $C$ is also not needed here.


Let us reconsider the system of differential equations \eqref{e:shearexample}
\[
\frac{dX}{dt} = \mattwo{1}{-1}{9}{-5} X = CX
\]
with initial value
\[
X_0 = \vectwo{2}{3}.
\]
The eigenvalues of $C$ are real and equal to $\lambda_1=-2$.

We may write
\[
C = \lambda_1 I_2 + N = -2I_2+N,
\]
where
\[
N = \mattwo{3}{-1}{9}{-3}.
\]
It follows from \eqref{E:exeq} that
\begin{equation}  \label{e:solntob}
e^{tC} =  e^{-2t}\left(I_2+t\mattwo{3}{-1}{9}{-3}\right)
= e^{-2t}\mattwoc{1+3t}{-t}{9t}{1-3t}.
\end{equation}
Hence the solution to the initial value problem is:
\begin{align*}
X(t) &= e^{tC}X_0 = e^{-2t}\mattwoc{1+3t}{-t}{9t}{1-3t}\vectwo{2}{3} \\
  &= e^{-2t}\vectwo{2+3t}{3+9t}.
\end{align*}




\EXER

\includeexercises



\end{document}
