\documentclass{ximera}

 

\usepackage{epsfig}

\graphicspath{
  {./}
  {figures/}
}

\usepackage{morewrites}
\makeatletter
\newcommand\subfile[1]{%
\renewcommand{\input}[1]{}%
\begingroup\skip@preamble\otherinput{#1}\endgroup\par\vspace{\topsep}
\let\input\otherinput}
\makeatother

\newcommand{\includeexercises}{\directlua{dofile("/home/jim/linearAlgebra/laode/exercises.lua")}}

%\newcounter{ccounter}
%\setcounter{ccounter}{1}
%\newcommand{\Chapter}[1]{\setcounter{chapter}{\arabic{ccounter}}\chapter{#1}\addtocounter{ccounter}{1}}

%\newcommand{\section}[1]{\section{#1}\setcounter{thm}{0}\setcounter{equation}{0}}

%\renewcommand{\theequation}{\arabic{chapter}.\arabic{section}.\arabic{equation}}
%\renewcommand{\thefigure}{\arabic{chapter}.\arabic{figure}}
%\renewcommand{\thetable}{\arabic{chapter}.\arabic{table}}

%\newcommand{\Sec}[2]{\section{#1}\markright{\arabic{ccounter}.\arabic{section}.#2}\setcounter{equation}{0}\setcounter{thm}{0}\setcounter{figure}{0}}

\newcommand{\Sec}[2]{\section{#1}}

\setcounter{secnumdepth}{2}
%\setcounter{secnumdepth}{1} 

%\newcounter{THM}
%\renewcommand{\theTHM}{\arabic{chapter}.\arabic{section}}

\newcommand{\trademark}{{R\!\!\!\!\!\bigcirc}}
%\newtheorem{exercise}{}

\newcommand{\dfield}{{\sf dfield9}}
\newcommand{\pplane}{{\sf pplane9}}

\newcommand{\EXER}{\section*{Exercises}}%\vspace*{0.2in}\hrule\small\setcounter{exercise}{0}}
\newcommand{\CEXER}{}%\vspace{0.08in}\begin{center}Computer Exercises\end{center}}
\newcommand{\TEXER}{} %\vspace{0.08in}\begin{center}Hand Exercises\end{center}}
\newcommand{\AEXER}{} %\vspace{0.08in}\begin{center}Hand Exercises\end{center}}

% BADBAD: \newcommand{\Bbb}{\bf}

\newcommand{\R}{\mbox{$\Bbb{R}$}}
\newcommand{\C}{\mbox{$\Bbb{C}$}}
\newcommand{\Z}{\mbox{$\Bbb{Z}$}}
\newcommand{\N}{\mbox{$\Bbb{N}$}}
\newcommand{\D}{\mbox{{\bf D}}}
\usepackage{amssymb}
%\newcommand{\qed}{\hfill\mbox{\raggedright$\square$} \vspace{1ex}}
%\newcommand{\proof}{\noindent {\bf Proof:} \hspace{0.1in}}

\newcommand{\setmin}{\;\mbox{--}\;}
\newcommand{\Matlab}{{M\small{AT\-LAB}} }
\newcommand{\Matlabp}{{M\small{AT\-LAB}}}
\newcommand{\computer}{\Matlab Instructions}
\newcommand{\half}{\mbox{$\frac{1}{2}$}}
\newcommand{\compose}{\raisebox{.15ex}{\mbox{{\scriptsize$\circ$}}}}
\newcommand{\AND}{\quad\mbox{and}\quad}
\newcommand{\vect}[2]{\left(\begin{array}{c} #1_1 \\ \vdots \\
 #1_{#2}\end{array}\right)}
\newcommand{\mattwo}[4]{\left(\begin{array}{rr} #1 & #2\\ #3
&#4\end{array}\right)}
\newcommand{\mattwoc}[4]{\left(\begin{array}{cc} #1 & #2\\ #3
&#4\end{array}\right)}
\newcommand{\vectwo}[2]{\left(\begin{array}{r} #1 \\ #2\end{array}\right)}
\newcommand{\vectwoc}[2]{\left(\begin{array}{c} #1 \\ #2\end{array}\right)}

\newcommand{\ignore}[1]{}


\newcommand{\inv}{^{-1}}
\newcommand{\CC}{{\cal C}}
\newcommand{\CCone}{\CC^1}
\newcommand{\Span}{{\rm span}}
\newcommand{\rank}{{\rm rank}}
\newcommand{\trace}{{\rm tr}}
\newcommand{\RE}{{\rm Re}}
\newcommand{\IM}{{\rm Im}}
\newcommand{\nulls}{{\rm null\;space}}

\newcommand{\dps}{\displaystyle}
\newcommand{\arraystart}{\renewcommand{\arraystretch}{1.8}}
\newcommand{\arrayfinish}{\renewcommand{\arraystretch}{1.2}}
\newcommand{\Start}[1]{\vspace{0.08in}\noindent {\bf Section~\ref{#1}}}
\newcommand{\exer}[1]{\noindent {\bf \ref{#1}}}
\newcommand{\ans}{}
\newcommand{\matthree}[9]{\left(\begin{array}{rrr} #1 & #2 & #3 \\ #4 & #5 & #6
\\ #7 & #8 & #9\end{array}\right)}
\newcommand{\cvectwo}[2]{\left(\begin{array}{c} #1 \\ #2\end{array}\right)}
\newcommand{\cmatthree}[9]{\left(\begin{array}{ccc} #1 & #2 & #3 \\ #4 & #5 &
#6 \\ #7 & #8 & #9\end{array}\right)}
\newcommand{\vecthree}[3]{\left(\begin{array}{r} #1 \\ #2 \\
#3\end{array}\right)}
\newcommand{\cvecthree}[3]{\left(\begin{array}{c} #1 \\ #2 \\
#3\end{array}\right)}
\newcommand{\cmattwo}[4]{\left(\begin{array}{cc} #1 & #2\\ #3
&#4\end{array}\right)}

\newcommand{\Matrix}[1]{\ensuremath{\left(\begin{array}{rrrrrrrrrrrrrrrrrr} #1 \end{array}\right)}}

\newcommand{\Matrixc}[1]{\ensuremath{\left(\begin{array}{cccccccccccc} #1 \end{array}\right)}}



\renewcommand{\labelenumi}{\theenumi)}
\newenvironment{enumeratea}%
{\begingroup
 \renewcommand{\theenumi}{\alph{enumi}}
 \renewcommand{\labelenumi}{(\theenumi)}
 \begin{enumerate}}
 {\end{enumerate}\endgroup}



\newcounter{help}
\renewcommand{\thehelp}{\thesection.\arabic{equation}}

%\newenvironment{equation*}%
%{\renewcommand\endequation{\eqno (\theequation)* $$}%
%   \begin{equation}}%
%   {\end{equation}\renewcommand\endequation{\eqno \@eqnnum
%$$\global\@ignoretrue}}

%\input{psfig.tex}

\author{Martin Golubitsky and Michael Dellnitz}

%\newenvironment{matlabEquation}%
%{\renewcommand\endequation{\eqno (\theequation*) $$}%
%   \begin{equation}}%
%   {\end{equation}\renewcommand\endequation{\eqno \@eqnnum
% $$\global\@ignoretrue}}

\newcommand{\soln}{\textbf{Solution:} }
\newcommand{\exercap}[1]{\centerline{Figure~\ref{#1}}}
\newcommand{\exercaptwo}[1]{\centerline{Figure~\ref{#1}a\hspace{2.1in}
Figure~\ref{#1}b}}
\newcommand{\exercapthree}[1]{\centerline{Figure~\ref{#1}a\hspace{1.2in}
Figure~\ref{#1}b\hspace{1.2in}Figure~\ref{#1}c}}
\newcommand{\para}{\hspace{0.4in}}

\renewenvironment{solution}{\suppress}{\endsuppress}

\ifxake
\newenvironment{matlabEquation}{\begin{equation}}{\end{equation}}
\else
\newenvironment{matlabEquation}%
{\let\oldtheequation\theequation\renewcommand{\theequation}{\oldtheequation*}\begin{equation}}%
  {\end{equation}\let\theequation\oldtheequation}
\fi

\makeatother


\title{Solutions Using Matrix Exponentials}

\begin{document}
\begin{abstract}
\end{abstract}
\maketitle


\label{S:Matrixexp} \index{matrix!exponential}

In Section~\ref{S:growthmodels} we showed that the solution of the single
ordinary differential equation $\dot x(t) = \lambda x(t)$ with initial
condition $x(0)=x_0$ is $x(t) = e^{t\lambda}x_0$ (see \eqref{lin1} in
Chapter~\ref{chap:SolveOdes}).  In this section we show that we
may write solutions of systems of equations in a similar form.
In particular, we show that the solution to the linear system of ODEs
\begin{equation}   \label{eq:x=Mx}
\frac{dX}{dt} = CX
\end{equation}
with initial condition
\[
X(0) = X_0,
\]
where $C$ is an $n\times n$ matrix and $X_0\in\R^n$, is
\begin{equation}  \label{matrixsoln}
X(t) = e^{tC}X_0.
\end{equation}

In order to make sense of the solution \eqref{matrixsoln} we need
to understand matrix exponentials. More precisely, since $tC$ is
an $n\times n$ matrix for each $t\in\R$, we need to make sense
of the expression $e^L$ where $L$ is an $n\times n$ matrix.  For
this we recall the form of the exponential function as a power
series:
\[
     e^t = 1 + t + \frac{1}{2!} t^2 + \frac{1}{3!} t^3
     + \frac{1}{4!} t^4 + \cdots .
\]
In more compact notation we have
\[
     e^t = \sum\limits_{k=0}^\infty \frac{1}{k!} t^k.
\]
By analogy, define the {\em matrix exponential\/}\index{matrix!exponential}
$e^L$ by
\begin{eqnarray}
e^{L} & = & I_n + L + \frac{1}{2!} L^2 + \frac{1}{3!} L^3 +\cdots
\label{e:expL}\\
      & = & \sum\limits_{k=0}^\infty\frac{1}{k!} L^k. \nonumber
\end{eqnarray}
In this formula $L^2 = LL$ is the matrix product of $L$ with itself, and the
power $L^k$ is defined inductively by $L^k = LL^{k-1}$ for $k>1$.  Hence
$e^L$ is an $n\times n$ matrix and is the infinite sum of $n\times n$
matrices.

\noindent {\bf Remark:}   The infinite series for matrix exponentials
\eqref{e:expL} converges for all $n\times n$ matrices $L$.  This fact
is proved in Exercises~\ref{c6.2.7} and \ref{c6.2.8}.

Using \eqref{e:expL}, we can write the matrix exponential of $tC$
for each real number $t$.  Since $(tC)^k = t^k C^k$ we obtain
\arraystart
\begin{equation}  \label{eq:MatrixExp}
\begin{array}{rcl}
\dps e^{tC} & = & \dps I_n + tC + \frac{1}{2!} (tC)^2 + \frac{1}{3!} (tC)^3
+\cdots\\
\dps & = & \dps I_n + tC + \frac{t^2}{2!} C^2 + \frac{t^3}{3!} C^3 +\cdots.
\end{array}
\end{equation}
\arrayfinish
Next we claim that
\begin{equation}  \label {e:diffmatexp}
  \frac{d}{dt} e^{tC} = Ce^{tC}.
\end{equation}
We verify the claim by supposing that we can differentiate
\eqref{eq:MatrixExp} term by term with respect to $t$. Then
\begin{align*}
  \dps\frac{d}{dt} e^{tC} & = \frac{d}{dt}(I_n) + \frac{d}{dt}(tC)
  + \frac{d}{dt}\left(\frac{t^2}{2!} C^2\right) + 
                            \frac{d}{dt}\left(\frac{t^3}{3!} C^3\right) + \\
                          &\quad
  \frac{d}{dt}\left(\frac{t^4}{4!}C^4\right) + \cdots\\
     & = 0 + C + t C^2 + \frac{t^2}{2!} C^3 +
\frac{t^3}{3!} C^4 + \cdots\\
     & = C\left(I_n + tC + \frac{t^2}{2!} C^2 + \frac{t^3}{3!} C^3
+\cdots\right)\\
     & = Ce^{tC}.
\end{align*}
It follows that the function $X(t) = e^{tC}X_0$ is a solution of
\eqref{eq:x=Mx} for each $X_0\in\R^n$; that is,
\[
     \frac{d}{dt} X(t) =  \frac{d}{dt}  e^{tC}X_0
     = C e^{tC}X_0 = C X(t).
\]
Since \eqref{e:expL} implies that $e^{0C} = e^0 = I_n$, it follows
that $X(t) = e^{tC}X_0$ is a solution of \eqref{eq:x=Mx} with
initial condition $X(0)=X_0$.  This discussion shows that solving
\eqref{eq:x=Mx} in closed form is equivalent to finding a closed
form expression for the matrix exponential $e^{tC}$.

\begin{theorem}  \label{T:linODEsoln}
The unique solution\index{uniqueness of solutions} to the
initial value problem\index{initial value problem}
\arraystart
\[
\begin{array}{rcl}
\dps\frac{dX}{dt} & = & CX \\
X(0) & = & X_0
\end{array}
\]
\arrayfinish
is
\[
X(t)=e^{tC}X_0.
\]
\end{theorem}

\begin{proof}  Existence follows from the previous discussion.
For uniqueness, suppose that $Y(t)$ is a solution to $\dot{Y}=CY$
with $Y(0)=X_0$.  We claim that $Y(t)=X(t)$.  Let $Z(t) = e^{-tC}Y(t)$ 
and use the product rule to compute 
\[
\dps\frac{dZ}{dt} = -C e^{-tC}Y(t) + e^{-tC}\frac{dY}{dt}(t) = e^{-tC}(-CY(t) + C Y(t)) = 0
\]  
It follows that $Z$ is constant in $t$ and $Z(t) = Z(0) = Y(0) = X_0$ or 
$Y(t) = e^{tC}X_0 = X(t)$, as claimed.
\end{proof}

\subsubsection*{Similarity and Matrix Exponentials}

We introduce similarity at this juncture for the following reason:
if $C$ is a matrix that is similar to $B$, then $e^C$ can be computed
from $e^B$.  More precisely:

\begin{lemma} \label{L:similarexp}
Let $C$ and $B$ be $n\times n$ similar matrices, and let $P$ be
an invertible $n\times n$ matrix such that
\[
C=P\inv BP.
\]
Then
\begin{equation}  \label{e:similarexp}
e^C = P\inv e^BP.
\end{equation}
\end{lemma} \index{similar} \index{invertible}

\begin{proof} Note that for all powers of $k$ we have
\[
(P\inv BP)^k = P\inv B^kP.
\]
Next verify \eqref{e:similarexp} by computing
\begin{align*}
e^C &=\sum^{\infty}_{k=0} \frac{1}{k!}C^k
 =  \sum^{\infty}_{k=0} \frac{1}{k!}(P\inv BP)^k \\
&=  \sum^{\infty}_{k=0} \frac{1}{k!}P\inv B^kP
= P\inv\left(\sum^{\infty}_{k=0} \frac{1}{k!}B^k\right)P
= P\inv e^B P.
\end{align*}
\end{proof}



\subsection*{Explicit Computation of Matrix Exponentials}
\index{matrix!exponential!computation}

We begin with the simplest computation of a matrix exponential.

\noindent (a) \quad Let $L$ be a multiple of the identity; that
is, let $L = \alpha I_n$ where $\alpha$ is a real number.  Then
\begin{equation} \label{ex:expm}
e^{\alpha I_n} = e^{\alpha} I_n.
\end{equation}
That is, $e^{\alpha I_n}$ is a scalar multiple of the
identity.  To verify \eqref{ex:expm}, compute
\begin{align*}
e^{\alpha I_n} &= I_n + \alpha I_n + \frac{\alpha^2}{2!} I_n^2 +
                 \frac{\alpha^3}{3!} I_n^3 +\cdots \\
  &= (1+\alpha+\frac{\alpha^2}{2!}
+\frac{\alpha^3}{3!}+\cdots)I_n = e^{\alpha} I_n.
\end{align*}

\noindent (b) \quad Let $C$ be a $2\times 2$ diagonal matrix,
     \[
          C = \mattwo{\lambda_1}{0}{0}{\lambda_2},
     \]
where $\lambda_1$ and $\lambda_2$ are real constants.  Then
\begin{equation}  \label{e:expdiag}
e^{tC} = \mattwo{e^{\lambda_1 t}}{0}{0}{e^{\lambda_2 t}}.
\end{equation}
To verify \eqref{e:expdiag} compute
\begin{eqnarray*}
   e^{tC} & = & I_2 + tC + \frac{t^2}{2!} C^2 +  \frac{t^3}{3!} C^3 +\cdots\\
        & = & \mattwo{1}{0}{0}{1} + \mattwo{\lambda_1 t}{0}{0}{\lambda_2 t} +
\mattwo{\frac{t^2}{2!}\lambda_1^2}{0}{0}{\frac{t^2}{2!}\lambda_2^2} +\cdots\\
        & = & \mattwo{e^{\lambda_1 t}}{0}{0}{e^{\lambda_2 t}}.
\end{eqnarray*}

\noindent (c) \quad Suppose that
     \[
                C = \mattwo{0}{-1}{1}{0}.
      \]
Then
\begin{equation} \label{e:exprotate}
e^{tC} = \mattwo{\cos t}{-\sin t}{\sin t}{\cos t}.
\end{equation}
We begin this computation by observing that
\[
C^2 = -I_2, \quad C^3 = -C, \AND C^4 = I_n.
\]
Therefore, by collecting terms of odd and even power in the series
expansion for the matrix exponential we obtain
\begin{align*}
e^{tC} & =  I_2 + tC + \frac{t^2}{2!} C^2 +  \frac{t^3}{3!}C^3 +\cdots\\
     & =  I_2 + tC - \frac{t^2}{2!}I_2 - \frac{t^3}{3!}C +\cdots\\
     & =  \left(1 - \frac{t^2}{2!} + \frac{t^4}{4!} - \frac{t^6}{6!} +
		\cdots \right)I_2 + \\
	 &\quad \left(t - \frac{t^3}{3!} + \frac{t^5}{5!} - \frac{t^7}{7!} +
	\cdots \right)C \\
     & =  (\cos t)I_2 + (\sin t)C \\
     & =  \mattwo{\cos t}{-\sin t}{\sin t}{\cos t}.
     \end{align*}
In this computation we have used the fact that the trigonometric
functions $\cos t$ and $\sin t$ have the power series expansions:
\begin{eqnarray*}
\cos t & = & 1-\frac{1}{2!}t^2+\frac{1}{4!} t^4 + \cdots =
\sum\limits_{k=0}^\infty\frac{(-1)^k}{(2k)!} t^{2k},\\
\sin t & = & t-\frac{1}{3!} t^3 + \frac{1}{5!} t^5 + \cdots
   = \sum\limits_{k=0}^\infty \frac{(-1)^k}{(2k+1)!} t^{2k+1}.
\end{eqnarray*}
See Exercise~\ref{c6.2.5C} for an alternative proof of \eqref{e:exprotate}.

To compute the matrix exponential
\Matlab\index{matrix!exponential!in \protect\Matlab} provides the command
{\tt expm}\index{\computer!expm}.  We use this command to compute
the matrix exponential $e^{tC}$ for
\[
C=\mattwo{0}{-1}{1}{0} \AND t=\frac{\pi}{4}.
\]
Type
\begin{verbatim}
C = [0, -1; 1, 0];
t = pi/4;
expm(t*C)
\end{verbatim}
that gives the answer
\begin{verbatim}
ans =
    0.7071   -0.7071
    0.7071    0.7071
\end{verbatim}
Indeed, this is precisely what we expect by \eqref{e:exprotate},
since
\[
\cos\left(\frac{\pi}{4}\right)=\sin\left(\frac{\pi}{4}\right)=
\frac{1}{\sqrt{2}}\approx 0.70710678.
\]

\noindent (d) \quad Let
\[
C = \mattwo{0}{1}{0}{0}.
\]
Then
\begin{equation}  \label{e:nilpotent}
e^{tC} = I_2 + tC = \mattwo{1}{t}{0}{1},
\end{equation}
since $C^2=0$.

\EXER

\CEXER

\begin{exercise} \label{c6.2.1}
Let $L$ be the $3\times 3$ matrix
\[
     L = \left(\begin{array}{rrr}
    2 & 0 & -1\\
    0 & -1 & 3\\
    1 & 0 & 1
               \end{array}\right).
\]
Find the smallest integer $m$ such that
\[
  I_3+L+\frac{1}{2!} L^2 + \frac{1}{3!} L^3 + \cdots
  + \frac{1}{m!} L^m
\]
is equal to $e^L$ up to a precision of two decimal places.  More
exactly, use the \Matlab command {\tt expm} to compute $e^L$ and
use \Matlab commands to compute the series expansion to order $m$.  Note
that the command for computing $n!$ in \Matlab is
{\tt prod(1:n)}\index{\computer!prod}.

\begin{solution}

\ans When $m = 7$, the power series is accurate up to two decimal places.

\soln Enter {\tt L} into \Matlab, then use the command {\tt expm(L)} to
find the value of $e^L$.
\begin{verbatim}
ans = 
    4.8746         0   -3.9421
    3.1370    0.3679    2.4154
    3.9421         0    0.9324
\end{verbatim}
Then, add elements of the power series to $I_3$ until this answer is
is equal to the \Matlab generated answer up to a precision of two
digits.

\end{solution}
\end{exercise}

\begin{exercise} \label{c6.2.2}
Use \Matlab to compute the matrix exponential $e^{tC}$ for
\[
     C =\mattwo{1}{1}{2}{-1}
\]
by choosing for $t$ the values $1.0,1.5$ and $2.5$.  Does $e^{C}
e^{1.5C}=e^{2.5C}$?

\begin{solution}

Enter {\tt C} into \Matlab, then use {\tt expm} to obtain
\begin{verbatim}
expm(C) =                 expm(1.5*C) =              expm(2.5*C) =
    4.4952    1.5806          10.6138    3.8577          59.9058   21.9222
    3.1612    1.3340           7.7154    2.8984          43.8444   16.0613
\end{verbatim}
Typing {\tt expm(C)*expm(1.5*C)} confirms that $e^Ce^{1.5C} =
e^{2.5C}$ is indeed valid.

\end{solution}
\end{exercise}

\begin{exercise} \label{c6.2.3}
For the scalar exponential function $e^{t}$ it is well known
that for any pair of real numbers $t_1,t_2$ the following
equality holds:
\[
     e^{t_1+t_2} = e^{t_1}e^{t_2}.
\]
Use \Matlab to find two $2\times 2$ matrices $C_1$ and $C_2$ such that
\[
     e^{C_1+C_2} \not= e^{C_1}e^{C_2}.
\]

\begin{solution}

One example of matrices for which $e^{C_1 + C_2} \neq e^{C_1}e^{C_2}$ is
\[ C_1 = \mattwo{1}{-2}{3}{1} \AND C_2 = \mattwo{-2}{3}{-1}{-2}. \]
For this case, using \Matlabp,
\begin{verbatim}
expm(C1+C2) =                        expm(C1)*expm(C2) =
    0.8013    0.5034                     0.1547   -0.4534
    1.0067    0.8013                     0.1152    0.5370
\end{verbatim}

\end{solution}
\end{exercise}

\TEXER

\noindent In Exercises~\ref{c6.2.4a} -- \ref{c6.2.4c} compute the matrix
exponential $e^{tC}$ for the matrix.
\begin{exercise} \label{c6.2.4a}
                $\mattwo{0}{1}{0}{0}$.

\begin{solution}
\ans $e^{tC} = \mattwo{1}{t}{0}{1}$.

\soln In general,
\[
e^{tC} = I + tC + \frac{t^2}{2!}C^2 + \frac{t^3}{3!}C^3 + \cdots
\]
Note that $C^2 = 0$, so $C^k = 0$, for $k \geq 2$, so we need only
calculate the first two terms:
\[
e^{tC} = I_2 + tC = \mattwo{1}{0}{0}{1} + t\mattwo{0}{1}{0}{0} =
\mattwo{1}{t}{0}{1}.
\]

\end{solution}
\end{exercise}
\begin{exercise} \label{c6.2.4b}
                $\left(\begin{array}{ccc}
                0 & 1 & 0\\
                0 & 0 & 1\\
                0 & 0 & 0 \end{array}\right)$.

\begin{solution}
\ans $e^{tC} =
\cmatthree{1}{t}{\frac{t^2}{2}}{0}{1}{t}{0}{0}{1}$.

\soln Since $D^3 = 0$,
\[
e^{tD} = I_3 + tD + \frac{t^2}{2}D^2 =
\matthree{1}{0}{0}{0}{1}{0}{0}{0}{1} +
\matthree{0}{t}{0}{0}{0}{t}{0}{0}{0} +
\cmatthree{0}{0}{\frac{t^2}{2}}{0}{0}{0}{0}{0}{0} =
\cmatthree{1}{t}{\frac{t^2}{2}}{0}{1}{t}{0}{0}{1}.
\]

\end{solution}
\end{exercise}
\begin{exercise} \label{c6.2.4c}
                $\mattwo{0}{-2}{2}{0}$.

\begin{solution}
\ans $e^{tC} =
\mattwo{\cos 2t}{-\sin 2t}{\sin 2t}{\cos 2t}$.

\soln 
First, write $C$ as
\[
\mattwo{0}{-2}{2}{0} = 2E, \hbox{ where } E = \mattwo{0}{-1}{1}{0}.
\]
Then find $e^{2tE}$ by equation~\eqref{e:exprotate}.

\end{solution}
\end{exercise}

\begin{exercise} \label{c6.2.5}
Let $\alpha,\beta$ be real numbers and let $\alpha I$ and $\beta
I$ be corresponding $n\times n$ diagonal matrices.  Use
properties of the scalar exponential function to show that
\[
     e^{(\alpha + \beta)I} = e^{\alpha I}e^{\beta I}.
\]

\begin{solution}

By equation~\eqref{ex:expm}, $e^{\alpha I} = e^\alpha I$.  Therefore,
\[ e^{(\alpha + \beta)I} = e^{\alpha + \beta}I = e^\alpha e^\beta I
= e^\alpha I e^\beta I = e^{\alpha I}e^{\beta I}. \]

\end{solution}
\end{exercise}

\noindent In Exercises~\ref{c6.2.5A} -- \ref{c6.2.5C} we use
Theorem~\ref{T:linODEsoln}, the uniqueness of solutions to initial value
problems, in perhaps a surprising way.
\begin{exercise}  \label{c6.2.5A}
Prove that
\[
e^{t+s} = e^te^s
\]
for all real numbers $s$ and $t$.  {\bf Hint:}
\begin{itemize}
\item[(a)]  Fix $s$ and verify that $y(t) = e^{t+s}$ is a solution to the
initial value problem
\begin{equation}  \label{E:init1}
\begin{array}{rcl}
\frac{dx}{dt} & = & x \\
x(0) & = & e^s
\end{array}
\end{equation}
\item[(b)] Fix $s$ and verify that $z(t) = e^te^s$ is also a solution to
\eqref{E:init1}.
\item[(c)]  Use Theorem~\ref{T:linODEsoln} to conclude that $y(t)=z(t)$ for
every $s$.
\end{itemize}

\begin{solution}

(a) To verify that $y(t) = e^{t + s}$ is a solution to the initial value
problem, first substitute $y(t)$ into the left hand side of the equation.
Using the chain rule, obtain
\[
\frac{dy}{dt}(t) = \frac{d}{dt}(e^{t + s}) = \frac{d}{dt}(t + s)e^{t + s}
= e^{t + s}.
\]
Then substitute $y(t)$ into the right hand side of the equation, obtaining
\[
y(t) = e^{t + s}.
\]
Thus, the left hand and right hand sides are equal, so $y(t)$ is a
solution to the differential equation.  Finally, check to see that
$y(t)$ satisfies the initial value:
\[
y(0) = e^{0 + s} = e^s
\]
as desired.

(b) Similarly, verify that $z(t) = e^te^s$ is a solution to the initial
value problem by substituting $z(t)$ into each side of the differential
equation:
\[
\frac{dz}{dt}(t) = \frac{d}{dt}(e^te^s) = e^te^s \AND
z(t) = e^te^s.
\]
Note that the results are equal, so that $z(t)$ is a solution to the
differential equation.  Since
\[
z(0) = e^0e^s = e^s,
\]
it follows that $z(t)$ is also a solution to the initial value problem.

(c) By Theorem~\ref{exist&unique}, if
$f(x)$ is differentiable near $x_0$, and if $\frac{df}{dx}$ is continuous,
then there is a unique solution to the differential equation $\dot{x} = f(x)$
with initial condition $x(0) = x_0$.  Let $f(x) = x$ and let $x(0) = e^s$.
Then, as shown in (a) and (b) of this problem, $e^{t + s}$ and $e^te^s$ are
both solutions.  However, by Theorem~\ref{exist&unique}, there is only one
solution, so $e^{t + s} = e^te^s$.

\end{solution}
\end{exercise}
\begin{exercise}  \label{c6.2.5B}
Let $A$ be an $n\times n$ matrix.  Prove that
\[
e^{(t+s)A} = e^{tA}e^{sA}
\]
for all real numbers $s$ and $t$.  {\bf Hint:}
\begin{itemize}
\item[(a)]  Fix $s\in\R$ and $X_0\in\R^n$ and verify that
$Y(t) = e^{(t+s)A}X_0$ is a solution to the initial value problem
\begin{equation}  \label{E:init2}
\begin{array}{rcl}
\frac{dX}{dt} & = & AX \\
X(0) & = & e^{sA}X_0
\end{array}
\end{equation}
\item[(b)] Fix $s$ and verify that $Z(t) = e^{tA}\left(e^{sA}X_0\right)$ is
also a solution to \eqref{E:init2}.
\item[(c)]  Use the $n$ dimensional version of Theorem~\ref{T:linODEsoln} to
conclude that $Y(t)=Z(t)$ for every $s$ and every $X_0$.
\end{itemize}
{\bf Remark:}  Compare the result in this exercise with the calculation in
Exercise~\ref{c6.2.5}.

\begin{solution}

(a) To verify that $Y(t)$ is a solution to the initial value problem
\eqref{E:init2}, first substitute $Y(t)$ into the left hand side of the
equation.  Using the chain rule, obtain
\[
\frac{dY}{dt}(t) = \frac{d}{dt}(e^{(t + s)A}X_0) =
\frac{d}{dt}((t + s)A)e^{(t + s)A}X_0 = Ae^{(t + s)A}X_0.
\]
Then substitute $Y(t)$ into the right hand side of the equation, obtaining
\[
AY(t) = Ae^{(t + s)A}X_0.
\]
Thus, the left hand and right hand sides of the equation are equal, so
$Y(t)$ is a solution to the differential equation.  Finally, check to see
that $Y(t)$ satisfies the initial value:
\[
Y(0) = e^{(0 + s)A}X_0 = e^{sA}X_0,
\]
as desired.

(b) Similarly, verify that $Z(t) = e^{tA}(e^{sA}X_0)$ is a solution to the
initial value problem by substituting $Z(t)$ into each side of the
differential equation:
\[
\frac{dZ}{dt}(t) = \frac{d}{dt}(e^{tA}(e^{sA}X_0))
= Ae^{tA}(e^{sA}X_0) \AND
AZ(t) = Ae^{tA}(e^{sA}X_0).
\]
Since the results are equal, $Z(t)$ is a solution to the differential
equation.  Evaluating at $t = 0$, we find
\[
Z(0) = e^{0}(e^{sA}X_0) = e^{sA}X_0,
\]
from which it follows that $Z(t)$ is also a solution to the initial
value problem.

(c) Since $Y(t)$ and $Z(t)$ are both solutions to the initial value problem
\[
\begin{array}{rcl}
\frac{dX}{dt} & = & AX \\
X(0) & = & e^{sA}X_0,
\end{array}
\]
it follows from the uniqueness part of Theorem~\ref{exist&unique} that
$Y(t) = Z(t)$.  Thus, $e^{(t + s)A} = e^{tA}e^{sA}$, as desired.


\end{solution}
\end{exercise}
\begin{exercise}  \label{c6.2.5C}
Prove that
\begin{equation}  \label{E:0-110E}
\exp\left(t\mattwo{0}{-1}{1}{0}\right) =
\mattwo{\cos t}{-\sin t}{\sin t}{\cos t}.
\end{equation}
{\bf Hint:}
\begin{itemize}
\item[(a)] Verify that $X_1(t) = \vectwo{\cos t}{\sin t}$ and
$X_2(t) = \vectwo{-\sin t}{\cos t}$ are solutions to the initial value problems
\begin{equation}  \label{E:init3}
\begin{array}{rcl}
\dps\frac{dX}{dt} & = & \mattwo{0}{-1}{1}{0}X \\
X(0) & = & e_j
\end{array}
\end{equation}
for $j=1,2$.
\item[(b)] Since $X_j(0)=e_j$, use Theorem~\ref{T:linODEsoln} to verify that
\begin{equation}   \label{E:0-110}
X_j(t) = \exp\left(t\mattwo{0}{-1}{1}{0}\right)e_j.
\end{equation}
\item[(c)]  Show that \eqref{E:0-110} proves \eqref{E:0-110E}
\end{itemize}

\begin{solution}

(a) To verify that $X_1(t) = (\cos t,\sin t)^t$ is a solution to the
initial value problem \eqref{E:init3}, substitute $X_1(t)$ into the left
hand side of the differential equation, obtaining
\[
\frac{dX_1}{dt} = \frac{d}{dt}\vectwo{\cos t}{\sin t} =
\vectwo{-\sin t}{\cos t}.
\]
Then substitute $X_1(t)$ into the right hand side of the differential
equation, obtaining
\[
\mattwo{0}{-1}{1}{0}X_1 = \mattwo{0}{-1}{1}{0}\vectwo{\cos t}{\sin t}
= \vectwo{-\sin t}{\cos t}.
\]
Since the two sides are equal, $X_1(t)$ is a solution to
\eqref{E:init3}.  Further, since
\[
X_1(0) = \vectwo{\cos 0}{\sin 0} = \vectwo{1}{0} = e_1,
\]
$X_1(t)$ is a solution to the given initial value problem.

\para Similarly, to verify that $X_2(t) = (-\sin t,\cos t)^t$ is a
solution to the initial value problem, substitute $X_2(t)$ into the
left hand side of the differential equation, obtaining
\[
\frac{dX_2}{dt} = \frac{d}{dt}\vectwo{-\sin t}{\cos t} =
\vectwo{-\cos t}{-\sin t}.
\]
Then substitute $X_2(t)$ into the right hand side of the differential
equation, obtaining
\[
\mattwo{0}{-1}{1}{0}X_2 = \mattwo{0}{-1}{1}{0}\vectwo{-\sin t}{\cos t}
= \vectwo{-\cos t}{-\sin t}.
\]
Since the two sides of the differential equation are equal, and since
\[
X_2(0) = \vectwo{-\sin 0}{\cos 0} = \vectwo{0}{1} = e_2,
\]
$X_2(t)$ is a solution to the initial value problem \eqref{E:init3}.

(b) By Theorem~\ref{T:linODEsoln}, the unique
solution to \eqref{E:init3} with initial condition $X(0) = e_j$ is
\[
Y_j(t) \equiv \exp\left(t\mattwo{0}{-1}{1}{0}\right)e_j.
\]
We showed in part (a) that $X_j(t)$ satisfies this initial value
problem.  Thus, by the uniqueness part of Theorem~\ref{exist&unique},
$X_j(t) = Y_j(t)$, as desired.

(c) By part (b), $X_j(t)$ is equal to $Y_j(t)$, which is defined as
the $j^{th}$ column of the matrix
\[
\exp\left(t\mattwo{0}{-1}{1}{0}\right).
\]
By definition, $X_j(t)$ is equal to the $j^{th}$ column of the matrix
\[
\mattwo{\cos t}{-\sin t}{\sin t}{\cos t}.
\]
Thus,
\[
\exp\left(t\mattwo{0}{-1}{1}{0}\right) =
\mattwo{\cos t}{-\sin t}{\sin t}{\cos t}.
\]

\end{solution}
\end{exercise}

\begin{exercise}  \label{c6.2.6A}
Let $C$ be an $n\times n$ matrix.  Use Theorem~\ref{T:linODEsoln} to show
that the $n$ columns of the $n\times n$ matrix $e^{tC}$ give a basis of
solutions for the system of differential equations $\dot{X}=CX$.

\begin{solution}
By Theorem~\ref{T:linODEsoln},
the unique solution to the differential equation $\dot{X} = CX$ with
initial condition $X(0) = X_0$ is $X(t) = e^{tC}X_0$.  Let $X(0) =
e_j$.  Then the vector $X_j(t) = e^{tC}e_j$ is a solution to the
differential equation, and is the $j^{th}$ column of the matrix
$e^{tC}$.  Thus, each column of $e^{tC}$ is a solution to the
differential equation, and the set of columns forms a basis of
solutions.


\end{solution}
\end{exercise}

\noindent {\bf Remark:}  The completion of Exercises~\ref{c6.2.7} and
\ref{c6.2.8} constitutes a proof that the infinite series definition of
the matrix exponential is a convergent series for all $n\times n$ matrices.

\begin{exercise}  \label{c6.2.7}
Let $A=(a_{ij})$ be an $n\times n$ matrix.  Define
\[
||A||_m = \max_{1\leq i\leq n} (|a_{i1}|+\cdots+|a_{in}|)
= \max_{1\leq i\leq n} \left(\sum_{j=1}^n|a_{ij}|\right).
\]
That is, to compute $||A||_m$, first sum the absolute values of the entries
in each row of $A$, and then take the maximum of these sums.  Prove that:
\[
||AB||_m \leq ||A||_m ||B||_m.
\]
{\bf Hint:} Begin by noting that
\begin{align*}
||AB||_m &=
\max_{1\leq i\leq n}\left(\sum_{j=1}^n\left|\sum_{k=1}^na_{ik}b_{kj}\right|
\right)\leq \max_{1\leq i\leq n}\left(\sum_{j=1}^n\sum_{k=1}^n\left|a_{ik}b_{kj}
           \right|\right) \\
  &= \max_{1\leq i\leq n}\left(\sum_{k=1}^n\sum_{j=1}^n
\left|a_{ik}b_{kj}\right|\right).
\end{align*}

\begin{solution}
Using the hint from the text:
\[
||AB||_m \leq \max_{1\leq i\leq n}
\left(\sum_k\sum_j\left|a_{ik}b_{kj}\right|\right) \leq
\max_{1\leq i\leq n} \left(\sum_k|a_{ik}|\sum_j|b_{kj}|\right).
\]
Then, because $\sum_j|b_{kj}|$ is the sum of the entries in the $k^{th}$
row of $B$:
\[
\max_{1\leq i\leq n}
\left(\sum_k|a_{ik}|\sum_j|b_{kj}|\right) \leq
\max_{1 \leq i \leq n}
\left(||B||_m \sum_k|a_{ik}|\right) =
||A||_m ||B||_m.
\]
Thus $||AB||_m \leq ||A||_m ||B||_m$.

\end{solution}
\end{exercise}

\begin{exercise} \label{c6.2.8}
Recall that an infinite series of real numbers
\[
c_1+c_2 +\cdots+c_N + \cdots
\]
converges absolutely if there is a constant $K$ such that for every $N$
the partial sum satisfies:
\[
|c_1| + |c_2| + \cdots + |c_N| \leq K.
\]

Let $A$ be an $n\times n$ matrix.  To prove that the matrix exponential $e^A$
is an absolutely convergent infinite series use Exercise~\ref{c6.2.7} and the
following steps.  Let $a_N$ be the $(i,j)^{th}$ entry in the matrix $A^N$
where $A^0=I_n$.
\begin{itemize}
\item[(a)]  $|a_N| \leq ||A^N||_m$.
\item[(b)]  $||A^N||_m \leq ||A||_m^N$.
\item[(c)]  $|a_0| + |a_1| + \cdots + \frac{1}{N!}|a_N| \leq e^{||A||_m}$.
\end{itemize}

\begin{solution}

(a) Let $a_{ij}$ be the $(i,j)^{th}$ entry in $n \times n$ matrix $A$. 
Then, since $|a_{ij}| \geq 0$ for all $(i,j)$,
\[ |a_{ij}| \leq |a_{i1}| + \cdots + |a_{in}| \leq
\max_i(|a_{i1}| + \cdots + |a_{in}|) = ||A||_m. \]
This statement is valid for the matrix $A^N$, so $|a_N| \leq ||A^N||_m$.

(b) To show this, we use the fact that $||A^N||_m = ||A^{N - 1}A||_m$.
According to Exercise~\ref{c6.2.7},
\[
||A^{N - 1}A||_m \leq ||A^{N - 1}||_m ||A||_m.
\]
Expanding $||A^N||_m$ in this way, we find that
\[
||A^N||_m \leq ||A||_m \cdots ||A||_m = ||A||^N_m.
\]

(c) According to our result in (a)
\[ |a_0| + |a_1| + \cdots + \frac{1}{N!}|a_N| = \sum_N
\frac{1}{N!}|a_N| \leq \sum_N \frac{1}{N!}||A^N||_m. \]
According to the result in (b)
\[ \sum_N \frac{1}{N!}||A^N||_m \leq \sum_N \frac{1}{N!}||A||^N_m. \]
We know that
\[ \sum_{N = 0}^{\infty} \frac{1}{N!}||A||^N_m = e^{||A||_m}.  \]
Therefore,
\[ \sum_N \frac{1}{N!}|a_N| \leq e^{||A||_m}, \]
so $e^A$ is absolutely convergent.





\end{solution}
\end{exercise}

\AEXER
\begin{exercise} \label{c6.3.14}
When the eigenvalues $\lambda_1$ and $\lambda_2$ of the $2\times 2$ 
matrix $C$ are real and distinct, $e^{tC}$ can be computed without determining 
the associated eigenvectors.  To see this, prove that
\begin{equation}  \label{E:exdist}
e^{tC} = \frac{1}{\lambda_2-\lambda_1}\left(e^{\lambda_1 t}(C-\lambda_2I_2) -
e^{\lambda_2 t}(C-\lambda_1I_2)\right).
\end{equation}
Hint:  The left and right hand sides of \eqref{E:exdist} are linear maps.  Two 
linear maps are identical if they have the same values on a basis of vectors 
$v_1$ and $v_2$.  Verify that the maps in \eqref{E:exdist} are equal when 
applied to the linearly independent eigenvectors of $C$.
\end{exercise}


\end{document}
