\documentclass{ximera}
 

\usepackage{epsfig}

\graphicspath{
  {./}
  {figures/}
}

\usepackage{morewrites}
\makeatletter
\newcommand\subfile[1]{%
\renewcommand{\input}[1]{}%
\begingroup\skip@preamble\otherinput{#1}\endgroup\par\vspace{\topsep}
\let\input\otherinput}
\makeatother

\newcommand{\includeexercises}{\directlua{dofile("/home/jim/linearAlgebra/laode/exercises.lua")}}

%\newcounter{ccounter}
%\setcounter{ccounter}{1}
%\newcommand{\Chapter}[1]{\setcounter{chapter}{\arabic{ccounter}}\chapter{#1}\addtocounter{ccounter}{1}}

%\newcommand{\section}[1]{\section{#1}\setcounter{thm}{0}\setcounter{equation}{0}}

%\renewcommand{\theequation}{\arabic{chapter}.\arabic{section}.\arabic{equation}}
%\renewcommand{\thefigure}{\arabic{chapter}.\arabic{figure}}
%\renewcommand{\thetable}{\arabic{chapter}.\arabic{table}}

%\newcommand{\Sec}[2]{\section{#1}\markright{\arabic{ccounter}.\arabic{section}.#2}\setcounter{equation}{0}\setcounter{thm}{0}\setcounter{figure}{0}}

\newcommand{\Sec}[2]{\section{#1}}

\setcounter{secnumdepth}{2}
%\setcounter{secnumdepth}{1} 

%\newcounter{THM}
%\renewcommand{\theTHM}{\arabic{chapter}.\arabic{section}}

\newcommand{\trademark}{{R\!\!\!\!\!\bigcirc}}
%\newtheorem{exercise}{}

\newcommand{\dfield}{{\sf dfield9}}
\newcommand{\pplane}{{\sf pplane9}}

\newcommand{\EXER}{\section*{Exercises}}%\vspace*{0.2in}\hrule\small\setcounter{exercise}{0}}
\newcommand{\CEXER}{}%\vspace{0.08in}\begin{center}Computer Exercises\end{center}}
\newcommand{\TEXER}{} %\vspace{0.08in}\begin{center}Hand Exercises\end{center}}
\newcommand{\AEXER}{} %\vspace{0.08in}\begin{center}Hand Exercises\end{center}}

% BADBAD: \newcommand{\Bbb}{\bf}

\newcommand{\R}{\mbox{$\Bbb{R}$}}
\newcommand{\C}{\mbox{$\Bbb{C}$}}
\newcommand{\Z}{\mbox{$\Bbb{Z}$}}
\newcommand{\N}{\mbox{$\Bbb{N}$}}
\newcommand{\D}{\mbox{{\bf D}}}
\usepackage{amssymb}
%\newcommand{\qed}{\hfill\mbox{\raggedright$\square$} \vspace{1ex}}
%\newcommand{\proof}{\noindent {\bf Proof:} \hspace{0.1in}}

\newcommand{\setmin}{\;\mbox{--}\;}
\newcommand{\Matlab}{{M\small{AT\-LAB}} }
\newcommand{\Matlabp}{{M\small{AT\-LAB}}}
\newcommand{\computer}{\Matlab Instructions}
\newcommand{\half}{\mbox{$\frac{1}{2}$}}
\newcommand{\compose}{\raisebox{.15ex}{\mbox{{\scriptsize$\circ$}}}}
\newcommand{\AND}{\quad\mbox{and}\quad}
\newcommand{\vect}[2]{\left(\begin{array}{c} #1_1 \\ \vdots \\
 #1_{#2}\end{array}\right)}
\newcommand{\mattwo}[4]{\left(\begin{array}{rr} #1 & #2\\ #3
&#4\end{array}\right)}
\newcommand{\mattwoc}[4]{\left(\begin{array}{cc} #1 & #2\\ #3
&#4\end{array}\right)}
\newcommand{\vectwo}[2]{\left(\begin{array}{r} #1 \\ #2\end{array}\right)}
\newcommand{\vectwoc}[2]{\left(\begin{array}{c} #1 \\ #2\end{array}\right)}

\newcommand{\ignore}[1]{}


\newcommand{\inv}{^{-1}}
\newcommand{\CC}{{\cal C}}
\newcommand{\CCone}{\CC^1}
\newcommand{\Span}{{\rm span}}
\newcommand{\rank}{{\rm rank}}
\newcommand{\trace}{{\rm tr}}
\newcommand{\RE}{{\rm Re}}
\newcommand{\IM}{{\rm Im}}
\newcommand{\nulls}{{\rm null\;space}}

\newcommand{\dps}{\displaystyle}
\newcommand{\arraystart}{\renewcommand{\arraystretch}{1.8}}
\newcommand{\arrayfinish}{\renewcommand{\arraystretch}{1.2}}
\newcommand{\Start}[1]{\vspace{0.08in}\noindent {\bf Section~\ref{#1}}}
\newcommand{\exer}[1]{\noindent {\bf \ref{#1}}}
\newcommand{\ans}{}
\newcommand{\matthree}[9]{\left(\begin{array}{rrr} #1 & #2 & #3 \\ #4 & #5 & #6
\\ #7 & #8 & #9\end{array}\right)}
\newcommand{\cvectwo}[2]{\left(\begin{array}{c} #1 \\ #2\end{array}\right)}
\newcommand{\cmatthree}[9]{\left(\begin{array}{ccc} #1 & #2 & #3 \\ #4 & #5 &
#6 \\ #7 & #8 & #9\end{array}\right)}
\newcommand{\vecthree}[3]{\left(\begin{array}{r} #1 \\ #2 \\
#3\end{array}\right)}
\newcommand{\cvecthree}[3]{\left(\begin{array}{c} #1 \\ #2 \\
#3\end{array}\right)}
\newcommand{\cmattwo}[4]{\left(\begin{array}{cc} #1 & #2\\ #3
&#4\end{array}\right)}

\newcommand{\Matrix}[1]{\ensuremath{\left(\begin{array}{rrrrrrrrrrrrrrrrrr} #1 \end{array}\right)}}

\newcommand{\Matrixc}[1]{\ensuremath{\left(\begin{array}{cccccccccccc} #1 \end{array}\right)}}



\renewcommand{\labelenumi}{\theenumi)}
\newenvironment{enumeratea}%
{\begingroup
 \renewcommand{\theenumi}{\alph{enumi}}
 \renewcommand{\labelenumi}{(\theenumi)}
 \begin{enumerate}}
 {\end{enumerate}\endgroup}



\newcounter{help}
\renewcommand{\thehelp}{\thesection.\arabic{equation}}

%\newenvironment{equation*}%
%{\renewcommand\endequation{\eqno (\theequation)* $$}%
%   \begin{equation}}%
%   {\end{equation}\renewcommand\endequation{\eqno \@eqnnum
%$$\global\@ignoretrue}}

%\input{psfig.tex}

\author{Martin Golubitsky and Michael Dellnitz}

%\newenvironment{matlabEquation}%
%{\renewcommand\endequation{\eqno (\theequation*) $$}%
%   \begin{equation}}%
%   {\end{equation}\renewcommand\endequation{\eqno \@eqnnum
% $$\global\@ignoretrue}}

\newcommand{\soln}{\textbf{Solution:} }
\newcommand{\exercap}[1]{\centerline{Figure~\ref{#1}}}
\newcommand{\exercaptwo}[1]{\centerline{Figure~\ref{#1}a\hspace{2.1in}
Figure~\ref{#1}b}}
\newcommand{\exercapthree}[1]{\centerline{Figure~\ref{#1}a\hspace{1.2in}
Figure~\ref{#1}b\hspace{1.2in}Figure~\ref{#1}c}}
\newcommand{\para}{\hspace{0.4in}}

\renewenvironment{solution}{\suppress}{\endsuppress}

\ifxake
\newenvironment{matlabEquation}{\begin{equation}}{\end{equation}}
\else
\newenvironment{matlabEquation}%
{\let\oldtheequation\theequation\renewcommand{\theequation}{\oldtheequation*}\begin{equation}}%
  {\end{equation}\let\theequation\oldtheequation}
\fi

\makeatother

\begin{document}

In Exercises~\ref{c6.6.2a} -- \ref{c6.6.2d} compute the general solution for
the given system of differential equations.
\begin{exercise}  \label{c6.6.2a}
$\dps\frac{dX}{dt} = \mattwo{-1}{-4}{2}{3} X$.

\begin{solution}

\ans The general solution to the differential equation is
\[
X(t) =
\alpha_1\cvectwo{2e^t\cos(2t)}{e^t(\sin(2t) - \cos(2t))} +
\alpha_2\cvectwo{2e^t\sin(2t)}{-e^t(\sin(2t) + \cos(2t))}.
\]

\soln First, find the eigenvalues of $C$, which are the roots of the
characteristic polynomial
\[
p_C(\lambda) = \lambda^2 - 2\lambda + 5.
\]
The eigenvalues are $\lambda_1 = 1 + 2i$ and $\lambda_2 = 1 - 2i$.  Then,
find the eigenvector associated to $\lambda_1$ by solving the equation
\[
(C - \lambda_1I_2)v_1 =
\left(\mattwo{-1}{-4}{2}{3} - \cmattwo{1 + 2i}{0}{0}{1 + 2i}\right)v_1
= \cmattwo{-2 - 2i}{-4}{2}{2 - 2i}v_1 = 0.
\]
Solve this equation to find that
\[
v_1 = \cvectwo{2}{-1 - i} = \vectwo{2}{-1} + i\vectwo{0}{-1}
\]
is an eigenvector of $C$.  Since the eigenvalues of $C$ are complex, we
can find the general solution using \eqref{E:CC1} and \eqref{E:CC2}.  In this
case, since $\lambda_1 = 1 + 2i$ is an eigenvalue, let $\sigma = 1$ and
let $\tau = 2$.  Then $v_1 = v + iw$, where $v = (2,-1)^t$ and
$w = (0,-1)^t$.  By \eqref{E:CC1},
\[
X_1(t) = e^{\sigma t}(\cos(\tau t)v - \sin(\tau t)w) \AND
X_2(t) = e^{\sigma t}(\sin(\tau t)v + \cos(\tau t)w)
\]
are solutions to the differential equation.  In this case,
\[
\begin{array}{rcl}
X_1(t) & = & e^t\left(\cos(2t)\vectwo{2}{-1} -
\sin(2t)\vectwo{0}{-1}\right)
= e^t\cvectwo{2\cos(2t)}{\sin(2t) - \cos(2t)}. \\
X_2(t) & = & e^t\left(\sin(2t)\vectwo{2}{-1} +
\cos(2t)\vectwo{0}{-1}\right)
= e^t\cvectwo{2\sin(2t)}{-\sin(2t) - \cos(2t)}.
\end{array}
\]
The general solution consists of all linear combinations
$X(t) = \alpha_1X_1(t) + \alpha_2X_2(t)$.


\end{solution}
\end{exercise}
\begin{exercise}  \label{c6.6.2b}
$\dps\frac{dX}{dt} = \mattwo{8}{-15}{3}{-4} X$.

\begin{solution}

\ans The general solution to the differential equation is
\[
X(t) =
\alpha_1\cvectwo{5e^{2t}\cos(3t)}{e^{2t}(2\cos(3t) + \sin(3t))} +
\alpha_2\cvectwo{5e^{2t}\sin(3t)}{e^{2t}(2\sin(3t) - \cos(3t))}.
\]

\soln First, find the eigenvalues of $C$, which are the roots of the
characteristic polynomial
\[
p_C(\lambda) = \lambda^2 - 4\lambda + 13.
\]
The eigenvalues are $\lambda_1 = 2 + 3i$ and $\lambda_2 = 2 - 3i$.  Then,
find the eigenvector associated to $\lambda_1$ by solving the equation
\[
(C - \lambda_1I_2)v_1 =
\left(\mattwo{8}{-15}{3}{-4} - \cmattwo{2 + 3i}{0}{0}{2 + 3i}\right)v_1
= \cmattwo{6 - 3i}{-15}{3}{-6 - 3i}v_1 = 0.
\]
Solve this equation to find that
\[
v_1 = \cvectwo{5}{2 - i} = \vectwo{5}{2} + i\vectwo{0}{-1}
\]
is an eigenvector of $C$.  Since the eigenvalues of $C$ are complex, we
can find the general solution using \eqref{E:CC1} and \eqref{E:CC2}.  In this
case, since $\lambda_1 = 2 + 3i$ is
an eigenvalue, let $\sigma = 2$ and let $\tau = 3$.  Then $v_1 = v + iw$,
where $v = (5,2)^t$ and $w = (0,-1)^t$.  By \eqref{E:CC1},
\[
X_1(t) = e^{\sigma t}(\cos(\tau t)v - \sin(\tau t)w) \AND
X_2(t) = e^{\sigma t}(\sin(\tau t)v + \cos(\tau t)w)
\]
are solutions to the differential equation.  In this case,
\[
\begin{array}{rcl}
X_1(t) & = & e^{2t}\left(\cos(3t)\vectwo{5}{2} -
\sin(3t)\vectwo{0}{-1}\right)
= e^{2t}\cvectwo{5\cos(3t)}{\sin(3t) + 2\cos(3t)}. \\
X_2(t) & = & e^{2t}\left(\sin(3t)\vectwo{5}{2} +
\cos(3t)\vectwo{0}{-1}\right)
= e^{2t}\cvectwo{5\sin(3t)}{2\sin(3t) - \cos(3t)}.
\end{array}
\]
The general solution consists of all linear combinations
$X(t) = \alpha_1X_1(t) + \alpha_2X_2(t)$.


\end{solution}
\end{exercise}
\begin{exercise}  \label{c6.6.2c}
$\dps\frac{dX}{dt} = \mattwo{5}{-1}{1}{3} X$.

\begin{solution}
\ans The general solution to the differential equation is
\[
X(t) = \alpha e^{4t}\vectwo{1}{1} + \beta e^{4t}\cvectwo{t + \frac{1}{2}}
{t - \frac{1}{2}}.
\]

\soln First, find the eigenvalues of $C$, which are the roots of the
characteristic polynomial
\[
p_C(\lambda) = \lambda^2 - 8\lambda + 16 = (\lambda - 4)^2.
\]
Thus, $C$ has a double eigenvalue at $\lambda_1 = 4$.  Since $C$ is not
a multiple of $I_2$, $C$ has only one linearly independent eigenvector.
Find this eigenvector by solving the equation
\[
(C - \lambda_1I_2)v_1 = \left(\mattwo{5}{-1}{1}{3} - \mattwo{4}{0}{0}{4}
\right)v_1 = \mattwo{1}{-1}{1}{-1}v_1 = 0,
\]
obtaining $v_1 = (1,1)^t$.  Find the generalized eigenvector $w_1$ by
solving the equation $(C - \lambda_1 I_2)w_1 = v_1$, that is
\[
\mattwo{1}{-1}{1}{-1}w_1 = \vectwo{1}{1}.
\]
So $w_1 = (\frac{1}{2},-\frac{1}{2})^t$ is the generalized eigenvector.
Now, by \eqref{e:exp1eva}, we know that the
general solution to $\dot{X} = CX$ when $C$ has equal eigenvalues and only
one independent eigenvector is
\[
X(t) = e^{\lambda_1 t}(\alpha v_1 + \beta(w_1 + tv_1)).
\]
In this case,
\[
X(t) = e^{4t}\left(\alpha \vectwo{1}{1} + \beta\left(\vectwo{\frac{1}{2}}
{-\frac{1}{2}} + t\vectwo{1}{1}\right)\right).
\]


\end{solution}
\end{exercise}
\begin{exercise}  \label{c6.6.2d}
$\dps\frac{dX}{dt} = \mattwo{-4}{4}{-1}{0} X$.

\begin{solution}
\ans The general solution to the differential equation is
\[
X(t) = \alpha e^{-2t}\cvectwo{2}{1} + \beta e^{-2t}\cvectwo{2t + 1}{t + 1}.
\]

\soln First, find the eigenvalues of $C$, which are the roots of the
characteristic polynomial
\[
p_C(\lambda) = \lambda^2 + 4\lambda + 4 = (\lambda + 2)^2.
\]
Thus, $C$ has a double eigenvalue at $\lambda_1 = -2$.  Since $C$ is not
a multiple of $I_2$, $C$ has only one linearly independent eigenvector.
Find this eigenvector by solving the equation
\[
(C - \lambda_1I_2)v_1 = \left(\mattwo{-4}{4}{-1}{0} + \mattwo{2}{0}{0}{2}
\right)v_1 = \mattwo{-2}{4}{-1}{2}v_1 = 0,
\]
obtaining $v_1 = (2,1)^t$.  Find the generalized eigenvector $w_1$ by
solving the equation $(C - \lambda_1 I_2)w_1 = v_1$, that is,
\[
\mattwo{-2}{4}{-1}{2}w_1 = \vectwo{2}{1}.
\]
So $w_1 = (1,1)^t$ is the generalized eigenvector.
Now, by \eqref{e:exp1eva}, we know that the
general solution to $\dot{X} = CX$ when $C$ has equal eigenvalues and only
one independent eigenvector is
\[
X(t) = e^{\lambda_1 t}(\alpha v_1 + \beta(w_1 + tv_1)).
\]
In this case,
\[
X(t) = e^{-2t}\left(\alpha \vectwo{2}{1} + \beta\left(\vectwo{1}{1} +
t\vectwo{2}{1}\right)\right).
\]






\end{solution}
\end{exercise}
\end{document}
