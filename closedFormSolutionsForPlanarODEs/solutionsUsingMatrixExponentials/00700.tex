\documentclass{ximera}
 

\usepackage{epsfig}

\graphicspath{
  {./}
  {figures/}
}

\usepackage{morewrites}
\makeatletter
\newcommand\subfile[1]{%
\renewcommand{\input}[1]{}%
\begingroup\skip@preamble\otherinput{#1}\endgroup\par\vspace{\topsep}
\let\input\otherinput}
\makeatother

\newcommand{\includeexercises}{\directlua{dofile("/home/jim/linearAlgebra/laode/exercises.lua")}}

%\newcounter{ccounter}
%\setcounter{ccounter}{1}
%\newcommand{\Chapter}[1]{\setcounter{chapter}{\arabic{ccounter}}\chapter{#1}\addtocounter{ccounter}{1}}

%\newcommand{\section}[1]{\section{#1}\setcounter{thm}{0}\setcounter{equation}{0}}

%\renewcommand{\theequation}{\arabic{chapter}.\arabic{section}.\arabic{equation}}
%\renewcommand{\thefigure}{\arabic{chapter}.\arabic{figure}}
%\renewcommand{\thetable}{\arabic{chapter}.\arabic{table}}

%\newcommand{\Sec}[2]{\section{#1}\markright{\arabic{ccounter}.\arabic{section}.#2}\setcounter{equation}{0}\setcounter{thm}{0}\setcounter{figure}{0}}

\newcommand{\Sec}[2]{\section{#1}}

\setcounter{secnumdepth}{2}
%\setcounter{secnumdepth}{1} 

%\newcounter{THM}
%\renewcommand{\theTHM}{\arabic{chapter}.\arabic{section}}

\newcommand{\trademark}{{R\!\!\!\!\!\bigcirc}}
%\newtheorem{exercise}{}

\newcommand{\dfield}{{\sf dfield9}}
\newcommand{\pplane}{{\sf pplane9}}

\newcommand{\EXER}{\section*{Exercises}}%\vspace*{0.2in}\hrule\small\setcounter{exercise}{0}}
\newcommand{\CEXER}{}%\vspace{0.08in}\begin{center}Computer Exercises\end{center}}
\newcommand{\TEXER}{} %\vspace{0.08in}\begin{center}Hand Exercises\end{center}}
\newcommand{\AEXER}{} %\vspace{0.08in}\begin{center}Hand Exercises\end{center}}

% BADBAD: \newcommand{\Bbb}{\bf}

\newcommand{\R}{\mbox{$\Bbb{R}$}}
\newcommand{\C}{\mbox{$\Bbb{C}$}}
\newcommand{\Z}{\mbox{$\Bbb{Z}$}}
\newcommand{\N}{\mbox{$\Bbb{N}$}}
\newcommand{\D}{\mbox{{\bf D}}}
\usepackage{amssymb}
%\newcommand{\qed}{\hfill\mbox{\raggedright$\square$} \vspace{1ex}}
%\newcommand{\proof}{\noindent {\bf Proof:} \hspace{0.1in}}

\newcommand{\setmin}{\;\mbox{--}\;}
\newcommand{\Matlab}{{M\small{AT\-LAB}} }
\newcommand{\Matlabp}{{M\small{AT\-LAB}}}
\newcommand{\computer}{\Matlab Instructions}
\newcommand{\half}{\mbox{$\frac{1}{2}$}}
\newcommand{\compose}{\raisebox{.15ex}{\mbox{{\scriptsize$\circ$}}}}
\newcommand{\AND}{\quad\mbox{and}\quad}
\newcommand{\vect}[2]{\left(\begin{array}{c} #1_1 \\ \vdots \\
 #1_{#2}\end{array}\right)}
\newcommand{\mattwo}[4]{\left(\begin{array}{rr} #1 & #2\\ #3
&#4\end{array}\right)}
\newcommand{\mattwoc}[4]{\left(\begin{array}{cc} #1 & #2\\ #3
&#4\end{array}\right)}
\newcommand{\vectwo}[2]{\left(\begin{array}{r} #1 \\ #2\end{array}\right)}
\newcommand{\vectwoc}[2]{\left(\begin{array}{c} #1 \\ #2\end{array}\right)}

\newcommand{\ignore}[1]{}


\newcommand{\inv}{^{-1}}
\newcommand{\CC}{{\cal C}}
\newcommand{\CCone}{\CC^1}
\newcommand{\Span}{{\rm span}}
\newcommand{\rank}{{\rm rank}}
\newcommand{\trace}{{\rm tr}}
\newcommand{\RE}{{\rm Re}}
\newcommand{\IM}{{\rm Im}}
\newcommand{\nulls}{{\rm null\;space}}

\newcommand{\dps}{\displaystyle}
\newcommand{\arraystart}{\renewcommand{\arraystretch}{1.8}}
\newcommand{\arrayfinish}{\renewcommand{\arraystretch}{1.2}}
\newcommand{\Start}[1]{\vspace{0.08in}\noindent {\bf Section~\ref{#1}}}
\newcommand{\exer}[1]{\noindent {\bf \ref{#1}}}
\newcommand{\ans}{}
\newcommand{\matthree}[9]{\left(\begin{array}{rrr} #1 & #2 & #3 \\ #4 & #5 & #6
\\ #7 & #8 & #9\end{array}\right)}
\newcommand{\cvectwo}[2]{\left(\begin{array}{c} #1 \\ #2\end{array}\right)}
\newcommand{\cmatthree}[9]{\left(\begin{array}{ccc} #1 & #2 & #3 \\ #4 & #5 &
#6 \\ #7 & #8 & #9\end{array}\right)}
\newcommand{\vecthree}[3]{\left(\begin{array}{r} #1 \\ #2 \\
#3\end{array}\right)}
\newcommand{\cvecthree}[3]{\left(\begin{array}{c} #1 \\ #2 \\
#3\end{array}\right)}
\newcommand{\cmattwo}[4]{\left(\begin{array}{cc} #1 & #2\\ #3
&#4\end{array}\right)}

\newcommand{\Matrix}[1]{\ensuremath{\left(\begin{array}{rrrrrrrrrrrrrrrrrr} #1 \end{array}\right)}}

\newcommand{\Matrixc}[1]{\ensuremath{\left(\begin{array}{cccccccccccc} #1 \end{array}\right)}}



\renewcommand{\labelenumi}{\theenumi)}
\newenvironment{enumeratea}%
{\begingroup
 \renewcommand{\theenumi}{\alph{enumi}}
 \renewcommand{\labelenumi}{(\theenumi)}
 \begin{enumerate}}
 {\end{enumerate}\endgroup}



\newcounter{help}
\renewcommand{\thehelp}{\thesection.\arabic{equation}}

%\newenvironment{equation*}%
%{\renewcommand\endequation{\eqno (\theequation)* $$}%
%   \begin{equation}}%
%   {\end{equation}\renewcommand\endequation{\eqno \@eqnnum
%$$\global\@ignoretrue}}

%\input{psfig.tex}

\author{Martin Golubitsky and Michael Dellnitz}

%\newenvironment{matlabEquation}%
%{\renewcommand\endequation{\eqno (\theequation*) $$}%
%   \begin{equation}}%
%   {\end{equation}\renewcommand\endequation{\eqno \@eqnnum
% $$\global\@ignoretrue}}

\newcommand{\soln}{\textbf{Solution:} }
\newcommand{\exercap}[1]{\centerline{Figure~\ref{#1}}}
\newcommand{\exercaptwo}[1]{\centerline{Figure~\ref{#1}a\hspace{2.1in}
Figure~\ref{#1}b}}
\newcommand{\exercapthree}[1]{\centerline{Figure~\ref{#1}a\hspace{1.2in}
Figure~\ref{#1}b\hspace{1.2in}Figure~\ref{#1}c}}
\newcommand{\para}{\hspace{0.4in}}

\renewenvironment{solution}{\suppress}{\endsuppress}

\ifxake
\newenvironment{matlabEquation}{\begin{equation}}{\end{equation}}
\else
\newenvironment{matlabEquation}%
{\let\oldtheequation\theequation\renewcommand{\theequation}{\oldtheequation*}\begin{equation}}%
  {\end{equation}\let\theequation\oldtheequation}
\fi

\makeatother

\begin{document}

\noindent In Exercises~\ref{c6.2.5A} -- \ref{c6.2.5C} we use
Theorem~\ref{T:linODEsoln}, the uniqueness of solutions to initial value
problems, in perhaps a surprising way.
\begin{computerExercise}  \label{c6.2.5A}
Prove that
\[
e^{t+s} = e^te^s
\]
for all real numbers $s$ and $t$.  {\bf Hint:}
\begin{itemize}
\item[(a)]  Fix $s$ and verify that $y(t) = e^{t+s}$ is a solution to the
initial value problem
\begin{equation}  \label{E:init1}
\begin{array}{rcl}
\frac{dx}{dt} & = & x \\
x(0) & = & e^s
\end{array}
\end{equation}
\item[(b)] Fix $s$ and verify that $z(t) = e^te^s$ is also a solution to
\eqref{E:init1}.
\item[(c)]  Use Theorem~\ref{T:linODEsoln} to conclude that $y(t)=z(t)$ for
every $s$.
\end{itemize}

\begin{solution}

(a) To verify that $y(t) = e^{t + s}$ is a solution to the initial value
problem, first substitute $y(t)$ into the left hand side of the equation.
Using the chain rule, obtain
\[
\frac{dy}{dt}(t) = \frac{d}{dt}(e^{t + s}) = \frac{d}{dt}(t + s)e^{t + s}
= e^{t + s}.
\]
Then substitute $y(t)$ into the right hand side of the equation, obtaining
\[
y(t) = e^{t + s}.
\]
Thus, the left hand and right hand sides are equal, so $y(t)$ is a
solution to the differential equation.  Finally, check to see that
$y(t)$ satisfies the initial value:
\[
y(0) = e^{0 + s} = e^s
\]
as desired.

(b) Similarly, verify that $z(t) = e^te^s$ is a solution to the initial
value problem by substituting $z(t)$ into each side of the differential
equation:
\[
\frac{dz}{dt}(t) = \frac{d}{dt}(e^te^s) = e^te^s \AND
z(t) = e^te^s.
\]
Note that the results are equal, so that $z(t)$ is a solution to the
differential equation.  Since
\[
z(0) = e^0e^s = e^s,
\]
it follows that $z(t)$ is also a solution to the initial value problem.

(c) By Theorem~\ref{exist&unique}, if
$f(x)$ is differentiable near $x_0$, and if $\frac{df}{dx}$ is continuous,
then there is a unique solution to the differential equation $\dot{x} = f(x)$
with initial condition $x(0) = x_0$.  Let $f(x) = x$ and let $x(0) = e^s$.
Then, as shown in (a) and (b) of this problem, $e^{t + s}$ and $e^te^s$ are
both solutions.  However, by Theorem~\ref{exist&unique}, there is only one
solution, so $e^{t + s} = e^te^s$.

\end{solution}
\end{computerExercise}
\begin{computerExercise}  \label{c6.2.5B}
Let $A$ be an $n\times n$ matrix.  Prove that
\[
e^{(t+s)A} = e^{tA}e^{sA}
\]
for all real numbers $s$ and $t$.  {\bf Hint:}
\begin{itemize}
\item[(a)]  Fix $s\in\R$ and $X_0\in\R^n$ and verify that
$Y(t) = e^{(t+s)A}X_0$ is a solution to the initial value problem
\begin{equation}  \label{E:init2}
\begin{array}{rcl}
\frac{dX}{dt} & = & AX \\
X(0) & = & e^{sA}X_0
\end{array}
\end{equation}
\item[(b)] Fix $s$ and verify that $Z(t) = e^{tA}\left(e^{sA}X_0\right)$ is
also a solution to \eqref{E:init2}.
\item[(c)]  Use the $n$ dimensional version of Theorem~\ref{T:linODEsoln} to
conclude that $Y(t)=Z(t)$ for every $s$ and every $X_0$.
\end{itemize}
{\bf Remark:}  Compare the result in this exercise with the calculation in
Exercise~\ref{c6.2.5}.

\begin{solution}

(a) To verify that $Y(t)$ is a solution to the initial value problem
\eqref{E:init2}, first substitute $Y(t)$ into the left hand side of the
equation.  Using the chain rule, obtain
\[
\frac{dY}{dt}(t) = \frac{d}{dt}(e^{(t + s)A}X_0) =
\frac{d}{dt}((t + s)A)e^{(t + s)A}X_0 = Ae^{(t + s)A}X_0.
\]
Then substitute $Y(t)$ into the right hand side of the equation, obtaining
\[
AY(t) = Ae^{(t + s)A}X_0.
\]
Thus, the left hand and right hand sides of the equation are equal, so
$Y(t)$ is a solution to the differential equation.  Finally, check to see
that $Y(t)$ satisfies the initial value:
\[
Y(0) = e^{(0 + s)A}X_0 = e^{sA}X_0,
\]
as desired.

(b) Similarly, verify that $Z(t) = e^{tA}(e^{sA}X_0)$ is a solution to the
initial value problem by substituting $Z(t)$ into each side of the
differential equation:
\[
\frac{dZ}{dt}(t) = \frac{d}{dt}(e^{tA}(e^{sA}X_0))
= Ae^{tA}(e^{sA}X_0) \AND
AZ(t) = Ae^{tA}(e^{sA}X_0).
\]
Since the results are equal, $Z(t)$ is a solution to the differential
equation.  Evaluating at $t = 0$, we find
\[
Z(0) = e^{0}(e^{sA}X_0) = e^{sA}X_0,
\]
from which it follows that $Z(t)$ is also a solution to the initial
value problem.

(c) Since $Y(t)$ and $Z(t)$ are both solutions to the initial value problem
\[
\begin{array}{rcl}
\frac{dX}{dt} & = & AX \\
X(0) & = & e^{sA}X_0,
\end{array}
\]
it follows from the uniqueness part of Theorem~\ref{exist&unique} that
$Y(t) = Z(t)$.  Thus, $e^{(t + s)A} = e^{tA}e^{sA}$, as desired.


\end{solution}
\end{computerExercise}
\begin{computerExercise}  \label{c6.2.5C}
Prove that
\begin{equation}  \label{E:0-110E}
\exp\left(t\mattwo{0}{-1}{1}{0}\right) =
\mattwo{\cos t}{-\sin t}{\sin t}{\cos t}.
\end{equation}
{\bf Hint:}
\begin{itemize}
\item[(a)] Verify that $X_1(t) = \vectwo{\cos t}{\sin t}$ and
$X_2(t) = \vectwo{-\sin t}{\cos t}$ are solutions to the initial value problems
\begin{equation}  \label{E:init3}
\begin{array}{rcl}
\dps\frac{dX}{dt} & = & \mattwo{0}{-1}{1}{0}X \\
X(0) & = & e_j
\end{array}
\end{equation}
for $j=1,2$.
\item[(b)] Since $X_j(0)=e_j$, use Theorem~\ref{T:linODEsoln} to verify that
\begin{equation}   \label{E:0-110}
X_j(t) = \exp\left(t\mattwo{0}{-1}{1}{0}\right)e_j.
\end{equation}
\item[(c)]  Show that \eqref{E:0-110} proves \eqref{E:0-110E}
\end{itemize}

\begin{solution}

(a) To verify that $X_1(t) = (\cos t,\sin t)^t$ is a solution to the
initial value problem \eqref{E:init3}, substitute $X_1(t)$ into the left
hand side of the differential equation, obtaining
\[
\frac{dX_1}{dt} = \frac{d}{dt}\vectwo{\cos t}{\sin t} =
\vectwo{-\sin t}{\cos t}.
\]
Then substitute $X_1(t)$ into the right hand side of the differential
equation, obtaining
\[
\mattwo{0}{-1}{1}{0}X_1 = \mattwo{0}{-1}{1}{0}\vectwo{\cos t}{\sin t}
= \vectwo{-\sin t}{\cos t}.
\]
Since the two sides are equal, $X_1(t)$ is a solution to
\eqref{E:init3}.  Further, since
\[
X_1(0) = \vectwo{\cos 0}{\sin 0} = \vectwo{1}{0} = e_1,
\]
$X_1(t)$ is a solution to the given initial value problem.

\para Similarly, to verify that $X_2(t) = (-\sin t,\cos t)^t$ is a
solution to the initial value problem, substitute $X_2(t)$ into the
left hand side of the differential equation, obtaining
\[
\frac{dX_2}{dt} = \frac{d}{dt}\vectwo{-\sin t}{\cos t} =
\vectwo{-\cos t}{-\sin t}.
\]
Then substitute $X_2(t)$ into the right hand side of the differential
equation, obtaining
\[
\mattwo{0}{-1}{1}{0}X_2 = \mattwo{0}{-1}{1}{0}\vectwo{-\sin t}{\cos t}
= \vectwo{-\cos t}{-\sin t}.
\]
Since the two sides of the differential equation are equal, and since
\[
X_2(0) = \vectwo{-\sin 0}{\cos 0} = \vectwo{0}{1} = e_2,
\]
$X_2(t)$ is a solution to the initial value problem \eqref{E:init3}.

(b) By Theorem~\ref{T:linODEsoln}, the unique
solution to \eqref{E:init3} with initial condition $X(0) = e_j$ is
\[
Y_j(t) \equiv \exp\left(t\mattwo{0}{-1}{1}{0}\right)e_j.
\]
We showed in part (a) that $X_j(t)$ satisfies this initial value
problem.  Thus, by the uniqueness part of Theorem~\ref{exist&unique},
$X_j(t) = Y_j(t)$, as desired.

(c) By part (b), $X_j(t)$ is equal to $Y_j(t)$, which is defined as
the $j^{th}$ column of the matrix
\[
\exp\left(t\mattwo{0}{-1}{1}{0}\right).
\]
By definition, $X_j(t)$ is equal to the $j^{th}$ column of the matrix
\[
\mattwo{\cos t}{-\sin t}{\sin t}{\cos t}.
\]
Thus,
\[
\exp\left(t\mattwo{0}{-1}{1}{0}\right) =
\mattwo{\cos t}{-\sin t}{\sin t}{\cos t}.
\]

\end{solution}
\end{computerExercise}
\end{document}
