\chapter{Determinants and Eigenvalues}

\subsection*{Section~\protect{\ref{S:det}} Determinants}
\rhead{S:det}{DETERMINANTS}

\exer{c10.1.1a}
\ans The determinant of the matrix is $-28$.

\soln Expand along the third column, obtaining:
\[
\det\matthree{-2}{1}{0}{4}{5}{0}{1}{0}{2} = 2\det\mattwo{-2}{1}{4}{5}
= 2(-14) = -28.
\]

\exer{c10.1.1b}
\ans The determinant of the matrix is $-110$.

\soln First row reduce:
\[
\det\left(\begin{array}{rrrr}
1 & 0 & 2 & 3 \\ 
-1 & -2 & 3 & 2 \\
4 & -2 & 0 & 3 \\
1 & 2 & 0 & -3 \end{array}\right) =
\det\left(\begin{array}{rrrr}
1 & 0 & 2 & 3 \\ 
0 & -2 & 5 & 5 \\
0 & -2 & -8 & -9 \\
0 & 2 & -2 & -6 \end{array}\right).
\]
Then, use formula \Ref{e:inductdet}:
\[ \begin{array}{rcl}
\det\matthree{-2}{5}{5}{-2}{-8}{-9}{2}{-2}{-6} & = &
\begin{array}{l}
(-2)(-8)(-6) + 5(-9)2 + 5(-2)(-2) - 5(-8)2 \\
- 5(-2)(-6) - (-2)(-9)(-2) \end{array} \\
& = & -96 - 90 + 20 + 80 - 60 + 36 \\
& = & -110.
\end{array}
\]

\exer{c10.1.1c}
\ans The determinant of the matrix is $14$.

\soln Using Lemma~\ref{L:detblockdiag}, compute
\[ 
\begin{array}{rcl}
\det(A) & = & 
\det\matthree{2}{1}{-1}{1}{-2}{3}{-3}{2}{-2}
\det\mattwo{2}{4}{-1}{-3} \\
& = & \big(2(-2)(-2) + 1 \cdot 3(-3) + (-1)1 \cdot 2
- (-1)(-2)(-3) - 1 \cdot 1(-2) \\
& & - 2 \cdot 3 \cdot 2\big)
(-2) \\
& = & (-7)(-2) \\
& = & 14.
\end{array}
\]

\exer{c10.1.2}
\ans The solution is $\det(A^{-1}) = \frac{1}{35}$.

\soln By Definition~\ref{D:determinants}(c),
$\det(A)\det(A^{-1}) = \det(I_3) = 1$.  Therefore,
\[
\det(A^{-1}) = \frac{1}{\det(A)}.
\]
Now compute $\det(A)$ using \Ref{e:inductdet}:
\[
\det(A) = -2\mattwo{1}{3}{1}{1} + 3\mattwo{4}{3}{-1}{1}
+ 2\mattwo{4}{1}{-1}{1} = 35.
\]

\exer{c10.1.3}
Two $n \times n$ matrices $B$ and $C$ are similar if there exists
an $n \times n$ matrix $P$ such that $B = P^{-1}CP$.  Therefore, by
Definition~\ref{D:determinants}(c),
\[
\det(B) = \det(P^{-1}CP)
= \det(P^{-1})\det(C)\det(P)
= \det(P)^{-1}\det(P)\det(C)
= \det(C).
\]

\exer{c10.1.4}
\ans The determinant is $\det(A) = 18$. 

\soln Compute by row reduction as follows:
\[
\begin{array}{rcl}
\det\matthree{-1}{-2}{1}{3}{1}{3}{-1}{1}{1}
& = & -\det\matthree{1}{2}{-1}{0}{-5}{6}{0}{3}{0} \\
& = & 3\det\matthree{1}{2}{-1}{0}{1}{0}{0}{-5}{6} \\
& = & 3\det\matthree{1}{2}{-1}{0}{1}{0}{0}{0}{6} = 18.
\end{array}
\]

\exer{c10.1.5a}
\ans The determinant is $\det(B) = -4$.

\soln Compute by row reduction:
\[
\det\left(\begin{array}{rrrr}
1 & 0 & 1 & 0 \\
0 & 1 & 0 & -1 \\
1 & 0 & -1 & 0 \\
0 & 1 & 0 & 1 \end{array}\right)
= \det\left(\begin{array}{rrrr}
1 & 0 & 1 & 0 \\
0 & 1 & 0 & -1 \\
0 & 0 & -2 & 0 \\
0 & 0 & 0 & 2 \end{array}\right)
= 1(1)(-2)(2).
\]

\exer{c10.1.5b}
\ans The determinant is $\det(C) = -7$.

\soln Compute by row reduction:
\[
\det\left(\begin{array}{rrrr}
1 & 2 & 0 & 1 \\
0 & 2 & 1 & 0 \\
-2 & -3 & 3 & -1 \\
1 & 0 & 5 & 2 \end{array}\right)
= 5\det\left(\begin{array}{rrrr}
1 & 2 & 0 & 1 \\
0 & 1 & 3 & 1 \\
0 & 0 & 1 & \frac{2}{5} \\
0 & 0 & 0 & -\frac{7}{5} \end{array}\right)
= 5\left(-\frac{7}{5}\right).
\]

\exer{c10.1.6}
(a) \ans When $\lambda = 1$ or $\lambda = -\frac{1}{3}$,
$\det(\lambda A - B) = 0$.

\soln Compute
\[
\begin{array}{rcl}
\det(\lambda A - B) & = & \det\cmatthree{2\lambda - 2}{-\lambda}{0}
{0}{3\lambda + 1}{0}{\lambda}{5\lambda}{3\lambda - 3} \\
& = & 3(\lambda - 1)\det\cmattwo{2(\lambda - 1)}{-\lambda}{0}
{3\lambda + 1} \\
& = & 6(\lambda - 1)^2(3\lambda - 1).
\end{array}
\]

(b) \ans Yes, there exists a vector $x \in \R^3$ such that $Ax = Bx$.

\soln Let $\lambda = 1$.  Then, part (a) of this exercise implies that
$\det(A - B) = 0$.  Therefore, there exists a vector $x$ such that
$(A - B)x = 0$, that is, $Ax = Bx$.  You can solve this equation for $x$
to obtain $x = (0,0,1)^t$.

\exer{c10.1.a7a}
By swapping rows $2$ and $3$ of matrix $A$, we find that:
\[
\det\matthree{1}{0}{0}{0}{0}{1}{0}{1}{0} =
-\det\matthree{1}{0}{0}{0}{1}{0}{0}{0}{1} = -1. \\
\]

\exer{c10.1.a7b}
By swapping rows $1$ and $3$ of matrix $B$, we find that:
\[
\det\matthree{0}{0}{1}{0}{1}{0}{1}{0}{0} =
-\det\matthree{1}{0}{0}{0}{1}{0}{0}{0}{1} = -1.
\]

\newpage
\exer{c10.1.b7a}
$A_{13} = \mattwo{0}{1}{0}{0}$;
$A_{22} = \mattwo{3}{-4}{0}{6}$;
$A_{21} = \mattwo{2}{-4}{0}{6}$.

\exer{c10.1.b7b}
$B_{11} = \matthree{7}{-2}{10}{0}{0}{-1}{4}{2}{-10}$;
$B_{23} = \matthree{0}{2}{5}{0}{0}{-1}{3}{4}{-10}$;
$B_{43} = \matthree{0}{2}{5}{-1}{7}{10}{0}{0}{-1}$.

\exer{c10.1.c7}
\ans The determinant of $A_\lambda$ vanishes at $\lambda = -1$,
$\lambda = 1$, and $\lambda = 2$.

\soln Compute $\det(A_\lambda)$ by expanding along the first column
to obtain:
\[
\begin{array}{rcl}
\det(A_\lambda) & = &
(\lambda - 1)((\lambda - 1)\lambda - 1) - 1(\lambda - 1) \\
& = & (\lambda - 1)(\lambda - 2)(\lambda + 1).
\end{array}
\]

\exer{c10.1.c8}
By Proposition~\ref{P:ERO}, every
elementary row operation on $A$ can be represented by an invertible $n
\times n$ matrix $E$.  That is, the matrix $EA$ is row equivalent to
$A$.  If $A$ and $B$ are row equivalent, then there exist matrices
$E_j$ such that $B = E_k\ldots E_1A$.  The product of invertible $n
\times n$ matrices is an invertible $n \times n$ matrix.  Thus $P =
E_k\ldots E_1$ is an invertible $n \times n$ matrix such that $B =
PA$.

\exer{c10.1.c9}
Let $A$ be an invertible $n\times n$ matrix and let $b\in\R^n$ be a column 
vector. Let $B_j$ be the $n\times n$ matrix obtained from $A$ by replacing 
the $j^{th}$ column of $A$ by the vector $b$.  Let $x=(x_1,\ldots,x_n)^t$ be 
the unique solution to $Ax=b$.  We claim that
\[
x_j = \frac{\det(B_j)}{\det(A)}.
\]
Let $A_j$ be the $j^{th}$ column of $A$ so that $Ae_j=A_j$.  It follows that 
\begin{eqnarray*}
A (e_1|\cdots|e_{j-1}|x|e_{j+1}|\cdots|e_n) & = & 
(A e_1|\cdots|A e_{j-1}|A x|A e_{j+1}|\cdots|A e_n)\\
& = & (A_1|\cdots|A_{j-1}|b|A_{j+1}|\cdots|A_n) \\
& = & B_j.
\end{eqnarray*}
Therefore,
\[
\det(B_j) = \det(A)\det(e_1|\cdots|e_{j-1}|x|e_{j+1}|\cdots|e_n).
\]
We claim that 
\[
\det(e_1|\cdots|e_{j-1}|x|e_{j+1}|\cdots|e_n)=x_j,
\]
from which Cramer's rule follows.  Since $\det(C)=\det(C^t)$, we can perform
elementary column operations on $C$ while computing determinants.  In
particular, we can subtract $x_ke_k$ (for $k>j$) from the $j^{th}$ column of
\[
(e_1|\cdots|e_{j-1}|x|e_{j+1}|\cdots|e_n)
\]
without changing its determinant.  The end result is that 
\[
\det(e_1|\cdots|e_{j-1}|x|e_{j+1}|\cdots|e_n)=
\det(e_1|\cdots|e_{j-1}|y|e_{j+1}|\cdots|e_n),
\]
where $y=(x_1,\ldots,x_j,0,\ldots,0)^t$.  Since this last matrix is upper
triangular, its determinant is the product of its diagonal elements ---
which equals $x_j$. 

\subsection*{Section~\protect{\ref{S:eig}} Eigenvalues}
\rhead{S:eig}{EIGENVALUES}

\exer{c10.2.1a}
\ans The characteristic polynomial of $A$ is $p_A(\lambda) =
-\lambda^3 + 2\lambda^2 + \lambda - 2$, and the eigenvalues are
$\lambda_1 = 1$, $\lambda_2 = -1$, and $\lambda_3 = 2$.

\soln Compute:
\[
\begin{array}{rcl}
p_A(\lambda) & = & \det(A - \lambda I_3) \\
& = & \cmatthree{-9 - \lambda}{-2}{-10}{3}{2 - \lambda}{3}
{8}{2}{9 - \lambda} \\
& = & (-9 - \lambda)\det\cmattwo{2 - \lambda}{3}{2}{9 - \lambda}
- 3\det\cmattwo{-2}{-10}{2}{9 - \lambda} + \\
& & 8\det\cmattwo{-2}{-10}{2 - \lambda}{3} \\
& = & (-9 - \lambda)(\lambda^2 - 11\lambda + 12)
- 3(2 + 2\lambda) + 8(14 - 10\lambda) \\
& = & -\lambda^3 + 2\lambda^2 + \lambda - 2 \\
& = & (\lambda - 1)(\lambda + 1)(\lambda - 2). \end{array}
\]
The eigenvalues of $A$ are the roots of the characteristic polynomial.

\exer{c10.2.1b}
\ans The characteristic polynomial of $B$ is $p_B(\lambda) = \lambda^4
- 8\lambda^3 + 23\lambda^2 - 28\lambda + 12$.  The matrix $B$ has single
eigenvalues at $\lambda = 1$ and $\lambda = 3$ and a double eigenvalue at
$\lambda = 2$.

\soln Using Lemma~\ref{L:detblockdiag},
compute:
\[
\begin{array}{rcl}
p_B(\lambda) & = & \det(B - \lambda I_3) \\
& = & \det\left(\begin{array}{cccc}
2 - \lambda & 1 & -5 & 2 \\
1 & 2 - \lambda & 13 & 2 \\
0 & 0 & 3 - \lambda & -1 \\
0 & 0 & 1 & 1 - \lambda \end{array}\right) \\
& = & \det\cmattwo{2 - \lambda}{1}{1}{2 - \lambda}
\det\cmattwo{3 - \lambda}{-1}{1}{1 - \lambda} \\
& = & ((2 - \lambda)^2 - 1)((3 - \lambda)(1 - \lambda) + 1) \\
& = & (\lambda - 3)(\lambda - 1)(\lambda - 2)^2.
\end{array}
\]

\exer{c10.2.2}
\ans A basis for the eigenspace of $A$ corresponding to the eigenvalue
$\lambda = 2$ is:
\[
\left\{\vecthree{-1}{1}{0}, \vecthree{1}{0}{1}\right\}.
\]

\soln First, find all eigenvectors of $A$ with eigenvalue
$\lambda = 2$, that is, all vectors $v = (v_1,v_2,v_3)$ such
that $(A - 2I_3)v = 0$.  Solve the system
\[
\matthree{1}{1}{-1}{-1}{-1}{1}{2}{2}{-2}\vecthree{v_1}{v_2}{v_3} =
\vecthree{0}{0}{0}.
\]
All solutions $v$ to this system satisfy $v_1 = v_3 - v_2$.  Thus:
\[
v = \cvecthree{v_3 - v_2}{v_2}{v_3} = v_2\vecthree{-1}{1}{0} +
v_3\vecthree{1}{0}{1}.
\]
Therefore, the vectors $(-1,1,0)^t$ and $(1,0,1)^t$ form a basis
for this eigenspace.

\exer{c10.2.3}
(a) Find the characteristic polynomial by solving
\[
\begin{array}{rcl}
p_A(\lambda) & = & \det(A - \lambda I_3) \\
& = & \cmatthree{-1 - \lambda}{1}{1}{1}{-1 - \lambda}{1}
{1}{1}{-1 - \lambda} \\
& = & (-1 - \lambda)\det\cmattwo{-1 - \lambda}{1}{1}{-1 - \lambda}
- \det\cmattwo{1}{1}{1}{-1 - \lambda} + \\
& & \det\cmattwo{1}{1}{-1 - \lambda}{1} \\
& = & -(\lambda^3 + 3\lambda - 4) \\
& = & -(\lambda - 1)(\lambda + 2)^2. \end{array}
\]

(b) Verify by computation:
\[
\matthree{-1}{1}{1}{1}{-1}{1}{1}{1}{-1}\vecthree{1}{1}{1} =
\vecthree{1}{1}{1}.
\]

(c) Find the space of eigenvectors $v = (v_1,v_2,v_3)$ corresponding
to $\lambda = -2$ by solving $(A - \lambda I_3)v = 0$ for $\lambda = -2$.
That is, solve
\[
\matthree{1}{1}{1}{1}{1}{1}{1}{1}{1}\vecthree{v_1}{v_2}{v_3} = 0
\]
to obtain
\[
v = v_2\vecthree{-1}{1}{0} + v_3\vecthree{-1}{0}{1}.
\]
Then compute
\[
\vecthree{1}{1}{1} \cdot \vecthree{-1}{1}{0} = 0 \AND
\vecthree{1}{1}{1} \cdot \vecthree{-1}{0}{1} = 0.
\]
Since $(-1,1,0)^t$ and $(-1,0,1)^t$ form a basis for the space of
eigenvectors of $A$ corresponding to $\lambda = -2$ and since
$(1,1,1)^t$ is orthogonal to these vectors, $(1,1,1)^t$ is
orthogonal to every eigenvector of $A$ corresponding to
$\lambda = -2$.

\exer{c10.2.4}
(a) \ans The eigenvalues of $A$ are $\lambda_1 = 3$ and $\lambda_2 = -2$,
with corresponding eigenvectors $v_1 = (1,-1)^t$ and
$v_2 = (1,-2)^t$, respectively.

\soln The characteristic polynomial is $p_A(\lambda) =
\lambda^2 - \lambda - 6 = (\lambda - 3)(\lambda + 2)$.  Then, solve
$Av = \lambda v$ for each eigenvalue to find the corresponding eigenvectors.

(b) Two linearly independent vectors in $\R^2$ form a basis for $\R^2$.
Note that $v_1 \neq \alpha v_2$ for any scalar $\alpha$.  Therefore,
$v_1$ and $v_2$ form a basis for $\R^2$.

(c) \ans The coordinates of $(x_1,x_2)$ in the basis $\{v_1,v_2\}$ are
$(2x_1 + x_2, -x_1 - x_2)$. 

\soln Find $\alpha_1$ and $\alpha_2$ such that $\alpha_1v_1 +
\alpha_2v_2 = (x_1,x_2)^t$.  That is, solve:
\[
\mattwo{1}{1}{-1}{-2}\vectwo{\alpha_1}{\alpha_2} = \vectwo{x_1}{x_2}
\]
to obtain $\alpha_1 = 2x_1 + x_2$ and $\alpha_2 = -x_1 - x_2$.

\exer{c10.2.5}
\ans The characteristic polynomial of $A$ is $p_A(\lambda) = -(\lambda^3
+ 5\lambda^2 + 6\lambda) = -\lambda(\lambda + 2)(\lambda + 3)$. 
The eigenvalues are $\lambda_1 = 0$, $\lambda_2 = -2$, and $\lambda_3 = -3$,
with eigenvectors $v_1 = (0,1,-1)^t$, $v_2 = (2,-1,0)^t$, and 
$v_3 = (1,0,-1)^t$, respectively.

\soln The eigenvalues are the roots of the characteristic polynomial
$p_A(\lambda) = \det(A - \lambda I_3)$.  The eigenvectors are vectors
$v$ such that $Av = \lambda v$, where $\lambda$ is an eigenvalue of
$A$.  Find them by solving the system $(A - \lambda I_3)v = 0$.

\exer{c10.2.6}
We are given $A^2 + A + I_n = 0$.  Therefore, $I_n = -A^2 - A =
A(-A - I_n)$.  Thus, $A^{-1} = -A - I_n$ exists.

\exer{c10.2.7a}
\ans The statement is false. 

\soln  A counterexample is the matrix $A = \mattwo{1}{500}{0}{1}$.

\exer{c10.2.7b} \ans The statement is false.

\soln For example, let
\[
A = \mattwo{1}{-1}{0}{1} \AND B = \mattwo{1}{-1}{2}{0}.
\]
Then $\trace(A)\trace(B) = 2(1) = 2$, and $\trace(AB) = -1$.

\exer{c10.2.8}
By Theorem~\ref{T:eigens}, every
$n \times n$ matrix has exactly $n$ eigenvalues, which are either
real or complex conjugate pairs.  Since complex eigenvalues are
paired, the number of complex eigenvalues must be even.  Since $n$ is
odd, there can be no more than $n - 1$ complex eigenvalues; so the
matrix has at least one real eigenvalue.

\exer{c10.2.9a}
(a) By calculation in \Matlab using the {\tt eig}, {\tt trace}, and
{\tt poly} commands, the eigenvalues of $A$ are 
\[
\lambda = -0.5861 \pm 20.2517, \quad
\lambda = -12.9416, \quad
\lambda = -9.1033, \AND
\lambda = 5.2171.
\]
The trace of $A$ is $-18$.  The characteristic polynomial of $A$ is
\[
p_A = \lambda^5 + 18\lambda^4 + 433\lambda^3 + 6296\lambda^2 +
429\lambda - 252292.
\]
Note that in order to obtain an accurate value for the characteristic
polynomial, it may be necessary to use the {\tt format} command.

(b) Theorem~\ref{T:inveig} states that the eigenvalues of $A^{-1}$ are
the inverses of the eigenvalues of $A$.  In \Matlab, compute
\begin{verbatim}
eig(inv(A)) =
  -0.1098    
  -0.0773    
  -0.0014 + 0.0493i
  -0.0014 - 0.0493i
   0.1917
\end{verbatim}
Then, compute the inverse of each eigenvalue of $A$ to find that if
$\lambda$ is an eigenvalue of $A$, then $\lambda^{-1}$ is indeed an
eigenvalue of $A^{-1}$. 

\exer{c10.2.9b}
(a) The eigenvalues of $B$ are
\[
\lambda = 32.6273, \quad
\lambda = -12.1564 \pm 5.8787i, \quad
\lambda = 18.0009, \quad
\lambda = -3.4878, \AND 
\lambda = 11.1723.
\]
The trace of $B$ is $34$.  The characteristic polynomial of $B$ is
\[
p_B = \lambda^6 - 24\lambda^5 - 298\lambda^4 + 9618\lambda^3
+ 86273\lambda^2 - 1019656\lambda - 4172976.
\]

(b) Theorem~\ref{T:inveig} states that the eigenvalues of $A^{-1}$ are
the inverses of the eigenvalues of $A$.  In \Matlab, compute
\begin{verbatim}
eig(inv(B)) =
  -0.2867
  -0.0667 + 0.0322i
  -0.0667 - 0.0322i
   0.0895
   0.0556
   0.0306
\end{verbatim}
Then, compute the inverse of each eigenvalue of $B$ to find that if
$\lambda$ is an eigenvalue of $B$, then $\lambda^{-1}$
is indeed an eigenvalue of $B^{-1}$.

\exer{c10.2.10}
\ans The matrix $B$ is the zero matrix.

\soln First use \Matlab to compute
$p_A(\lambda) = \lambda^3 - 9\lambda^2 + 9\lambda - 348$.  Then
$B = p_A(A) = A^3 - 9A^2 + 9A - 348I_3$ is the zero matrix.  To
see why this is true, see the Cayley-Hamilton Theorem
(Theorem~\ref{T:CH}).

