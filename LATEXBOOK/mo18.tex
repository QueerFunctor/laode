\chapter{Numerical Solutions of ODEs}

\subsection*{Section~\protect{\ref{sec:DNM}} A Description of Numerical Methods}
\rhead{sec:DNM}{A DESCRIPTION OF NUMERICAL METHODS}

\exer{c15.1.1} The solution to the initial value problem is
$x(t)=-2(t-t_0)$.  On the other hand, an application of Euler's method
with step size $h$ leads to
\[
\begin{array}{rcl}
x_0 & = & 0 \\
x_{k+1} & = & x_k + h (-2) = x_k - 2h\\
t_k & = & t_0 + kh.
\end{array}
\]
Hence
\[
\begin{array}{rcl}
x_0 & = & 0\\
x_1 & = & -2h\\
x_2 & = & -4h\\
 & \vdots & \\
x_k & = & -2kh,
\end{array}
\]
and
\[
x(t_k) = -2(t_k-t_0) = -2kh = x_k.
\]

\exer{c15.1.3a} \ans Yes.

\soln Set
\[
s=2,\quad b_1=0,\quad b_2=1,\quad c_1=0,\quad c_2=\frac{1}{2},\quad
a_{21}=\frac{1}{2}.
\]

\exer{c15.1.3c} \ans Yes.

\soln Set
\[
s=2,\quad b_1=\frac{1}{4},\quad b_2=\frac{3}{4},\quad
c_1=0,\quad c_2=\frac{2}{3},\quad a_{21}=\frac{2}{3}.
\]

\exer{c15.1.4a} The first step in Euler's method is given by
\[
x_1 = x_0 + hf(t_0,x_0),
\]
and hence formula \Ref{eq:tfeul} implies that
\begin{eqnarray*}
x(t_1) &=&  x(t_0)+h\frac{dx}{dt}(t_0)+
\frac{h^2}{2}\frac{d^2x}{dt^2}(t_0+\theta h)\\
&=& x_0 + hf(t_0,x_0) +\frac{h^2}{2}\frac{d^2x}{dt^2}(t_0+\theta h)\\
&=& x_1 +\frac{h^2}{2}\frac{d^2x}{dt^2}(t_0+\theta h).
\end{eqnarray*}
The exact solution is $x(t)= 4e^{t-1}-t-1$ with the second order
derivative
\[
\frac{d^2x}{dt^2}(t) = 4e^{t-1}>0\quad \mbox{for all $t$.}
\]
It follows that $x(t_1) > x_1$.

\exer{c15.1.5a} The following \Matlab commands approximate the
solution by

(i) Euler's method:
\begin{verbatim}
h    = 0.1;
t(1) = 2;
x(1) = 3;
K    = 20;
for k = 1:K
   t(k+1) = t(k)+h;
   x(k+1) = x(k)+h*1;
end
plot(t,x,'o')
hold on
plot(t,x,'--')
\end{verbatim}

(ii) The modified Euler method:
\begin{verbatim}
h    = 0.1;
t(1) = 2;
x(1) = 3;
K    = 20;
for k = 1:K
   t(k+1) = t(k)+h;
   y(k)   = x(k)+h*1;
   x(k+1) = x(k)+(h/2)*(1+1);
end
plot(t,x,'o')
hold on
plot(t,x,'--')
\end{verbatim}

\exer{c15.1.5c} The following \Matlab commands approximate the solution by

(i) Euler's method:
\begin{verbatim}
h    = 0.05;
t(1) = 1;
x(1) = -1;
K    = 40;
for k = 1:K
   t(k+1) = t(k)+h;
   x(k+1) = x(k)+h*t(k)*x(k);
end
plot(t,x,'o')
hold on
plot(t,x,'--')
\end{verbatim}

(ii) The modified Euler method:
\begin{verbatim}
h    = 0.05;
t(1) = 1;
x(1) = -1;
K    = 40;
for k = 1:K
   t(k+1) = t(k)+h;
   y(k)   = x(k)+h*t(k)*x(k);
   x(k+1) = x(k)+(h/2)*(t(k)*x(k)+t(k+1)*y(k));
end
plot(t,x,'o')
hold on
plot(t,x,'--')
\end{verbatim}

\exer{c15.1.8} For the initial value problem \Ref{eq:eulexivp} the
fourth order Runge-Kutta method is given by
\[
t_{k+1} = t_k+h,\quad f_1= x_k + t_k,\quad
f_2 = t_k+\frac{h}{2} + x_k + \frac{h}{2}f_1,
\]
\[
f_3 = t_k+\frac{h}{2} + x_k + \frac{h}{2}f_2,\quad f_4 = t_k+h+x_k+hf_3,
\]
and
\[
x_{k+1} = x_k + \frac{h}{6}(f_1+2f_2+2f_3+f_4).
\]
Hence we can reproduce Figure~\ref{fig:rk1} using the following
\Matlab commands:
\begin{verbatim}
h    = 0.6;
t(1) = 1;
x(1) = 2;
K    = 20;
for k = 1:K
   t(k+1) = t(k)+h;
   f1 = x(k)+t(k);
   f2 = t(k)+h/2 + x(k)+h/2*f1;
   f3 = t(k)+h/2 + x(k)+h/2*f2;
   f4 = t(k)+h + x(k)+h*f3;
   x(k+1) = x(k)+h/6*(f1+2*f2+2*f3+f4);
end
plot(t,x,'o')
hold on
plot(t,x,'--')
s = 1 : 0.1 : 13;
y = 4*exp(s-1)-s-1;
plot(s,y)
\end{verbatim}
Using the m-file {\tt fexam.m},
\begin{verbatim}
function f=fexam(t,x)
f = x + t;
\end{verbatim}
we superimpose the approximate solution using {\tt ode45} by
\begin{verbatim}
[t,y] = ode45('fexam',[1 13],2);
plot(t,y)
\end{verbatim}
This result is also very accurate.



\subsection*{Section~\protect{\ref{sec:EEEM}} Error Bounds for Euler's Method}
\rhead{sec:EEEM}{ERROR BOUNDS FOR EULER'S METHOD}

\exer{c15.2.1a} \ans Local discretization error:
\[
\delta(k+1) \le
e^{3T}\frac{(3h)^2}{2} \quad \mbox{for all $h\ge 0$.}
\]
Global discretization error:
\[
|\epsilon(k)| \le e^{3T}(e^{3T}-1)\frac{3h}{2}.
\]

\soln Euler's method applied to the initial value problem takes the form
\[
\begin{array}{rclcl}
t_{k+1} & = & t_k+h & & \\
x_{k+1} & = & x_k + h 3 x_k & = & (1+3h)x_k
\end{array}
\]
where $k=0,1,\ldots,K-1$.

{\em Local error:} we assume that the
numerical solution is exact up to step $k$, that is,
we start in $x(t_k)=e^{3t_k}$.  Then the local discretization error
$\delta(k+1)$ is given by
\[
\delta(k+1) = x(t_{k+1}) - (x(t_k) + h 3x(t_k))=
e^{3t_{k+1}} - (1+3h)e^{t_k}.
\]
Since $t_k = kh$ we obtain
\[
\delta(k+1) = e^{3t_{k+1}} - (1+3h)e^{t_k} =
e^{3(k+1)h} - (1+3h)e^{3kh} = e^{3kh}(e^{3h}-(1+3h)).
\]
By definition of the exponential function
\[
e^{3h}-(1+3h) = \frac{(3h)^2}{2}+\frac{(3h)^3}{6}+\cdots =
\frac{(3h)^2}{2}\left[ 1+\frac{(3h)}{3}+\frac{(3h)^2}{12}+\cdots\right]
\le \frac{(3h)^2}{2}e^{3h}.
\]
Since $(k+1)h=(k+1)/K\le T$ for $k=0,1,\ldots,K-1$ we finally have the
desired bound on the local discretization error:
\[
\delta(k+1) = e^{3kh}(e^{3h}-(1+3h)) \le
e^{3(k+1)h}\left(\frac{(3h)^2}{2}\right)\le
e^{3T}\frac{(3h)^2}{2} \quad \mbox{for all $h\ge 0$.}
\]

{\em Global error:} we have for $k=1,2,\ldots,K$
\[
x(t_k)=(1+3h)x(t_{k-1})+\delta(k),
\]
and subtracting
\[
x_k = (1+3h)x_{k-1}
\]
leads to
\[
x(t_k) - x_k = (1+3h)(x(t_{k-1})-x_{k-1})+\delta(k).
\]
Therefore
\begin{eqnarray*}
|\epsilon(k)| & = & |x(t_k) - x_k| =
|(1+3h)(x(t_{k-1})-x_{k-1})+\delta(k)|\\
& \le & (1+3h)|\epsilon(k-1)|+\delta_h
\end{eqnarray*}
since
\[
\delta(k)\le \delta_h = e^{3T}\frac{(3h)^2}{2}.
\]
Apply this formula repeatedly until $k$ is zero,
\[
\begin{array}{rcl}
|\epsilon(k)|&\le&(1+3h)|\epsilon(k-1)|+\delta_h\\
&\le& (1+3h)[(1+3h)|\epsilon(k-2)|+\delta_h]+\delta_h\\
&=& (1+3h)^2|\epsilon(k-2)| + ((1+3h) + 1)\delta_h\\
&\le& (1+3h)^2[(1+3h)|\epsilon(k-3)|+\delta_h] + ((1+3h) + 1)\delta_h\\
&=& (1+3h)^3|\epsilon(k-3)| + ((1+3h)^2 + (1+3h) + 1)\delta_h\\
&\vdots& \\
&\le & (1+3h)^k|\epsilon(0)| + ((1+3h)^{k-1} +\cdots + (1+3h) + 1)\delta_h.
\end{array}
\]
Since $\epsilon(0)=x(t_0) - x_0=0$
\[
|\epsilon(k)| \le \frac{(1+3h)^k -1}{3h}\delta_h.
\]
With $1+3h\le e^{3h}$ we have the bound on the global discretization error:
\[
|\epsilon(k)| \le \frac{1}{3h} (e^{3kh}-1)\delta_h=
\frac{1}{3h}(e^{3kh}-1)e^{3T}\frac{(3h)^2}{2} =
e^{3T}(e^{3kh}-1)\frac{3h}{2}
\le e^{3T}(e^{3T}-1)\frac{3h}{2}.
\]

\exer{c15.2.1c} \ans Local discretization error:
\[
\delta(k+1) \le
2e^{3T}\frac{(3h)^2}{2} \quad \mbox{for all $h\ge 0$.}
\]
Global discretization error:
\[
|\epsilon(k)| \le 2e^{3T}(e^{3T}-1)\frac{3h}{2}.
\]

\soln Euler's method applied to the initial value problem takes the form
\[
\begin{array}{rclcl}
t_{k+1} & = & t_k+h & & \\
x_{k+1} & = & x_k + h 3x_k & = & (1+3h)x_k
\end{array}
\]
where $k=0,1,\ldots,K-1$.

{\em Local error:} we assume that the
numerical solution is exact up to step $k$, that is,
we start in $x(t_k)=2e^{3t_k}$.  Then the local discretization error
$\delta(k+1)$ is given by
\[
\delta(k+1) = x(t_{k+1}) - (x(t_k) + h 3 x(t_k))=
2e^{3t_{k+1}} - (1+3h)2e^{3t_k}.
\]
Since $t_k = kh$ we obtain
\[
\delta(k+1) = 2e^{3t_{k+1}} - (1+3h)2e^{3t_k} =
2e^{3(k+1)h} - (1+3h)2e^{3kh} = 2e^{3kh}(e^{3h}-(1+3h)).
\]
By definition of the exponential function
\[
e^{3h}-(1+3h) = \frac{(3h)^2}{2}+\frac{(3h)^3}{6}+\cdots =
\frac{(3h)^2}{2}\left[ 1+\frac{(3h)}{3}+\frac{(3h)^2}{12}+\cdots\right]
\le \frac{(3h)^2}{2}e^{3h}.
\]
Since $(k+1)h=(k+1)/K\le T$ for $k=0,1,\ldots,K-1$ we finally have the
desired bound on the local discretization error:
\[
\delta(k+1) = 2e^{3kh}(e^{3h}-(1+3h)) \le
2e^{3(k+1)h}\left(\frac{(3h)^2}{2}\right)\le
2e^{3T}\frac{(3h)^2}{2} \quad \mbox{for all $h\ge 0$.}
\]

{\em Global error:} we have for $k=1,2,\ldots,K$
\[
x(t_k)=(1+3h)x(t_{k-1})+\delta(k),
\]
and subtracting
\[
x_k = (1+3h)x_{k-1}
\]
leads to
\[
x(t_k) - x_k = (1+3h)(x(t_{k-1})-x_{k-1})+\delta(k).
\]
Therefore
\begin{eqnarray*}
|\epsilon(k)| & = & |x(t_k) - x_k| =
|(1+3h)(x(t_{k-1})-x_{k-1})+\delta(k)|\\
& \le & (1+3h)|\epsilon(k-1)|+\delta_h
\end{eqnarray*}
since
\[
\delta(k)\le \delta_h = 2e^{3T}\frac{(3h)^2}{2}.
\]
Apply this formula repeatedly until $k$ is zero,
\[
\begin{array}{rcl}
|\epsilon(k)|&\le&(1+3h)|\epsilon(k-1)|+\delta_h\\
&\le& (1+3h)[(1+3h)|\epsilon(k-2)|+\delta_h]+\delta_h\\
&=& (1+3h)^2|\epsilon(k-2)| + ((1+3h) + 1)\delta_h\\
&\le& (1+3h)^2[(1+3h)|\epsilon(k-3)|+\delta_h] + ((1+3h) + 1)\delta_h\\
&=& (1+3h)^3|\epsilon(k-3)| + ((1+3h)^2 + (1+3h) + 1)\delta_h\\
&\vdots& \\
&\le & (1+3h)^k|\epsilon(0)| + ((1+3h)^{k-1} +\cdots + (1+3h) + 1)\delta_h.
\end{array}
\]
Since $\epsilon(0)=x(t_0) - x_0=0$
\[
|\epsilon(k)| \le \frac{(1+3h)^k -1}{3h}\delta_h.
\]
With $1+3h\le e^{3h}$ we have the bound on the global discretization error:
\[
|\epsilon(k)| \le \frac{1}{3h} (e^{3kh}-1)\delta_h=
\frac{1}{3h}(e^{3kh}-1)2e^{3T}\frac{(3h)^2}{2} =
2e^{3T}(e^{3kh}-1)\frac{3h}{2}
\le 2e^{3T}(e^{3T}-1)\frac{3h}{2}.
\]

\exer{c15.2.3a} \ans $h=0.0014$.

\soln Using the bound on the global discretization error
given in Proposition~\ref{prop:globerr1} we can choose a
step length $h$ satisfying
\[
e^4(e^4-1)\frac{h}{2} = 2,
\]
or, equivalently,
\[
h = \frac{4}{e^4(e^4-1)} \approx 0.0014.
\]
Now $4/h \approx 2927$, and we can verify our result using the
following \Matlab commands:
\begin{verbatim}
h      = 0.0014;
t(1)   = 0;
x(1)   = 1;
err(1) = 0;
est(1) = 0;
K      = 2927;
for k = 1:K
    t(k+1) = t(k)+h;
    x(k+1) = (1+h)*x(k);
  err(k+1) = exp(t(k+1))-x(k+1);
end
plot(t,err)
xlabel('t')
ylabel('global error')
\end{verbatim}
Indeed, by this choice of $h$ the error is always smaller than $0.18$.

\exer{c15.2.4} \ans $T=1.609$.

\soln Using the bound on the global discretization error
given in Proposition~\ref{prop:globerr1} we can choose
$T$ satisfying
\[
e^{T}(e^{T}-1)\frac{0.005}{2} = 0.05,
\]
or, equivalently,
\[
e^{2T} - e^T - 20 = 0.
\]
Solving this quadratic equation in $e^T$ leads to the positive
solution
\[
e^T = 5 \Rightarrow T \approx 1.609.
\]
Now $T/h \approx 321$, and we can verify our result using the
following \Matlab commands:
\begin{verbatim}
h      = 0.005;
t(1)   = 0;
x(1)   = 1;
err(1) = 0;
est(1) = 0;
K      = 321;
for k = 1:K
    t(k+1) = t(k)+h;
    x(k+1) = (1+h)*x(k);
  err(k+1) = exp(t(k+1))-x(k+1);
end
plot(t,err)
xlabel('t')
ylabel('global error')
\end{verbatim}
Indeed, by this choice of $T$ the error is smaller than $0.02$.



\subsection*{Section~\protect{\ref{sec:LGEE}} Local and Global Error Bounds}
\rhead{sec:LGEE}{LOCAL AND GLOBAL ERROR BOUNDS}

\exer{c15.3.1} \ans
\begin{eqnarray*}
\Phi(t_k,x_k,h) & = & \frac{1}{6}\Big[f(t_k,x_k)+\\
&&2f\left(t_k+\frac{h}{2},x_k+\frac{h}{2}f(t_k,x_k)\right) +\\
&&2f\left(t_k+\frac{h}{2},x_k+
  \frac{h}{2}f\left(t_k+\frac{h}{2},x_k+\frac{h}{2}f(t_k,x_k)\right)\right)+\\
&&\left.f\left(t_k+h,x_k+hf\left(t_k+\frac{h}{2},x_k+
  \frac{h}{2}f\left(t_k+\frac{h}{2},x_k+
  \frac{h}{2}f(t_k,x_k)\right)\right)\right)\right]
\end{eqnarray*}

\soln The fourth order Runge-Kutta method is given by
\[
x_{k+1} = x_k+\frac{h}{6}(f_1+2f_2+2f_3+f_4),
\]
where
\begin{eqnarray*}
f_1 &=& f(t_k,x_k)\\
f_2 &=& f\left(t_k+\frac{h}{2},x_k+\frac{h}{2}f_1\right)\\
f_3 &=& f\left(t_k+\frac{h}{2},x_k+\frac{h}{2}f_2\right)\\
f_4 &=& f(t_k+h,x_k+hf_3).
\end{eqnarray*}
The result follows by inserting these expressions for $f_1,\ldots,f_4$.

\newpage
\exer{c15.3.2b} \ans $C_E \le \frac{1}{2}x(0)=\frac{1}{2}$.

\soln We want to use \Ref{eq:CE} and compute
\[
\frac{\partial f}{\partial t}(t,x)=0\AND
\frac{\partial f}{\partial x}(t,x)=-1.
\]
Since the solution of the initial value problem is monotonically
decreasing while staying positive we find that
\[
C_E\le \frac{1}{2}\max\left\{ \left\vert 0+y\right\vert,
0<y\le x(0) \right\} = \frac{1}{2}x(0)=\frac{1}{2}.
\]
