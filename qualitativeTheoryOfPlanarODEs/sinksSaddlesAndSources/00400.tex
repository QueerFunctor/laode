\documentclass{ximera}
 

\usepackage{epsfig}

\graphicspath{
  {./}
  {figures/}
}

\usepackage{morewrites}
\makeatletter
\newcommand\subfile[1]{%
\renewcommand{\input}[1]{}%
\begingroup\skip@preamble\otherinput{#1}\endgroup\par\vspace{\topsep}
\let\input\otherinput}
\makeatother

\newcommand{\includeexercises}{\directlua{dofile("/home/jim/linearAlgebra/laode/exercises.lua")}}

%\newcounter{ccounter}
%\setcounter{ccounter}{1}
%\newcommand{\Chapter}[1]{\setcounter{chapter}{\arabic{ccounter}}\chapter{#1}\addtocounter{ccounter}{1}}

%\newcommand{\section}[1]{\section{#1}\setcounter{thm}{0}\setcounter{equation}{0}}

%\renewcommand{\theequation}{\arabic{chapter}.\arabic{section}.\arabic{equation}}
%\renewcommand{\thefigure}{\arabic{chapter}.\arabic{figure}}
%\renewcommand{\thetable}{\arabic{chapter}.\arabic{table}}

%\newcommand{\Sec}[2]{\section{#1}\markright{\arabic{ccounter}.\arabic{section}.#2}\setcounter{equation}{0}\setcounter{thm}{0}\setcounter{figure}{0}}

\newcommand{\Sec}[2]{\section{#1}}

\setcounter{secnumdepth}{2}
%\setcounter{secnumdepth}{1} 

%\newcounter{THM}
%\renewcommand{\theTHM}{\arabic{chapter}.\arabic{section}}

\newcommand{\trademark}{{R\!\!\!\!\!\bigcirc}}
%\newtheorem{exercise}{}

\newcommand{\dfield}{{\sf dfield9}}
\newcommand{\pplane}{{\sf pplane9}}

\newcommand{\EXER}{\section*{Exercises}}%\vspace*{0.2in}\hrule\small\setcounter{exercise}{0}}
\newcommand{\CEXER}{}%\vspace{0.08in}\begin{center}Computer Exercises\end{center}}
\newcommand{\TEXER}{} %\vspace{0.08in}\begin{center}Hand Exercises\end{center}}
\newcommand{\AEXER}{} %\vspace{0.08in}\begin{center}Hand Exercises\end{center}}

% BADBAD: \newcommand{\Bbb}{\bf}

\newcommand{\R}{\mbox{$\Bbb{R}$}}
\newcommand{\C}{\mbox{$\Bbb{C}$}}
\newcommand{\Z}{\mbox{$\Bbb{Z}$}}
\newcommand{\N}{\mbox{$\Bbb{N}$}}
\newcommand{\D}{\mbox{{\bf D}}}
\usepackage{amssymb}
%\newcommand{\qed}{\hfill\mbox{\raggedright$\square$} \vspace{1ex}}
%\newcommand{\proof}{\noindent {\bf Proof:} \hspace{0.1in}}

\newcommand{\setmin}{\;\mbox{--}\;}
\newcommand{\Matlab}{{M\small{AT\-LAB}} }
\newcommand{\Matlabp}{{M\small{AT\-LAB}}}
\newcommand{\computer}{\Matlab Instructions}
\newcommand{\half}{\mbox{$\frac{1}{2}$}}
\newcommand{\compose}{\raisebox{.15ex}{\mbox{{\scriptsize$\circ$}}}}
\newcommand{\AND}{\quad\mbox{and}\quad}
\newcommand{\vect}[2]{\left(\begin{array}{c} #1_1 \\ \vdots \\
 #1_{#2}\end{array}\right)}
\newcommand{\mattwo}[4]{\left(\begin{array}{rr} #1 & #2\\ #3
&#4\end{array}\right)}
\newcommand{\mattwoc}[4]{\left(\begin{array}{cc} #1 & #2\\ #3
&#4\end{array}\right)}
\newcommand{\vectwo}[2]{\left(\begin{array}{r} #1 \\ #2\end{array}\right)}
\newcommand{\vectwoc}[2]{\left(\begin{array}{c} #1 \\ #2\end{array}\right)}

\newcommand{\ignore}[1]{}


\newcommand{\inv}{^{-1}}
\newcommand{\CC}{{\cal C}}
\newcommand{\CCone}{\CC^1}
\newcommand{\Span}{{\rm span}}
\newcommand{\rank}{{\rm rank}}
\newcommand{\trace}{{\rm tr}}
\newcommand{\RE}{{\rm Re}}
\newcommand{\IM}{{\rm Im}}
\newcommand{\nulls}{{\rm null\;space}}

\newcommand{\dps}{\displaystyle}
\newcommand{\arraystart}{\renewcommand{\arraystretch}{1.8}}
\newcommand{\arrayfinish}{\renewcommand{\arraystretch}{1.2}}
\newcommand{\Start}[1]{\vspace{0.08in}\noindent {\bf Section~\ref{#1}}}
\newcommand{\exer}[1]{\noindent {\bf \ref{#1}}}
\newcommand{\ans}{}
\newcommand{\matthree}[9]{\left(\begin{array}{rrr} #1 & #2 & #3 \\ #4 & #5 & #6
\\ #7 & #8 & #9\end{array}\right)}
\newcommand{\cvectwo}[2]{\left(\begin{array}{c} #1 \\ #2\end{array}\right)}
\newcommand{\cmatthree}[9]{\left(\begin{array}{ccc} #1 & #2 & #3 \\ #4 & #5 &
#6 \\ #7 & #8 & #9\end{array}\right)}
\newcommand{\vecthree}[3]{\left(\begin{array}{r} #1 \\ #2 \\
#3\end{array}\right)}
\newcommand{\cvecthree}[3]{\left(\begin{array}{c} #1 \\ #2 \\
#3\end{array}\right)}
\newcommand{\cmattwo}[4]{\left(\begin{array}{cc} #1 & #2\\ #3
&#4\end{array}\right)}

\newcommand{\Matrix}[1]{\ensuremath{\left(\begin{array}{rrrrrrrrrrrrrrrrrr} #1 \end{array}\right)}}

\newcommand{\Matrixc}[1]{\ensuremath{\left(\begin{array}{cccccccccccc} #1 \end{array}\right)}}



\renewcommand{\labelenumi}{\theenumi)}
\newenvironment{enumeratea}%
{\begingroup
 \renewcommand{\theenumi}{\alph{enumi}}
 \renewcommand{\labelenumi}{(\theenumi)}
 \begin{enumerate}}
 {\end{enumerate}\endgroup}



\newcounter{help}
\renewcommand{\thehelp}{\thesection.\arabic{equation}}

%\newenvironment{equation*}%
%{\renewcommand\endequation{\eqno (\theequation)* $$}%
%   \begin{equation}}%
%   {\end{equation}\renewcommand\endequation{\eqno \@eqnnum
%$$\global\@ignoretrue}}

%\input{psfig.tex}

\author{Martin Golubitsky and Michael Dellnitz}

%\newenvironment{matlabEquation}%
%{\renewcommand\endequation{\eqno (\theequation*) $$}%
%   \begin{equation}}%
%   {\end{equation}\renewcommand\endequation{\eqno \@eqnnum
% $$\global\@ignoretrue}}

\newcommand{\soln}{\textbf{Solution:} }
\newcommand{\exercap}[1]{\centerline{Figure~\ref{#1}}}
\newcommand{\exercaptwo}[1]{\centerline{Figure~\ref{#1}a\hspace{2.1in}
Figure~\ref{#1}b}}
\newcommand{\exercapthree}[1]{\centerline{Figure~\ref{#1}a\hspace{1.2in}
Figure~\ref{#1}b\hspace{1.2in}Figure~\ref{#1}c}}
\newcommand{\para}{\hspace{0.4in}}

\renewenvironment{solution}{\suppress}{\endsuppress}

\ifxake
\newenvironment{matlabEquation}{\begin{equation}}{\end{equation}}
\else
\newenvironment{matlabEquation}%
{\let\oldtheequation\theequation\renewcommand{\theequation}{\oldtheequation*}\begin{equation}}%
  {\end{equation}\let\theequation\oldtheequation}
\fi

\makeatother

\begin{document}


\noindent In Exercises~\ref{E:sima} -- \ref{E:simb} the given matrices $B$
and $C$ are similar.  Observe that the phase portraits of the systems
$\dot{X}=BX$ and $\dot{X}=CX$ are qualitatively the same in two steps.
\begin{itemize}
\item[(a)]  Use \Matlab to find the $2\times 2$ matrix $P$ such that
$B=P\inv CP$.  Use {\tt map} to understand how the matrix $P$ moves points 
in the plane.
\item[(b)]  Use {\pplane} to observe that $P$ moves solutions of
$\dot{X}=BX$ to the solution of $\dot{X}=CX$.  Write a sentence or two describing your results.
\end{itemize}
\begin{computerExercise} \label{E:sima}
$C = \mattwo{2}{3}{-1}{-3} \AND B = \frac{1}{2}\mattwo{1}{-1}{-9}{-3}$.

\begin{solution}

(a) \ans
\[
P = \frac{\sqrt{2}}{2}\mattwo{1}{-1}{1}{1}.
\]
The matrix $P$ rotates vectors in the plane by $45^\circ$ counterclockwise.

\soln Enter matrices $B$ and $C$ into \Matlabp.  Then type
\begin{verbatim}
[Q,D] = eig(C);
\end{verbatim}
Since $C$ has distinct eigenvalues (you can check this using
\Matlabp), the matrix $D$ is a diagonal matrix with the eigenvalues of
$C$ along its diagonal.  This matrix is similar to $C$.  Indeed, $D =
Q^{-1}CQ$.  The matrix $D$ is also similar to $B$, and $D= R^{-1}BR$.
Find the matrix $R$ by typing
\begin{verbatim}
[R,D] = eig(B);
\end{verbatim}
We know that $D = R^{-1}BR$ and $D = Q^{-1}CQ$ for the same diagonal
matrix $D$.  Therefore,
\[
B = RDR^{-1} = R(Q^{-1}CQ)R^{-1} = (QR^{-1})^{-1}C(QR).
\]
Thus, $P = QR^{-1}$, and typing
\begin{verbatim}
P = Q*inv(R)
\end{verbatim}
in \Matlab yields $P$.

(b) \ans The solutions of $\dot{X} = CX$ are found by rotating the
solutions of $\dot{X} = BX$ by $45^\circ$ counterclockwise.

\soln Enter the system $\dot{X} = BX$ into {\tt pplane5}.  Then enter the
system $\dot{X} = CX$.  Note that both systems are saddles.  You can
plot the stable and unstable trajectories at the origin in each system
to see that the trajectories for $\dot{X} = CX$ appear to be about
$45^\circ$ counterclockwise of those for $\dot{X} = BX$.  Thus, if
$X(t)$ is a solution to $\dot{X} = BX$, then $PX(t)$ is a solution to
$\dot{X} = CX$, verifying Lemma~\ref{L:simsoln}.

\end{solution}

\end{computerExercise}

\begin{computerExercise} \label{E:simb}
$C = \mattwo{-1}{5}{-5}{-1} \AND B = \mattwo{-1}{0.5}{-50}{-1}$.

\begin{solution}

(a) \ans
\[
P \approx \mattwo{7.1063}{0}{0}{0.7106}.
\]
The matrix $P$ stretches the $x$-coordinate of a vector and shrinks the
$y$-coordinate.

\soln Enter matrices $B$ and $C$ into \Matlabp.  Then type
\begin{verbatim}
[Q,D] = eig(C);
\end{verbatim}
Since $C$ has distinct eigenvalues (you can check this using
\Matlabp), the matrix $D$ is a diagonal matrix with the eigenvalues of
$C$ along its diagonal.  This matrix is similar to $C$.  Indeed, $D =
Q^{-1}CQ$.  The matrix $D$ is also similar to $B$, and $D= R^{-1}BR$.
Find the matrix $R$ by typing
\begin{verbatim}
[R,D] = eig(B);
\end{verbatim}
We know that $D = R^{-1}BR$ and $D = Q^{-1}CQ$ for the same diagonal
matrix $D$.  Therefore,
\[
B = RDR^{-1} = R(Q^{-1}CQ)R^{-1} = (QR^{-1})^{-1}C(QR).
\]
Thus, $P = QR^{-1}$, and typing
\begin{verbatim}
P = Q*inv(R)
\end{verbatim}
in \Matlab yields $P$.

(b) \ans The solutions of $\dot{X} = CX$ are obtained from the solutions
of $\dot{X} = BX$ by stretching the $x$-coordinate by a factor of $7.1063$
and the $y$-coordinate by a factor of $0.7106$.

\soln Enter the system $\dot{X} = BX$ into {\tt pplane5}.  Then enter the
system $\dot{X} = CX$.  Note that both systems are spirals.  You can
plot trajectories in each system to see that the trajectories for
$\dot{X} = CX$ appear to be similar to those for $\dot{X} = BX$, but
stretched in one direction and contracted in the other.  Thus, if
$X(t)$ is a solution to $\dot{X} = BX$, then $PX(t)$ is a solution to
$\dot{X} = CX$, verifying Lemma~\ref{L:simsoln}.

\end{solution}
\end{computerExercise}
\end{document}
