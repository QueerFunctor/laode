\documentclass{ximera}

 

\usepackage{epsfig}

\graphicspath{
  {./}
  {figures/}
}

\usepackage{morewrites}
\makeatletter
\newcommand\subfile[1]{%
\renewcommand{\input}[1]{}%
\begingroup\skip@preamble\otherinput{#1}\endgroup\par\vspace{\topsep}
\let\input\otherinput}
\makeatother

\newcommand{\includeexercises}{\directlua{dofile("/home/jim/linearAlgebra/laode/exercises.lua")}}

%\newcounter{ccounter}
%\setcounter{ccounter}{1}
%\newcommand{\Chapter}[1]{\setcounter{chapter}{\arabic{ccounter}}\chapter{#1}\addtocounter{ccounter}{1}}

%\newcommand{\section}[1]{\section{#1}\setcounter{thm}{0}\setcounter{equation}{0}}

%\renewcommand{\theequation}{\arabic{chapter}.\arabic{section}.\arabic{equation}}
%\renewcommand{\thefigure}{\arabic{chapter}.\arabic{figure}}
%\renewcommand{\thetable}{\arabic{chapter}.\arabic{table}}

%\newcommand{\Sec}[2]{\section{#1}\markright{\arabic{ccounter}.\arabic{section}.#2}\setcounter{equation}{0}\setcounter{thm}{0}\setcounter{figure}{0}}

\newcommand{\Sec}[2]{\section{#1}}

\setcounter{secnumdepth}{2}
%\setcounter{secnumdepth}{1} 

%\newcounter{THM}
%\renewcommand{\theTHM}{\arabic{chapter}.\arabic{section}}

\newcommand{\trademark}{{R\!\!\!\!\!\bigcirc}}
%\newtheorem{exercise}{}

\newcommand{\dfield}{{\sf dfield9}}
\newcommand{\pplane}{{\sf pplane9}}

\newcommand{\EXER}{\section*{Exercises}}%\vspace*{0.2in}\hrule\small\setcounter{exercise}{0}}
\newcommand{\CEXER}{}%\vspace{0.08in}\begin{center}Computer Exercises\end{center}}
\newcommand{\TEXER}{} %\vspace{0.08in}\begin{center}Hand Exercises\end{center}}
\newcommand{\AEXER}{} %\vspace{0.08in}\begin{center}Hand Exercises\end{center}}

% BADBAD: \newcommand{\Bbb}{\bf}

\newcommand{\R}{\mbox{$\Bbb{R}$}}
\newcommand{\C}{\mbox{$\Bbb{C}$}}
\newcommand{\Z}{\mbox{$\Bbb{Z}$}}
\newcommand{\N}{\mbox{$\Bbb{N}$}}
\newcommand{\D}{\mbox{{\bf D}}}
\usepackage{amssymb}
%\newcommand{\qed}{\hfill\mbox{\raggedright$\square$} \vspace{1ex}}
%\newcommand{\proof}{\noindent {\bf Proof:} \hspace{0.1in}}

\newcommand{\setmin}{\;\mbox{--}\;}
\newcommand{\Matlab}{{M\small{AT\-LAB}} }
\newcommand{\Matlabp}{{M\small{AT\-LAB}}}
\newcommand{\computer}{\Matlab Instructions}
\newcommand{\half}{\mbox{$\frac{1}{2}$}}
\newcommand{\compose}{\raisebox{.15ex}{\mbox{{\scriptsize$\circ$}}}}
\newcommand{\AND}{\quad\mbox{and}\quad}
\newcommand{\vect}[2]{\left(\begin{array}{c} #1_1 \\ \vdots \\
 #1_{#2}\end{array}\right)}
\newcommand{\mattwo}[4]{\left(\begin{array}{rr} #1 & #2\\ #3
&#4\end{array}\right)}
\newcommand{\mattwoc}[4]{\left(\begin{array}{cc} #1 & #2\\ #3
&#4\end{array}\right)}
\newcommand{\vectwo}[2]{\left(\begin{array}{r} #1 \\ #2\end{array}\right)}
\newcommand{\vectwoc}[2]{\left(\begin{array}{c} #1 \\ #2\end{array}\right)}

\newcommand{\ignore}[1]{}


\newcommand{\inv}{^{-1}}
\newcommand{\CC}{{\cal C}}
\newcommand{\CCone}{\CC^1}
\newcommand{\Span}{{\rm span}}
\newcommand{\rank}{{\rm rank}}
\newcommand{\trace}{{\rm tr}}
\newcommand{\RE}{{\rm Re}}
\newcommand{\IM}{{\rm Im}}
\newcommand{\nulls}{{\rm null\;space}}

\newcommand{\dps}{\displaystyle}
\newcommand{\arraystart}{\renewcommand{\arraystretch}{1.8}}
\newcommand{\arrayfinish}{\renewcommand{\arraystretch}{1.2}}
\newcommand{\Start}[1]{\vspace{0.08in}\noindent {\bf Section~\ref{#1}}}
\newcommand{\exer}[1]{\noindent {\bf \ref{#1}}}
\newcommand{\ans}{}
\newcommand{\matthree}[9]{\left(\begin{array}{rrr} #1 & #2 & #3 \\ #4 & #5 & #6
\\ #7 & #8 & #9\end{array}\right)}
\newcommand{\cvectwo}[2]{\left(\begin{array}{c} #1 \\ #2\end{array}\right)}
\newcommand{\cmatthree}[9]{\left(\begin{array}{ccc} #1 & #2 & #3 \\ #4 & #5 &
#6 \\ #7 & #8 & #9\end{array}\right)}
\newcommand{\vecthree}[3]{\left(\begin{array}{r} #1 \\ #2 \\
#3\end{array}\right)}
\newcommand{\cvecthree}[3]{\left(\begin{array}{c} #1 \\ #2 \\
#3\end{array}\right)}
\newcommand{\cmattwo}[4]{\left(\begin{array}{cc} #1 & #2\\ #3
&#4\end{array}\right)}

\newcommand{\Matrix}[1]{\ensuremath{\left(\begin{array}{rrrrrrrrrrrrrrrrrr} #1 \end{array}\right)}}

\newcommand{\Matrixc}[1]{\ensuremath{\left(\begin{array}{cccccccccccc} #1 \end{array}\right)}}



\renewcommand{\labelenumi}{\theenumi)}
\newenvironment{enumeratea}%
{\begingroup
 \renewcommand{\theenumi}{\alph{enumi}}
 \renewcommand{\labelenumi}{(\theenumi)}
 \begin{enumerate}}
 {\end{enumerate}\endgroup}



\newcounter{help}
\renewcommand{\thehelp}{\thesection.\arabic{equation}}

%\newenvironment{equation*}%
%{\renewcommand\endequation{\eqno (\theequation)* $$}%
%   \begin{equation}}%
%   {\end{equation}\renewcommand\endequation{\eqno \@eqnnum
%$$\global\@ignoretrue}}

%\input{psfig.tex}

\author{Martin Golubitsky and Michael Dellnitz}

%\newenvironment{matlabEquation}%
%{\renewcommand\endequation{\eqno (\theequation*) $$}%
%   \begin{equation}}%
%   {\end{equation}\renewcommand\endequation{\eqno \@eqnnum
% $$\global\@ignoretrue}}

\newcommand{\soln}{\textbf{Solution:} }
\newcommand{\exercap}[1]{\centerline{Figure~\ref{#1}}}
\newcommand{\exercaptwo}[1]{\centerline{Figure~\ref{#1}a\hspace{2.1in}
Figure~\ref{#1}b}}
\newcommand{\exercapthree}[1]{\centerline{Figure~\ref{#1}a\hspace{1.2in}
Figure~\ref{#1}b\hspace{1.2in}Figure~\ref{#1}c}}
\newcommand{\para}{\hspace{0.4in}}

\renewenvironment{solution}{\suppress}{\endsuppress}

\ifxake
\newenvironment{matlabEquation}{\begin{equation}}{\end{equation}}
\else
\newenvironment{matlabEquation}%
{\let\oldtheequation\theequation\renewcommand{\theequation}{\oldtheequation*}\begin{equation}}%
  {\end{equation}\let\theequation\oldtheequation}
\fi

\makeatother


\title{Orthonormal Bases}

\begin{document}
\begin{abstract}
\end{abstract}
\maketitle


\label{S:orthonormal}

In Section~\ref{S:coordinates} we discussed how to write the coordinates of
a vector in a basis.  We now show that finding coordinates of vectors in
certain bases is a very simple task --- these bases are called orthonormal
bases.

Nonzero vectors $v_1,\ldots,v_k$ in $\R^n$ are
{\em orthogonal\/}\index{orthogonal} if the
dot products\index{dot product}
\[
v_i\cdot v_j  =  0
\]
when $i\neq j$.  These vectors are
{\em orthonormal\/}\index{orthonormal} if they are
orthogonal and of unit length, that is,
\[
v_i\cdot v_i=1.
\]
The standard example of a set of orthonormal vectors in $\R^n$ is the
standard basis $e_1,\ldots,e_n$.

\begin{lemma} \label{L:orthog}
Nonzero orthogonal vectors are
linearly independent\index{linearly!independent}.
\end{lemma}

\begin{proof}  Let $v_1,\ldots,v_k$ be a set of nonzero orthogonal vectors in $\R^n$
and suppose that
\[
\alpha_1v_1 + \cdots + \alpha_kv_k = 0.
\]
To prove the lemma we must show that each $\alpha_j=0$.  Since
$v_i\cdot v_j = 0$ for $i\not= j$,
\[
\alpha_jv_j\cdot v_j = \alpha_1v_1\cdot v_j + \cdots + \alpha_kv_k\cdot v_j =
(\alpha_1v_1 + \cdots +\alpha_kv_k)\cdot v_j = 0\cdot v_j = 0.
\]
Since $v_j\cdot v_j = ||v_j||^2> 0$, it follows that $\alpha_j=0$.  \end{proof}

\begin{corollary}
A set of $n$ nonzero orthogonal vectors in $\R^n$ is a basis.
\end{corollary}

\begin{proof}  Lemma~\ref{L:orthog} implies that the $n$ vectors are linearly
independent, and Chapter~\ref{C:vectorspaces}, Corollary~\ref{C:dim=n} states
that $n$ linearly independent vectors in $\R^n$ form a basis.  \end{proof}

Next we discuss how to find coordinates of a vector in an
{\em orthonormal basis}\index{basis!orthonormal},
that is, a basis consisting of orthonormal vectors.

\begin{theorem}  \label{T:orthocoord}
Let $V\subset\R^n$ be a subspace\index{subspace} and
let $\{v_1,\ldots,v_k\}$ be an
orthonormal basis of $V$.  Let $v\in V$ be a vector.   Then
\[
v = \alpha_1v_1 + \cdots + \alpha_kv_k.
\]
where
\[
\alpha_i = v\cdot v_i.
\]
\end{theorem}

\begin{proof}  Since $\{v_1,\ldots,v_k\}$ is a basis of $V$, we can write
\[
v = \alpha_1v_1 + \cdots + \alpha_kv_k
\]
for some scalars $\alpha_j$.  It follows that
\[
v\cdot v_j = (\alpha_1v_1 + \cdots + \alpha_kv_k)\cdot v_j = \alpha_j,
\]
as claimed.   \end{proof}

\subsubsection{An Example in $\R^3$}

Let
\[
v_1 = \frac{1}{\sqrt{3}}(1,1,1), \quad v_2 = \frac{1}{\sqrt{6}}(1,-2,1)
\AND v_3 = \frac{1}{\sqrt{2}}(1,0,-1).
\]
It is a straightforward calculation to verify that these vectors have
unit length and are pairwise orthogonal.  Let $v=(1,2,3)$ be a vector
and determine the coordinates of $v$ in the basis ${\cal V}=\{v_1,v_2,v_3\}$.
Theorem~\ref{T:orthocoord} states that these coordinates are:
\[
[v]_{\cal V} = (v\cdot v_1, v\cdot v_2, v\cdot v_3)
= (2\sqrt{3},\frac{7}{\sqrt{6}},-\sqrt{2}).
\]


\subsection*{Matrices in Orthonormal Coordinates}

Next we discuss how to find the matrix associated with a linear map in an
orthonormal basis.  Let $L:\R^n\to\R^n$ be a linear map and let
${\cal V} = \{v_1,\ldots,v_n\}$ be an orthonormal basis for $\R^n$.  Then
the matrix associated to $L$ in the basis ${\cal V}$ is easy to calculate
in terms of dot product.  That matrix is:
\begin{equation}  \label{e:coordorthomat}
[L]_{\cal V} = (L(v_j)\cdot v_i).
\end{equation}
To verify this claim, recall from Definition~\ref{D:matrixincoord} of
Chapter~\ref{Chap:PlanarQ} that the $(i,j)^{th}$ entry of $[L]_{\cal V}$ is
the $i^{th}$ entry in the vector $[L(v_j)]_{\cal V}$ which is
$L(v_j)\cdot v_i$ by Theorem~\ref{T:orthocoord}.

\subsubsection{An Example in $\R^2$}

Let ${\cal V}=\{v_1,v_2\}\subset\R^2$ where
\[
v_1=\frac{1}{\sqrt{2}}\vectwo{1}{1} \AND
v_2=\frac{1}{\sqrt{2}}\vectwo{1}{-1}.
\]
The set ${\cal V}$ is an orthonormal basis of $\R^2$.  Using
\Ref{e:coordorthomat} we can find the matrix associated to the linear map
\[
L_A(x) = \mattwo{2}{1}{-1}{3}x
\]
in the basis ${\cal V}$ by straightforward calculation.  That is, compute
\[
[L]_{\cal V} =
\mattwo{Av_1\cdot v_1}{Av_2\cdot v_1}{Av_1\cdot v_2}{Av_2\cdot v_2}
=\frac{1}{2}\mattwo{5}{-3}{1}{5}.
\]

\subsection*{Remarks Concerning \Matlab}

In the next section we prove that every vector subspace of $\R^n$ has an
orthonormal basis (see Theorem~\ref{T:orthobasis}), and we present a method
for constructing such a basis (the Gram-Schmidt orthonormalization process).
Here we note that certain commands in \Matlab produce bases for vector spaces.
For those commands \Matlab always produces an orthonormal basis.  For example,
{\tt null(A)}\index{\computer!null} produces a basis for the null space
\index{null space} of $A$.  Take the $3\times 5$ matrix
\begin{equation*}
\label{eq:Anull1}
A = \left(\begin{array}{rrrrr} 1 & 2 & 3 & 4 & 5\\ 0 & 1 & 2 & 3 & 4\\
2 & 3 & 4 & 0 & 0 \end{array}\right).
\end{equation*}
Since $\rank(A)=3$, it follows that the null space of $A$ is two-dimensional.
Typing {\tt B = null(A)} in \Matlab produces
\begin{verbatim}
B =
   -0.4666         0
    0.6945    0.4313
   -0.2876   -0.3235
    0.3581   -0.6470
   -0.2984    0.5392
\end{verbatim}
The columns of $B$ form an orthonormal basis for the null space of $A$.
This assertion can be checked by first typing
\begin{verbatim}
v1 = B(:,1);
v2 = B(:,2);
\end{verbatim}
and then typing
\begin{verbatim}
norm(v1)
norm(v2)
dot(v1,v2)
A*v1
A*v2
\end{verbatim}\index{\computer!norm}
yields answers $1,1,0$, $(0,0,0)^t,(0,0,0)^t$
(to within numerical accuracy).  Recall that the \Matlab
command {\tt norm(v)} computes the norm of a vector {\tt v}.





\EXER

\TEXER

\begin{exercise} \label{c7.4.1}
Find an orthonormal basis for the solutions to the linear equation
\[
2x_1-x_2+x_3=0.
\]
\end{exercise}

\begin{exercise} \label{c7.4.2}
\begin{itemize}
\item[(a)] Find the coordinates of the vector $v=(1,4)$ in the orthonormal
basis ${\cal V}$
\[
v_1 = \frac{1}{\sqrt{5}}(1,2) \AND v_2 = \frac{1}{\sqrt{5}}(2,-1).
\]
\item[(b)]  Let $A=\mattwo{1}{1}{2}{-3}$. Find $[A]_{\cal V}$.
\end{itemize}
\end{exercise}




\CEXER

\begin{exercise} \label{c7.4.3}
Load the matrix
\[
A=\left(\begin{array}{rrr} 1 & 2 & 0\\ 0 & 1 & 0\\
0 & 0 & 0\end{array}\right)
\]
into \Matlabp.  Then type the command {\tt orth(A)}\index{\computer!orth}.
Verify that the result is an orthonormal basis for the column space of $A$.
\end{exercise}



\end{document}
