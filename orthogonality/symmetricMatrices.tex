\documentclass{ximera}

 

\usepackage{epsfig}

\graphicspath{
  {./}
  {figures/}
}

\usepackage{morewrites}
\makeatletter
\newcommand\subfile[1]{%
\renewcommand{\input}[1]{}%
\begingroup\skip@preamble\otherinput{#1}\endgroup\par\vspace{\topsep}
\let\input\otherinput}
\makeatother

\newcommand{\includeexercises}{\directlua{dofile("/home/jim/linearAlgebra/laode/exercises.lua")}}

%\newcounter{ccounter}
%\setcounter{ccounter}{1}
%\newcommand{\Chapter}[1]{\setcounter{chapter}{\arabic{ccounter}}\chapter{#1}\addtocounter{ccounter}{1}}

%\newcommand{\section}[1]{\section{#1}\setcounter{thm}{0}\setcounter{equation}{0}}

%\renewcommand{\theequation}{\arabic{chapter}.\arabic{section}.\arabic{equation}}
%\renewcommand{\thefigure}{\arabic{chapter}.\arabic{figure}}
%\renewcommand{\thetable}{\arabic{chapter}.\arabic{table}}

%\newcommand{\Sec}[2]{\section{#1}\markright{\arabic{ccounter}.\arabic{section}.#2}\setcounter{equation}{0}\setcounter{thm}{0}\setcounter{figure}{0}}

\newcommand{\Sec}[2]{\section{#1}}

\setcounter{secnumdepth}{2}
%\setcounter{secnumdepth}{1} 

%\newcounter{THM}
%\renewcommand{\theTHM}{\arabic{chapter}.\arabic{section}}

\newcommand{\trademark}{{R\!\!\!\!\!\bigcirc}}
%\newtheorem{exercise}{}

\newcommand{\dfield}{{\sf dfield9}}
\newcommand{\pplane}{{\sf pplane9}}

\newcommand{\EXER}{\section*{Exercises}}%\vspace*{0.2in}\hrule\small\setcounter{exercise}{0}}
\newcommand{\CEXER}{}%\vspace{0.08in}\begin{center}Computer Exercises\end{center}}
\newcommand{\TEXER}{} %\vspace{0.08in}\begin{center}Hand Exercises\end{center}}
\newcommand{\AEXER}{} %\vspace{0.08in}\begin{center}Hand Exercises\end{center}}

% BADBAD: \newcommand{\Bbb}{\bf}

\newcommand{\R}{\mbox{$\Bbb{R}$}}
\newcommand{\C}{\mbox{$\Bbb{C}$}}
\newcommand{\Z}{\mbox{$\Bbb{Z}$}}
\newcommand{\N}{\mbox{$\Bbb{N}$}}
\newcommand{\D}{\mbox{{\bf D}}}
\usepackage{amssymb}
%\newcommand{\qed}{\hfill\mbox{\raggedright$\square$} \vspace{1ex}}
%\newcommand{\proof}{\noindent {\bf Proof:} \hspace{0.1in}}

\newcommand{\setmin}{\;\mbox{--}\;}
\newcommand{\Matlab}{{M\small{AT\-LAB}} }
\newcommand{\Matlabp}{{M\small{AT\-LAB}}}
\newcommand{\computer}{\Matlab Instructions}
\newcommand{\half}{\mbox{$\frac{1}{2}$}}
\newcommand{\compose}{\raisebox{.15ex}{\mbox{{\scriptsize$\circ$}}}}
\newcommand{\AND}{\quad\mbox{and}\quad}
\newcommand{\vect}[2]{\left(\begin{array}{c} #1_1 \\ \vdots \\
 #1_{#2}\end{array}\right)}
\newcommand{\mattwo}[4]{\left(\begin{array}{rr} #1 & #2\\ #3
&#4\end{array}\right)}
\newcommand{\mattwoc}[4]{\left(\begin{array}{cc} #1 & #2\\ #3
&#4\end{array}\right)}
\newcommand{\vectwo}[2]{\left(\begin{array}{r} #1 \\ #2\end{array}\right)}
\newcommand{\vectwoc}[2]{\left(\begin{array}{c} #1 \\ #2\end{array}\right)}

\newcommand{\ignore}[1]{}


\newcommand{\inv}{^{-1}}
\newcommand{\CC}{{\cal C}}
\newcommand{\CCone}{\CC^1}
\newcommand{\Span}{{\rm span}}
\newcommand{\rank}{{\rm rank}}
\newcommand{\trace}{{\rm tr}}
\newcommand{\RE}{{\rm Re}}
\newcommand{\IM}{{\rm Im}}
\newcommand{\nulls}{{\rm null\;space}}

\newcommand{\dps}{\displaystyle}
\newcommand{\arraystart}{\renewcommand{\arraystretch}{1.8}}
\newcommand{\arrayfinish}{\renewcommand{\arraystretch}{1.2}}
\newcommand{\Start}[1]{\vspace{0.08in}\noindent {\bf Section~\ref{#1}}}
\newcommand{\exer}[1]{\noindent {\bf \ref{#1}}}
\newcommand{\ans}{}
\newcommand{\matthree}[9]{\left(\begin{array}{rrr} #1 & #2 & #3 \\ #4 & #5 & #6
\\ #7 & #8 & #9\end{array}\right)}
\newcommand{\cvectwo}[2]{\left(\begin{array}{c} #1 \\ #2\end{array}\right)}
\newcommand{\cmatthree}[9]{\left(\begin{array}{ccc} #1 & #2 & #3 \\ #4 & #5 &
#6 \\ #7 & #8 & #9\end{array}\right)}
\newcommand{\vecthree}[3]{\left(\begin{array}{r} #1 \\ #2 \\
#3\end{array}\right)}
\newcommand{\cvecthree}[3]{\left(\begin{array}{c} #1 \\ #2 \\
#3\end{array}\right)}
\newcommand{\cmattwo}[4]{\left(\begin{array}{cc} #1 & #2\\ #3
&#4\end{array}\right)}

\newcommand{\Matrix}[1]{\ensuremath{\left(\begin{array}{rrrrrrrrrrrrrrrrrr} #1 \end{array}\right)}}

\newcommand{\Matrixc}[1]{\ensuremath{\left(\begin{array}{cccccccccccc} #1 \end{array}\right)}}



\renewcommand{\labelenumi}{\theenumi)}
\newenvironment{enumeratea}%
{\begingroup
 \renewcommand{\theenumi}{\alph{enumi}}
 \renewcommand{\labelenumi}{(\theenumi)}
 \begin{enumerate}}
 {\end{enumerate}\endgroup}



\newcounter{help}
\renewcommand{\thehelp}{\thesection.\arabic{equation}}

%\newenvironment{equation*}%
%{\renewcommand\endequation{\eqno (\theequation)* $$}%
%   \begin{equation}}%
%   {\end{equation}\renewcommand\endequation{\eqno \@eqnnum
%$$\global\@ignoretrue}}

%\input{psfig.tex}

\author{Martin Golubitsky and Michael Dellnitz}

%\newenvironment{matlabEquation}%
%{\renewcommand\endequation{\eqno (\theequation*) $$}%
%   \begin{equation}}%
%   {\end{equation}\renewcommand\endequation{\eqno \@eqnnum
% $$\global\@ignoretrue}}

\newcommand{\soln}{\textbf{Solution:} }
\newcommand{\exercap}[1]{\centerline{Figure~\ref{#1}}}
\newcommand{\exercaptwo}[1]{\centerline{Figure~\ref{#1}a\hspace{2.1in}
Figure~\ref{#1}b}}
\newcommand{\exercapthree}[1]{\centerline{Figure~\ref{#1}a\hspace{1.2in}
Figure~\ref{#1}b\hspace{1.2in}Figure~\ref{#1}c}}
\newcommand{\para}{\hspace{0.4in}}

\renewenvironment{solution}{\suppress}{\endsuppress}

\ifxake
\newenvironment{matlabEquation}{\begin{equation}}{\end{equation}}
\else
\newenvironment{matlabEquation}%
{\let\oldtheequation\theequation\renewcommand{\theequation}{\oldtheequation*}\begin{equation}}%
  {\end{equation}\let\theequation\oldtheequation}
\fi

\makeatother


\title{Symmetric and Orthogonal Matrices}

\begin{document}
\begin{abstract}
\end{abstract}
\maketitle


\label{S:symmetric}

Symmetric matrices have some remarkable properties that can be
summarized by:
\begin{theorem}  \label{T:symmetricmat}
Let $A$ be an $n\times n$ symmetric matrix\index{matrix!symmetric}.
Then
\begin{enumerate}
\item[(a)] every eigenvalue\index{eigenvalue!of symmetric matrix}
of $A$ is real, and
\item[(b)] there is an orthonormal basis\index{basis!orthonormal}
of $\R^n$ consisting of
	eigenvectors of $A$.
\end{enumerate}
\end{theorem}

\subsubsection*{Hermitian Inner Products}

The proof of Theorem~\ref{T:symmetricmat} uses the {\em Hermitian inner
product}\index{Hermitian inner product} --- a generalization of
dot product\index{dot product} to complex vectors\index{vector!complex}.
Let $v,w\in\C^n$ be two complex $n$-vectors.  Define
\[
\langle v,w \rangle = v_1\overline{w}_1 + \cdots + v_n\overline{w}_n.
\]
Note that the coordinates $w_i$ of the second vector enter this formula
with a complex conjugate.  However, if $v$ and $w$ are real vectors, then
\[
\langle v,w \rangle = v\cdot w.
\]
A more compact notation for the Hermitian inner product is given by
matrix multiplication\index{matrix!multiplication}.
Suppose that $v$ and $w$ are column $n$-vectors.
Then
\[
\langle v,w \rangle = v^t\overline{w}.
\]

The properties of the Hermitian inner product are similar to those of dot
product.  We note three.  Let $c\in\C$ be a complex scalar.  Then
\begin{eqnarray*}
\langle v,v \rangle & = & ||v||^2\ge 0\\
\langle cv,w \rangle & = & c\langle v,w \rangle \\
\langle v,cw \rangle & = & \overline{c} \langle v,w \rangle
\end{eqnarray*}
Note the complex conjugation of the complex scalar $c$ in the previous
formula.

Let $C$ be a complex $n\times n$ matrix.  Then the main observation
concerning Hermitian inner products that we shall use is:
\[
\langle Cv,w \rangle = \langle v,\overline{C}^tw \rangle.
\]
This fact is verified by calculating
\[
\langle Cv,w \rangle = (Cv)^t\overline{w} = (v^tC^t)\overline{w}
= v^t(C^t\overline{w}) = v^t(\overline{\overline{C}^tw})
= \langle v,\overline{C}^tw \rangle.
\]
So if $A$ is a $n\times n$ real symmetric matrix, then
\begin{equation}   \label{e:symminv}
\langle Av,w \rangle = \langle v,Aw \rangle,
\end{equation}
since $\overline{A}^t= A^t = A$.

\begin{proof}[Proof of Theorem~\ref{T:symmetricmat}(a)]  Let $\lambda$
be an eigenvalue of $A$ and let $v$ be the associated eigenvector. Since
$Av=\lambda v$ we can use \eqref{e:symminv} to compute
\[
\lambda \langle v,v \rangle = \langle Av,v \rangle = \langle v,Av \rangle
= \overline{\lambda} \langle v,v \rangle.
\]
Since $\langle v,v \rangle = ||v||^2 > 0$, it follows that
$\lambda=\overline{\lambda}$ and $\lambda$ is real.  \end{proof}


\begin{proof}[Proof of Theorem~\ref{T:symmetricmat}(b)]
Let $A$ be a real symmetric $n\times n$ matrix.  We want to show that there
is an orthonormal basis of $\R^n$ consisting of eigenvectors of $A$.  The
proof proceeds inductively on $n$.   The theorem is trivially valid for
$n=1$; so we assume that it is valid for $n-1$.

Theorem~\ref{T:eigens} of Chapter~\ref{C:D&E} implies that $A$ has an 
eigenvalue $\lambda_1$ and Theorem~\ref{T:symmetricmat}(a) states that 
this eigenvalue is real.
Let $v_1$ be a unit length eigenvector corresponding to the eigenvalue
$\lambda_1$.  Extend $v_1$ to an orthonormal basis $v_1,w_2,\ldots,w_n$ of
$\R^n$ and let $P=(v_1|w_2|\cdots|w_n)$ be the matrix whose columns are the
vectors in this orthonormal basis.  Orthonormality and direct multiplication
implies that
\begin{equation}  \label{e:orthosym}
P^tP=I_n.
\end{equation}
Therefore $P$ is invertible; indeed $P\inv=P^t$.

Next, let
\[
B= P\inv AP.
\]
By direct computation
\[
Be_1 = P\inv APe_1 = P\inv Av_1 = \lambda_1 P\inv v_1=\lambda_1e_1.
\]
It follows that that $B$ has the form
\[
B = \mattwo{\lambda_1}{*}{0}{C}
\]
where $C$ is an $(n-1)\times (n-1)$ matrix.   Since $P\inv=P^t$, it follows
that $B$ is a symmetric matrix; to verify this point compute
\[
B^t = (P^t AP)^t = P^t A^t (P^t)^t = P^tAP = B.
\]
It follows that
\[
B =\mattwo{\lambda_1}{0}{0}{C}
\]
where $C$ is a symmetric matrix.  By induction we can choose an orthonormal
basis $z_2,\ldots,z_n$ in $\{0\}\times\R^{n-1}$ consisting of eigenvectors
of $C$.  It follows that $e_1,z_2,\ldots,z_n$ is an orthonormal basis for
$\R^n$ consisting of eigenvectors of $B$.

Finally, let $v_j=P\inv z_j$ for $j=2,\ldots,n$.  Since $v_1=P\inv e_1$,
it follows that  $v_1,v_2,\ldots,v_n$ is a basis of $\R^n$ consisting of
eigenvectors of $A$.  We need only show that the $v_j$ form an orthonormal
basis of $\R^n$.   This is done using \eqref{e:symminv}.  For notational
convenience let $z_1=e_1$ and compute
\begin{align*}
\langle v_i,v_j \rangle  &=\langle P\inv z_i,P\inv z_j\rangle =
                           \langle P^tz_i, P^tz_j \rangle \\
  &= \langle z_i, PP^t z_j \rangle =
\langle z_i,z_j \rangle,
\end{align*}
since $PP^t= I_n$.  Thus the vectors $v_j$ form an orthonormal basis since
the vectors $z_j$ form an orthonormal basis.  \end{proof}


\subsection*{Orthogonal Matrices}

\begin{definition} \label{def:orthmat}
\index{matrix!orthogonal}
An $n\times n$ matrix $Q$ is {\em orthogonal\/} if its columns form an
orthonormal basis\index{basis!orthonormal}
of $\R^n$.
\end{definition}



The following lemma states elementary properties of orthogonal matrices:
\begin{lemma} \label{lem:orthprop}
Let $Q$ be an $n\times n$ matrix.  Then
\begin{itemize}
\item[(a)] $Q$ is orthogonal if and only if $Q^tQ=I_n$;
\item[(b)] $Q$ is orthogonal if and only if $Q^{-1} = Q^t$;
\item[(c)] If $Q_1,Q_2$ are orthogonal matrices, then $Q_1Q_2$ is
an orthogonal matrix.
\end{itemize}
\end{lemma}
\begin{proof}  (a) Let $Q=(v_1|\cdots|v_n)$.  Since $Q$ is orthogonal, the $v_j$
form an orthonormal basis.  By direct computation note that
$Q^tQ=\{(v_i\cdot v_j)\}=I_n$, since the $v_j$ are orthonormal. Note that
(b) is simply a restatement of (a).

\noindent (c) Now let $Q_1,Q_2$ be orthogonal. Then (a) implies
\[
(Q_1Q_2)^t(Q_1Q_2) = Q_2^tQ_1^tQ_1Q_2 = Q_2^tQ_2 = I_n,
\]
thus proving (c).  \end{proof}

As a consequence of Theorem~\ref{T:symmetricmat}, let
${\cal V}=\{v_1,\ldots,v_n\}$ be an orthonormal basis for $\R^n$
consisting of eigenvectors of $A$.  Indeed, suppose
\[
Av_j = \lambda_jv_j
\]
where $\lambda_j\in\R$.  Note that
\[
Av_j\cdot v_i =  \left\{\begin{array}{rl} \lambda_j & \qquad i=j\\
			0 & \qquad i\neq j \end{array}\right.
\]
It follows from \eqref{e:coordorthomat} that
\[
[A]_{\cal V}= \left(\begin{array}{ccc} \lambda_1 & & 0 \\  & \ddots & \\
	0 &  & \lambda_n \end{array}\right)
\]
is a diagonal matrix.  So every symmetric matrix is similar to a diagonal
matrix.  We have in fact proved:


%The previous lemma together with \eqref{e:orthosym} in the proof of Theorem~\ref{T:symmetricmat}(b) lead to the following result:

\begin{proposition}  For each symmetric $n\times n$ matrix $A$, there exists an
orthogonal matrix $P$ such that $P^tAP$ is a diagonal matrix.
\end{proposition}





\includeexercises



\end{document}

%%% Local Variables:
%%% mode: latex
%%% TeX-master: "../linearAlgebra"
%%% End:
