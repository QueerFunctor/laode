\documentclass{ximera}

 

\usepackage{epsfig}

\graphicspath{
  {./}
  {figures/}
}

\usepackage{morewrites}
\makeatletter
\newcommand\subfile[1]{%
\renewcommand{\input}[1]{}%
\begingroup\skip@preamble\otherinput{#1}\endgroup\par\vspace{\topsep}
\let\input\otherinput}
\makeatother

\newcommand{\includeexercises}{\directlua{dofile("/home/jim/linearAlgebra/laode/exercises.lua")}}

%\newcounter{ccounter}
%\setcounter{ccounter}{1}
%\newcommand{\Chapter}[1]{\setcounter{chapter}{\arabic{ccounter}}\chapter{#1}\addtocounter{ccounter}{1}}

%\newcommand{\section}[1]{\section{#1}\setcounter{thm}{0}\setcounter{equation}{0}}

%\renewcommand{\theequation}{\arabic{chapter}.\arabic{section}.\arabic{equation}}
%\renewcommand{\thefigure}{\arabic{chapter}.\arabic{figure}}
%\renewcommand{\thetable}{\arabic{chapter}.\arabic{table}}

%\newcommand{\Sec}[2]{\section{#1}\markright{\arabic{ccounter}.\arabic{section}.#2}\setcounter{equation}{0}\setcounter{thm}{0}\setcounter{figure}{0}}

\newcommand{\Sec}[2]{\section{#1}}

\setcounter{secnumdepth}{2}
%\setcounter{secnumdepth}{1} 

%\newcounter{THM}
%\renewcommand{\theTHM}{\arabic{chapter}.\arabic{section}}

\newcommand{\trademark}{{R\!\!\!\!\!\bigcirc}}
%\newtheorem{exercise}{}

\newcommand{\dfield}{{\sf dfield9}}
\newcommand{\pplane}{{\sf pplane9}}

\newcommand{\EXER}{\section*{Exercises}}%\vspace*{0.2in}\hrule\small\setcounter{exercise}{0}}
\newcommand{\CEXER}{}%\vspace{0.08in}\begin{center}Computer Exercises\end{center}}
\newcommand{\TEXER}{} %\vspace{0.08in}\begin{center}Hand Exercises\end{center}}
\newcommand{\AEXER}{} %\vspace{0.08in}\begin{center}Hand Exercises\end{center}}

% BADBAD: \newcommand{\Bbb}{\bf}

\newcommand{\R}{\mbox{$\Bbb{R}$}}
\newcommand{\C}{\mbox{$\Bbb{C}$}}
\newcommand{\Z}{\mbox{$\Bbb{Z}$}}
\newcommand{\N}{\mbox{$\Bbb{N}$}}
\newcommand{\D}{\mbox{{\bf D}}}
\usepackage{amssymb}
%\newcommand{\qed}{\hfill\mbox{\raggedright$\square$} \vspace{1ex}}
%\newcommand{\proof}{\noindent {\bf Proof:} \hspace{0.1in}}

\newcommand{\setmin}{\;\mbox{--}\;}
\newcommand{\Matlab}{{M\small{AT\-LAB}} }
\newcommand{\Matlabp}{{M\small{AT\-LAB}}}
\newcommand{\computer}{\Matlab Instructions}
\newcommand{\half}{\mbox{$\frac{1}{2}$}}
\newcommand{\compose}{\raisebox{.15ex}{\mbox{{\scriptsize$\circ$}}}}
\newcommand{\AND}{\quad\mbox{and}\quad}
\newcommand{\vect}[2]{\left(\begin{array}{c} #1_1 \\ \vdots \\
 #1_{#2}\end{array}\right)}
\newcommand{\mattwo}[4]{\left(\begin{array}{rr} #1 & #2\\ #3
&#4\end{array}\right)}
\newcommand{\mattwoc}[4]{\left(\begin{array}{cc} #1 & #2\\ #3
&#4\end{array}\right)}
\newcommand{\vectwo}[2]{\left(\begin{array}{r} #1 \\ #2\end{array}\right)}
\newcommand{\vectwoc}[2]{\left(\begin{array}{c} #1 \\ #2\end{array}\right)}

\newcommand{\ignore}[1]{}


\newcommand{\inv}{^{-1}}
\newcommand{\CC}{{\cal C}}
\newcommand{\CCone}{\CC^1}
\newcommand{\Span}{{\rm span}}
\newcommand{\rank}{{\rm rank}}
\newcommand{\trace}{{\rm tr}}
\newcommand{\RE}{{\rm Re}}
\newcommand{\IM}{{\rm Im}}
\newcommand{\nulls}{{\rm null\;space}}

\newcommand{\dps}{\displaystyle}
\newcommand{\arraystart}{\renewcommand{\arraystretch}{1.8}}
\newcommand{\arrayfinish}{\renewcommand{\arraystretch}{1.2}}
\newcommand{\Start}[1]{\vspace{0.08in}\noindent {\bf Section~\ref{#1}}}
\newcommand{\exer}[1]{\noindent {\bf \ref{#1}}}
\newcommand{\ans}{}
\newcommand{\matthree}[9]{\left(\begin{array}{rrr} #1 & #2 & #3 \\ #4 & #5 & #6
\\ #7 & #8 & #9\end{array}\right)}
\newcommand{\cvectwo}[2]{\left(\begin{array}{c} #1 \\ #2\end{array}\right)}
\newcommand{\cmatthree}[9]{\left(\begin{array}{ccc} #1 & #2 & #3 \\ #4 & #5 &
#6 \\ #7 & #8 & #9\end{array}\right)}
\newcommand{\vecthree}[3]{\left(\begin{array}{r} #1 \\ #2 \\
#3\end{array}\right)}
\newcommand{\cvecthree}[3]{\left(\begin{array}{c} #1 \\ #2 \\
#3\end{array}\right)}
\newcommand{\cmattwo}[4]{\left(\begin{array}{cc} #1 & #2\\ #3
&#4\end{array}\right)}

\newcommand{\Matrix}[1]{\ensuremath{\left(\begin{array}{rrrrrrrrrrrrrrrrrr} #1 \end{array}\right)}}

\newcommand{\Matrixc}[1]{\ensuremath{\left(\begin{array}{cccccccccccc} #1 \end{array}\right)}}



\renewcommand{\labelenumi}{\theenumi)}
\newenvironment{enumeratea}%
{\begingroup
 \renewcommand{\theenumi}{\alph{enumi}}
 \renewcommand{\labelenumi}{(\theenumi)}
 \begin{enumerate}}
 {\end{enumerate}\endgroup}



\newcounter{help}
\renewcommand{\thehelp}{\thesection.\arabic{equation}}

%\newenvironment{equation*}%
%{\renewcommand\endequation{\eqno (\theequation)* $$}%
%   \begin{equation}}%
%   {\end{equation}\renewcommand\endequation{\eqno \@eqnnum
%$$\global\@ignoretrue}}

%\input{psfig.tex}

\author{Martin Golubitsky and Michael Dellnitz}

%\newenvironment{matlabEquation}%
%{\renewcommand\endequation{\eqno (\theequation*) $$}%
%   \begin{equation}}%
%   {\end{equation}\renewcommand\endequation{\eqno \@eqnnum
% $$\global\@ignoretrue}}

\newcommand{\soln}{\textbf{Solution:} }
\newcommand{\exercap}[1]{\centerline{Figure~\ref{#1}}}
\newcommand{\exercaptwo}[1]{\centerline{Figure~\ref{#1}a\hspace{2.1in}
Figure~\ref{#1}b}}
\newcommand{\exercapthree}[1]{\centerline{Figure~\ref{#1}a\hspace{1.2in}
Figure~\ref{#1}b\hspace{1.2in}Figure~\ref{#1}c}}
\newcommand{\para}{\hspace{0.4in}}

\renewenvironment{solution}{\suppress}{\endsuppress}

\ifxake
\newenvironment{matlabEquation}{\begin{equation}}{\end{equation}}
\else
\newenvironment{matlabEquation}%
{\let\oldtheequation\theequation\renewcommand{\theequation}{\oldtheequation*}\begin{equation}}%
  {\end{equation}\let\theequation\oldtheequation}
\fi

\makeatother


\title{Existence of Determinants}

\begin{document}
\begin{abstract}
\end{abstract}
\maketitle


\label{A:det}

The purpose of this appendix is to verify the inductive
definition of determinant \Ref{e:inductdet}. We have already
shown that if a determinant function exists, then it is unique.
We also know that the determinant function exists for $1\times
1$ matrices. So we assume by induction that the determinant function
exists for $(n-1)\times(n-1)$ matrices and prove that the
inductive definition gives a determinant function for $n\times
n$ matrices.  \index{determinant}

Recall that $A_{ij}$ is the cofactor matrix obtained from $A$ by
deleting the $i^{th}$ row and $j^{th}$ column --- so $A_{ij}$ is
an $(n-1)\times(n-1)$ matrix.  The inductive definition is:
\index{determinant!inductive formula for}
\[
D(A) = a_{11}\det(A_{11})-a_{12}\det(A_{12})+\cdots 
+(-1)^{n+1}a_{1n}\det(A_{1n}).
\] \index{cofactor}
We use the notation $D(A)$ to remind us that we have not yet
verified that this definition satisfies properties (a)-(c) of
Definition~\ref{D:determinants}.  In this appendix we verify these
properties after assuming that the inductive definition
satisfies properties (a)-(c) for $(n-1)\times (n-1)$ matrices.
For emphasis, we use the notation $\det$ to indicate the
determinant of square matrices of size less than $n$.

Property (a) is easily verified for $D(A)$ since if $A$ is lower
triangular, then
\[
D(A) = a_{11}\det(A_{11}) = a_{11}a_{22}\cdots a_{nn}
\]
by induction.

Before verifying that $D$ satisfies properties (b) and (c) of a
determinant, we prove:
\begin{lemma}
Let $E$ be a elementary row matrix and let $B$ be any $n\times
n$ matrix.   Then
\begin{equation} \label{e:proddetE}
D(EB) = D(E) D(B)
\end{equation} 
\end{lemma}

\begin{proof} We verify \Ref{e:proddetE} for each of the three types of
elementary row operations. \index{elementary row operations}

\noindent (I) Suppose that $E$ multiplies the $i^{th}$ row by a
nonzero scalar $c$.  If $i>1$, then the cofactor matrix
$(EA)_{1j}$ is obtained from the cofactor matrix $A_{1j}$ by
multiplying the $(i-1)^{st}$ row by $c$.  By induction,
$\det(EA)_{1j}= c\det(A_{1j})$ and $D(EA)=cD(A)$.  On the other
hand, $D(E)=\det(E_{11})=c$.  So \Ref{e:proddetE} is verified in
this instance.  If $i=1$, then the $1^{st}$ row of $EA$ is
$(ca_{11},\ldots,ca_{1n})$ from which it is easy to verify
\Ref{e:proddetE}.

\noindent (II) Next suppose that $E$ adds a multiple $c$ of the
$i^{th}$ row to the $j^{th}$ row.  We note that $D(E)=1$.  When
$j>1$ then $D(E)=\det(E_{11})=1$ by induction.  When $j=1$ then
$D(E)= \det(E_{11})\pm c\det(E_{1i})=\det(I_{n-1})\pm
c\det(E_{1i})$. But $E_{1i}$ is strictly upper triangular and
$\det(E_{1i})=0$.  Thus $D(E)=1$. 

If $i>1$ and $j>1$, then the result $D(EA)=D(A)=D(E)D(A)$
follows by induction.  

If $i=1$, then
\begin{eqnarray*}
D(EB) & = & b_{11}\det((EB)_{11})+\cdots+(-1)^{n+1}b_{1n}\det((EB)_{1n})\\
& = & D(B) + cD(C)
\end{eqnarray*}
where the $1^{st}$ and $i^{th}$ row of $C$ are equal.

If $j=1$, then 
\begin{eqnarray*}
D(EB) & = & (b_{11}+cb_{i1})\det(B_{11}) +\cdots+ 
(-1)^{n+1}(b_{1n}+cb_{in})\det(B_{1n})\\
& = &
\left[b_{11}\det(B_{11})+\cdots+(-1)^{n+1}b_{1n}\det(B_{1n})\right]
+ 
\\ & & 
        c\left[b_{i1}\det(B_{11})+\cdots+(-1)^{n+1}b_{i1}\det(B_{1n})\right]\\
& = & D(B) + cD(C)
\end{eqnarray*}
where the $1^{st}$ and $i^{th}$ row of $C$ are equal.  

The hardest part of this proof is a calculation that shows that
if the $1^{st}$ and $i^{th}$ rows of $C$ are equal, then
$D(C)=0$.  By induction, we can swap the $i^{th}$ row with the
$2^{nd}$.  Hence we need only verify this fact when $i=2$. 

\noindent (III) $E$ is the matrix that swaps two rows.   

As we saw earlier \Ref{e:swapdecomp}, $E$ is the product of four
matrices of types (I) and (II).  It follows that $D(E)=-1$ and
$D(EA)=-D(A)=D(E)D(A)$.  

We now verify that when the $1^{st}$ and $2^{nd}$ rows of an
$n\times n$ matrix $C$ are equal, then $D(C)=0$.  This is a
tedious calculation that requires some facility with indexes and
summations.  Rather than do this proof for general $n$, we
present the proof for $n=4$.  This case contains all of the
ideas of the general proof.  

We begin with the definition of $D(C)$
\begin{eqnarray*}
D(C) & = & c_{11}\det\left(\begin{array}{ccc} c_{22} & c_{23} & c_{24}
\\ c_{32} & c_{33} & c_{34} \\ c_{42} & c_{43} & c_{44}
\end{array}\right) 
-c_{12}\det\left(\begin{array}{ccc} c_{21} & c_{23} & c_{24}
\\ c_{31} & c_{33} & c_{34} \\ c_{41} & c_{43} & c_{44}
\end{array}\right) + \\ & &  
c_{13}\det\left(\begin{array}{ccc} c_{21} & c_{22} & c_{24}
\\ c_{31} & c_{32} & c_{34} \\ c_{41} & c_{42} & c_{44}
\end{array}\right) 
-c_{14}\det\left(\begin{array}{ccc} c_{21} & c_{22} & c_{23}
\\ c_{31} & c_{32} & c_{33} \\ c_{41} & c_{42} & c_{43}
\end{array}\right).
\end{eqnarray*}
Next we expand each of the four $3\times 3$ matrices along their
$1^{st}$ rows, obtaining
\begin{eqnarray*}
D(C) & = & 
c_{11}\left(c_{22}\det\mattwo{c_{33}}{c_{34}}{c_{43}}{c_{44}}
-c_{23}\det\mattwo{c_{32}}{c_{34}}{c_{42}}{c_{44}}
+c_{24}\det\mattwo{c_{32}}{c_{33}}{c_{42}}{c_{43}}\right)\\ & &
-c_{12}\left(c_{21}\det\mattwo{c_{33}}{c_{34}}{c_{43}}{c_{44}}
-c_{23}\det\mattwo{c_{31}}{c_{34}}{c_{41}}{c_{44}}
+c_{24}\det\mattwo{c_{31}}{c_{33}}{c_{41}}{c_{43}}\right)\\ & &
+c_{13}\left(c_{21}\det\mattwo{c_{32}}{c_{34}}{c_{42}}{c_{44}}
-c_{22}\det\mattwo{c_{31}}{c_{34}}{c_{41}}{c_{44}}
+c_{24}\det\mattwo{c_{31}}{c_{32}}{c_{41}}{c_{42}}\right)\\ & &
-c_{14}\left(c_{21}\det\mattwo{c_{32}}{c_{33}}{c_{42}}{c_{43}}
-c_{22}\det\mattwo{c_{31}}{c_{33}}{c_{41}}{c_{43}}
+c_{23}\det\mattwo{c_{31}}{c_{32}}{c_{41}}{c_{42}}\right)
\end{eqnarray*}
Combining the $2\times 2$ determinants leads to:
\begin{eqnarray*}
D(C) & = &
(c_{11}c_{22}-c_{12}c_{21})\det\mattwo{c_{33}}{c_{34}}{c_{43}}{c_{44}}
+(c_{11}c_{24}-c_{14}c_{21})\det\mattwo{c_{32}}{c_{33}}{c_{42}}{c_{43}}
\\ & & 
+(c_{12}c_{23}-c_{13}c_{22})\det\mattwo{c_{31}}{c_{34}}{c_{41}}{c_{44}}
+(c_{13}c_{21}-c_{11}c_{23})\det\mattwo{c_{32}}{c_{34}}{c_{42}}{c_{44}}
\\ & & 
+(c_{13}c_{24}-c_{14}c_{23})\det\mattwo{c_{31}}{c_{32}}{c_{41}}{c_{42}}
+(c_{14}c_{22}-c_{12}c_{24})\det\mattwo{c_{31}}{c_{33}}{c_{41}}{c_{43}}
\end{eqnarray*}
Supposing that 
\[
c_{21}=c_{11} \quad  c_{22}=c_{12} \quad c_{23}=c_{13} \quad
c_{24}=c_{14} 
\]
it is now easy to check that $D(C)=0$.

We now return to verifying that $D(A)$ satisfies properties (b)
and (c) of being a determinant.  We begin by showing that
$D(A)=0$ if $A$ has a row that is identically zero.  Suppose
that the zero row is the $i^{th}$ row and let $E$ be the matrix
that multiplies the $i^{th}$ row of $A$ by $c$.  Then $EA=A$.
Using \Ref{e:proddetE} we see that
\[
D(A)=D(EA)=D(E)D(A)=cD(A),
\]
which implies that $D(A)=0$ since $c$ is arbitrary.

Next we prove that $D(A)=0$ when $A$ is singular.  Using row 
reduction we can write
\[
A=E_s\cdots E_1R
\]
where the $E_j$ are elementary row matrices and $R$ is in
reduced echelon form\index{echelon form!reduced}.  
Since $A$ is singular, the last row of
$R$ is identically zero.  Hence $D(R)=0$ and \Ref{e:proddetE}
implies that $D(A)=0$.  

We now verify property (b).  Suppose that $A$ is singular; we
show that $D(A^t)=D(A)=0$.  Since the row rank of $A$ equals the
column rank of $A$, it follows that $A^t$ is singular when $A$
is singular.  Next assume that $A$ is nonsingular.  Then $A$ is
row equivalent to $I_n$ and we can write
\begin{equation}  \label{e:Adecomp}
A=E_s\cdots E_1,
\end{equation}
where the $E_j$ are elementary row matrices.  Since  
\[
A^t = E_1^t\cdots E_s^t
\]
and $D(E)=D(E^t)$, property (b) follows. 

We now verify property (c): $D(AB)=D(A)D(B)$.  Recall that $A$
is singular if and only if there exists a nonzero vector $v$
such that $Av=0$.  Now if $A$ is singular, then so is $A^t$.
Therefore $(AB)^t=B^tA^t$ is also singular.  To verify this
point, let $w$ be the nonzero vector such that $A^tw=0$.  Then
$B^tA^tw=0$.  Thus $AB$ is singular since $(AB)^t$ is singular.
Thus $D(AB)=0=D(A)D(B)$ when $A$ is singular.  Suppose now that
$A$ is nonsingular.  It follows that 
\[
AB = E_s\cdots E_1B.
\]
Using \Ref{e:proddetE} we see that
\[
D(AB)=D(E_s)\cdots D(E_1)D(B) = D(E_s\cdots E_1)D(B) = D(A)D(B),
\]
as desired. We have now completed the proof that a determinant
function exists. \end{proof}
\end{document}
