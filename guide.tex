\documentclass{article}
\usepackage{hyperref}
\usepackage{fullpage}

\setlength{\parindent}{0in}
\setlength{\parskip}{1ex}


\title{Instructor guide}
\author{LAODE}

\usepackage{xcolor,verbatimbox}
\catcode`>=\active %
\catcode`<=\active %
\def\openesc{\color{red}}
\def\closeesc{\color{black}}
\def\vbdelim{\catcode`<=\active\catcode`>=\active%
\def<{\openesc}
\def>{\closeesc}}
\catcode`>=12 %
\catcode`<=12 %

\begin{document}

\maketitle
\tableofcontents

\section{How to create worksheets}\label{how-to-create-worksheets}

In order to create worksheets, you will need access to the exercise
files (which include short answers and solutions).  To access these, you
will need to know the \verb|\label| attached to the exercise.  At
\url{BROKEN LINK}, you will find a copy of the textbook with the
exercise \verb|\label|s displayed in the margin.

The following instructions assume you are using a Mac.

\subsection{Get the .cls files.}\label{get-the-.cls-files.}

In the Terminal, run

\begin{verbatim}
mkdir -p ~/Library/texmf/tex/latex
git clone https://github.com/XimeraProject/ximeraLatex ~/Library/texmf/tex/latex/ximeraLatex
\end{verbatim}

\subsection{Download resources}\label{download-resources}

There are three steps:
\begin{itemize}
\item First, download the \LaTeX\ files for the textbook.

\begin{verbatim}
mkdir ~/linear-algebra
git clone https://github.com/mooculus/laode ~/linear-algebra/laode
\end{verbatim}

\item Then, download the worksheet builder using
\begin{verbatim}
git clone https://github.com/kisonecat/worksheet-builder ~/linear-algebra/worksheet-builder
\end{verbatim}
The result will be a directory
\verb|~/linear-algebra/worksheet-builder| populated with scripts like
\verb|build-worksheet.rb| which copy and rearrange the \LaTeX\ source
of the textbook to provide worksheets.

\item Finally, install the other software needed to run the worksheet builder.

\begin{verbatim}
gem install optparse-pathname
\end{verbatim}

If you lack permissions to run \texttt{gem}, try

\begin{verbatim}
gem install --user-install optparse-pathname
\end{verbatim}

to install it locally.
\end{itemize}

\subsection{Build the book}\label{build-the-book}

Compile the book. This step is needed because various \texttt{aux} files
are created that will be read by the worksheet builder.

\begin{verbatim}
cd ~/linear-algebra/laode
lualatex --shell-escape linearAlgebra
lualatex --shell-escape linearAlgebra
\end{verbatim}

\subsection{Build a worksheet}\label{try-it}

Now you are ready to build a worksheet.

\begin{description}
\item[Preliminaries.]
Go to the worksheet builder directory.

\begin{verbatim}
cd ~/linear-algebra/worksheet-builder
\end{verbatim}

There is a sample worksheet source at \verb|homework-sample.tex|.
To build the associated sample worksheet, run the command
\begin{verbatim}
./build-both.sh --root=$HOME/linear-algebra/laode homework-sample.tex
\end{verbatim}
Among many output files, you should see two PDF files
\begin{enumerate}
\item one with solutions called \texttt{homework-samples.pdf} and
\item one without solutions called \texttt{homework-samplee.pdf}.
\end{enumerate}
Here is a helpful mnemonic: the filename
\texttt{homework-sample\textcolor{red}{s}.pdf} has an \texttt{s} for
\texttt{s}olutions and the filename
\texttt{homework-sample\textcolor{red}{e}.pdf} has an \texttt{e} for
\texttt{e}xercises.  It is helpful to provide the PDF with solutions
to the grader, and to provide the PDF with solutions to the students
after the due date.

You can create your own file with a name like \texttt{homework1.tex}
with the following content:

\vspace{2ex}\hrule

\begin{verbnobox}[\vbdelim\mbox{}]
\documentclass{article}

 

\usepackage{epsfig}

\graphicspath{
  {./}
  {figures/}
}

\usepackage{morewrites}
\makeatletter
\newcommand\subfile[1]{%
\renewcommand{\input}[1]{}%
\begingroup\skip@preamble\otherinput{#1}\endgroup\par\vspace{\topsep}
\let\input\otherinput}
\makeatother

\newcommand{\includeexercises}{\directlua{dofile("/home/jim/linearAlgebra/laode/exercises.lua")}}

%\newcounter{ccounter}
%\setcounter{ccounter}{1}
%\newcommand{\Chapter}[1]{\setcounter{chapter}{\arabic{ccounter}}\chapter{#1}\addtocounter{ccounter}{1}}

%\newcommand{\section}[1]{\section{#1}\setcounter{thm}{0}\setcounter{equation}{0}}

%\renewcommand{\theequation}{\arabic{chapter}.\arabic{section}.\arabic{equation}}
%\renewcommand{\thefigure}{\arabic{chapter}.\arabic{figure}}
%\renewcommand{\thetable}{\arabic{chapter}.\arabic{table}}

%\newcommand{\Sec}[2]{\section{#1}\markright{\arabic{ccounter}.\arabic{section}.#2}\setcounter{equation}{0}\setcounter{thm}{0}\setcounter{figure}{0}}

\newcommand{\Sec}[2]{\section{#1}}

\setcounter{secnumdepth}{2}
%\setcounter{secnumdepth}{1} 

%\newcounter{THM}
%\renewcommand{\theTHM}{\arabic{chapter}.\arabic{section}}

\newcommand{\trademark}{{R\!\!\!\!\!\bigcirc}}
%\newtheorem{exercise}{}

\newcommand{\dfield}{{\sf dfield9}}
\newcommand{\pplane}{{\sf pplane9}}

\newcommand{\EXER}{\section*{Exercises}}%\vspace*{0.2in}\hrule\small\setcounter{exercise}{0}}
\newcommand{\CEXER}{}%\vspace{0.08in}\begin{center}Computer Exercises\end{center}}
\newcommand{\TEXER}{} %\vspace{0.08in}\begin{center}Hand Exercises\end{center}}
\newcommand{\AEXER}{} %\vspace{0.08in}\begin{center}Hand Exercises\end{center}}

% BADBAD: \newcommand{\Bbb}{\bf}

\newcommand{\R}{\mbox{$\Bbb{R}$}}
\newcommand{\C}{\mbox{$\Bbb{C}$}}
\newcommand{\Z}{\mbox{$\Bbb{Z}$}}
\newcommand{\N}{\mbox{$\Bbb{N}$}}
\newcommand{\D}{\mbox{{\bf D}}}
\usepackage{amssymb}
%\newcommand{\qed}{\hfill\mbox{\raggedright$\square$} \vspace{1ex}}
%\newcommand{\proof}{\noindent {\bf Proof:} \hspace{0.1in}}

\newcommand{\setmin}{\;\mbox{--}\;}
\newcommand{\Matlab}{{M\small{AT\-LAB}} }
\newcommand{\Matlabp}{{M\small{AT\-LAB}}}
\newcommand{\computer}{\Matlab Instructions}
\newcommand{\half}{\mbox{$\frac{1}{2}$}}
\newcommand{\compose}{\raisebox{.15ex}{\mbox{{\scriptsize$\circ$}}}}
\newcommand{\AND}{\quad\mbox{and}\quad}
\newcommand{\vect}[2]{\left(\begin{array}{c} #1_1 \\ \vdots \\
 #1_{#2}\end{array}\right)}
\newcommand{\mattwo}[4]{\left(\begin{array}{rr} #1 & #2\\ #3
&#4\end{array}\right)}
\newcommand{\mattwoc}[4]{\left(\begin{array}{cc} #1 & #2\\ #3
&#4\end{array}\right)}
\newcommand{\vectwo}[2]{\left(\begin{array}{r} #1 \\ #2\end{array}\right)}
\newcommand{\vectwoc}[2]{\left(\begin{array}{c} #1 \\ #2\end{array}\right)}

\newcommand{\ignore}[1]{}


\newcommand{\inv}{^{-1}}
\newcommand{\CC}{{\cal C}}
\newcommand{\CCone}{\CC^1}
\newcommand{\Span}{{\rm span}}
\newcommand{\rank}{{\rm rank}}
\newcommand{\trace}{{\rm tr}}
\newcommand{\RE}{{\rm Re}}
\newcommand{\IM}{{\rm Im}}
\newcommand{\nulls}{{\rm null\;space}}

\newcommand{\dps}{\displaystyle}
\newcommand{\arraystart}{\renewcommand{\arraystretch}{1.8}}
\newcommand{\arrayfinish}{\renewcommand{\arraystretch}{1.2}}
\newcommand{\Start}[1]{\vspace{0.08in}\noindent {\bf Section~\ref{#1}}}
\newcommand{\exer}[1]{\noindent {\bf \ref{#1}}}
\newcommand{\ans}{}
\newcommand{\matthree}[9]{\left(\begin{array}{rrr} #1 & #2 & #3 \\ #4 & #5 & #6
\\ #7 & #8 & #9\end{array}\right)}
\newcommand{\cvectwo}[2]{\left(\begin{array}{c} #1 \\ #2\end{array}\right)}
\newcommand{\cmatthree}[9]{\left(\begin{array}{ccc} #1 & #2 & #3 \\ #4 & #5 &
#6 \\ #7 & #8 & #9\end{array}\right)}
\newcommand{\vecthree}[3]{\left(\begin{array}{r} #1 \\ #2 \\
#3\end{array}\right)}
\newcommand{\cvecthree}[3]{\left(\begin{array}{c} #1 \\ #2 \\
#3\end{array}\right)}
\newcommand{\cmattwo}[4]{\left(\begin{array}{cc} #1 & #2\\ #3
&#4\end{array}\right)}

\newcommand{\Matrix}[1]{\ensuremath{\left(\begin{array}{rrrrrrrrrrrrrrrrrr} #1 \end{array}\right)}}

\newcommand{\Matrixc}[1]{\ensuremath{\left(\begin{array}{cccccccccccc} #1 \end{array}\right)}}



\renewcommand{\labelenumi}{\theenumi)}
\newenvironment{enumeratea}%
{\begingroup
 \renewcommand{\theenumi}{\alph{enumi}}
 \renewcommand{\labelenumi}{(\theenumi)}
 \begin{enumerate}}
 {\end{enumerate}\endgroup}



\newcounter{help}
\renewcommand{\thehelp}{\thesection.\arabic{equation}}

%\newenvironment{equation*}%
%{\renewcommand\endequation{\eqno (\theequation)* $$}%
%   \begin{equation}}%
%   {\end{equation}\renewcommand\endequation{\eqno \@eqnnum
%$$\global\@ignoretrue}}

%\input{psfig.tex}

\author{Martin Golubitsky and Michael Dellnitz}

%\newenvironment{matlabEquation}%
%{\renewcommand\endequation{\eqno (\theequation*) $$}%
%   \begin{equation}}%
%   {\end{equation}\renewcommand\endequation{\eqno \@eqnnum
% $$\global\@ignoretrue}}

\newcommand{\soln}{\textbf{Solution:} }
\newcommand{\exercap}[1]{\centerline{Figure~\ref{#1}}}
\newcommand{\exercaptwo}[1]{\centerline{Figure~\ref{#1}a\hspace{2.1in}
Figure~\ref{#1}b}}
\newcommand{\exercapthree}[1]{\centerline{Figure~\ref{#1}a\hspace{1.2in}
Figure~\ref{#1}b\hspace{1.2in}Figure~\ref{#1}c}}
\newcommand{\para}{\hspace{0.4in}}

\renewenvironment{solution}{\suppress}{\endsuppress}

\ifxake
\newenvironment{matlabEquation}{\begin{equation}}{\end{equation}}
\else
\newenvironment{matlabEquation}%
{\let\oldtheequation\theequation\renewcommand{\theequation}{\oldtheequation*}\begin{equation}}%
  {\end{equation}\let\theequation\oldtheequation}
\fi

\makeatother


\title{<Homework 1>}
\author{<Instructor Name>}
\date{Due: <Jan 1, 2078> at <9:10am>}

\begin{document}
\maketitle

\exercise{<label.for.problem.1>}

\exercise{<label.for.problem.2>}

\end{document}
\end{verbnobox}

\hrule\vspace{2ex}

Now if you run

\begin{verbatim}
./build-both.sh --root=$HOME/linear-algebra/laode homework1.tex
\end{verbatim}

two PDF files will be created:
\begin{enumerate}
\item one with solutions called \texttt{homework1s.pdf} and
\item one without solutions called \texttt{homework1e.pdf}.
\end{enumerate}

Two \LaTeX\ files are also created:
\begin{enumerate}
\item one with solutions called \texttt{homework1s.tex} and
\item one without solutions called \texttt{homework1e.tex}.
\end{enumerate}
You may edit the \LaTeX\ files by hand if you wish.  For example, you
may wish to insert \verb|\clearpage| to provide more logical
page-breaks.  Be warned that these auto-generated \LaTeX\ files may be
challenging to read.
\end{description}

\subsection{Send us new exercises}

Our goal is to increase the number of available exercises.  Your
contributions will be a tremendous help in reaching this goal.  A good
source of new exercises will be your exams.

It is easiest for us if you submit a separate \LaTeX\ file for each
exercise.  Include a suggestion for the most appropriate section of
the textbook, and which existing exercise in that section your problem
ought to follow.

It would be helpful if you use the format provided at
\href{https://github.com/mooculus/laode/blob/master/howToContribute/template.tex}{\texttt{template.tex}}.

As in the template, your contribution should include the exercise
written in \LaTeX, and below it the solution (preferably including an
answer).

MATLAB exercises should be marked as such.  If you wish for new
\verb|.m|-files to be included in the next edition of the textbook,
please include those contributions as well.

\section{Installing MATLAB}\label{installing-matlab}

\subsection{For Ohio State faculty and
students}\label{for-ohio-state-faculty-and-students}

Computers not only enable the completion of complicated calculations,
but can also improve the conceptual understanding of mathematics. This
course relies on MATLAB, a popular software package for matrix
computations. MATLAB is available free-of-charge to Ohio State students.
To download and activate your free copy of MATLAB, please follow the
steps below.

\begin{enumerate}
\def\labelenumi{\arabic{enumi})}
\item
  Launch Ohio State's IT Self-Service. Links to an external site. tool
  and click Login at the upper right.
\item
  Click Order Services. It is marked with a shopping cart but MATLAB
  will not cost you anything.
\item
  Click the Software Request link located under the ``Software
  Services'' heading.
\item
  Complete the Software Request Form, choosing MATLAB. Agree to terms
  and conditions.
\item
  Click Check Out.
\item
  Click Submit Order at the lower right.
\item
  Your order confirmation page lists your selections and their costs, as
  well as provides a Request Number (i.e., REQ12345) for tracking
  purposes.
\end{enumerate}

The rest of the instructions for installing MATLAB will automatically be
emailed to your Ohio State email account. Instructions will vary as to
whether you are on Windows, Mac OS, or Linux.

\subsection{Getting the m\_files}\label{getting-the-m_files}

To produce m\_files with filenames coming from the numeric labels in the
textbook, run

\begin{verbatim}
cd ~/linear-algebra/laode/m_files
ruby numeric-labels.rb
\end{verbatim}

Then to see the m\_files, run

\begin{verbatim}
cd ~/linear-algebra/laode/m_files/linearAlgebra
ls
\end{verbatim}

\section{Set up canvas}\label{set-up-canvas}

To set up a course shell in Canvas (``Carmen'') populated with linear
algebra content, email fowler@math.osu.edu.

\section{Syllabus}\label{syllabus}

To receive a sample syllabus and calendar, email fowler@math.osu.edu.

\end{document}
