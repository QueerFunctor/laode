\documentclass{ximera}

 

\usepackage{epsfig}

\graphicspath{
  {./}
  {figures/}
}

\usepackage{morewrites}
\makeatletter
\newcommand\subfile[1]{%
\renewcommand{\input}[1]{}%
\begingroup\skip@preamble\otherinput{#1}\endgroup\par\vspace{\topsep}
\let\input\otherinput}
\makeatother

\newcommand{\includeexercises}{\directlua{dofile("/home/jim/linearAlgebra/laode/exercises.lua")}}

%\newcounter{ccounter}
%\setcounter{ccounter}{1}
%\newcommand{\Chapter}[1]{\setcounter{chapter}{\arabic{ccounter}}\chapter{#1}\addtocounter{ccounter}{1}}

%\newcommand{\section}[1]{\section{#1}\setcounter{thm}{0}\setcounter{equation}{0}}

%\renewcommand{\theequation}{\arabic{chapter}.\arabic{section}.\arabic{equation}}
%\renewcommand{\thefigure}{\arabic{chapter}.\arabic{figure}}
%\renewcommand{\thetable}{\arabic{chapter}.\arabic{table}}

%\newcommand{\Sec}[2]{\section{#1}\markright{\arabic{ccounter}.\arabic{section}.#2}\setcounter{equation}{0}\setcounter{thm}{0}\setcounter{figure}{0}}

\newcommand{\Sec}[2]{\section{#1}}

\setcounter{secnumdepth}{2}
%\setcounter{secnumdepth}{1} 

%\newcounter{THM}
%\renewcommand{\theTHM}{\arabic{chapter}.\arabic{section}}

\newcommand{\trademark}{{R\!\!\!\!\!\bigcirc}}
%\newtheorem{exercise}{}

\newcommand{\dfield}{{\sf dfield9}}
\newcommand{\pplane}{{\sf pplane9}}

\newcommand{\EXER}{\section*{Exercises}}%\vspace*{0.2in}\hrule\small\setcounter{exercise}{0}}
\newcommand{\CEXER}{}%\vspace{0.08in}\begin{center}Computer Exercises\end{center}}
\newcommand{\TEXER}{} %\vspace{0.08in}\begin{center}Hand Exercises\end{center}}
\newcommand{\AEXER}{} %\vspace{0.08in}\begin{center}Hand Exercises\end{center}}

% BADBAD: \newcommand{\Bbb}{\bf}

\newcommand{\R}{\mbox{$\Bbb{R}$}}
\newcommand{\C}{\mbox{$\Bbb{C}$}}
\newcommand{\Z}{\mbox{$\Bbb{Z}$}}
\newcommand{\N}{\mbox{$\Bbb{N}$}}
\newcommand{\D}{\mbox{{\bf D}}}
\usepackage{amssymb}
%\newcommand{\qed}{\hfill\mbox{\raggedright$\square$} \vspace{1ex}}
%\newcommand{\proof}{\noindent {\bf Proof:} \hspace{0.1in}}

\newcommand{\setmin}{\;\mbox{--}\;}
\newcommand{\Matlab}{{M\small{AT\-LAB}} }
\newcommand{\Matlabp}{{M\small{AT\-LAB}}}
\newcommand{\computer}{\Matlab Instructions}
\newcommand{\half}{\mbox{$\frac{1}{2}$}}
\newcommand{\compose}{\raisebox{.15ex}{\mbox{{\scriptsize$\circ$}}}}
\newcommand{\AND}{\quad\mbox{and}\quad}
\newcommand{\vect}[2]{\left(\begin{array}{c} #1_1 \\ \vdots \\
 #1_{#2}\end{array}\right)}
\newcommand{\mattwo}[4]{\left(\begin{array}{rr} #1 & #2\\ #3
&#4\end{array}\right)}
\newcommand{\mattwoc}[4]{\left(\begin{array}{cc} #1 & #2\\ #3
&#4\end{array}\right)}
\newcommand{\vectwo}[2]{\left(\begin{array}{r} #1 \\ #2\end{array}\right)}
\newcommand{\vectwoc}[2]{\left(\begin{array}{c} #1 \\ #2\end{array}\right)}

\newcommand{\ignore}[1]{}


\newcommand{\inv}{^{-1}}
\newcommand{\CC}{{\cal C}}
\newcommand{\CCone}{\CC^1}
\newcommand{\Span}{{\rm span}}
\newcommand{\rank}{{\rm rank}}
\newcommand{\trace}{{\rm tr}}
\newcommand{\RE}{{\rm Re}}
\newcommand{\IM}{{\rm Im}}
\newcommand{\nulls}{{\rm null\;space}}

\newcommand{\dps}{\displaystyle}
\newcommand{\arraystart}{\renewcommand{\arraystretch}{1.8}}
\newcommand{\arrayfinish}{\renewcommand{\arraystretch}{1.2}}
\newcommand{\Start}[1]{\vspace{0.08in}\noindent {\bf Section~\ref{#1}}}
\newcommand{\exer}[1]{\noindent {\bf \ref{#1}}}
\newcommand{\ans}{}
\newcommand{\matthree}[9]{\left(\begin{array}{rrr} #1 & #2 & #3 \\ #4 & #5 & #6
\\ #7 & #8 & #9\end{array}\right)}
\newcommand{\cvectwo}[2]{\left(\begin{array}{c} #1 \\ #2\end{array}\right)}
\newcommand{\cmatthree}[9]{\left(\begin{array}{ccc} #1 & #2 & #3 \\ #4 & #5 &
#6 \\ #7 & #8 & #9\end{array}\right)}
\newcommand{\vecthree}[3]{\left(\begin{array}{r} #1 \\ #2 \\
#3\end{array}\right)}
\newcommand{\cvecthree}[3]{\left(\begin{array}{c} #1 \\ #2 \\
#3\end{array}\right)}
\newcommand{\cmattwo}[4]{\left(\begin{array}{cc} #1 & #2\\ #3
&#4\end{array}\right)}

\newcommand{\Matrix}[1]{\ensuremath{\left(\begin{array}{rrrrrrrrrrrrrrrrrr} #1 \end{array}\right)}}

\newcommand{\Matrixc}[1]{\ensuremath{\left(\begin{array}{cccccccccccc} #1 \end{array}\right)}}



\renewcommand{\labelenumi}{\theenumi)}
\newenvironment{enumeratea}%
{\begingroup
 \renewcommand{\theenumi}{\alph{enumi}}
 \renewcommand{\labelenumi}{(\theenumi)}
 \begin{enumerate}}
 {\end{enumerate}\endgroup}



\newcounter{help}
\renewcommand{\thehelp}{\thesection.\arabic{equation}}

%\newenvironment{equation*}%
%{\renewcommand\endequation{\eqno (\theequation)* $$}%
%   \begin{equation}}%
%   {\end{equation}\renewcommand\endequation{\eqno \@eqnnum
%$$\global\@ignoretrue}}

%\input{psfig.tex}

\author{Martin Golubitsky and Michael Dellnitz}

%\newenvironment{matlabEquation}%
%{\renewcommand\endequation{\eqno (\theequation*) $$}%
%   \begin{equation}}%
%   {\end{equation}\renewcommand\endequation{\eqno \@eqnnum
% $$\global\@ignoretrue}}

\newcommand{\soln}{\textbf{Solution:} }
\newcommand{\exercap}[1]{\centerline{Figure~\ref{#1}}}
\newcommand{\exercaptwo}[1]{\centerline{Figure~\ref{#1}a\hspace{2.1in}
Figure~\ref{#1}b}}
\newcommand{\exercapthree}[1]{\centerline{Figure~\ref{#1}a\hspace{1.2in}
Figure~\ref{#1}b\hspace{1.2in}Figure~\ref{#1}c}}
\newcommand{\para}{\hspace{0.4in}}

\renewenvironment{solution}{\suppress}{\endsuppress}

\ifxake
\newenvironment{matlabEquation}{\begin{equation}}{\end{equation}}
\else
\newenvironment{matlabEquation}%
{\let\oldtheequation\theequation\renewcommand{\theequation}{\oldtheequation*}\begin{equation}}%
  {\end{equation}\let\theequation\oldtheequation}
\fi

\makeatother


\title{Row Rank Equals Column Rank}

\begin{document}
\begin{abstract}
\end{abstract}
\maketitle

 \label{S:5.8}

Let $A$ be an $m\times n$ matrix.  The {\em row space\/}
\index{row!space} of $A$ is the span of the row vectors of $A$
and is a subspace of $\R^n$.  The {\em column space\/}
\index{column!space} of $A$ is the span of the columns of $A$
and is a subspace of $\R^m$.
\begin{definition} 
The {\em row rank\/} of $A$ is the dimension of the
row space of $A$ and the {\em column rank\/} of $A$ is the
dimension of the column space of $A$.
\end{definition} \index{row!rank}  \index{column!rank}
Lemma~\ref{L:computerank} of Chapter~\ref{C:vectorspaces} states that
\[
\mbox{row rank}(A) = \rank(A).
\]
We show below that row ranks and column ranks are equal.  We
begin by continuing the discussion of the previous section on linear maps
between vector spaces.

\subsection*{Null Space and Range}

Each linear map between vector spaces defines two subspaces.  Let $V$ and $W$ 
be vector spaces and let $L:V\to W$ be a linear map.  Then
\[
\mbox{null space}(L) = \{v\in V: L(v)=0\} \subset V
\]
\index{null space} and \index{range}
\[
\mbox{range}(L) = \{L(v)\in W: v\in V \} \subset W.
\]

\begin{lemma} \label{L:nsr}
Let $L:V\to W$ be a linear map between vector spaces.  Then the null space of
$L$ is a subspace of $V$ and the range of $L$ is a subspace of $W$.
\end{lemma}\index{subspace}

\begin{proof}  The proof that the null space of $L$ is a subspace of $V$ follows
from linearity in precisely the same way that the null space of an
$m\times n$ matrix is a subspace of $\R^n$.  That is, if $v_1$ and $v_2$ are
in the null space of $L$, then
\[
L(v_1+v_2) = L(v_1) + L(v_2) = 0 + 0 = 0,
\]
and for $c\in\R$
\[
L(cv_1) = cL(v_1) = c0 = 0.
\]
So the null space of $L$ is closed under addition and scalar multiplication
and is a subspace of $V$.

To prove that the range of $L$ is a subspace of $W$, let $w_1$ and $w_2$ be
in the range of $L$.  Then, by definition, there exist $v_1$ and $v_2$ in $V$
such that $L(v_j)=w_j$.  It follows that
\[
L(v_1+v_2) = L(v_1) + L(v_2) = w_1 + w_2.
\]
Therefore, $w_1+w_2$ is in the range of $L$.  Similarly,
\[
L(cv_1) = cL(v_1) = cw_1.
\]
So the range of $L$ is closed under addition and scalar multiplication and is
a subspace of $W$.  \end{proof}

Suppose that $A$ is an $m\times n$ matrix and $L_A:\R^n\to\R^m$ is the
associated linear map.  Then the null space of $L_A$ is precisely the null
space of $A$, as defined in Definition~\ref{D:nullspace} of 
Chapter~\ref{C:vectorspaces}.  Moreover, the range of $L_A$ is the column 
space of $A$.  To verify this, write $A=(A_1|\cdots|A_n)$ where $A_j$ is the 
$j^{th}$ column of $A$ and let $v=(v_1,\ldots v_n)^t$.  Then, $L_A(v)$ is the 
linear combination of columns of $A$
\[
L_A(v)=Av = v_1A_1+\cdots+v_nA_n.
\]

There is a theorem that relates the dimensions of the null space and range
with the dimension of $V$.
\begin{theorem}  \label{T:nsr}
Let $V$ and $W$ be vector spaces with $V$ finite dimensional and let
$L:V\to W$ be a linear map.  Then
\[
\dim(V) = \dim(\mbox{\rm null space}(L)) + \dim({\rm range}(L)).
\]\index{dimension}\index{null space} \index{range}
\end{theorem}

\begin{proof}   Since $V$ is finite dimensional, the null space of $L$ is finite 
dimensional (since the null space is a subspace of $V$) and the range of $L$ 
is finite dimensional (since it is spanned by the vectors $L(v_j)$ where 
$v_1,\ldots,v_n$ is a basis for $V$).  Let $u_1,\ldots,u_k$ be a basis for 
the null space of $L$ and let $w_1,\ldots,w_\ell$ be a basis for the range of
$L$.   Choose vectors $y_j\in V$ such that $L(y_j)=w_j$.  We claim that
$u_1,\ldots,u_k,y_1,\ldots,y_\ell$ is a basis for $V$, which proves the
theorem.

To verify that $u_1,\ldots,u_k,y_1,\ldots,y_\ell$ are linear independent,
suppose that
\begin{equation}  \label{E:uy}
\alpha_1u_1+\cdots+\alpha_ku_k+\beta_1y_1+\cdots+\beta_\ell y_\ell = 0.
\end{equation}
Apply $L$ to both sides of \eqref{E:uy} to obtain
\[
\beta_1w_1+\cdots+\beta_\ell w_\ell = 0.
\]
Since the $w_j$ are linearly independent, it follows that $\beta_j=0$ for all
$j$.  Now  \eqref{E:uy} implies that
\[
\alpha_1u_1+\cdots+\alpha_ku_k = 0.
\]
Since the $u_j$ are linearly independent, it follows that $\alpha_j=0$ for
all $j$.

To verify that $u_1,\ldots,u_k,y_1,\ldots,y_\ell$ span $V$, let $v$ be in
$V$.  Since $w_1,\ldots,w_\ell$ span $W$, it follows that there exist scalars
$\beta_j$ such that
\[
L(v) = \beta_1w_1+\cdots+\beta_\ell w_\ell.
\]
Note that by choice of the $y_j$
\[
L(\beta_1y_1+\cdots+\beta_\ell y_\ell) = \beta_1w_1+\cdots+\beta_\ell w_\ell.
\]
It follows by linearity that
\[
u = v - (\beta_1y_1+\cdots+\beta_\ell y_\ell)
\]
is in the null space of $L$.  Hence there exist scalars $\alpha_j$ such that
\[
u = \alpha_1u_1+\cdots+\alpha_ku_k.
\]
Thus, $v$ is in the span of $u_1,\ldots,u_k,y_1,\ldots,y_\ell$, as desired.
\end{proof}

\subsection*{Row Rank and Column Rank}

Recall Theorem~\ref{T:dimsoln} of Chapter~\ref{C:vectorspaces} that states
that the nullity plus the rank of an $m\times n$ matrix equals $n$.  At first 
glance it might seem that this theorem and Theorem~\ref{T:nsr} contain the 
same information, but they do not.  Theorem~\ref{T:dimsoln} of 
Chapter~\ref{C:vectorspaces} is proved using a detailed analysis of solutions 
of linear equations based on Gaussian elimination, back substitution, and 
reduced echelon form, while Theorem~\ref{T:nsr} is proved using abstract 
properties of linear maps.

Let $A$ be an $m\times n$ matrix.  Theorem~\ref{T:dimsoln} of
Chapter~\ref{C:vectorspaces} states that 
\[
{\rm nullity}(A) + \rank(A) = n.
\]
Meanwhile, Theorem~\ref{T:nsr} states that 
\[
\dim(\mbox{\rm null space}(L_A)) + \dim({\rm range}(L_A)) = n.
\]
But the dimension of the null space of $L_A$ equals the nullity of $A$ 
and the dimension of the range of $A$ equals the dimension of the column 
space of $A$.  Therefore, 
\[
{\rm nullity}(A) + \dim(\mbox{column space}(A)) = n.
\]
Hence, the rank of $A$ equals the column rank of $A$.  Since rank and row rank 
are identical, we have proved:
\begin{theorem} \label{T:rowrank=columnrank}
Let $A$ be an $m\times n$ matrix.  Then
\[
\mbox{row rank } A=\mbox{column rank } A.
\]
\end{theorem}\index{row!rank}\index{column!rank}

Since the row rank of $A$ equals the column rank of $A^t$, we have:
\begin{corollary}
Let $A$ be an $m\times n$ matrix.  Then
\[
\rank(A) = \rank(A^t).
\]
\end{corollary}\index{matrix!transpose}


\EXER

\includeexercises


\end{document}
