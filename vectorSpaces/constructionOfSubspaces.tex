\documentclass{ximera}

 

\usepackage{epsfig}

\graphicspath{
  {./}
  {figures/}
}

\usepackage{morewrites}
\makeatletter
\newcommand\subfile[1]{%
\renewcommand{\input}[1]{}%
\begingroup\skip@preamble\otherinput{#1}\endgroup\par\vspace{\topsep}
\let\input\otherinput}
\makeatother

\newcommand{\includeexercises}{\directlua{dofile("/home/jim/linearAlgebra/laode/exercises.lua")}}

%\newcounter{ccounter}
%\setcounter{ccounter}{1}
%\newcommand{\Chapter}[1]{\setcounter{chapter}{\arabic{ccounter}}\chapter{#1}\addtocounter{ccounter}{1}}

%\newcommand{\section}[1]{\section{#1}\setcounter{thm}{0}\setcounter{equation}{0}}

%\renewcommand{\theequation}{\arabic{chapter}.\arabic{section}.\arabic{equation}}
%\renewcommand{\thefigure}{\arabic{chapter}.\arabic{figure}}
%\renewcommand{\thetable}{\arabic{chapter}.\arabic{table}}

%\newcommand{\Sec}[2]{\section{#1}\markright{\arabic{ccounter}.\arabic{section}.#2}\setcounter{equation}{0}\setcounter{thm}{0}\setcounter{figure}{0}}

\newcommand{\Sec}[2]{\section{#1}}

\setcounter{secnumdepth}{2}
%\setcounter{secnumdepth}{1} 

%\newcounter{THM}
%\renewcommand{\theTHM}{\arabic{chapter}.\arabic{section}}

\newcommand{\trademark}{{R\!\!\!\!\!\bigcirc}}
%\newtheorem{exercise}{}

\newcommand{\dfield}{{\sf dfield9}}
\newcommand{\pplane}{{\sf pplane9}}

\newcommand{\EXER}{\section*{Exercises}}%\vspace*{0.2in}\hrule\small\setcounter{exercise}{0}}
\newcommand{\CEXER}{}%\vspace{0.08in}\begin{center}Computer Exercises\end{center}}
\newcommand{\TEXER}{} %\vspace{0.08in}\begin{center}Hand Exercises\end{center}}
\newcommand{\AEXER}{} %\vspace{0.08in}\begin{center}Hand Exercises\end{center}}

% BADBAD: \newcommand{\Bbb}{\bf}

\newcommand{\R}{\mbox{$\Bbb{R}$}}
\newcommand{\C}{\mbox{$\Bbb{C}$}}
\newcommand{\Z}{\mbox{$\Bbb{Z}$}}
\newcommand{\N}{\mbox{$\Bbb{N}$}}
\newcommand{\D}{\mbox{{\bf D}}}
\usepackage{amssymb}
%\newcommand{\qed}{\hfill\mbox{\raggedright$\square$} \vspace{1ex}}
%\newcommand{\proof}{\noindent {\bf Proof:} \hspace{0.1in}}

\newcommand{\setmin}{\;\mbox{--}\;}
\newcommand{\Matlab}{{M\small{AT\-LAB}} }
\newcommand{\Matlabp}{{M\small{AT\-LAB}}}
\newcommand{\computer}{\Matlab Instructions}
\newcommand{\half}{\mbox{$\frac{1}{2}$}}
\newcommand{\compose}{\raisebox{.15ex}{\mbox{{\scriptsize$\circ$}}}}
\newcommand{\AND}{\quad\mbox{and}\quad}
\newcommand{\vect}[2]{\left(\begin{array}{c} #1_1 \\ \vdots \\
 #1_{#2}\end{array}\right)}
\newcommand{\mattwo}[4]{\left(\begin{array}{rr} #1 & #2\\ #3
&#4\end{array}\right)}
\newcommand{\mattwoc}[4]{\left(\begin{array}{cc} #1 & #2\\ #3
&#4\end{array}\right)}
\newcommand{\vectwo}[2]{\left(\begin{array}{r} #1 \\ #2\end{array}\right)}
\newcommand{\vectwoc}[2]{\left(\begin{array}{c} #1 \\ #2\end{array}\right)}

\newcommand{\ignore}[1]{}


\newcommand{\inv}{^{-1}}
\newcommand{\CC}{{\cal C}}
\newcommand{\CCone}{\CC^1}
\newcommand{\Span}{{\rm span}}
\newcommand{\rank}{{\rm rank}}
\newcommand{\trace}{{\rm tr}}
\newcommand{\RE}{{\rm Re}}
\newcommand{\IM}{{\rm Im}}
\newcommand{\nulls}{{\rm null\;space}}

\newcommand{\dps}{\displaystyle}
\newcommand{\arraystart}{\renewcommand{\arraystretch}{1.8}}
\newcommand{\arrayfinish}{\renewcommand{\arraystretch}{1.2}}
\newcommand{\Start}[1]{\vspace{0.08in}\noindent {\bf Section~\ref{#1}}}
\newcommand{\exer}[1]{\noindent {\bf \ref{#1}}}
\newcommand{\ans}{}
\newcommand{\matthree}[9]{\left(\begin{array}{rrr} #1 & #2 & #3 \\ #4 & #5 & #6
\\ #7 & #8 & #9\end{array}\right)}
\newcommand{\cvectwo}[2]{\left(\begin{array}{c} #1 \\ #2\end{array}\right)}
\newcommand{\cmatthree}[9]{\left(\begin{array}{ccc} #1 & #2 & #3 \\ #4 & #5 &
#6 \\ #7 & #8 & #9\end{array}\right)}
\newcommand{\vecthree}[3]{\left(\begin{array}{r} #1 \\ #2 \\
#3\end{array}\right)}
\newcommand{\cvecthree}[3]{\left(\begin{array}{c} #1 \\ #2 \\
#3\end{array}\right)}
\newcommand{\cmattwo}[4]{\left(\begin{array}{cc} #1 & #2\\ #3
&#4\end{array}\right)}

\newcommand{\Matrix}[1]{\ensuremath{\left(\begin{array}{rrrrrrrrrrrrrrrrrr} #1 \end{array}\right)}}

\newcommand{\Matrixc}[1]{\ensuremath{\left(\begin{array}{cccccccccccc} #1 \end{array}\right)}}



\renewcommand{\labelenumi}{\theenumi)}
\newenvironment{enumeratea}%
{\begingroup
 \renewcommand{\theenumi}{\alph{enumi}}
 \renewcommand{\labelenumi}{(\theenumi)}
 \begin{enumerate}}
 {\end{enumerate}\endgroup}



\newcounter{help}
\renewcommand{\thehelp}{\thesection.\arabic{equation}}

%\newenvironment{equation*}%
%{\renewcommand\endequation{\eqno (\theequation)* $$}%
%   \begin{equation}}%
%   {\end{equation}\renewcommand\endequation{\eqno \@eqnnum
%$$\global\@ignoretrue}}

%\input{psfig.tex}

\author{Martin Golubitsky and Michael Dellnitz}

%\newenvironment{matlabEquation}%
%{\renewcommand\endequation{\eqno (\theequation*) $$}%
%   \begin{equation}}%
%   {\end{equation}\renewcommand\endequation{\eqno \@eqnnum
% $$\global\@ignoretrue}}

\newcommand{\soln}{\textbf{Solution:} }
\newcommand{\exercap}[1]{\centerline{Figure~\ref{#1}}}
\newcommand{\exercaptwo}[1]{\centerline{Figure~\ref{#1}a\hspace{2.1in}
Figure~\ref{#1}b}}
\newcommand{\exercapthree}[1]{\centerline{Figure~\ref{#1}a\hspace{1.2in}
Figure~\ref{#1}b\hspace{1.2in}Figure~\ref{#1}c}}
\newcommand{\para}{\hspace{0.4in}}

\renewenvironment{solution}{\suppress}{\endsuppress}

\ifxake
\newenvironment{matlabEquation}{\begin{equation}}{\end{equation}}
\else
\newenvironment{matlabEquation}%
{\let\oldtheequation\theequation\renewcommand{\theequation}{\oldtheequation*}\begin{equation}}%
  {\end{equation}\let\theequation\oldtheequation}
\fi

\makeatother


\title{Construction of Subspaces}

\begin{document}
\begin{abstract}
\end{abstract}
\maketitle

 \label{S:5.2}

The principle of superposition shows that the set of all
solutions to a homogeneous system of linear equations is closed under 
addition and scalar multiplication and is a
subspace.  Indeed, there are two ways to describe subspaces:
first as solutions to linear systems, and second as the span of
a set of vectors.  We shall see that solving a homogeneous linear system of
equations just means writing the solution set as the span of a
finite set of vectors.

\subsection*{Solutions to Homogeneous Systems Form Subspaces}
\index{homogeneous} \index{subspace}

\begin{definition} \label{D:nullspace}
Let $A$ be an $m\times n$ matrix.  The {\em null space\/} of $A$
is the set of solutions to the homogeneous system of linear equations
\begin{equation} \label{Ax=0}
Ax=0.
\end{equation}
\end{definition} \index{null space}

\begin{lemma}
Let $A$ be an $m\times n$ matrix.  Then the null space of $A$
is a subspace of $\R^n$.
\end{lemma}

\begin{proof}
Suppose that $x$ and $y$ are solutions to \eqref{Ax=0}.  Then
\[
A(x+y) = Ax+Ay = 0+0 = 0;
\]
so $x+y$ is a solution of \eqref{Ax=0}.  Similarly, for $r\in\R$
\[
A(rx) = rAx = r0 = 0;
\]
so $rx$ is a solution of \eqref{Ax=0}.  Thus, $x+y$ and $rx$ are
in the null space of $A$, and the null space is closed under addition 
and scalar multiplication.  So Theorem~\ref{T:subspaces} implies that
the null space is a subspace of the vector space $\R^n$.   \end{proof}

\subsubsection*{Solutions to Linear Systems of Differential Equations Form
Subspaces}

Let $C$ be an $n\times n$ matrix and let $W$ be the set of solutions to
the linear system of ordinary differential equations
\begin{equation} \label{Cx(t)}
\frac{dx}{dt}(t) = Cx(t).
\end{equation}
We will see later that a solution to \eqref{Cx(t)} has coordinate
functions $x_j(t)$ in $\CCone$.  The principle of superposition
then shows that $W$ is a subspace of $(\CCone)^n$.  Suppose
$x(t)$ and $y(t)$ are solutions of \eqref{Cx(t)}.  Then
\[
\frac{d}{dt}(x(t)+y(t)) = \frac{dx}{dt}(t) + \frac{dy}{dt}(t) =
Cx(t) + Cy(t) = C(x(t)+y(t));
\]
so $x(t)+y(t)$ is a solution of \eqref{Cx(t)} and in $W$.  A
similar calculation shows that $rx(t)$ is also in $W$ and
that $W\subset(\CCone)^n$ is a subspace.


\subsection*{Writing Solution Subspaces as a Span}
\index{homogeneous}\index{span}

The way we solve homogeneous systems of equations gives a second
method for defining subspaces.  For example, consider the system
\[
Ax=0,
\]
where
\[
A=\left(\begin{array}{rccc} 2 & 1 & 4 & 0 \\ -1 & 0 & 2 & 1
        \end{array}\right).
\]
The matrix $A$ is row equivalent to the reduced echelon form matrix
\[
E=\left(\begin{array}{ccrr} 1 & 0 & -2 & -1 \\ 0 & 1 & 8 & 2
        \end{array}\right).
\]
Therefore $x=(x_1,x_2,x_3,x_4)$ is a solution of $Ex=0$ if and
only if $x_1 = 2x_3+x_4$ and $x_2 = -8x_3 - 2x_4$.
It follows that every solution of $Ex=0$ can be written as:
\[
x = x_3\left(\begin{array}{r} 2 \\ -8\\ 1\\ 0 \end{array}\right)
+x_4\left(\begin{array}{r} 1 \\ -2\\ 0\\ 1 \end{array}\right).
\]
Since row operations do not change the set of solutions, it
follows that every solution of $Ax=0$ has this form. We have
also shown that every solution is generated by two vectors by
use of vector addition\index{vector!addition} and
scalar multiplication\index{scalar multiplication}.  We say that
this subspace is {\em spanned\/} by the two vectors
\[
\left(\begin{array}{r} 2 \\ -8\\ 1\\ 0 \end{array}\right)
\AND
\left(\begin{array}{r} 1 \\ -2\\ 0\\ 1 \end{array}\right).
\]
For example, a calculation verifies that the vector
\[
\left(\begin{array}{r} -1 \\ -2\\ 1\\ -3 \end{array}\right)
\]
is also a solution of $Ax=0$.  Indeed, we may write it as
\begin{equation}
\label{eq:SpanSol}
\left(\begin{array}{r} -1 \\ -2\\ 1\\ -3 \end{array}\right)
=
\left(\begin{array}{r} 2 \\ -8\\ 1\\ 0 \end{array}\right) -
3 \left(\begin{array}{r} 1 \\ -2\\ 0\\ 1 \end{array}\right).
\end{equation}


\subsection*{Spans}

Let $v_1,\ldots,v_k$ be a set of vectors in a vector space $V$.  A vector
$v\in V$ is a {\em linear combination\/} of $v_1,\ldots,v_k$
if
\[
v = r_1v_1 + \cdots + r_kv_k
\]
for some scalars $r_1,\ldots,r_k$.  \index{linear!combination}


\begin{definition}  \label{span}
The set of all linear combinations of the vectors $v_1,\ldots,v_k$
in a vector space $V$ is the {\em span\/} of $v_1,\ldots,v_k$ and is
denoted by $\Span\{v_1,\ldots,v_k\}$.
\end{definition} \index{span}

For example, the vector on the left hand side in \eqref{eq:SpanSol}
is a linear combination of the two vectors on the right hand side.

The simplest example of a span is $\R^n$ itself.  Let $v_j=e_j$
where $e_j\in\R^n$ is the vector with a $1$ in the $j^{th}$
coordinate and $0$ in all other coordinates.  Then every vector
$x=(x_1,\ldots,x_n)\in\R^n$ can be written as
\[
x = x_1e_1 + \cdots + x_ne_n.
\]
It follows that
\[
\R^n=\Span\{e_1,\ldots,e_n\}.
\]
Similarly, the set $\Span\{e_1,e_3\}\subset\R^3$ is just the
$x_1x_3$-plane, since vectors in this span are
\[
x_1e_1+x_3e_3 = x_1(1,0,0) + x_3(0,0,1) = (x_1,0,x_3).
\]

\begin{proposition} \label{spansubspace} Let $V$ be a vector space and
let $w_1,\ldots,w_k\in V$. Then $W=\Span\{w_1,\ldots,w_k\}
\subset V$ is a subspace.
\end{proposition} \index{span} \index{subspace}\index{vector!space}

\begin{proof}  Suppose $x,y\in W$.  Then
\begin{eqnarray*}
x & = & r_1w_1 + \cdots + r_kw_k \\
y & = & s_1w_1 + \cdots + s_kw_k
\end{eqnarray*}
for some scalars $r_1,\ldots,r_k$ and $s_1,\ldots,s_k$.  It
follows that
\[
x+y = (r_1+s_1)w_1 + \cdots + (r_k+s_k)w_k
\]
and
\[
rx = (rr_1)w_1 + \cdots + (rr_k)w_k
\]
are both in $\Span\{w_1,\ldots,w_k\}$. Hence $W\subset V$ is
closed under addition and scalar multiplication, and is a
subspace by Theorem~\ref{T:subspaces}. \end{proof}

For example, let
\begin{equation}  \label{e:vandw}
v=(2,1,0) \AND w=(1,1,1)
\end{equation}
be vectors in $\R^3$. Then linear combinations of the vectors
$v$ and $w$ have the form
\[
\alpha v + \beta w = (2\alpha+\beta, \alpha+\beta, \beta)
\]
for real numbers $\alpha$ and $\beta$.  Note that every one of
these vectors is a solution to the linear equation
\begin{equation} \label{ex1}
x_1 - 2x_2 + x_3 = 0,
\end{equation}
that is, the $1^{st}$ coordinate minus twice the $2^{nd}$ coordinate 
plus the $3^{rd}$ coordinate equals zero.  Moreover, you may verify 
that every solution of \eqref{ex1} is a linear combination
of the vectors $v$ and $w$ in \eqref{e:vandw}.  Thus, the set of
solutions to the homogeneous\index{homogeneous} linear equation
\eqref{ex1} is a
subspace, and that subspace can be written as the span of
all linear combinations of the vectors $v$ and $w$.

In this language we see that the process of solving a
homogeneous system of linear equations is just the process of
finding a set of vectors that span the subspace of all
solutions.  Indeed,
we can now restate Theorem~\ref{number} of Chapter~\ref{lineq}.
Recall that a matrix $A$ has {\em rank\/} $\ell$ if it is row
equivalent to a matrix in echelon form with $\ell$ nonzero rows.
\index{rank}

\begin{proposition}  \label{P:n-rank}
Let $A$ be an $m\times n$ matrix with rank $\ell$. Then the
null space of $A$ is the span of $n-\ell$ vectors.
\end{proposition} \index{null space} \index{span}

We have now seen that there are two ways to describe subspaces ---
as solutions of homogeneous systems of linear equations and as a
span of a set of vectors, the {\em spanning set}\index{spanning set}.
Much of linear algebra is concerned
with determining how one goes from one description of a subspace
to the other.

\EXER

\TEXER

\noindent In Exercises~\ref{c5.2.1a} -- \ref{c5.2.1d} a single equation in
three variables is given.  For each equation write the subspace of solutions
in $\R^3$ as the span of two vectors in $\R^3$.
\begin{exercise} \label{c5.2.1a}
$4x - 2y + z = 0$.

\begin{solution}
\ans The subspace of solutions can be spanned by the vectors 
$(1,0,-4)^t$ and $(0,1,2)^t$.

\soln All solutions to $4x - 2y + z = 0$ can be written in the form
\[
\vecthree{x}{y}{z} = \cvecthree{x}{y}{2y - 4x}
= x\vecthree{1}{0}{-4} + y\vecthree{0}{1}{2}.
\]

\end{solution}
\end{exercise}
\begin{exercise} \label{c5.2.1b}
$x - y + 3z = 0$.

\begin{solution}
\ans The subspace of solutions can be spanned by the vectors 
$(1,1,0)^t$ and $(-3,0,1)^t$.

\soln All solutions to $x - y + 3z = 0$ can be written in the form
\[
\vecthree{x}{y}{z} = \cvecthree{y-3z}{y}{z}
= y\vecthree{1}{1}{0} + z\vecthree{-3}{0}{1}.
\]

\end{solution}
\end{exercise}
\begin{exercise} \label{c5.2.1c}
$x + y + z = 0$.

\begin{solution}
\ans The subspace of solutions can be spanned by the vectors 
$(1,0,-1)^t$ and $(0,1,-1)^t$.

\soln All solutions to $x + y + z = 0$ can be written in the form
\[
\vecthree{x}{y}{z} = \cvecthree{x}{y}{-x-y}
= x\vecthree{1}{0}{-1} + z\vecthree{0}{1}{-1}.
\]



\end{solution}
\end{exercise}
\begin{exercise} \label{c5.2.1d}
$y=z$.

\begin{solution}
\ans The subspace of solutions can be spanned by the vectors 
$(1,0,0)^t$ and $(0,1,1)^t$.

\soln All solutions to $y = z$ can be written in the form
\[
\vecthree{x}{y}{z} = \cvecthree{x}{y}{y}
= x\vecthree{1}{0}{0} + z\vecthree{0}{1}{1}.
\]


\end{solution}
\end{exercise}

\noindent In Exercises~\ref{c5.2.2a} -- \ref{c5.2.2d} each of the
given matrices is in reduced echelon form.  Write solutions of the
corresponding homogeneous system of linear equations as a span of vectors.
\begin{exercise} \label{c5.2.2a}
$A = \left(\begin{array}{rrrrr} 1 & 2 & 0 & 1 & 0 \\
	0 & 0 & 1 & 4 & 0 \\ 0 & 0 & 0 & 0 & 1 \end{array}\right)$.

\begin{solution}

\ans The subspace of solutions is spanned by the vectors
\[
(-2,1,0,0,0)^t \AND (-1,0,-4,1,0)^t.
\]

\soln Let $x = (x_1,\dots ,x_5)$ be a solution to $Ax = 0$.  All
solutions to this equation have the form
\[
\left(\begin{array}{r} x_1 \\ x_2 \\ x_3 \\ x_4 \\ x_5
\end{array}\right) = \left(\begin{array}{c} -2x_2 - x_4 \\ x_2 \\
-4x_4 \\ x_4 \\ 0 \end{array}\right) = x_2\left(\begin{array}{r}
-2 \\ 1 \\ 0 \\ 0 \\ 0 \end{array}\right) +
x_4\left(\begin{array}{r} -1 \\ 0 \\ -4 \\ 1 \\ 0
\end{array}\right).
\]

\end{solution}
\end{exercise}
\begin{exercise} \label{c5.2.2b}
$B = \left(\begin{array}{rrrr} 1 & 3 & 0 & 5 \\
	0 & 0 & 1 & 2 \end{array}\right)$.

\begin{solution}

\ans The subspace of solutions to $Bx = 0$ is spanned by the vectors
\[
(-3,1,0,0)^t \AND (-5,0,-2,1)^t.
\]

\soln Let $x = (x_1,x_2,x_3,x_4)$ be a solution to $Bx = 0$.  All
solutions to this equation have the form
\[
\left(\begin{array}{r} x_1 \\ x_2 \\ x_3 \\ x_4 \end{array}\right)
= \left(\begin{array}{c} -3x_2 - 5x_4 \\ x_2 \\ -2x_4 \\ x_4
\end{array}\right) = x_2\left(\begin{array}{r} -3 \\ 1 \\ 0 \\ 0
\end{array}\right) + x_4\left(\begin{array}{r} -5 \\ 0 \\ -2 \\ 1
\end{array}\right).
\]

\end{solution}
\end{exercise}

\begin{exercise} \label{c5.2.2c}
$A = \left(\begin{array}{rrr} 1 & 0 & 2 \\
        0 & 1 & 1\end{array}\right)$.

\begin{solution}

\ans The subspace of solutions to $Ax = 0$ is spanned by the vector
$(-2,-1,1)^t$.

\soln Let $x = (x_1,x_2,x_3)$ be a solution to $Ax = 0$.  All solutions
to this equation have the form
\[
\vecthree{x_1}{x_2}{x_3} = \vecthree{-2x_3}{-x_3}{x_3} =
x_3\vecthree{-2}{-1}{1}.
\]

\end{solution}
\end{exercise}
\begin{exercise} \label{c5.2.2d}
$B = \left(\begin{array}{rrrrrr} 1 & -1 & 0 & 5 & 0 & 0\\
        0 & 0 & 1 & 2 & 0 & 2\\
        0 & 0 & 0 & 0 & 1 & 2\end{array}\right)$.

\begin{solution}

\ans The subspace of solutions to $Bx = 0$ is spanned by the vectors
\[
\left(\begin{array}{r} 1 \\ 1 \\ 0 \\ 0 \\ 0 \\ 0 \end{array}\right), \quad
\left(\begin{array}{r} -5 \\ 0 \\ -2 \\ 1 \\ 0 \\ 0 \end{array}\right), \quad
\left(\begin{array}{r} 0 \\ 0 \\ -2 \\ 0 \\ -2 \\ 1 \end{array}\right).
\]

\soln Let $x = (x_1,\dots,x_6)$ be a solution to $Bx = 0$.  All solutions
to this equation have the form
\[
\left(\begin{array}{r} x_1 \\ x_2 \\ x_3 \\ x_4 \\ x_5 \\ x_6
\end{array}\right) =
\left(\begin{array}{c} x_2 - 5x_4 \\ x_2 \\ -2x_4 - 2x_6 \\ x_4 \\ -2x_6
\\ x_6 \end{array}\right) =
x_2\left(\begin{array}{r} 1 \\ 1 \\ 0 \\ 0 \\ 0 \\ 0 \end{array}\right) +
x_4\left(\begin{array}{r} -5 \\ 0 \\ -2 \\ 1 \\ 0 \\ 0 \end{array}\right) +
x_6\left(\begin{array}{r} 0 \\ 0 \\ -2 \\ 0 \\ -2 \\ 1 \end{array}\right).
\]


\end{solution}
\end{exercise}

\begin{exercise} \label{c5.2.3}
Write a system of two linear equations of the form $Ax=0$ where
$A$ is a $2\times 4$ matrix whose subspace of solutions in $\R^4$
is the span of the two vectors
\[
v_1 = \left(\begin{array}{r} 1 \\ -1 \\ 0 \\  0 \end{array}\right) \AND
v_2 = \left(\begin{array}{r} 0 \\  0 \\ 1 \\ -1 \end{array}\right).
\]

\begin{solution}

\ans The matrix $A$ whose subspace of solutions in $\R^4$ is the span of
$v_1$ and $v_2$ is
\[
A = \left(\begin{array}{rrrr} 1 & 1 & 0 & 0 \\ 0 & 0 & 1 & 1
\end{array}\right).
\]

\soln Note that all vectors $x$ in the spanning set of $v_1$ and $v_2$
are of the form:
\[
x = \left(\begin{array}{r} x_1 \\ x_2 \\ x_3 \\ x_4
\end{array}\right)
= a\left(\begin{array}{r} 1 \\ -1 \\ 0 \\ 0 \end{array}\right) + 
b\left(\begin{array}{r} 0 \\ 0 \\ 1 \\ -1 \end{array}\right) =
\left(\begin{array}{r} a \\ -a \\ b \\ -b \end{array}\right).
\]
Therefore, $x_1 = -x_2$ and $x_3 = -x_4$.  So,
\[
\begin{array}{rrrrrrrrl}
x_1 & + & x_2 & & & & & = & 0 \\
& & & & x_3 & + & x_4 & = & 0. \end{array}
\]
The matrix of this system is $A$.

\end{solution}
\end{exercise}

\begin{exercise} \label{c5.2.4}
Write the matrix $A=\mattwo{2}{2}{-3}{0}$ as a linear combination
of the matrices
\[
B=\mattwo{1}{1}{0}{0} \AND C=\mattwo{0}{0}{1}{0}.
\]

\begin{solution}

\[ A = \mattwo{2}{2}{-3}{0} = 2\mattwo{1}{1}{0}{0} -
3\mattwo{0}{0}{1}{0} = 2B - 3C. \]

\end{solution}
\end{exercise}

\begin{exercise} \label{c5.2.5}
Is $(2,20,0)$ in the span of $w_1=(1,1,3)$ and $w_2=(1,4,2)$?
Answer this question by setting up a system of linear equations
and solving that system by row reducing the associated augmented
matrix\index{matrix!augmented}.

\begin{solution}

\ans The vector $(2,20,0)^t$ is in the span of $w_1$ and $w_2$. 
Specifically, $v = -4w_1 + 6w_2$.

\soln Note that, for some real numbers $a$ and $b$,
\[
(2,20,0)^t = aw_1 + bw_2 = a(1,1,3)^t + b(1,4,2)^t
\]
if $v$ is in the span of $w_1$ and $w_2$.
This corresponds to the linear system
\[
\begin{array}{rrrrr}
a & + & b & = 2 \\
a & + & 4b & = 20 \\
3a & + & 2b & = 0 \end{array}
\]
To find $a$ and $b$, row reduce the augmented matrix of the system:
\[
\left(\begin{array}{rr|r} 1 & 1 & 2 \\ 1 & 4 & 20 \\
3 & 2 & 0 \end{array}\right) \longrightarrow
\left(\begin{array}{rr|r} 1 & 0 & -4 \\ 0 & 1 & 6 \\
0 & 0 & 0 \end{array}\right).
\]
The system is consistent; $a = -4$ and $b = 6$.

\end{solution}
\end{exercise}

\noindent In Exercises~\ref{c5.2.6a} -- \ref{c5.2.6d} let $W\subset\CCone$
be the subspace spanned by the two polynomials $x_1(t) = 1$ and
$x_2(t)=t^2$.  For the given function $y(t)$ decide whether or not $y(t)$
is an element of $W$.  Furthermore, if $y(t)\in W$, determine whether the set
$\{y(t),x_2(t)\}$ is a spanning set for $W$.
\begin{exercise} \label{c5.2.6a}
$y(t) = 1-t^2$,

\begin{solution}

\ans The function $y(t) = 1 - t^2$ is an element of $W$ and the set
$\{y(t),x_2(t)\}$ is a spanning set for $W$.



\soln The space $W$ equals $\Span\{x_1(t),x_2(t)\}$ where $x_1(t)=1$ and 
$x_2(t)=t^2$.  To show that $y(t)$ is an element of $W$, let
$a = 1$ and $b = -1$, and compute
\[
ax_1(t) + bx_2(t) = x_1(t) - x_2(t) = 1 - t^2 = y(t). 
\]
To show that $\{y(t),x_2(t)\}$ is a spanning set for $W$, rewrite every
linear combination of $x_1(t)$ and $x_2(t)$ in terms of $y(t)$ and $x_2(t)$, 
as follows:
\[ 
ax_1(t) + bx_2(t) = a + bt^2 = a(1 - t^2) + (a + b)t^2
= ay(t) + (a + b)x_2(t). 
\]

\end{solution}
\end{exercise}
\begin{exercise} \label{c5.2.6b}
$y(t) = t^4$,

\begin{solution}
The function $y(t) = t^4$ is not in $W$.

\end{solution}
\end{exercise}
\begin{exercise} \label{c5.2.6c}
$y(t) = \sin t$,

\begin{solution}
The function $y(t) = \sin(t)$ is not in $W$.

\end{solution}
\end{exercise}
\begin{exercise} \label{c5.2.6d}
$y(t) = 0.5 t^2$

\begin{solution}

\ans The function $y(t) = 0.5t^2$ is an element of $W$, but the set
$\{y(t),x_2(t)\}$ does not span $W$.

\soln When $a = 0$ and $b = 0.5$,
\[ 
ax_1(t) + bx_2(t) = 0.5x_2(t) = 0.5t^2 = y(t). 
\]
In this case, there exist functions in $W$ that are not in 
$\Span\{y(t),x_2(t)\}$.  For example, the function $x_1(t) = 1$ cannot
be written as a linear combination of $x_2(t)$ and $y(t)$.

\end{solution}
\end{exercise}

\begin{exercise} \label{c5.2.7}
Let $W\subset\R^4$ be the subspace that is spanned by the vectors
\[
        w_1=(-1,2,1,5)\AND w_2=(2,1,3,0).
\]
Find a linear system of two equations such that
$W=\Span\{w_1,w_2\}$ is the set of solutions of this system.

\begin{solution}

\ans The span of $W$ is the set of solutions to the system
\[ 
\begin{array}{rrrrrrrrr}
x_1 & + & x_2 & - & x_3 & & & = & 0 \\
& & 3x_2 & - & x_3 & - & x_4 & = & 0 \end{array}. 
\]
where $x = (x_1,x_2,x_3,x_4) \in W$.  Row reduction of the associated
matrix demonstrates that this system is a valid solution set.

\soln Solve for $x$ as a linear combination of $w_1$ and $w_2$
by creating the matrix whose columns are $w_1$ and $w_2$,
then setting up the equation:
\[ 
\left(\begin{array}{rr} -1 & 2 \\ 2 & 1 \\ 1 & 3 \\ 5 & 0 
\end{array}\right) \vectwo{a}{b} = \left(\begin{array}{r} x_1 \\
x_2 \\ x_3 \\ x_4 \end{array}\right) 
\]
where $a$ and $b$ are scalars.  Then row reduce the associated
augmented matrix:
\[ 
\left(\begin{array}{rr|r} -1 & 2 & x_1 \\ 2 & 1 & x_2 \\ 1 & 3
& x_3 \\ 5 & 0 & x_4 \end{array}\right) \longrightarrow
\left(\begin{array}{rr|l} 1 & 3 & x_3 \\ 0 & -5 & x_2 - 2x_3 \\
0 & 0 & x_1 + x_2 - x_3 \\ 0 & 0 & -3x_2 + x_3 + x_4
\end{array}\right). 
\]
Extract from this solution the values that are independent of $a$
and $b$ to obtain the linear system above.


\end{solution}
\end{exercise}

\begin{exercise} \label{c5.2.8a}
Let $V$ be a vector space and let $v\in V$ be a nonzero vector.  Show that
\[
\Span\{v,v\}=\Span\{v\}.
\]

\begin{solution}

Every vector $x \in \Span\{v\}$ is of the form $x = av
= av + 0v$, so $x \in \Span\{v,v\}$.  Therefore, $\Span\{v\}
\subset \Span\{v,v\}$.  Every vector $y \in \Span\{v,v\}$ is of the
form 
\[
y = bv + cv = (b + c)v \in \Span\{v\}.
\]
Therefore $\Span\{v,v\} \subset \Span\{v,v\}$, so the two spans are equal.

\end{solution}
\end{exercise}
\begin{exercise} \label{c5.2.8b}
Let $V$ be a vector space and let $v,w\in V$ be vectors.  Show that
\[
\Span\{v,w\}=\Span\{v,w,v+3w\}.
\]

\begin{solution}
Every vector $x \in \Span\{v,w\}$ is of the form 
\[
x = av + bw = av + bw + 0(v + 3w) \in \Span\{v,w,v+3w\}.
\]
  Also, every vector $y \in \Span\{v,w,v+3w\}$ is of the form
\[
y = cv + dw + f(v + 3w) = (c + f)v + (d + 3f)w \in \Span\{v,w\}.
\]
Therefore, $\Span\{v,w\} = \Span\{v,w,v+3w\}$.

\end{solution}
\end{exercise}

\begin{exercise} \label{c5.2.9}
Let $W=\Span\{w_1,\ldots,w_k\}$ be a subspace of the vector
space $V$ and let $w_{k+1}\in W$ be another vector.  Prove that
$W=\Span\{w_1,\ldots,w_{k+1}\}$.

\begin{solution}

Since $W = \Span\{w_1,\dots ,w_k\}$, every vector $x \in W$ can be
written as the linear combination
\[ x = a_1w_1 + \cdots + a_kw_k \]
for some choice of $a_1 \dots a_k$.  Since $w_{k + 1}$ is a vector in
$W$, it can therefore be written as
\[ w_{k + 1} = b_1w_1 + \cdots + b_kw_k \]
and any vector $x \in W$ can be written as
\[ \begin{array}{rcl}
x & = &
a_1w_1 + \cdots + a_kw_k + a_{k+1}w_{k+1} \\
& = & a_1w_1 + \cdots + a_kw_k + a_{k+1}b_1w_1 + \cdots + a_{k+1}b_kw_k
\\ & = & (a_1 + a_{k+1}b_1)w_1 + \cdots + (a_k + a_{k+1}b_k)w_k.
\end{array} \]
So, $W = \Span\{w_1,\dots ,w_{k+1}\}$.

\end{solution}
\end{exercise}

\begin{exercise} \label{c5.2.10}
Let $Ax=b$ be a system of $m$ linear equations in $n$ unknowns,
and let $r=\rank(A)$ and $s=\rank(A|b)$.  Suppose that this system
has a unique solution.  What can you say about the relative
magnitudes of $m,n,r,s$?

\begin{solution}

\ans The relationship of the constants is $m \geq n = r = s$.

\soln The rank of matrix $A$ cannot be greater than the rank of matrix
$(A|b)$, since $(A|b)$ consists of $A$ plus one column.  The rank of $A$
is the number of pivots in the row reduced matrix.  $(A|b)$ can be row 
reduced through the same operations, and will have either the same number
of pivots as $A$ or, if there is a pivot in the last column, one more
pivot than $A$.  Since the system has a unique solution, it is consistent,
and therefore $(A|b)$ cannot have a pivot in the $(n + 1)^{st}$ column, so
$r = \rank(A) = \rank(A|b) = s$.

\para The set of solutions is parameterized by $n - r$ parameters,
where $n$ is the number of columns of $A$.  Since there is a unique
solution, the set of solutions is parameterized by $0$ parameters,
so $n = r$.

\para The number $m$ of rows of the matrix must be greater than or
equal to $n$ in order for the system to have a unique solution, since
there must be $n$ pivots, and each pivot must be in a separate row.



\end{solution}
\end{exercise}


\end{document}
