\documentclass{ximera}

 

\usepackage{epsfig}

\graphicspath{
  {./}
  {figures/}
}

\usepackage{morewrites}
\makeatletter
\newcommand\subfile[1]{%
\renewcommand{\input}[1]{}%
\begingroup\skip@preamble\otherinput{#1}\endgroup\par\vspace{\topsep}
\let\input\otherinput}
\makeatother

\newcommand{\includeexercises}{\directlua{dofile("/home/jim/linearAlgebra/laode/exercises.lua")}}

%\newcounter{ccounter}
%\setcounter{ccounter}{1}
%\newcommand{\Chapter}[1]{\setcounter{chapter}{\arabic{ccounter}}\chapter{#1}\addtocounter{ccounter}{1}}

%\newcommand{\section}[1]{\section{#1}\setcounter{thm}{0}\setcounter{equation}{0}}

%\renewcommand{\theequation}{\arabic{chapter}.\arabic{section}.\arabic{equation}}
%\renewcommand{\thefigure}{\arabic{chapter}.\arabic{figure}}
%\renewcommand{\thetable}{\arabic{chapter}.\arabic{table}}

%\newcommand{\Sec}[2]{\section{#1}\markright{\arabic{ccounter}.\arabic{section}.#2}\setcounter{equation}{0}\setcounter{thm}{0}\setcounter{figure}{0}}

\newcommand{\Sec}[2]{\section{#1}}

\setcounter{secnumdepth}{2}
%\setcounter{secnumdepth}{1} 

%\newcounter{THM}
%\renewcommand{\theTHM}{\arabic{chapter}.\arabic{section}}

\newcommand{\trademark}{{R\!\!\!\!\!\bigcirc}}
%\newtheorem{exercise}{}

\newcommand{\dfield}{{\sf dfield9}}
\newcommand{\pplane}{{\sf pplane9}}

\newcommand{\EXER}{\section*{Exercises}}%\vspace*{0.2in}\hrule\small\setcounter{exercise}{0}}
\newcommand{\CEXER}{}%\vspace{0.08in}\begin{center}Computer Exercises\end{center}}
\newcommand{\TEXER}{} %\vspace{0.08in}\begin{center}Hand Exercises\end{center}}
\newcommand{\AEXER}{} %\vspace{0.08in}\begin{center}Hand Exercises\end{center}}

% BADBAD: \newcommand{\Bbb}{\bf}

\newcommand{\R}{\mbox{$\Bbb{R}$}}
\newcommand{\C}{\mbox{$\Bbb{C}$}}
\newcommand{\Z}{\mbox{$\Bbb{Z}$}}
\newcommand{\N}{\mbox{$\Bbb{N}$}}
\newcommand{\D}{\mbox{{\bf D}}}
\usepackage{amssymb}
%\newcommand{\qed}{\hfill\mbox{\raggedright$\square$} \vspace{1ex}}
%\newcommand{\proof}{\noindent {\bf Proof:} \hspace{0.1in}}

\newcommand{\setmin}{\;\mbox{--}\;}
\newcommand{\Matlab}{{M\small{AT\-LAB}} }
\newcommand{\Matlabp}{{M\small{AT\-LAB}}}
\newcommand{\computer}{\Matlab Instructions}
\newcommand{\half}{\mbox{$\frac{1}{2}$}}
\newcommand{\compose}{\raisebox{.15ex}{\mbox{{\scriptsize$\circ$}}}}
\newcommand{\AND}{\quad\mbox{and}\quad}
\newcommand{\vect}[2]{\left(\begin{array}{c} #1_1 \\ \vdots \\
 #1_{#2}\end{array}\right)}
\newcommand{\mattwo}[4]{\left(\begin{array}{rr} #1 & #2\\ #3
&#4\end{array}\right)}
\newcommand{\mattwoc}[4]{\left(\begin{array}{cc} #1 & #2\\ #3
&#4\end{array}\right)}
\newcommand{\vectwo}[2]{\left(\begin{array}{r} #1 \\ #2\end{array}\right)}
\newcommand{\vectwoc}[2]{\left(\begin{array}{c} #1 \\ #2\end{array}\right)}

\newcommand{\ignore}[1]{}


\newcommand{\inv}{^{-1}}
\newcommand{\CC}{{\cal C}}
\newcommand{\CCone}{\CC^1}
\newcommand{\Span}{{\rm span}}
\newcommand{\rank}{{\rm rank}}
\newcommand{\trace}{{\rm tr}}
\newcommand{\RE}{{\rm Re}}
\newcommand{\IM}{{\rm Im}}
\newcommand{\nulls}{{\rm null\;space}}

\newcommand{\dps}{\displaystyle}
\newcommand{\arraystart}{\renewcommand{\arraystretch}{1.8}}
\newcommand{\arrayfinish}{\renewcommand{\arraystretch}{1.2}}
\newcommand{\Start}[1]{\vspace{0.08in}\noindent {\bf Section~\ref{#1}}}
\newcommand{\exer}[1]{\noindent {\bf \ref{#1}}}
\newcommand{\ans}{}
\newcommand{\matthree}[9]{\left(\begin{array}{rrr} #1 & #2 & #3 \\ #4 & #5 & #6
\\ #7 & #8 & #9\end{array}\right)}
\newcommand{\cvectwo}[2]{\left(\begin{array}{c} #1 \\ #2\end{array}\right)}
\newcommand{\cmatthree}[9]{\left(\begin{array}{ccc} #1 & #2 & #3 \\ #4 & #5 &
#6 \\ #7 & #8 & #9\end{array}\right)}
\newcommand{\vecthree}[3]{\left(\begin{array}{r} #1 \\ #2 \\
#3\end{array}\right)}
\newcommand{\cvecthree}[3]{\left(\begin{array}{c} #1 \\ #2 \\
#3\end{array}\right)}
\newcommand{\cmattwo}[4]{\left(\begin{array}{cc} #1 & #2\\ #3
&#4\end{array}\right)}

\newcommand{\Matrix}[1]{\ensuremath{\left(\begin{array}{rrrrrrrrrrrrrrrrrr} #1 \end{array}\right)}}

\newcommand{\Matrixc}[1]{\ensuremath{\left(\begin{array}{cccccccccccc} #1 \end{array}\right)}}



\renewcommand{\labelenumi}{\theenumi)}
\newenvironment{enumeratea}%
{\begingroup
 \renewcommand{\theenumi}{\alph{enumi}}
 \renewcommand{\labelenumi}{(\theenumi)}
 \begin{enumerate}}
 {\end{enumerate}\endgroup}



\newcounter{help}
\renewcommand{\thehelp}{\thesection.\arabic{equation}}

%\newenvironment{equation*}%
%{\renewcommand\endequation{\eqno (\theequation)* $$}%
%   \begin{equation}}%
%   {\end{equation}\renewcommand\endequation{\eqno \@eqnnum
%$$\global\@ignoretrue}}

%\input{psfig.tex}

\author{Martin Golubitsky and Michael Dellnitz}

%\newenvironment{matlabEquation}%
%{\renewcommand\endequation{\eqno (\theequation*) $$}%
%   \begin{equation}}%
%   {\end{equation}\renewcommand\endequation{\eqno \@eqnnum
% $$\global\@ignoretrue}}

\newcommand{\soln}{\textbf{Solution:} }
\newcommand{\exercap}[1]{\centerline{Figure~\ref{#1}}}
\newcommand{\exercaptwo}[1]{\centerline{Figure~\ref{#1}a\hspace{2.1in}
Figure~\ref{#1}b}}
\newcommand{\exercapthree}[1]{\centerline{Figure~\ref{#1}a\hspace{1.2in}
Figure~\ref{#1}b\hspace{1.2in}Figure~\ref{#1}c}}
\newcommand{\para}{\hspace{0.4in}}

\renewenvironment{solution}{\suppress}{\endsuppress}

\ifxake
\newenvironment{matlabEquation}{\begin{equation}}{\end{equation}}
\else
\newenvironment{matlabEquation}%
{\let\oldtheequation\theequation\renewcommand{\theequation}{\oldtheequation*}\begin{equation}}%
  {\end{equation}\let\theequation\oldtheequation}
\fi

\makeatother


\title{Dimension and Bases}

\begin{document}
\begin{abstract}
\end{abstract}
\maketitle

 \label{S:5.5}

The minimum number of vectors that span a vector space has special
significance.

\begin{definition}
The vector space $V$ has {\em finite dimension\/} if $V$ is the
span of a finite number of vectors.  If $V$ has finite dimension, then the
smallest number of vectors that span $V$ is called the {\em dimension\/} of
$V$ and is denoted by $\dim V$.
\end{definition} \index{dimension}\index{dimension!finite} \index{span}

For example, recall that $e_j$ is the vector in $\R^n$ whose $j^{th}$
component is $1$ and all of whose other components are $0$.
Let $x=(x_1,\ldots,x_n)$ be in $\R^n$. Then
\begin{equation}  \label{e:spanrn}
x =  x_1e_1 + \cdots + x_ne_n.
\end{equation}
Since every vector in $\R^n$ is a linear combination of the
vectors $e_1,\ldots,e_n$, it follows that
$\R^n=\Span\{e_1,\ldots,e_n\}$.  Thus, $\R^n$ is finite
dimensional.  Moreover, the dimension of $\R^n$ is at most $n$,
since $\R^n$ is spanned by $n$ vectors. It seems unlikely that
$\R^n$ could be spanned by fewer than $n$ vectors--- but this
point needs to be proved.

\subsubsection*{An Example of a Vector Space that is Not Finite Dimensional}
\index{dimension!infinite}

Next we discuss an example of a vector space that does not have finite 
dimension.  Consider the subspace ${\cal P}\subset\CCone$ consisting of 
polynomials \index{subspace!of polynomials} of all degrees.  We show that 
${\cal P}$ is not the span of a finite number of vectors and hence that 
${\cal P}$ does not have finite dimension.  Let $p_1(t),p_2(t),\ldots,p_k(t)$ 
be a set of $k$ polynomials and let $d$ be the maximum degree of these $k$ 
polynomials.  Then every polynomial in the span of $p_1(t),\ldots,p_k(t)$ has 
degree less than or equal to $d$.  In particular, $p(t)=t^{d+1}$ is a 
polynomial that is not in the span of $p_1(t),\ldots,p_k(t)$ and ${\cal P}$ 
is not spanned by finitely many vectors.


\subsection*{Bases and The Main Theorem}

\begin{definition} \label{basis}
Let ${\cal B} = \{w_1,\ldots,w_k\}$ be a set of vectors in a
vector space $W$.  The subset ${\cal B}$ is a {\em basis\/} for $W$
if ${\cal B}$ is a spanning set for $W$ with the smallest number
of elements in a spanning set for $W$.
\end{definition} \index{basis} \index{spanning set}

It follows that if $\{w_1,\ldots,w_k\}$ is a basis for $W$, then
$k=\dim W$. The main theorem about bases is:

\begin{theorem}  \label{basis=span+indep}
A set of vectors ${\cal B} =\{w_1,\ldots,w_k\}$ in a vector space $W$
is a basis for $W$ if and only if the set ${\cal B}$ is linearly
independent and spans $W$.
\end{theorem} \index{linearly!independent} \index{basis} \index{span}

\noindent {\bf Remark:}  The importance of Theorem~\ref{basis=span+indep} is
that we can show that a set of vectors is a basis by verifying spanning
and linear independence.   We never have to check directly that the spanning
set has the minimum number of vectors for a spanning set.


For example, we have shown previously that the set of vectors
$\{e_1,\ldots,e_n\}$ in $\R^n$ is linearly independent and spans $\R^n$.  It
follows from Theorem~\ref{basis=span+indep} that this set is a basis,
and that the dimension of $\R^n$ is $n$\index{dimension!of $\R^n$}.
In particular, $\R^n$ cannot be spanned by fewer than $n$ vectors.

The proof of Theorem~\ref{basis=span+indep} is given in Section~\ref{S:5.6}.


\subsection*{Consequences of Theorem~\protect\ref{basis=span+indep}}

We discuss two applications of Theorem~\ref{basis=span+indep}.  First,
we use this theorem to derive a way of determining the dimension of the
subspace spanned by a finite number of vectors.  Second, we show that the
dimension of the subspace of solutions to a homogeneous system of linear
equation $Ax=0$ is $n-\rank(A)$ where $A$ is an $m\times n$ matrix.

\subsubsection*{Computing the Dimension of a Span} \index{dimension}

We show that the dimension of a span of vectors can be found using
elementary row operations on $M$. \index{elementary row operations}

\begin{lemma}  \label{L:computerank}
Let $w_1,\ldots,w_k$ be $k$ row vectors in $\R^n$ and let
$W=\Span\{w_1,\ldots,w_k\}\subset\R^n$.  Define
\[
M =\left(\begin{array}{c} w_1\\ \vdots \\w_k \end{array}\right)
\]
to be the matrix whose rows are the $w_j$s.  Then
\begin{equation}  \label{e:dimW=rankM}
\dim(W) = \rank(M).
\end{equation}
\end{lemma}\index{dimension}\index{rank}

\begin{proof} To verify \Ref{e:dimW=rankM}, observe that the span of
$w_1,\ldots,w_k$ is unchanged by
\begin{itemize}
\item[(a)] swapping $w_i$ and $w_j$,
\item[(b)] multiplying $w_i$ by a nonzero scalar, and
\item[(c)] adding a multiple of $w_i$ to $w_j$.
\end{itemize}
That is, if we perform elementary row operations on $M$, the
vector space spanned by the rows of $M$ does not change. So we
may perform elementary row operations on $M$ until we arrive at
the matrix $E$ in reduced echelon form.  \index{echelon form!reduced} 
Suppose that $\ell=\rank(M)$; that is, suppose that $\ell$
is the number of nonzero rows in $E$.  Then
\[
E =\left(\begin{array}{c} v_1\\ \vdots \\v_\ell\\ 0 \\ \vdots
\\ 0 \end{array}\right),
\]
where the $v_j$ are the nonzero rows in the reduced echelon form
matrix.

We claim that the vectors $v_1,\ldots,v_\ell$ are linearly
independent.  It then follows from Theorem~\ref{basis=span+indep} that
$\{v_1,\ldots,v_\ell\}$ is a basis for $W$ and that the dimension of
$W$ is $\ell$.  To verify the claim, suppose
\begin{equation} \label{e:rowsums}
a_1v_1 + \cdots + a_\ell v_\ell = 0.
\end{equation}
We show that $a_i$ must equal $0$ as follows.  In the $i^{th}$
row, the pivot\index{pivot} must occur in some column --- say in the $j^{th}$
column.  It follows that the $j^{th}$ entry in the vector of the
left hand side of \Ref{e:rowsums} is
\[
0a_1 + \cdots + 0a_{i-1} +1a_i + 0a_{i+1} + \cdots + 0a_\ell =
a_i,
\]
since all entries in the $j^{th}$ column of $E$ other than the
pivot must be zero, as $E$ is in reduced echelon form.  \end{proof}

For instance, let $W=\Span\{w_1,w_2,w_3\}$ in $\R^4$ where
\begin{align*} \label{eq:vectors}
  w_1 &= (3, -2, 1,-1), \\
  w_2 &= (1,5,10,12), \\
  w_3 &= (1,-12,-19,-25).
\end{align*}%
To compute $\dim W$ in \Matlab, type \verb+e5_5_4+ to load the
vectors and type
\begin{verbatim}
M = [w1; w2; w3]
\end{verbatim}
Row reduction\index{row!reduction} of the matrix {\tt M} in \Matlab
leads to the reduced echelon form matrix
\begin{verbatim}
ans =
     1.0000         0    1.4706    1.1176
         0     1.0000    1.7059    2.1765
         0          0         0         0
\end{verbatim}
indicating that the dimension of the subspace $W$ is two, and
therefore $\{w_1,w_2,w_3\}$ is not a basis of $W$. Alternatively,
we can use the \Matlab command {\tt rank(M)}\index{\computer!rank}
to compute the rank of $M$ and the dimension of the span $W$.

However, if we change one of the entries in $w_3$, for instance
{\tt w3(3)=-18} then indeed the command {\tt rank([w1;w2;w3])}
gives the answer three indicating that for this choice of vectors
$\{w1,w2,w3\}$ is a basis for $\Span\{w1,w2,w3\}$.

\subsubsection*{Solutions to Homogeneous Systems Revisited}
\index{homogeneous}

We return to our discussions in Chapter~\ref{lineq} on solving
linear equations.  Recall that we can write all solutions to
the system of homogeneous equations $Ax=0$ in terms of a few
parameters, and that the null space of $A$ is the subspace of
solutions (See Definition~\ref{D:nullspace}).
More precisely, Proposition~\ref{P:n-rank} states that the number of
parameters needed is $n-\rank(A)$ where $n$ is the number of
variables in the homogeneous system.  We claim that the dimension
of the null space\index{dimension!of null space} is exactly
$n - \rank(A)$.

For example, consider the reduced echelon form $3\times 7$ matrix
\begin{equation}  \label{E:nullityexamp}
A=\left(\begin{array}{rrrrrrr}
1 & -4 & 0 &  2 & -3 & 0 & 8 \\
0 &  0 & 1 &  3 &  2 & 0 & 4 \\
0 &  0 & 0 &  0 &  0 & 1 & 2  \end{array}\right)
\end{equation}
that has rank three. Suppose that the unknowns for this system of
equations are
$x_1,\ldots,x_7$.  We can solve the equations associated with
$A$ by solving the first equation for $x_1$, the second equation
for $x_3$, and the third equation for $x_6$, as follows:
\begin{eqnarray*}
x_1 & = & 4x_2 - 2x_4 + 3x_5 - 8x_7 \\
x_3 & = &      - 3x_4 - 2x_5 - 4x_7 \\
x_6 & = &                    - 2x_7
\end{eqnarray*}
Thus, all solutions to this system of equations have the form
\begin{equation}   \label{e:expandsoln}
\left(\begin{array}{c} 4x_2 - 2x_4 + 3x_5 - 8x_7 \\ x_2 \\
        -3x_4 - 2x_5 - 4x_7 \\ x_4 \\ x_5 \\  - 2x_7 \\ x_7 \end{array} \right)\end{equation}
  which equals
  \begin{equation*}
x_2 \left(\begin{array}{r}  4 \\ 1 \\  0 \\ 0 \\ 0 \\  0 \\ 0
\end{array} \right) +
x_4 \left(\begin{array}{r} -2 \\ 0 \\ -3 \\ 1 \\ 0 \\  0 \\ 0
\end{array} \right) +
x_5 \left(\begin{array}{r}  3 \\ 0 \\ -2 \\ 0 \\ 1 \\  0 \\ 0
\end{array} \right) +
x_7 \left(\begin{array}{r} -8 \\ 0 \\ -4 \\ 0 \\ 0 \\ -2 \\ 1
\end{array} \right).
\end{equation*}
\noindent We can rewrite the right hand side of \Ref{e:expandsoln}
as a linear combination\index{linear!combination} of four
vectors $w_2,w_4,w_5,w_7$
\begin{equation}   \label{e:w'scomb}
x_2w_2 + x_4w_4 + x_5w_5 + x_7w_7.
\end{equation}

This calculation shows that the null space of $A$, which is
$W=\{x\in\R^7:Ax=0\}$, is spanned by the four vectors
$w_2,w_4,w_5,w_7$.  Moreover, this same calculation shows that
the four vectors  are linearly independent.
From the left hand side of \Ref{e:expandsoln} we see that if this
linear combination sums to zero, then $x_2=x_4=x_5=x_7=0$.  It
follows from Theorem~\ref{basis=span+indep} that $\dim W = 4$.

\begin{definition}  \label{D:nullity}
The {\em nullity\/} of $A$ is the dimension of the null space of $A$.
\end{definition} \index{nullity}\index{null space!dimension}

\begin{theorem}  \label{T:dimsoln}
Let $A$ be an $m\times n$ matrix. Then
\[
{\rm nullity}(A) + \rank(A) = n.
\]
\end{theorem} \index{rank}

\begin{proof}	Neither the rank nor the null space of $A$ are changed by
elementary row operations.  So we can assume that $A$ is in reduced
echelon form.  The rank of $A$ is the number of nonzero rows in
the reduced echelon form matrix.  Proposition~\ref{P:n-rank} states that
the null space is spanned by $p$ vectors where $p=n-\rank(A)$.  We
must show that these vectors are linearly independent.

Let $j_1,\ldots,j_p$ be the columns of $A$ that do not contain pivots.
In example \Ref{E:nullityexamp} $p=4$ and
\[
j_1 = 2, \qquad j_2 = 4, \qquad j_3 = 5, \qquad j_4 = 7.
\]
After solving for the variables corresponding to pivots, we find that
the spanning set of the null space consists of $p$ vectors in $\R^n$,
which we label as $\{w_{j_1},\ldots,w_{j_p}\}$.  See \Ref{e:expandsoln}.
Note that the $j_m$$^{th}$  entry of $w_{j_m}$ is $1$ while the
$j_m$$^{th}$ entry in all of the other $p-1$ vectors is $0$.  Again,
see \Ref{e:expandsoln} as an example that supports this statement.  It
follows that the set of spanning vectors is a linearly independent set.
That is, suppose that
\[
r_1w_{j_1} + \cdots + r_pw_{j_p} = 0.
\]
From the $j_m$$^{th}$ entry in this equation, it follows that $r_m=0$;
and the vectors are linearly independent.  \end{proof}

Theorem~\ref{T:dimsoln} has an interesting and useful interpretation.
We have seen in the previous subsection that the rank of a matrix $A$
is just the number of linearly independent rows in $A$.
In linear systems each row of the coefficient matrix corresponds
to a linear equation.  Thus, the rank of $A$ may be thought of as the
number of independent equations in a system of linear equations.
This theorem just states that the space of solutions loses a dimension
for each independent equation.


\EXER

\TEXER

\begin{exercise} \label{c5.5.1}
Show that ${\cal U}=\{u_1,u_2,u_3\}$ where
\[
u_1=(1,1,0) \quad u_2=(0,1,0) \quad u_3=(-1,0,1)
\]
is a basis for $\R^3$.
\end{exercise}


\begin{exercise} \label{c5.5.2}
Let $S=\Span\{v_1,v_2,v_3\}$ where
\[
v_1=(1,0,-1,0) \quad v_2=(0,1,1,1) \quad v_3=(5,4,-1,4).
\]
Find the dimension of $S$ and find a basis for $S$.
\end{exercise}

\begin{exercise} \label{c5.5.3}
Find a basis for the null space of
\[
A =\left(\begin{array}{rrrr} 1 & 0 & -1 & 2\\ 1 & -1 & 0 & 0\\
4 & -5 & 1 & -2 \end{array} \right).
\]
What is the dimension of the null space of $A$?
\end{exercise}

\begin{exercise} \label{c5.5.4}
Show that the set $V$ of all $2\times 2$ matrices is a vector space.
Show that the dimension of $V$ is four by finding a basis of $V$
with four elements.  Show that the space $M(m,n)$ of all $m\times n$
matrices is also a vector space.  What is $\dim M(m,n)$?
\end{exercise}

\begin{exercise} \label{c5.5.5}
Show that the set ${\cal P}_n$ of all polynomials of degree less than
or equal to $n$ is a subspace of $\CCone$.  What is $\dim {\cal P}_2$?
What is $\dim {\cal P}_n$?
\end{exercise}

\begin{exercise} \label{c5.5.6}
Let ${\cal P}_3$ be the vector space of polynomials of degree at
most three in one variable $t$.  Let $p(t)=t^3+a_2t^2+a_1t+a_0$ where
$a_0,a_1,a_2\in\R$ are fixed constants.  Show that
\[
\left\{ p, \frac{dp}{dt}, \frac{d^2p}{dt^2}, \frac{d^3p}{dt^3}\right\}
\]
is a basis for ${\cal P}_3$.
\end{exercise}

\begin{exercise} \label{c5.5.7}
Let $u\in\R^n$ be a nonzero row vector.
\begin{itemize}
\item[(a)]  Show that the $n\times n$ matrix $A=u^tu$ is symmetric and that
$\rank(A)=1$.  {\bf Hint:}  Begin by showing that $Av^t=0$ for every vector
$v\in\R^n$ that is perpendicular to $u$ and that $Au^t$ is a nonzero multiple
of $u^t$.
\item[(b)]  Show that the matrix $P=I_n+u^tu$ is invertible.  {\bf Hint:}
Show that $\rank(P)=n$.
\end{itemize}
\end{exercise}


\end{document}

%%% Local Variables:
%%% mode: latex
%%% TeX-master: "../linearAlgebra"
%%% End:
