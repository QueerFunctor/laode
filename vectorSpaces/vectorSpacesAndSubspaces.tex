\documentclass{ximera}

 

\usepackage{epsfig}

\graphicspath{
  {./}
  {figures/}
}

\usepackage{morewrites}
\makeatletter
\newcommand\subfile[1]{%
\renewcommand{\input}[1]{}%
\begingroup\skip@preamble\otherinput{#1}\endgroup\par\vspace{\topsep}
\let\input\otherinput}
\makeatother

\newcommand{\includeexercises}{\directlua{dofile("/home/jim/linearAlgebra/laode/exercises.lua")}}

%\newcounter{ccounter}
%\setcounter{ccounter}{1}
%\newcommand{\Chapter}[1]{\setcounter{chapter}{\arabic{ccounter}}\chapter{#1}\addtocounter{ccounter}{1}}

%\newcommand{\section}[1]{\section{#1}\setcounter{thm}{0}\setcounter{equation}{0}}

%\renewcommand{\theequation}{\arabic{chapter}.\arabic{section}.\arabic{equation}}
%\renewcommand{\thefigure}{\arabic{chapter}.\arabic{figure}}
%\renewcommand{\thetable}{\arabic{chapter}.\arabic{table}}

%\newcommand{\Sec}[2]{\section{#1}\markright{\arabic{ccounter}.\arabic{section}.#2}\setcounter{equation}{0}\setcounter{thm}{0}\setcounter{figure}{0}}

\newcommand{\Sec}[2]{\section{#1}}

\setcounter{secnumdepth}{2}
%\setcounter{secnumdepth}{1} 

%\newcounter{THM}
%\renewcommand{\theTHM}{\arabic{chapter}.\arabic{section}}

\newcommand{\trademark}{{R\!\!\!\!\!\bigcirc}}
%\newtheorem{exercise}{}

\newcommand{\dfield}{{\sf dfield9}}
\newcommand{\pplane}{{\sf pplane9}}

\newcommand{\EXER}{\section*{Exercises}}%\vspace*{0.2in}\hrule\small\setcounter{exercise}{0}}
\newcommand{\CEXER}{}%\vspace{0.08in}\begin{center}Computer Exercises\end{center}}
\newcommand{\TEXER}{} %\vspace{0.08in}\begin{center}Hand Exercises\end{center}}
\newcommand{\AEXER}{} %\vspace{0.08in}\begin{center}Hand Exercises\end{center}}

% BADBAD: \newcommand{\Bbb}{\bf}

\newcommand{\R}{\mbox{$\Bbb{R}$}}
\newcommand{\C}{\mbox{$\Bbb{C}$}}
\newcommand{\Z}{\mbox{$\Bbb{Z}$}}
\newcommand{\N}{\mbox{$\Bbb{N}$}}
\newcommand{\D}{\mbox{{\bf D}}}
\usepackage{amssymb}
%\newcommand{\qed}{\hfill\mbox{\raggedright$\square$} \vspace{1ex}}
%\newcommand{\proof}{\noindent {\bf Proof:} \hspace{0.1in}}

\newcommand{\setmin}{\;\mbox{--}\;}
\newcommand{\Matlab}{{M\small{AT\-LAB}} }
\newcommand{\Matlabp}{{M\small{AT\-LAB}}}
\newcommand{\computer}{\Matlab Instructions}
\newcommand{\half}{\mbox{$\frac{1}{2}$}}
\newcommand{\compose}{\raisebox{.15ex}{\mbox{{\scriptsize$\circ$}}}}
\newcommand{\AND}{\quad\mbox{and}\quad}
\newcommand{\vect}[2]{\left(\begin{array}{c} #1_1 \\ \vdots \\
 #1_{#2}\end{array}\right)}
\newcommand{\mattwo}[4]{\left(\begin{array}{rr} #1 & #2\\ #3
&#4\end{array}\right)}
\newcommand{\mattwoc}[4]{\left(\begin{array}{cc} #1 & #2\\ #3
&#4\end{array}\right)}
\newcommand{\vectwo}[2]{\left(\begin{array}{r} #1 \\ #2\end{array}\right)}
\newcommand{\vectwoc}[2]{\left(\begin{array}{c} #1 \\ #2\end{array}\right)}

\newcommand{\ignore}[1]{}


\newcommand{\inv}{^{-1}}
\newcommand{\CC}{{\cal C}}
\newcommand{\CCone}{\CC^1}
\newcommand{\Span}{{\rm span}}
\newcommand{\rank}{{\rm rank}}
\newcommand{\trace}{{\rm tr}}
\newcommand{\RE}{{\rm Re}}
\newcommand{\IM}{{\rm Im}}
\newcommand{\nulls}{{\rm null\;space}}

\newcommand{\dps}{\displaystyle}
\newcommand{\arraystart}{\renewcommand{\arraystretch}{1.8}}
\newcommand{\arrayfinish}{\renewcommand{\arraystretch}{1.2}}
\newcommand{\Start}[1]{\vspace{0.08in}\noindent {\bf Section~\ref{#1}}}
\newcommand{\exer}[1]{\noindent {\bf \ref{#1}}}
\newcommand{\ans}{}
\newcommand{\matthree}[9]{\left(\begin{array}{rrr} #1 & #2 & #3 \\ #4 & #5 & #6
\\ #7 & #8 & #9\end{array}\right)}
\newcommand{\cvectwo}[2]{\left(\begin{array}{c} #1 \\ #2\end{array}\right)}
\newcommand{\cmatthree}[9]{\left(\begin{array}{ccc} #1 & #2 & #3 \\ #4 & #5 &
#6 \\ #7 & #8 & #9\end{array}\right)}
\newcommand{\vecthree}[3]{\left(\begin{array}{r} #1 \\ #2 \\
#3\end{array}\right)}
\newcommand{\cvecthree}[3]{\left(\begin{array}{c} #1 \\ #2 \\
#3\end{array}\right)}
\newcommand{\cmattwo}[4]{\left(\begin{array}{cc} #1 & #2\\ #3
&#4\end{array}\right)}

\newcommand{\Matrix}[1]{\ensuremath{\left(\begin{array}{rrrrrrrrrrrrrrrrrr} #1 \end{array}\right)}}

\newcommand{\Matrixc}[1]{\ensuremath{\left(\begin{array}{cccccccccccc} #1 \end{array}\right)}}



\renewcommand{\labelenumi}{\theenumi)}
\newenvironment{enumeratea}%
{\begingroup
 \renewcommand{\theenumi}{\alph{enumi}}
 \renewcommand{\labelenumi}{(\theenumi)}
 \begin{enumerate}}
 {\end{enumerate}\endgroup}



\newcounter{help}
\renewcommand{\thehelp}{\thesection.\arabic{equation}}

%\newenvironment{equation*}%
%{\renewcommand\endequation{\eqno (\theequation)* $$}%
%   \begin{equation}}%
%   {\end{equation}\renewcommand\endequation{\eqno \@eqnnum
%$$\global\@ignoretrue}}

%\input{psfig.tex}

\author{Martin Golubitsky and Michael Dellnitz}

%\newenvironment{matlabEquation}%
%{\renewcommand\endequation{\eqno (\theequation*) $$}%
%   \begin{equation}}%
%   {\end{equation}\renewcommand\endequation{\eqno \@eqnnum
% $$\global\@ignoretrue}}

\newcommand{\soln}{\textbf{Solution:} }
\newcommand{\exercap}[1]{\centerline{Figure~\ref{#1}}}
\newcommand{\exercaptwo}[1]{\centerline{Figure~\ref{#1}a\hspace{2.1in}
Figure~\ref{#1}b}}
\newcommand{\exercapthree}[1]{\centerline{Figure~\ref{#1}a\hspace{1.2in}
Figure~\ref{#1}b\hspace{1.2in}Figure~\ref{#1}c}}
\newcommand{\para}{\hspace{0.4in}}

\renewenvironment{solution}{\suppress}{\endsuppress}

\ifxake
\newenvironment{matlabEquation}{\begin{equation}}{\end{equation}}
\else
\newenvironment{matlabEquation}%
{\let\oldtheequation\theequation\renewcommand{\theequation}{\oldtheequation*}\begin{equation}}%
  {\end{equation}\let\theequation\oldtheequation}
\fi

\makeatother


\title{Vector Spaces and Subspaces}

\begin{document}
\begin{abstract}
\end{abstract}
\maketitle

 \label{S:5.1}

Vector spaces abstract the arithmetic properties of addition and
scalar multiplication of vectors.  In $\R^n$ we know how to add
vectors and to multiply vectors by scalars.  Indeed, it is
straightforward to verify that each of the eight
properties listed in Table~\ref{vectorspacelist} is valid for
vectors in $V=\R^n$.  Remarkably, sets that satisfy these eight
properties have much in common with $\R^n$.  So we define:
\begin{Def}  \label{vectorspace}
Let $V$ be a set having the two operations of addition and scalar
multiplication.  Then $V$ is a {\em vector space\/} if the eight
properties listed in Table~\ref{vectorspace} hold.  The elements
of a vector space are called {\em vectors}\index{vector}.
\end{Def} \index{vector!space}\index{vector!space}

The vector $0$ mentioned in (A3) in Table~\ref{vectorspacelist}
is called the {\em zero vector}\index{zero vector}.

\begin{table}
\caption{Properties of Vector Spaces: suppose $u,v,w\in V$ and
$r,s\in\R$.}
\begin{center}
\begin{tabular}{|c|c|c|}
\hline
(A1) & Addition is commutative & $v+w=w+v$ \\
\hline
(A2) & Addition is associative & $(u+v)+w = u+(v+w)$ \\
\hline
(A3) & Additive identity $0$ exists & $v+0=v$ \\
\hline
(A4) & Additive inverse $-v$ exists & $v+(-v) = 0$ \\
\hline
(M1) & Multiplication is associative & $(rs)v = r(sv)$ \\
\hline
(M2) & Multiplicative identity exists & $1v=v$ \\
\hline
(D1) & Distributive law for scalars & $(r+s)v = rv+sv$ \\
\hline
(D2) & Distributive law for vectors & $r(v+w) = rv+rw$ \\
\hline
\end{tabular}
\end{center} \index{scalar multiplication} \index{vector!addition}
\index{associative} \index{distributive}
\index{commutative} \index{inverse}
\label{vectorspacelist}
\end{table}

When we say that a vector space $V$ has the two operations of
addition and scalar multiplication we mean that the sum of two
vectors in $V$ is again a vector in $V$ and the scalar product
of a vector with a number is again a vector in $V$.  These two
properties are called {\em closure under addition\/} and
{\em closure under scalar multiplication}.
\index{closure!under addition}
\index{closure!under scalar multiplication}

In this discussion we focus on just two types of vector spaces:
$\R^n$ and function spaces.  The reason that we make this choice
is that solutions to linear equations are vectors in $\R^n$ while
solutions to linear systems of differential equations are vectors
of functions.

\subsubsection*{An Example of a Function Space}
\index{function space}

For example, let ${\cal F}$ denote the set of all functions $f:\R\to\R$.
Note that functions like $f_1(t)=t^2-2t+7$ and $f_2(t)=\sin t$ are in
${\cal F}$ since they are defined for all real numbers $t$, but that functions
like $g_1(t)=\frac{1}{t}$ and $g_2(t)=\tan t$ are not in ${\cal F}$ since
they are not defined for all $t$.

We can add two functions $f$ and $g$ by defining the function $f+g$ to be:
\[
(f+g)(t) = f(t) + g(t).
\]
We can also multiply a function $f$ by a scalar $c\in\R$ by defining the
function $cf$ to be:
\[
(cf)(t) = cf(t).
\]
With these operations of addition and scalar multiplication, ${\cal F}$
is a vector space; that is, ${\cal F}$ satisfies the eight vector space
properties in Table~\ref{vectorspacelist}.  More precisely:
\begin{itemize}
\item[(A3)] Define the zero function ${\cal O}$ by
\[
     {\cal O}(t) = 0 \quad \mbox{for all $t\in\R$.}
\]
For every $x$ in ${\cal F}$ the function ${\cal O}$ satisfies:
\[
     (x+{\cal O})(t) = x(t) + {\cal O}(t) = x(t) + 0 = x(t).
\]
Therefore, $x+{\cal O}=x$ and ${\cal O}$ is the additive
identity in ${\cal F}$.
\item[(A4)] Let $x$ be a function in ${\cal F}$ and define $y(t)=-x(t)$.
Then $y$ is also a function in ${\cal F}$, and
\[
    (x+y)(t) =  x(t) + y(t) = x(t) + (-x(t)) = 0 = {\cal O}(t).
\]
Thus, $x$ has the additive inverse $-x$.
\end{itemize}
After these comments it is straightforward to verify that the
remaining six properties in Table~\ref{vectorspacelist} are
satisfied by functions in ${\cal F}$.

\subsubsection*{Sets that are not Vector Spaces}
\index{vector!space}\index{vector!space}

It is worth considering how closure under vector addition and
scalar multiplication can fail.  Consider the following three
examples.

\begin{enumerate}
\item[(i)] Let $V_1$ be the set that consists of just the $x$
and $y$ axes in the plane.  Since $(1,0)$ and $(0,1)$ are in
$V_1$ but
\[
(1,0) + (0,1) = (1,1)
\]
is not in $V_1$, we see that $V_1$ is not closed under vector
addition. On the other hand, $V_1$ is closed under scalar
multiplication.  \index{closure!under addition}
\index{closure!under scalar multiplication}


\item[(ii)] Let $V_2$ be the set of all vectors $(k,\ell)\in\R^2$
where $k$ and $\ell$ are integers.  The set $V_2$ is closed under
addition but not under scalar multiplication since
$\half(1,0)=(\half,0)$ is not in $V_2$.

\item[(iii)] Let $V_3=[1,2]$ be the closed interval in $\R$. The
set $V_3$ is neither closed under addition ($1+1.5=2.5\not\in
V_3$) nor under scalar multiplication ($4\cdot 1.5 = 6\not\in
V_3$).  Hence the set $V_3$ is not closed under vector addition
and not closed under scalar multiplication.
\end{enumerate}

\subsection*{Subspaces}

\begin{Def} \label{subspaces}
Let $V$ be a vector space.  A nonempty subset $W\subset V$ is a
{\em subspace\/} if $W$ is a vector space using the operations of 
addition and scalar multiplication defined on $V$.
\end{Def} \index{subspace}

Note that in order for a subset $W$ of a vector space $V$ to be a 
subspace it must be {\em closed under addition\/} and {\em closed under
scalar multiplication\/}.  That is, suppose $w_1,w_2\in W$ and 
$r\in\R$.  Then
\begin{itemize}
\item[(i)]  $w_1+w_2 \in W$, and
\item[(ii)]  $rw_1\in W$.
\end{itemize}
\index{vector!addition} \index{scalar multiplication}

The $x$-axis and the $xz$-plane are examples of subsets of $\R^3$
that are closed under addition and closed under scalar multiplication.
Every vector on the $x$-axis has the form $(a,0,0)\in\R^3$.
The sum of two vectors $(a,0,0)$ and $(b,0,0)$ on the $x$-axis is
$(a+b,0,0)$ which is also on the $x$-axis.  The $x$-axis
is also closed under scalar multiplication as $r(a,0,0)=(ra,0,0)$,
and the $x$-axis is a subspace of $\R^3$.  Similarly, every vector
in the $xz$-plane in $\R^3$ has the form $(a_1,0,a_3)$. As in the
case of the $x$-axis, it is easy to verify that this set of vectors
is closed under addition and scalar multiplication.  Thus, the
$xz$-plane is also a subspace of $\R^3$.

In Theorem~\ref{T:subspaces} we show that every subset of a vector space 
that is closed under addition and scalar multiplication is a subspace.  
To verify this statement, we need the following lemma in which some special 
notation is used.  Typically, we use the same notation $0$ to denote the
real number zero and the zero vector\index{zero vector}.  In the following
lemma it is convenient to distinguish the two different uses of $0$, and we 
write the zero vector in boldface.

\begin{lemma}  \label{lem:AddId}
Let $V$ be a vector space, and let ${\bf 0}\in V$ be the zero vector.  Then
\[
0v={\bf 0} \quad \mbox{and}\quad (-1)v=-v
\]
for every vector in $v\in V$.
\end{lemma}

\begin{proof} Let $v$ be a vector in $V$ and use (D1) to compute
\[
     0v + 0v  = (0+0)v = 0v.
\]
By (A4) the vector $0v$ has an additive inverse $-0v$.  Adding $-0v$ 
to both sides yields 
\[
(0v + 0v) + (-0v) = 0v + (-0v) = {\bf 0}.
\] 
Associativity of addition (A2) now implies 
\[
0v + (0v + (-0v)) = {\bf 0}.
\]
A second application of (A4) implies that 
\[
0v + {\bf 0} = {\bf 0}
\]
and (A3) implies that $0v={\bf 0}$.

Next, we show that the additive inverse $-v$ of a vector $v$ is unique. 
That is, if $v + a = {\bf 0}$, then $a=-v$.  

Before beginning the proof, note that commutativity of addition (A1) together 
with (A3) implies that ${\bf 0}+v=v$.  Similarly, (A1) and (A4) imply that 
$-v+v={\bf 0}$.  

To prove uniqueness of additive inverses, add $-v$ to both sides of the 
equation $v + a = {\bf 0}$ yielding 
\[
-v + (v + a) = -v + {\bf 0}. 
\]
Properties (A2) and (A3) imply
\[
(-v + v) + a = -v.
\]
But 
\[
(-v + v) + a = {\bf 0} + a = a.
\]
Therefore $a = -v$, as claimed.

To verify that $(-1)v = -v$, we show that $(-1)v$ is the
additive inverse of $v$.  Using (M1), (D1), and the fact that $0v = {\bf 0}$, 
calculate
\[
     v + (-1)v = 1v + (-1)v =(1-1)v = 0v = {\bf 0}.
\]
Thus, $(-1)v$ is the additive inverse of $v$ and must equal $-v$, as
claimed.  \end{proof}

\begin{thm}  \label{T:subspaces}
Let $W$ be a subset of the vector space $V$. If $W$ is closed under addition 
and closed under scalar multiplication, then $W$ is a subspace.
\end{thm}\index{subspace}
\begin{proof}  We have to show that $W$ is a vector space using the operations 
of addition and scalar multiplication defined on $V$.  That is, we need to
verify that the eight properties listed in Table~\ref{vectorspacelist} are 
satisfied.  Note that properties (A1), (A2), (M1), (M2), (D1), and (D2) are 
valid for vectors in $W$ since they are valid for vectors in $V$.

It remains to verify (A3) and (A4).  Let $w\in W$ be any vector.
Since $W$ is closed under scalar multiplication, it follows that
$0w$ and $(-1)w$ are in $W$. Lemma~\ref{lem:AddId} states that
$0w=0$ and $(-1)w=-w$; it follows that $0$ and $-w$ are in $W$.
Hence, properties (A3) and (A4) are valid for vectors in $W$,
since they are valid for vectors in $V$.  \end{proof}

\subsubsection*{Examples of Subspaces of $\R^n$}

\begin{exam}  \label{EX:subspaces}
{\rm
\begin{itemize}
\item[(a)] Let $V$ be a vector space.  Then the subsets $V$ and $\{0\}$ are
always subspaces of $V$.  A subspace $W\subset V$ is {\em proper\/} if
$W\neq 0$ and $W\neq V$. \index{subspace!proper}
\item[(b)] Lines through the origin are subspaces of $\R^n$.  Let $w\in \R^n$
be a nonzero vector and let $W=\{rw:r\in\R\}$.  The set $W$ is closed under
addition and scalar multiplication and is a subspace of $\R^n$ by 
Theorem~\ref{T:subspaces}.
The subspace $W$ is just a {\em line through the origin\/} in $\R^n$, since
the vector $rw$ points in the same direction as $w$ when $r>0$ and the exact
opposite direction when $r<0$.
\item[(c)]  Planes containing the origin are subspaces of $\R^3$.  To verify
this point, let $P$ be a plane through the origin and let $N$ be a vector
perpendicular\index{perpendicular} to $P$.  Then $P$ consists of all vectors
$v\in\R^3$ perpendicular to $N$; using the dot-product (see Chapter~\ref{lineq},
\Ref{XX_0}) we recall that such vectors satisfy the linear equation
$N\cdot v = 0$.  By superposition\index{superposition}, the set of all
solutions to this equation is closed under addition and scalar multiplication
and is therefore a subspace by Theorem~\ref{T:subspaces}.
\end{itemize}
}
\end{exam}
In a sense that will be made precise all subspaces of $\R^n$ can be written
as the span of a finite number of vectors generalizing
Example~\ref{EX:subspaces}(b) or as solutions to a system of linear equations
generalizing Example~\ref{EX:subspaces}(c).


\subsubsection*{Examples of Subspaces of the Function Space ${\cal F}$}
\index{function space!subspace of}\index{subspace!of function space}

Let ${\cal P}$ be the set of all polynomials in ${\cal F}$.  The sum of
two polynomials is a polynomial and the scalar multiple of a polynomial is a
polynomial.  Thus, ${\cal P}$ is closed under addition and scalar
multiplication, and ${\cal P}$ is a subspace of ${\cal F}$.

As a second example of a subspace of ${\cal F}$, let $\CCone$ be the set of
all continuously differentiable functions $u:\R\to\R$.  A function $u$ is
in $\CCone$ if $u$ and $u'$ exist and are continuous for all $t\in\R$.
Examples of functions in $\CCone$ are:
\begin{enumerate}
\item[(i)] Every polynomial $p(t)=a_mt^m+a_{m-1}t^{m-1}+\cdots
+a_1t+a_0$ is in $\CCone$. \index{polynomial}
\item[(ii)]  The function $u(t)=e^{\lambda t}$ is in $\CCone$
for each constant $\lambda\in\R$.
\item[(iii)]  The trigonometric functions $u(t)=\sin(\lambda t)$
and $v(t)=\cos(\lambda t)$ are in $\CCone$ for each constant
$\lambda\in\R$. \index{trigonometric function}
\item[(iv)] $u(t)=t^{7/3}$ is twice differentiable everywhere and is
in $\CCone$.
\end{enumerate}
Equally there are many commonly used functions that are not in
$\CCone$.  Examples include:
\begin{enumerate}
\item[(i)] $u(t)=\frac{1}{t-5}$ is neither defined nor
continuous at $t=5$.
\item[(ii)] $u(t)=|t|$ is not differentiable (at $t=0$).
\item[(iii)] $u(t)=\csc(t)$ is neither defined nor continuous at
$t=k\pi$ for any integer $k$.
\end{enumerate}

The subset $\CCone\subset{\cal F}$  is a subspace and hence a vector space.
\index{vector!in $\CCone$}
The reason is simple.  If $x(t)$ and $y(t)$ are continuously differentiable,
then
\[
\frac{d}{dt}(x+y) = \frac{dx}{dt}+\frac{dy}{dt}.
\]
Hence $x+y$ is differentiable and is in $\CCone$ and $\CCone$ is closed under
addition.  Similarly, $\CCone$ is closed under scalar multiplication.  Let
$r\in\R$ and let $x\in\CCone$. Then
\[
\frac{d}{dt}(rx)(t) = r\frac{dx}{dt}(t).
\]
Hence $rx$ is differentiable and is in $\CCone$.

\subsubsection*{The Vector Space $(\CCone)^n$}

Another example of a vector space that combines the features of
both $\R^n$ and $\CCone$ is $(\CCone)^n$.  Vectors
$u\in (\CCone)^n$ have the form
\[
u(t) = (u_1(t),\ldots,u_n(t)),
\]
where each coordinate function $u_j(t)\in\CCone$.  Addition and
scalar multiplication in $(\CCone)^n$ are defined coordinatewise
--- just like addition and scalar multiplication in $\R^n$.
That is, let $u,v$ be in $(\CCone)^n$ and let $r$ be in $\R$, then
\begin{eqnarray*}
(u+v)(t) & = & (u_1(t)+v_1(t),\ldots,u_n(t)+v_n(t)) \\
(ru)(t) & = & (ru_1(t),\ldots,ru_n(t)).
\end{eqnarray*}
The set $(\CCone)^n$ satisfies the eight properties of vector spaces and
is a vector space.  Solutions to systems of $n$ linear ordinary differential
equations are vectors in $(\CCone)^n$.


\EXER

\TEXER

\begin{exercise} \label{c5.1.1}
Verify that the set $V_1$ consisting of all scalar multiples of
$(1,-1,-2)$ is a subspace of $\R^3$.
\end{exercise}

\begin{exercise} \label{c5.1.2}
Let $V_2$ be the set of all $2\times 3$ matrices.   Verify that
$V_2$ is a vector space.
\end{exercise}

\begin{exercise} \label{c5.1.3}
Let
\[
A=\left(\begin{array}{rrr} 1 & 1 & 0\\ 1 & -1 & 1 \end{array}
\right).
\]
Let $V_3$ be the set of vectors $x\in\R^3$ such that $Ax=0$.
Verify that $V_3$ is a subspace of $\R^3$.  Compare $V_1$ with
$V_3$.
\end{exercise}

\noindent In Exercises~\ref{c5.1.4a} -- \ref{c5.1.4f} you are given a
vector space $V$ and a subset $W$.  For each pair, decide whether or
not $W$ is a subspace of $V$.
\begin{exercise} \label{c5.1.4a}
$V=\R^3$ and $W$ consists of vectors in $\R^3$
     that have a $0$ in their first component.
\end{exercise}
\begin{exercise} \label{c5.1.4b}
$V=\R^3$ and $W$ consists of vectors in $\R^3$
     that have a $1$ in their first component.
\end{exercise}
\begin{exercise} \label{c5.1.4d}
$V=\R^2$ and $W$ consists of vectors in $\R^2$
     for which the sum of the components is $1$.
\end{exercise}
\begin{exercise} \label{c5.1.4c}
$V=\R^2$ and $W$ consists of vectors in $\R^2$
     for which the sum of the components is $0$.
\end{exercise}
\begin{exercise} \label{c5.1.4g}
$V=\CCone$ and $W$ consists of functions
     $x(t)\in\CCone$ satisfying $\int_{-2}^4x(t)dt =0$.
\end{exercise}
\begin{exercise} \label{c5.1.4e}
$V=\CCone$ and $W$ consists of functions
     $x(t)\in\CCone$ satisfying $x(1)=0$.
\end{exercise}
\begin{exercise} \label{c5.1.4f}
$V=\CCone$ and $W$ consists of functions
     $x(t)\in\CCone$ satisfying $x(1)=1$.
\end{exercise}

\noindent In Exercises~\ref{c5.1.5a} -- \ref{c5.1.5e} which of the
sets $S$ are subspaces?
\begin{exercise} \label{c5.1.5a}
$S = \{(a,b,c)\in\R^3 : a\ge 0, \; b\ge 0,\; c\ge 0\}$.
\end{exercise}
\begin{exercise} \label{c5.1.5b}
$S = \{(x_1,x_2,x_3)\in\R^3 : a_1x_1+a_2x_2+a_3x_3=0
\mbox{ where } a_1,a_2,a_3\in\R \mbox{ are fixed}\}$.
\end{exercise}
\begin{exercise} \label{c5.1.5c}
$S = \{(x,y)\in\R^2: (x,y) \mbox{ is on the line through }
(1,1) \mbox{ with slope } 1\}$.
\end{exercise}
\begin{exercise} \label{c5.1.5d}
$S = \{x\in\R^2: Ax=0\}$ where $A$ is a $3\times 2$ matrix.
\end{exercise}
\begin{exercise} \label{c5.1.5e}
$S = \{x\in\R^2: Ax=b\}$ where $A$ is a $3\times 2$ matrix
	and $b\in\R^3$ is a fixed nonzero vector.
\end{exercise}

\begin{exercise} \label{c5.1.6}
Let $V$ be a vector space and let $W_1$ and $W_2$ be subspaces.
Show that the intersection $W_1\cap W_2$ is also a subspace of $V$.
\end{exercise}

\begin{exercise} \label{c5.1.7a}
For which scalars $a,b,c$ do the solutions to the equation
\[
ax+by = c
\]
form a subspace of $\R^2$?
\end{exercise}
\begin{exercise} \label{c5.1.7b}
For which scalars $a,b,c,d$ do the solutions to the equation
\[
ax+by+cz = d
\]
form a subspace of $\R^3$?
\end{exercise}

\begin{exercise} \label{c5.1.8}
Show that the set of all solutions to the differential equation
$\dot{x}=2x$ is a subspace of $\CCone$.
\end{exercise}

\begin{exercise} \label{c5.1.9}
Recall from equation~\Ref{e:solnODE} of Section~\ref{S:IVP&E}
that solutions to the system of differential equations
\[
\frac{dX}{dt} = \mattwo{-1}{3}{3}{-1} X
\]
are
\[
X(t) = \alpha e^{2t}\vectwo{1}{1} + \beta e^{-4t}\vectwo{1}{-1}.
\]
Use this formula for solutions to show that the set of solutions
to this system of differential equations is a vector subspace of
$(\CCone)^2$.
\end{exercise}

\end{document}
