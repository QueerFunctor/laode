\documentclass{ximera}

 

\usepackage{epsfig}

\graphicspath{
  {./}
  {figures/}
}

\usepackage{morewrites}
\makeatletter
\newcommand\subfile[1]{%
\renewcommand{\input}[1]{}%
\begingroup\skip@preamble\otherinput{#1}\endgroup\par\vspace{\topsep}
\let\input\otherinput}
\makeatother

\newcommand{\includeexercises}{\directlua{dofile("/home/jim/linearAlgebra/laode/exercises.lua")}}

%\newcounter{ccounter}
%\setcounter{ccounter}{1}
%\newcommand{\Chapter}[1]{\setcounter{chapter}{\arabic{ccounter}}\chapter{#1}\addtocounter{ccounter}{1}}

%\newcommand{\section}[1]{\section{#1}\setcounter{thm}{0}\setcounter{equation}{0}}

%\renewcommand{\theequation}{\arabic{chapter}.\arabic{section}.\arabic{equation}}
%\renewcommand{\thefigure}{\arabic{chapter}.\arabic{figure}}
%\renewcommand{\thetable}{\arabic{chapter}.\arabic{table}}

%\newcommand{\Sec}[2]{\section{#1}\markright{\arabic{ccounter}.\arabic{section}.#2}\setcounter{equation}{0}\setcounter{thm}{0}\setcounter{figure}{0}}

\newcommand{\Sec}[2]{\section{#1}}

\setcounter{secnumdepth}{2}
%\setcounter{secnumdepth}{1} 

%\newcounter{THM}
%\renewcommand{\theTHM}{\arabic{chapter}.\arabic{section}}

\newcommand{\trademark}{{R\!\!\!\!\!\bigcirc}}
%\newtheorem{exercise}{}

\newcommand{\dfield}{{\sf dfield9}}
\newcommand{\pplane}{{\sf pplane9}}

\newcommand{\EXER}{\section*{Exercises}}%\vspace*{0.2in}\hrule\small\setcounter{exercise}{0}}
\newcommand{\CEXER}{}%\vspace{0.08in}\begin{center}Computer Exercises\end{center}}
\newcommand{\TEXER}{} %\vspace{0.08in}\begin{center}Hand Exercises\end{center}}
\newcommand{\AEXER}{} %\vspace{0.08in}\begin{center}Hand Exercises\end{center}}

% BADBAD: \newcommand{\Bbb}{\bf}

\newcommand{\R}{\mbox{$\Bbb{R}$}}
\newcommand{\C}{\mbox{$\Bbb{C}$}}
\newcommand{\Z}{\mbox{$\Bbb{Z}$}}
\newcommand{\N}{\mbox{$\Bbb{N}$}}
\newcommand{\D}{\mbox{{\bf D}}}
\usepackage{amssymb}
%\newcommand{\qed}{\hfill\mbox{\raggedright$\square$} \vspace{1ex}}
%\newcommand{\proof}{\noindent {\bf Proof:} \hspace{0.1in}}

\newcommand{\setmin}{\;\mbox{--}\;}
\newcommand{\Matlab}{{M\small{AT\-LAB}} }
\newcommand{\Matlabp}{{M\small{AT\-LAB}}}
\newcommand{\computer}{\Matlab Instructions}
\newcommand{\half}{\mbox{$\frac{1}{2}$}}
\newcommand{\compose}{\raisebox{.15ex}{\mbox{{\scriptsize$\circ$}}}}
\newcommand{\AND}{\quad\mbox{and}\quad}
\newcommand{\vect}[2]{\left(\begin{array}{c} #1_1 \\ \vdots \\
 #1_{#2}\end{array}\right)}
\newcommand{\mattwo}[4]{\left(\begin{array}{rr} #1 & #2\\ #3
&#4\end{array}\right)}
\newcommand{\mattwoc}[4]{\left(\begin{array}{cc} #1 & #2\\ #3
&#4\end{array}\right)}
\newcommand{\vectwo}[2]{\left(\begin{array}{r} #1 \\ #2\end{array}\right)}
\newcommand{\vectwoc}[2]{\left(\begin{array}{c} #1 \\ #2\end{array}\right)}

\newcommand{\ignore}[1]{}


\newcommand{\inv}{^{-1}}
\newcommand{\CC}{{\cal C}}
\newcommand{\CCone}{\CC^1}
\newcommand{\Span}{{\rm span}}
\newcommand{\rank}{{\rm rank}}
\newcommand{\trace}{{\rm tr}}
\newcommand{\RE}{{\rm Re}}
\newcommand{\IM}{{\rm Im}}
\newcommand{\nulls}{{\rm null\;space}}

\newcommand{\dps}{\displaystyle}
\newcommand{\arraystart}{\renewcommand{\arraystretch}{1.8}}
\newcommand{\arrayfinish}{\renewcommand{\arraystretch}{1.2}}
\newcommand{\Start}[1]{\vspace{0.08in}\noindent {\bf Section~\ref{#1}}}
\newcommand{\exer}[1]{\noindent {\bf \ref{#1}}}
\newcommand{\ans}{}
\newcommand{\matthree}[9]{\left(\begin{array}{rrr} #1 & #2 & #3 \\ #4 & #5 & #6
\\ #7 & #8 & #9\end{array}\right)}
\newcommand{\cvectwo}[2]{\left(\begin{array}{c} #1 \\ #2\end{array}\right)}
\newcommand{\cmatthree}[9]{\left(\begin{array}{ccc} #1 & #2 & #3 \\ #4 & #5 &
#6 \\ #7 & #8 & #9\end{array}\right)}
\newcommand{\vecthree}[3]{\left(\begin{array}{r} #1 \\ #2 \\
#3\end{array}\right)}
\newcommand{\cvecthree}[3]{\left(\begin{array}{c} #1 \\ #2 \\
#3\end{array}\right)}
\newcommand{\cmattwo}[4]{\left(\begin{array}{cc} #1 & #2\\ #3
&#4\end{array}\right)}

\newcommand{\Matrix}[1]{\ensuremath{\left(\begin{array}{rrrrrrrrrrrrrrrrrr} #1 \end{array}\right)}}

\newcommand{\Matrixc}[1]{\ensuremath{\left(\begin{array}{cccccccccccc} #1 \end{array}\right)}}



\renewcommand{\labelenumi}{\theenumi)}
\newenvironment{enumeratea}%
{\begingroup
 \renewcommand{\theenumi}{\alph{enumi}}
 \renewcommand{\labelenumi}{(\theenumi)}
 \begin{enumerate}}
 {\end{enumerate}\endgroup}



\newcounter{help}
\renewcommand{\thehelp}{\thesection.\arabic{equation}}

%\newenvironment{equation*}%
%{\renewcommand\endequation{\eqno (\theequation)* $$}%
%   \begin{equation}}%
%   {\end{equation}\renewcommand\endequation{\eqno \@eqnnum
%$$\global\@ignoretrue}}

%\input{psfig.tex}

\author{Martin Golubitsky and Michael Dellnitz}

%\newenvironment{matlabEquation}%
%{\renewcommand\endequation{\eqno (\theequation*) $$}%
%   \begin{equation}}%
%   {\end{equation}\renewcommand\endequation{\eqno \@eqnnum
% $$\global\@ignoretrue}}

\newcommand{\soln}{\textbf{Solution:} }
\newcommand{\exercap}[1]{\centerline{Figure~\ref{#1}}}
\newcommand{\exercaptwo}[1]{\centerline{Figure~\ref{#1}a\hspace{2.1in}
Figure~\ref{#1}b}}
\newcommand{\exercapthree}[1]{\centerline{Figure~\ref{#1}a\hspace{1.2in}
Figure~\ref{#1}b\hspace{1.2in}Figure~\ref{#1}c}}
\newcommand{\para}{\hspace{0.4in}}

\renewenvironment{solution}{\suppress}{\endsuppress}

\ifxake
\newenvironment{matlabEquation}{\begin{equation}}{\end{equation}}
\else
\newenvironment{matlabEquation}%
{\let\oldtheequation\theequation\renewcommand{\theequation}{\oldtheequation*}\begin{equation}}%
  {\end{equation}\let\theequation\oldtheequation}
\fi

\makeatother


\title{Spanning Sets and MATLAB}

\begin{document}
\begin{abstract}
\end{abstract}
\maketitle

 \label{S:5.3}

In this section we discuss:
\begin{itemize}
\item	how to find a spanning set for the subspace of solutions to a
homogeneous system of linear equations using the \Matlab command {\tt null},
and
\item	how to determine when a vector is in the subspace spanned by a
set of vectors using the \Matlab command {\tt rref}.
\end{itemize}

\subsection*{Spanning Sets for Homogeneous Linear Equations}

In Chapter~\ref{lineq} we saw how to use Gaussian elimination,
back substitution, and \Matlab to compute solutions to a system
of linear equations.  For systems of
homogeneous equations\index{homogeneous}, \Matlab
provides a command to find a spanning set for the subspace of solutions.
That command is {\tt null}.  For example, if we type
\begin{verbatim}
A = [2 1 4 0; -1 0 2 1]
B = null(A)
\end{verbatim} \index{\computer!null}
then we obtain
\begin{verbatim}
B =
    0.4830         0
   -0.4140    0.8729
   -0.1380   -0.2182
    0.7591    0.4364
\end{verbatim}
The two columns of the matrix $B$ span the set of solutions of
the equation $Ax=0$.  In particular, the vector $(2,-8,1,0)$ is a
solution to $Ax=0$ and is therefore a
linear combination \index{linear!combination} of the
column vectors of {\tt B}.  Indeed, type
\begin{verbatim}
4.1404*B(:,1)-7.2012*B(:,2)
\end{verbatim}
and observe that this linear combination is the desired one.

Next we describe how to find the coefficients {\tt 4.1404} and
{\tt -7.2012} by showing that these coefficients themselves are
solutions to another system of linear equations.

\subsection*{When is a Vector in a Span?} \index{span}

Let $w_1,\ldots,w_k$ and $v$ be vectors in $\R^n$.  We now
describe a method that allows us to decide whether $v$ is in
$\Span\{w_1,\ldots,w_k\}$.  To answer this question one has
to solve a system of $n$ linear equations in $k$ unknowns.
The unknowns correspond to the coefficients in the linear
combination of the vectors $w_1,\ldots,w_k$ that gives $v$.

Let us be more precise.  The vector\index{vector} $v$ is in
$\Span\{w_1,\ldots,w_k\}$ if and only if there are constants
$r_1,\ldots,r_k$ such that the equation
\begin{equation}  \label{e:lindepeqn}
     r_1 w_1 + \cdots + r_k w_k = v
\end{equation}
is valid.  Define the $n\times k$ matrix $A$ as the one having
$w_1,\ldots,w_k$ as its columns; that is,
\begin{equation}  \label{E:Abycol}
A = (w_1| \cdots |w_k).
\end{equation}
Let  $r$ be the $k$-vector
\[
r= \left(\begin{array}{c} r_1 \\ \vdots \\ r_k\end{array}\right).
\]
Then we may rewrite equation \eqref{e:lindepeqn} as
\begin{equation}  \label{E:Ar=v}
   Ar=v.
\end{equation}
To summarize:
\begin{lemma}
Let $w_1,\ldots,w_k$ and $v$ be vectors in $\R^n$.  Then $v$
is in $\Span\{w_1,\ldots,w_k\}$ if and only if the system of linear
equations \eqref{E:Ar=v} has a solution where $A$ is the $n\times k$
defined in \eqref{E:Abycol}.
\end{lemma}

To solve \eqref{E:Ar=v} we row reduce the
augmented matrix\index{matrix!augmented} $(A|v)$.
For example, is $v=(2,1)$ in the span of $w_1=(1,1)$ and $w_2=(1,-1)$?
That is, do there exist scalars $r_1,r_2$ such that
\[
r_1\vectwo{1}{1} + r_2\vectwo{1}{-1} = \vectwo{2}{1}?
\]
As noted, we can rewrite this equation as
\[
\mattwo{1}{1}{1}{-1}\vectwo{r_1}{r_2} = \vectwo{2}{1}.
\]
We can solve this equation by row reducing the augmented
matrix
\[
\left(\begin{array}{rr|r}
1 & 1 & 2 \\ 1 & -1 & 1 \end{array}\right)
\]
to obtain
\[
\left(\begin{array}{rr|r}
1 & 0 & \frac{3}{2} \\ 0 & 1 & \frac{1}{2}
\end{array}\right).
\]
So $v = \frac{3}{2}w_1 + \frac{1}{2}w_2$.

Row reduction to reduced echelon form\index{echelon form!reduced}
has been preprogrammed in the
\Matlab command {\tt rref}. \index{\computer!rref}  Consider the
following example.  Let
\begin{equation}  \label{e:w1w2}
     w_1=(2,0,-1,4) \AND w_2=(2,-1,0,2)
\end{equation}
and ask the question whether $v=(-2,4,-3,4)$ is in $\Span\{w_1,w_2\}$.

In \Matlab load the matrix $A$ having $w_1$ and
$w_2$ as its columns and the vector $v$ by typing {\tt e5\_3\_5}
\begin{matlabEquation}  \label{e:Aandv}
A=\left(\begin{array}{rr} 2 & 2 \\ 0 & -1 \\ -1 & 0 \\ 4 & 2
\end{array}\right) \AND
v=\left(\begin{array}{r} -2 \\ 4 \\ -3 \\ 4 \end{array}\right).
\end{matlabEquation}%
We can solve the system of equations using \Matlabp.
First, form the augmented matrix by typing
\begin{verbatim}
aug = [A v]
\end{verbatim}
Then solve the system by typing {\tt rref(aug)} to obtain
\begin{verbatim}
ans =
     1   0   3
     0   1  -4
     0   0   0
     0   0   0
\end{verbatim}
It follows that $(r_1,r_2)=(3,-4)$ is a solution and $v=3w_1-4w_2$.

Now we change the $4^{th}$ entry in $v$ slightly by typing
{\tt v(4) = 4.01}.  There is no solution to the system of equations
\[
Ar = \left(\begin{array}{r} -2 \\ 4 \\ -3 \\ 4.01
\end{array}\right)
\]
as we now show.  Type
\begin{verbatim}
aug = [A v]
rref(aug)
\end{verbatim}
which yields
\begin{verbatim}
ans =
     1    0    0
     0    1    0
     0    0    1
     0    0    0
\end{verbatim}
This matrix corresponds to an inconsistent\index{inconsistent} system;
thus $v$ is no longer in the span\index{span} of $w_1$ and $w_2$.



\includeexercises

\end{document}
