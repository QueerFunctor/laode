\documentclass{ximera}

 

\usepackage{epsfig}

\graphicspath{
  {./}
  {figures/}
}

\usepackage{morewrites}
\makeatletter
\newcommand\subfile[1]{%
\renewcommand{\input}[1]{}%
\begingroup\skip@preamble\otherinput{#1}\endgroup\par\vspace{\topsep}
\let\input\otherinput}
\makeatother

\newcommand{\includeexercises}{\directlua{dofile("/home/jim/linearAlgebra/laode/exercises.lua")}}

%\newcounter{ccounter}
%\setcounter{ccounter}{1}
%\newcommand{\Chapter}[1]{\setcounter{chapter}{\arabic{ccounter}}\chapter{#1}\addtocounter{ccounter}{1}}

%\newcommand{\section}[1]{\section{#1}\setcounter{thm}{0}\setcounter{equation}{0}}

%\renewcommand{\theequation}{\arabic{chapter}.\arabic{section}.\arabic{equation}}
%\renewcommand{\thefigure}{\arabic{chapter}.\arabic{figure}}
%\renewcommand{\thetable}{\arabic{chapter}.\arabic{table}}

%\newcommand{\Sec}[2]{\section{#1}\markright{\arabic{ccounter}.\arabic{section}.#2}\setcounter{equation}{0}\setcounter{thm}{0}\setcounter{figure}{0}}

\newcommand{\Sec}[2]{\section{#1}}

\setcounter{secnumdepth}{2}
%\setcounter{secnumdepth}{1} 

%\newcounter{THM}
%\renewcommand{\theTHM}{\arabic{chapter}.\arabic{section}}

\newcommand{\trademark}{{R\!\!\!\!\!\bigcirc}}
%\newtheorem{exercise}{}

\newcommand{\dfield}{{\sf dfield9}}
\newcommand{\pplane}{{\sf pplane9}}

\newcommand{\EXER}{\section*{Exercises}}%\vspace*{0.2in}\hrule\small\setcounter{exercise}{0}}
\newcommand{\CEXER}{}%\vspace{0.08in}\begin{center}Computer Exercises\end{center}}
\newcommand{\TEXER}{} %\vspace{0.08in}\begin{center}Hand Exercises\end{center}}
\newcommand{\AEXER}{} %\vspace{0.08in}\begin{center}Hand Exercises\end{center}}

% BADBAD: \newcommand{\Bbb}{\bf}

\newcommand{\R}{\mbox{$\Bbb{R}$}}
\newcommand{\C}{\mbox{$\Bbb{C}$}}
\newcommand{\Z}{\mbox{$\Bbb{Z}$}}
\newcommand{\N}{\mbox{$\Bbb{N}$}}
\newcommand{\D}{\mbox{{\bf D}}}
\usepackage{amssymb}
%\newcommand{\qed}{\hfill\mbox{\raggedright$\square$} \vspace{1ex}}
%\newcommand{\proof}{\noindent {\bf Proof:} \hspace{0.1in}}

\newcommand{\setmin}{\;\mbox{--}\;}
\newcommand{\Matlab}{{M\small{AT\-LAB}} }
\newcommand{\Matlabp}{{M\small{AT\-LAB}}}
\newcommand{\computer}{\Matlab Instructions}
\newcommand{\half}{\mbox{$\frac{1}{2}$}}
\newcommand{\compose}{\raisebox{.15ex}{\mbox{{\scriptsize$\circ$}}}}
\newcommand{\AND}{\quad\mbox{and}\quad}
\newcommand{\vect}[2]{\left(\begin{array}{c} #1_1 \\ \vdots \\
 #1_{#2}\end{array}\right)}
\newcommand{\mattwo}[4]{\left(\begin{array}{rr} #1 & #2\\ #3
&#4\end{array}\right)}
\newcommand{\mattwoc}[4]{\left(\begin{array}{cc} #1 & #2\\ #3
&#4\end{array}\right)}
\newcommand{\vectwo}[2]{\left(\begin{array}{r} #1 \\ #2\end{array}\right)}
\newcommand{\vectwoc}[2]{\left(\begin{array}{c} #1 \\ #2\end{array}\right)}

\newcommand{\ignore}[1]{}


\newcommand{\inv}{^{-1}}
\newcommand{\CC}{{\cal C}}
\newcommand{\CCone}{\CC^1}
\newcommand{\Span}{{\rm span}}
\newcommand{\rank}{{\rm rank}}
\newcommand{\trace}{{\rm tr}}
\newcommand{\RE}{{\rm Re}}
\newcommand{\IM}{{\rm Im}}
\newcommand{\nulls}{{\rm null\;space}}

\newcommand{\dps}{\displaystyle}
\newcommand{\arraystart}{\renewcommand{\arraystretch}{1.8}}
\newcommand{\arrayfinish}{\renewcommand{\arraystretch}{1.2}}
\newcommand{\Start}[1]{\vspace{0.08in}\noindent {\bf Section~\ref{#1}}}
\newcommand{\exer}[1]{\noindent {\bf \ref{#1}}}
\newcommand{\ans}{}
\newcommand{\matthree}[9]{\left(\begin{array}{rrr} #1 & #2 & #3 \\ #4 & #5 & #6
\\ #7 & #8 & #9\end{array}\right)}
\newcommand{\cvectwo}[2]{\left(\begin{array}{c} #1 \\ #2\end{array}\right)}
\newcommand{\cmatthree}[9]{\left(\begin{array}{ccc} #1 & #2 & #3 \\ #4 & #5 &
#6 \\ #7 & #8 & #9\end{array}\right)}
\newcommand{\vecthree}[3]{\left(\begin{array}{r} #1 \\ #2 \\
#3\end{array}\right)}
\newcommand{\cvecthree}[3]{\left(\begin{array}{c} #1 \\ #2 \\
#3\end{array}\right)}
\newcommand{\cmattwo}[4]{\left(\begin{array}{cc} #1 & #2\\ #3
&#4\end{array}\right)}

\newcommand{\Matrix}[1]{\ensuremath{\left(\begin{array}{rrrrrrrrrrrrrrrrrr} #1 \end{array}\right)}}

\newcommand{\Matrixc}[1]{\ensuremath{\left(\begin{array}{cccccccccccc} #1 \end{array}\right)}}



\renewcommand{\labelenumi}{\theenumi)}
\newenvironment{enumeratea}%
{\begingroup
 \renewcommand{\theenumi}{\alph{enumi}}
 \renewcommand{\labelenumi}{(\theenumi)}
 \begin{enumerate}}
 {\end{enumerate}\endgroup}



\newcounter{help}
\renewcommand{\thehelp}{\thesection.\arabic{equation}}

%\newenvironment{equation*}%
%{\renewcommand\endequation{\eqno (\theequation)* $$}%
%   \begin{equation}}%
%   {\end{equation}\renewcommand\endequation{\eqno \@eqnnum
%$$\global\@ignoretrue}}

%\input{psfig.tex}

\author{Martin Golubitsky and Michael Dellnitz}

%\newenvironment{matlabEquation}%
%{\renewcommand\endequation{\eqno (\theequation*) $$}%
%   \begin{equation}}%
%   {\end{equation}\renewcommand\endequation{\eqno \@eqnnum
% $$\global\@ignoretrue}}

\newcommand{\soln}{\textbf{Solution:} }
\newcommand{\exercap}[1]{\centerline{Figure~\ref{#1}}}
\newcommand{\exercaptwo}[1]{\centerline{Figure~\ref{#1}a\hspace{2.1in}
Figure~\ref{#1}b}}
\newcommand{\exercapthree}[1]{\centerline{Figure~\ref{#1}a\hspace{1.2in}
Figure~\ref{#1}b\hspace{1.2in}Figure~\ref{#1}c}}
\newcommand{\para}{\hspace{0.4in}}

\renewenvironment{solution}{\suppress}{\endsuppress}

\ifxake
\newenvironment{matlabEquation}{\begin{equation}}{\end{equation}}
\else
\newenvironment{matlabEquation}%
{\let\oldtheequation\theequation\renewcommand{\theequation}{\oldtheequation*}\begin{equation}}%
  {\end{equation}\let\theequation\oldtheequation}
\fi

\makeatother


\title{Partial Fractions}

\begin{document}
\begin{abstract}
\end{abstract}
\maketitle

  \label{S:PF}
\index{partial fractions}

In Section~\ref{S:13.1} we saw that expansions into partial fractions is a 
necessary tool when applying the method of Laplace transforms.  In the
simplest case partial fractions work as follows.  Assume that $p(s)$ and 
$q(s)$ are two polynomials such that
\begin{enumerate}
\item[(a)] the degree of $p(s)$ is less or equal to the degree $d$ of $q(s)$;
\item[(b)] $q(s)=(s-r_1)\cdots(s-r_d)$ has no multiple roots.
\end{enumerate}
The roots $r_1,\ldots,r_d$ of $q(s)$ may be either real or complex.  Then the 
expansion into partial fractions of $p(s)/q(s)$ has the form
\begin{equation}  \label{E:PF}
\frac{p(s)}{q(s)} = \frac{c_1}{s-r_1}+\frac{c_2}{s-r_2}+\cdots
+\frac{c_d}{s-r_d},
\end{equation}
where $c_1,\ldots,c_d$ are  scalars determined from $p$ and $q$.

There is a simple way to compute the constant $c_j$. Define the degree $d-1$
polynomial
\[
q_j(s) = \frac{q(s)}{s-r_j} = 
(s-r_1)\cdots(s-r_{j-1})(s-r_{j+1})\cdots(s-r_d).
\]
Multiply both sides of \eqref{E:PF} by $s-r_j$ and evaluate at $s=r_j$ to 
obtain
\[
c_j = \frac{p(r_j)}{q_j(r_j)}.
\]   

For example, compute the partial fraction expansion for
\[
\frac{s^2+3s-5}{s(s-1)(s+2)} = \frac{c_1}{s}+\frac{c_2}{s-1}+\frac{c_3}{s+2}.
\]
In this example, $r_1=0$, $r_2=1$, and $r_3=-2$.  The relevant polynomials
are:
\[
p(s) = s^2+3s-5 \qquad q_1(s) = (s-1)(s+2) \qquad q_2(s) = s(s+2) \qquad
q_3(s) = s(s-1).
\]
It follows that 
\[
c_1 = \frac{p(0)}{q_1(0)} = \frac{5}{2} \qquad
c_2 = \frac{p(1)}{q_2(1)} = -\frac{1}{3} \qquad
c_3 = \frac{p(-2)}{q_3(-2)} = -\frac{7}{6}.
\]

\subsubsection*{Partial Fractions with Complex Roots}
\index{partial fractions!complex roots}


Suppose that the denominator $q(s)$ has complex conjugate roots $r$ and 
$\overline{r}$.  When $r$ is a complex simple root of $q(s)$, then the 
partial fractions expansion of $p(s)/q(s)$ contains the two terms
\[
\frac{c}{s-r} + \frac{d}{s-\overline{r}}
\]
where $c,d\in\C$.  Together these two terms must be real-valued, and it 
follows that $d=\overline{c}$.  Therefore the expansion is 
\[
\frac{c}{s-r} + \frac{\overline{c}}{s-\overline{r}}
\]
for some complex scalar $c$.  These terms combine as
\[
\frac{c(s-\overline{r})+ \overline{c}(s-r)}{s^2-2r_1s+r\overline{r}}
\]
where $r=r_1+ir_2$.  With inverse Laplace transforms in mind, we prefer 
to write this expression as
\begin{equation}  \label{e:pfreal}
\frac{2c_1(s-r_1)-2c_2r_2}{(s-r_1)^2+r_2^2}
\end{equation}
where $c=c_1+ic_2$.  

In the third part of Section~\ref{S:13.3} we saw how to compute inverse 
Laplace transforms of functions like those in \eqref{e:pfreal}.



\subsection*{Partial Fractions Using \Matlab}
\index{partial fractions!in \Matlab}

The \Matlab command {\tt residue}\index{\computer!residue} can be used to 
determine partial fractions
expansions.  We begin by discussing how polynomials are defined in \Matlabp.  
The polynomial $q(s)=a_ds^d+\cdots+a_1s+a_0$ is stored in \Matlab by the 
vector {\tt q = [ad $\cdots$ a1 a0]} consisting of the coefficients of $q(s)$ 
in {\em descending\/} order.  For instance, in \Matlabp, the polynomial 
$q(s)=2s^3 - 3s +5$ is identified with the vector $(2,0,-3,5)$.  

Suppose that the two vectors {\tt p} and {\tt q} represent the two polynomials 
$p(s)$ (of degree less than $D$) and $q(s)$ (of degree $d$).  Both the vector 
of roots $r=(r_1,\ldots,r_d)$ and the vectors of scalars $c=(c_1,\ldots,c_d)$ 
are determined using the command 
\begin{verbatim}
[c,r] = residue(p,q)
\end{verbatim}

To illustrate this command, find the partial fraction expansion of 
\[
\frac{4s+2}{s^2-s}
\]
(which we computed previously in \eqref{eq:Lapdxx2}) by typing
\begin{verbatim}
    p = [4 2];
    q = [1 -1 0];
[c,r] = residue(p,q)
\end{verbatim}
\Matlab responds with
\begin{verbatim}
c =
     6
    -2
r =
     1
     0
\end{verbatim}
Note that this result agrees with our previous calculation in \eqref{eq:Lapdxx2}.

Observe that the situation where the polynomial $q(s)$ has complex roots 
is not excluded.  Indeed, let $p(s) = s+1$ and $q(s) = s^3+s^2-4s+6$, and type
\begin{verbatim}
    p = [1 1];
    q = [1 1 -4 6];
[c,r] = residue(p,q)
\end{verbatim}
to obtain the answer
\begin{verbatim}
c =
  -0.1176          
   0.0588 - 0.2647i
   0.0588 + 0.2647i
r =
  -3.0000          
   1.0000 + 1.0000i
   1.0000 - 1.0000i
\end{verbatim}
In particular, $q(s)$ has the three roots $-3,1+i,1-i$ and we have
the expansion
\begin{equation}   \label{e:realfex}
\frac{s+1}{s^3+s^2-4s+6} = -\frac{0.1176}{s+3}+\frac{0.0588 - 0.2647i}{s-(1+i)}+
\frac{0.0588 + 0.2647i}{s-(1-i)}.
\end{equation}

\subsubsection*{The Return to Real Form in Partial Fractions}
\index{partial fractions!in \Matlab!complex roots}

We can return to a representation in real numbers by combining the terms 
corresponding to complex conjugate roots.  When $r$ is a complex simple root 
of $q(s)$, then the partial fractions expansion of $p(s)/q(s)$ contains 
the two terms
\[
\frac{c}{s-r} + \frac{\overline{c}}{s-\overline{r}}
\]
for some complex scalar $c$.  These terms combine to give
\[
\frac{2c_1(s-r_1)-2c_2r_2}{(s-r_1)^2+r_2^2}
\]
where $c=c_1+ic_2$, as in \eqref{e:pfreal}.

We can write a \Matlab m-file to perform the computations in \eqref{e:pfreal}
as follows:
\begin{verbatim}
function [cr,rr] = realform(c,r)
cr = [2*real(c), -2*imag(c)*imag(r)];
rr = [real(r), imag(r)^2];
\end{verbatim}\index{partial fractions!in \Matlab!{\tt realform}}
This m-file is accessed using the command 
\begin{verbatim}
[num,denom] = realform(c(i),r(i))
\end{verbatim}
where {\tt i} is the index corresponding to the complex conjugate root of 
$q(s)$.  For example, if we consider the expansion in \eqref{e:realfex}, then we 
type 
\begin{verbatim}
[num,denom] = realform(c(2),r(2))
\end{verbatim}
yielding the answer
\begin{verbatim}
num =
     0.1176    0.5294
denom = 
    1.0000    1.0000
\end{verbatim}
This output corresponds to the expression
\[
\frac{{\tt num}(1)(s-{\tt denom}(1)) + {\tt num}(2)}{(s-{\tt denom}(1))^2 + 
{\tt denom}(2)}.
\]
Combining the second and third term on the right hand side leads to
\begin{eqnarray*}
\frac{s+1}{s^3+s^2-4s+6} & = & 
-\frac{0.1176}{s+3} + \frac{0.1176(s-1)+0.5294}{(s-1)^2+1}.
\end{eqnarray*}

\subsubsection*{Repeating a Calculation Using \Matlab}

The partial fraction expansion of \eqref{e:exlappf}, that is,
\[
\frac{s-1}{s^2+2s+5}
\]
is found by typing
\begin{verbatim}
    p = [2 -2];
    q = [1 2 5];
[c,r] = residue(p,q)
\end{verbatim}
We obtain
\begin{verbatim}
c =
   0.5000 + 0.5000i
   0.5000 - 0.5000i
r =
  -1.0000 + 2.0000i
  -1.0000 - 2.0000i
\end{verbatim}
Hence we have the expansion
\[
\frac{s-1}{s^2+2s+5}= \frac{0.5+0.5i}{s-(-1+2i)}+\frac{0.5-0.5i}{s-(-1-2i)}.
\]
Now we use {\tt realform}\index{partial fractions!in \Matlab!{\tt realform}} 
to return to a representation avoiding complex numbers.  Type
\begin{verbatim}
[num,denom] = realform(c(1),r(1))
\end{verbatim}
to obtain
\begin{verbatim}
num = 
     1    -2
denom = 
    -1     4
\end{verbatim}
which corresponds to \eqref{e:pf1}.

\EXER

\TEXER

\noindent In Exercises \ref{c13.1.0C} -- \ref{c13.1.0D} use partial fractions
to find a function $x(t)$ whose Laplace transform is the given function
$Y(s)$.
\begin{exercise} \label{c13.1.0C}
$Y(s) = \dps\frac{10}{s^3+s^2-6s}$.
\end{exercise}
\begin{exercise} \label{c13.1.0D}
$Y(s) = \dps\frac{s+2}{s^3-s}$.
\end{exercise}



\CEXER



\noindent In Exercises~\ref{c13.3.3a} -- \ref{c13.3.3c} use the \Matlab 
command {\tt residue}\index{\computer!residue}
to compute the expansion into partial fractions of 
$p(s)/q(s)$ for the given polynomials $p$ and $q$.
\begin{exercise} \label{c13.3.3a}
$p(s)=2(s-1)$ and $q(s)=s^2-3s+2$
\end{exercise}
\begin{exercise} \label{c13.3.3b}
$p(s)=s^3-6s^2-45s+50$ and $q(s)=s^4-8s^3-21s^2+8s+20$.
\end{exercise}
\begin{exercise} \label{c13.3.3c}
$p(s)=3(s-1)$ and $q(s)=s^3-s^2+4s-4$.
\end{exercise}

\begin{exercise} \label{c13.3.4}
The {\tt residue} command in \Matlab also works when the degree of the
numerator $p(s)$ is greater than the degree of the denominator $q(s)$.
In this case the answer has the form `partial fractions + polynomial'.
The {\tt residue} command stores the polynomial part in a vector {\tt k}.
To explore this feature, enter the vectors {\tt p = [2,0,-2,0]} and
{\tt q = [1,0,-4]}, and type the command
\begin{verbatim}
[c,r,k] = residue(p,q)
\end{verbatim}
Explain the result of this calculation by comparing it to the expansion
\[
\frac{2(s^3-s)}{s^2-4} = \frac{3}{s-2} + \frac{3}{s+2} +2s.
\]
\end{exercise}

\begin{exercise} \label{c13.3.4b}
Use the command {\tt residue} as in Exercise~\ref{c13.3.4} to compute an 
expansion into partial fractions for
\[
\frac{s^3-4s^2+s+6}{s^2-3s+2}.
\]
\end{exercise}



\end{document}
