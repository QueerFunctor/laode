\documentclass{ximera}

 

\usepackage{epsfig}

\graphicspath{
  {./}
  {figures/}
}

\usepackage{morewrites}
\makeatletter
\newcommand\subfile[1]{%
\renewcommand{\input}[1]{}%
\begingroup\skip@preamble\otherinput{#1}\endgroup\par\vspace{\topsep}
\let\input\otherinput}
\makeatother

\newcommand{\includeexercises}{\directlua{dofile("/home/jim/linearAlgebra/laode/exercises.lua")}}

%\newcounter{ccounter}
%\setcounter{ccounter}{1}
%\newcommand{\Chapter}[1]{\setcounter{chapter}{\arabic{ccounter}}\chapter{#1}\addtocounter{ccounter}{1}}

%\newcommand{\section}[1]{\section{#1}\setcounter{thm}{0}\setcounter{equation}{0}}

%\renewcommand{\theequation}{\arabic{chapter}.\arabic{section}.\arabic{equation}}
%\renewcommand{\thefigure}{\arabic{chapter}.\arabic{figure}}
%\renewcommand{\thetable}{\arabic{chapter}.\arabic{table}}

%\newcommand{\Sec}[2]{\section{#1}\markright{\arabic{ccounter}.\arabic{section}.#2}\setcounter{equation}{0}\setcounter{thm}{0}\setcounter{figure}{0}}

\newcommand{\Sec}[2]{\section{#1}}

\setcounter{secnumdepth}{2}
%\setcounter{secnumdepth}{1} 

%\newcounter{THM}
%\renewcommand{\theTHM}{\arabic{chapter}.\arabic{section}}

\newcommand{\trademark}{{R\!\!\!\!\!\bigcirc}}
%\newtheorem{exercise}{}

\newcommand{\dfield}{{\sf dfield9}}
\newcommand{\pplane}{{\sf pplane9}}

\newcommand{\EXER}{\section*{Exercises}}%\vspace*{0.2in}\hrule\small\setcounter{exercise}{0}}
\newcommand{\CEXER}{}%\vspace{0.08in}\begin{center}Computer Exercises\end{center}}
\newcommand{\TEXER}{} %\vspace{0.08in}\begin{center}Hand Exercises\end{center}}
\newcommand{\AEXER}{} %\vspace{0.08in}\begin{center}Hand Exercises\end{center}}

% BADBAD: \newcommand{\Bbb}{\bf}

\newcommand{\R}{\mbox{$\Bbb{R}$}}
\newcommand{\C}{\mbox{$\Bbb{C}$}}
\newcommand{\Z}{\mbox{$\Bbb{Z}$}}
\newcommand{\N}{\mbox{$\Bbb{N}$}}
\newcommand{\D}{\mbox{{\bf D}}}
\usepackage{amssymb}
%\newcommand{\qed}{\hfill\mbox{\raggedright$\square$} \vspace{1ex}}
%\newcommand{\proof}{\noindent {\bf Proof:} \hspace{0.1in}}

\newcommand{\setmin}{\;\mbox{--}\;}
\newcommand{\Matlab}{{M\small{AT\-LAB}} }
\newcommand{\Matlabp}{{M\small{AT\-LAB}}}
\newcommand{\computer}{\Matlab Instructions}
\newcommand{\half}{\mbox{$\frac{1}{2}$}}
\newcommand{\compose}{\raisebox{.15ex}{\mbox{{\scriptsize$\circ$}}}}
\newcommand{\AND}{\quad\mbox{and}\quad}
\newcommand{\vect}[2]{\left(\begin{array}{c} #1_1 \\ \vdots \\
 #1_{#2}\end{array}\right)}
\newcommand{\mattwo}[4]{\left(\begin{array}{rr} #1 & #2\\ #3
&#4\end{array}\right)}
\newcommand{\mattwoc}[4]{\left(\begin{array}{cc} #1 & #2\\ #3
&#4\end{array}\right)}
\newcommand{\vectwo}[2]{\left(\begin{array}{r} #1 \\ #2\end{array}\right)}
\newcommand{\vectwoc}[2]{\left(\begin{array}{c} #1 \\ #2\end{array}\right)}

\newcommand{\ignore}[1]{}


\newcommand{\inv}{^{-1}}
\newcommand{\CC}{{\cal C}}
\newcommand{\CCone}{\CC^1}
\newcommand{\Span}{{\rm span}}
\newcommand{\rank}{{\rm rank}}
\newcommand{\trace}{{\rm tr}}
\newcommand{\RE}{{\rm Re}}
\newcommand{\IM}{{\rm Im}}
\newcommand{\nulls}{{\rm null\;space}}

\newcommand{\dps}{\displaystyle}
\newcommand{\arraystart}{\renewcommand{\arraystretch}{1.8}}
\newcommand{\arrayfinish}{\renewcommand{\arraystretch}{1.2}}
\newcommand{\Start}[1]{\vspace{0.08in}\noindent {\bf Section~\ref{#1}}}
\newcommand{\exer}[1]{\noindent {\bf \ref{#1}}}
\newcommand{\ans}{}
\newcommand{\matthree}[9]{\left(\begin{array}{rrr} #1 & #2 & #3 \\ #4 & #5 & #6
\\ #7 & #8 & #9\end{array}\right)}
\newcommand{\cvectwo}[2]{\left(\begin{array}{c} #1 \\ #2\end{array}\right)}
\newcommand{\cmatthree}[9]{\left(\begin{array}{ccc} #1 & #2 & #3 \\ #4 & #5 &
#6 \\ #7 & #8 & #9\end{array}\right)}
\newcommand{\vecthree}[3]{\left(\begin{array}{r} #1 \\ #2 \\
#3\end{array}\right)}
\newcommand{\cvecthree}[3]{\left(\begin{array}{c} #1 \\ #2 \\
#3\end{array}\right)}
\newcommand{\cmattwo}[4]{\left(\begin{array}{cc} #1 & #2\\ #3
&#4\end{array}\right)}

\newcommand{\Matrix}[1]{\ensuremath{\left(\begin{array}{rrrrrrrrrrrrrrrrrr} #1 \end{array}\right)}}

\newcommand{\Matrixc}[1]{\ensuremath{\left(\begin{array}{cccccccccccc} #1 \end{array}\right)}}



\renewcommand{\labelenumi}{\theenumi)}
\newenvironment{enumeratea}%
{\begingroup
 \renewcommand{\theenumi}{\alph{enumi}}
 \renewcommand{\labelenumi}{(\theenumi)}
 \begin{enumerate}}
 {\end{enumerate}\endgroup}



\newcounter{help}
\renewcommand{\thehelp}{\thesection.\arabic{equation}}

%\newenvironment{equation*}%
%{\renewcommand\endequation{\eqno (\theequation)* $$}%
%   \begin{equation}}%
%   {\end{equation}\renewcommand\endequation{\eqno \@eqnnum
%$$\global\@ignoretrue}}

%\input{psfig.tex}

\author{Martin Golubitsky and Michael Dellnitz}

%\newenvironment{matlabEquation}%
%{\renewcommand\endequation{\eqno (\theequation*) $$}%
%   \begin{equation}}%
%   {\end{equation}\renewcommand\endequation{\eqno \@eqnnum
% $$\global\@ignoretrue}}

\newcommand{\soln}{\textbf{Solution:} }
\newcommand{\exercap}[1]{\centerline{Figure~\ref{#1}}}
\newcommand{\exercaptwo}[1]{\centerline{Figure~\ref{#1}a\hspace{2.1in}
Figure~\ref{#1}b}}
\newcommand{\exercapthree}[1]{\centerline{Figure~\ref{#1}a\hspace{1.2in}
Figure~\ref{#1}b\hspace{1.2in}Figure~\ref{#1}c}}
\newcommand{\para}{\hspace{0.4in}}

\renewenvironment{solution}{\suppress}{\endsuppress}

\ifxake
\newenvironment{matlabEquation}{\begin{equation}}{\end{equation}}
\else
\newenvironment{matlabEquation}%
{\let\oldtheequation\theequation\renewcommand{\theequation}{\oldtheequation*}\begin{equation}}%
  {\end{equation}\let\theequation\oldtheequation}
\fi

\makeatother


\title{The Method of Laplace Transforms}

\begin{document}
\begin{abstract}
\end{abstract}
\maketitle

 \label{S:13.1}

In this section we describe the basic properties of Laplace transforms and 
show how these properties lead to a method for solving forced equations.   
We also discuss the kind of information that we will need about Laplace 
transforms in order to solve a general second order forced equation. 

\subsection*{The Basic Properties}

Suppose that there is an operation or transform that transforms functions $x$ 
defined on a variable $t$ to functions ${\cal L}[x]$ defined on a variable 
$s$ in such a way that:
\begin{equation}  \label{Tprops}
\begin{array}{cl}
(a) & {\cal L}\; \mbox{is linear},\\
(b) & {\cal L}\; \mbox{is invertible},\; \mbox{and}\\
(c) & {\cal L}[\dot{x}](s) = s{\cal L}[x](s) - x(0).
\end{array}
\end{equation}\index{Laplace transform!general properties}
Note that \eqref{Tprops}(c) relates the transform of the derivative of a 
function to the transform of the function itself.  We call a transform that 
satisfies \eqref{Tprops} the {\em Laplace transform\/}, as there is only one 
transform that satisfies these three properties.

\subsubsection*{The Laplace Transform of $e^{at}$}

We can use properties \eqref{Tprops} to compute the Laplace transform of the
function $x(t) = e^{at}$.  To perform this calculation we need only recall 
that $e^{at}$ is the unique solution to the differential equation 
$\dot{x}=ax$ with initial value $x(0)=1$.  In essence, we can compute the 
Laplace transform of $e^{at}$ without actually having defined the Laplace 
transform.

It follows from the differential equation $\dot{x}=ax$ and \eqref{Tprops}(a) 
that
\[
{\cal L}[\dot{x}](s) = {\cal L}[ae^{at}](s) = a{\cal L}[e^{at}](s)
\]
and from \eqref{Tprops}(c) that
\[
{\cal L}[\dot{x}](s) = s{\cal L}[e^{at}](s)-1.
\]
Equating these two expressions for ${\cal L}[\dot{x}]$ leads to
\[
s{\cal L}[e^{at}](s)-1 = a{\cal L}[e^{at}](s),
\]
and hence to 
\begin{equation} \label{eq:Lexp}
{\cal L}[e^{at}](s) = \frac{1}{s-a}.
\end{equation}

\subsubsection*{Solving an Inhomogeneous Equation by Laplace Transforms}

Properties \eqref{Tprops} and formula \eqref{eq:Lexp} allow us to solve the 
initial value problem 
\begin{equation} \label{eq:expODE}
\begin{array}{crcl}
(a) & \dot{x} - 3x & = & e^{2t}\\
(b) & x(0) & = & 1.
\end{array}
\end{equation}

Before proceeding, note that \eqref{eq:expODE} can be solved directly using
the method of undetermined coefficients.  \index{undetermined coefficients}  
Indeed, undetermined coefficients show that $x_p(t) = 2e^{2t}$ is a 
particular solution to \eqref{eq:expODE}(a).  Since the general solution to the 
homogeneous equation is $\alpha e^{3t}$, the general solution to the 
inhomogeneous equation \eqref{eq:expODE}(a) is:
\[
x(t) = 2e^{2t} + \alpha e^{3t}.
\]
Setting $\alpha=-1$ solves the initial value problem \eqref{eq:expODE}(b).

We illustrate the method of Laplace transforms by solving \eqref{eq:expODE} 
using Laplace transforms.  Apply the transform ${\cal L}$ to both sides of 
\eqref{eq:expODE}, obtaining 
\[
{\cal L}[\dot{x}-3x](s) = {\cal L}[e^{2t}](s).
\]
From \eqref{eq:Lexp} with $a=2$, we see that
\[
{\cal L}[e^{2t}](s) = \frac{1}{s-2}
\]
Using \eqref{Tprops}(c,a), we see that  
\[
{\cal L}[\dot{x}-3x](s) = (s-3){\cal L}[x](s) - 1.
\]
Hence 
\[
(s-3){\cal L}[x](s) = 1 + \frac{1}{s-2}.
\]

Next, we solve for ${\cal L}[x](s)$, obtaining
\[
{\cal L}[x](s) = \frac{1}{s-3} + \frac{1}{(s-2)(s-3)}.
\]
Using partial fractions\index{partial fractions}, we find  
\[
{\cal L}[x](s)  = \frac{2}{s-3} - \frac{1}{s-2}.
\]
We describe the method of partial fractions in more detail in 
Section~\ref{S:PF}.  The particular uses of partial fractions in this section 
all follow from Exercise~\ref{Ex:pf}. 

Finally, from \eqref{eq:Lexp} we see that 
\[
{\cal L}[2e^{3t}-e^{2t}](s) = \frac{2}{s-3} - \frac{1}{s-2},
\]
and using \eqref{Tprops}(b) we conclude that
\[
x(t) = 2e^{3t}-e^{2t} 
\]
is the unique solution to the initial value problem \eqref{eq:expODE}. 

\subsection*{Three Steps in Solving Equations by Laplace Transforms} 

We can summarize the method for solving ordinary differential equations by
Laplace transforms in three steps.  In this summary it will be useful to 
have defined the inverse Laplace transform.
\begin{definition}  \label{D:invLaplace}
The {\em inverse Laplace transform\/} of a function $Y(s)$ is the function 
$y(t)$ satisfying ${\cal L}[y(t)](s) = Y(s)$, and is denoted by  
${\cal L}\inv[Y(s)]$.
\end{definition}\index{Laplace transform!inverse}

The summary of the Laplace transform method is: 
\begin{enumerate}
\item	Compute the Laplace transforms of both sides of the differential 
equation in \eqref{e:2ndforced} using the linearity of Laplace transforms, the 
formulas for Laplace transforms of derivatives \eqref{Tprops}(c), and specific 
Laplace transforms such as in \eqref{eq:Lexp}.
\item	Explicitly solve the transformed equation for 
${\cal L}[x(t)](s)=Y(s)$.
\item	Find a function $x(t)$ whose Laplace transform is $Y(s)$; that is, 
find ${\cal L}\inv[Y(s)]$.
\end{enumerate}

\subsubsection*{An Example of a First Order Forced Equation}

As an example, we use this three step method to solve the initial value 
problem:
\begin{equation}  \label{eq:dxx2}
\begin{array}{rcl}
\dot x - x & = & 2\\
  x(0) & = & 4.
\end{array}
\end{equation}

The first step is to apply the Laplace transform to both sides of the 
differential equation.  Using formula \eqref{Tprops}(c) and the linearity of 
${\cal L}$, we obtain
\[
\frac{2}{s} = {\cal L}[2] = {\cal L}[\dot x]-{\cal L}[x] = 
(s-1){\cal L}[x]-4.
\]
Note that $e^{0t}=1$ so that ${\cal L}[1] = \frac{1}{s-0}=\frac{1}{s}$.

The second step requires solving this equation explicitly for ${\cal L}[x]$
obtaining
\begin{equation}  \label{eq:Lapdxx2}
{\cal L}[x] = \frac{\frac{2}{s}+4}{s-1} = \frac{2+4s}{s(s-1)}
= \frac{6}{s-1} - \frac{2}{s}.
\end{equation}
Note that we have simplified the right hand side by use of partial fractions.

The third step requires using \eqref{eq:Lexp} to see that
\[
x(t) = {\cal L}\inv\left[\frac{6}{s-1} - \frac{2}{s}\right](t) = 
6{\cal L}\inv\left[\frac{1}{s-1}\right](t) -
2{\cal L}\inv\left[\frac{1}{s}\right](t) = 6e^t-2.
\]


\subsection*{Laplace Transforms of Second Order Equations}

We end this section with a discussion of the type of information that
we will need to solve initial value problems of second order inhomogeneous 
\index{inhomogeneous} linear equations by the method of Laplace transforms. 
Consider the differential equation
\begin{equation}  \label{e:2ndforced}
\begin{array}{rcl} 
\ddot{x} + a\dot{x} + bx & = & g(t)\\
x(0) & = & x_0\\
\dot{x}(0) & = & \dot{x}_0,
\end{array}
\end{equation}
where $a,b\in\R$ are constants.  The function $g(t)$ is called the 
{\em forcing\/} term. 

\subsubsection*{A Second Order Homogeneous Example}
\index{differential equation!higher order}

We begin our discussion by considering a simple example of an unforced 
equation:
\begin{equation}  \label{Lexam2}
\begin{array}{rcl}
\ddot{x} + 3\dot{x} +2x & = & 0  \\
 x(0) & = & 1 \\
\dot{x}(0) & = & 2.
\end{array}
\end{equation}
We begin by applying \eqref{Tprops}(c) twice to obtain
\[
\begin{array}{rcl}
{\cal L}[\ddot{x}](s) & = & s{\cal L}[\dot{x}](s) - \dot{x}(0)\\
& = & s(s{\cal L}[x](s) - x(0)) - \dot{x}(0)\\
& = & s^2{\cal L}[x](s) - sx(0) - \dot{x}(0).
\end{array}
\]
Where there is no ambiguity we will drop the argument $(s)$ in 
${\cal L}[x](s)$; this change should make subsequent formulas easier 
to read.  For instance, the previous equation is:
\begin{equation}  \label{eq:dderLap}
{\cal L}[\ddot{x}] = s^2{\cal L}[x] - sx(0) - \dot{x}(0).
\end{equation}
Now apply \eqref{Tprops} and \eqref{eq:dderLap} to the differential
equation \eqref{Lexam2} to obtain
\begin{eqnarray*}
0 & = & {\cal L}[\ddot{x} + 3\dot{x} +2x] \\
& = & (s^2{\cal L}[x] - sx(0) - \dot{x}(0)) + 
3(s{\cal L}[x] - x(0)) + 2{\cal L}[x]\\
& = & (s^2+3s+2){\cal L}[x] -(sx(0)+\dot{x}(0)+3x(0)) \\
& = & (s^2+3s+2){\cal L}[x] - (s+5).
\end{eqnarray*}

Next, use this equation and partial fractions to solve for ${\cal L}[x]$ as
\[
\dps{\cal L}[x] = \frac{s+5}{(s+1)(s+2)} = 4\frac{1}{s+1} - 3\frac{1}{s+2}.
\]
Using linearity and \eqref{eq:Lexp}, we see that 
\[
{\cal L}[4e^{-t}-3e^{-2t}] = 4\frac{1}{s+1} - 3\frac{1}{s+2}.
\]
Hence
\[
x(t) = 4e^{-t}-3e^{-2t}
\]
is the solution to \eqref{Lexam2}.


\subsubsection*{Information Needed to Solve Second Order Equations}

What information about Laplace transforms will we need to solve the 
initial value problem \eqref{e:2ndforced} in general?  We can answer this 
question just by taking the Laplace transform of \eqref{e:2ndforced} and 
solving for ${\cal L}[x]$.  Using \eqref{eq:dderLap} and \eqref{Tprops}(c), 
compute 
\begin{eqnarray*}
{\cal L}[g(t)] & = & {\cal L}[\ddot{x}+a\dot{x} + bx] \\
& = & {\cal L}[\ddot{x}] + a{\cal L}[\dot{x}]+b {\cal L}[x] \\ 
& = & (s^2{\cal L}[x]-sx_0-\dot{x}_0) +a(s{\cal L}[x]-x_0)+b{\cal L}[x]\\
& = & (s^2+as+b){\cal L}[x]-(x_0s+ax_0+\dot{x}_0).
\end{eqnarray*}
Then solve for ${\cal L}[x]$ as
\begin{equation}  \label{e:laplace2nd}
{\cal L}[x] = \frac{x_0s + ax_0+\dot{x}_0}{s^2+as+b} + \frac{G(s)}{s^2+as+b}
\end{equation}
where $G = {\cal L}[g(t)]$.  The third step in solving \eqref{e:2ndforced}
by Laplace transforms is to compute the inverse Laplace transform
\index{Laplace transform!inverse} of the right hand side of 
\eqref{e:laplace2nd}.  

To find the inverse Laplace transform of the first term, we use partial 
fractions.  For example, suppose that the polynomial $s^2+as+b$ 
has real distinct roots $r_1$ and $r_2$, then we can rewrite the first 
term as 
\[
\frac{c_1}{s-r_1} + \frac{c_2}{s-r_2}
\]
for some real constants $c_1$ and $c_2$.  We can now use \eqref{eq:Lexp}
to find the inverse Laplace transform of this first term.  

If this polynomial has a double real root or a complex conjugate pair of 
roots, then we need to find inverse Laplace transforms of functions like
\begin{equation}  \label{e:sampleLT}
\frac{1}{(s-r_1)^2}  \AND \frac{1}{(s-B)^2+C^2} \AND \frac{s}{(s-B)^2+C^2}.
\end{equation}

Finding the inverse Laplace transform for the second term is in general 
more difficult, since it depends on the hitherto unspecified function $g(t)$.
As it happens this inverse Laplace transform can be computed for a number 
of important functions, as we discuss in the next section.

To summarize: in order to use the method of Laplace transforms successfully
to solve forced second order linear equations, we must be able to
\begin{itemize}
\item	compute partial fraction\index{partial fractions} expansions,
\item	compute the inverse Laplace transforms
\index{Laplace transform!inverse} for functions in \eqref{e:sampleLT}, and
\item   compute the inverse Laplace transforms of the second term in 
\eqref{e:laplace2nd}, which involves the forcing function $g(t)$.
\end{itemize}




\includeexercises


\end{document}
