\documentclass{ximera}

 

\usepackage{epsfig}

\graphicspath{
  {./}
  {figures/}
}

\usepackage{morewrites}
\makeatletter
\newcommand\subfile[1]{%
\renewcommand{\input}[1]{}%
\begingroup\skip@preamble\otherinput{#1}\endgroup\par\vspace{\topsep}
\let\input\otherinput}
\makeatother

\newcommand{\includeexercises}{\directlua{dofile("/home/jim/linearAlgebra/laode/exercises.lua")}}

%\newcounter{ccounter}
%\setcounter{ccounter}{1}
%\newcommand{\Chapter}[1]{\setcounter{chapter}{\arabic{ccounter}}\chapter{#1}\addtocounter{ccounter}{1}}

%\newcommand{\section}[1]{\section{#1}\setcounter{thm}{0}\setcounter{equation}{0}}

%\renewcommand{\theequation}{\arabic{chapter}.\arabic{section}.\arabic{equation}}
%\renewcommand{\thefigure}{\arabic{chapter}.\arabic{figure}}
%\renewcommand{\thetable}{\arabic{chapter}.\arabic{table}}

%\newcommand{\Sec}[2]{\section{#1}\markright{\arabic{ccounter}.\arabic{section}.#2}\setcounter{equation}{0}\setcounter{thm}{0}\setcounter{figure}{0}}

\newcommand{\Sec}[2]{\section{#1}}

\setcounter{secnumdepth}{2}
%\setcounter{secnumdepth}{1} 

%\newcounter{THM}
%\renewcommand{\theTHM}{\arabic{chapter}.\arabic{section}}

\newcommand{\trademark}{{R\!\!\!\!\!\bigcirc}}
%\newtheorem{exercise}{}

\newcommand{\dfield}{{\sf dfield9}}
\newcommand{\pplane}{{\sf pplane9}}

\newcommand{\EXER}{\section*{Exercises}}%\vspace*{0.2in}\hrule\small\setcounter{exercise}{0}}
\newcommand{\CEXER}{}%\vspace{0.08in}\begin{center}Computer Exercises\end{center}}
\newcommand{\TEXER}{} %\vspace{0.08in}\begin{center}Hand Exercises\end{center}}
\newcommand{\AEXER}{} %\vspace{0.08in}\begin{center}Hand Exercises\end{center}}

% BADBAD: \newcommand{\Bbb}{\bf}

\newcommand{\R}{\mbox{$\Bbb{R}$}}
\newcommand{\C}{\mbox{$\Bbb{C}$}}
\newcommand{\Z}{\mbox{$\Bbb{Z}$}}
\newcommand{\N}{\mbox{$\Bbb{N}$}}
\newcommand{\D}{\mbox{{\bf D}}}
\usepackage{amssymb}
%\newcommand{\qed}{\hfill\mbox{\raggedright$\square$} \vspace{1ex}}
%\newcommand{\proof}{\noindent {\bf Proof:} \hspace{0.1in}}

\newcommand{\setmin}{\;\mbox{--}\;}
\newcommand{\Matlab}{{M\small{AT\-LAB}} }
\newcommand{\Matlabp}{{M\small{AT\-LAB}}}
\newcommand{\computer}{\Matlab Instructions}
\newcommand{\half}{\mbox{$\frac{1}{2}$}}
\newcommand{\compose}{\raisebox{.15ex}{\mbox{{\scriptsize$\circ$}}}}
\newcommand{\AND}{\quad\mbox{and}\quad}
\newcommand{\vect}[2]{\left(\begin{array}{c} #1_1 \\ \vdots \\
 #1_{#2}\end{array}\right)}
\newcommand{\mattwo}[4]{\left(\begin{array}{rr} #1 & #2\\ #3
&#4\end{array}\right)}
\newcommand{\mattwoc}[4]{\left(\begin{array}{cc} #1 & #2\\ #3
&#4\end{array}\right)}
\newcommand{\vectwo}[2]{\left(\begin{array}{r} #1 \\ #2\end{array}\right)}
\newcommand{\vectwoc}[2]{\left(\begin{array}{c} #1 \\ #2\end{array}\right)}

\newcommand{\ignore}[1]{}


\newcommand{\inv}{^{-1}}
\newcommand{\CC}{{\cal C}}
\newcommand{\CCone}{\CC^1}
\newcommand{\Span}{{\rm span}}
\newcommand{\rank}{{\rm rank}}
\newcommand{\trace}{{\rm tr}}
\newcommand{\RE}{{\rm Re}}
\newcommand{\IM}{{\rm Im}}
\newcommand{\nulls}{{\rm null\;space}}

\newcommand{\dps}{\displaystyle}
\newcommand{\arraystart}{\renewcommand{\arraystretch}{1.8}}
\newcommand{\arrayfinish}{\renewcommand{\arraystretch}{1.2}}
\newcommand{\Start}[1]{\vspace{0.08in}\noindent {\bf Section~\ref{#1}}}
\newcommand{\exer}[1]{\noindent {\bf \ref{#1}}}
\newcommand{\ans}{}
\newcommand{\matthree}[9]{\left(\begin{array}{rrr} #1 & #2 & #3 \\ #4 & #5 & #6
\\ #7 & #8 & #9\end{array}\right)}
\newcommand{\cvectwo}[2]{\left(\begin{array}{c} #1 \\ #2\end{array}\right)}
\newcommand{\cmatthree}[9]{\left(\begin{array}{ccc} #1 & #2 & #3 \\ #4 & #5 &
#6 \\ #7 & #8 & #9\end{array}\right)}
\newcommand{\vecthree}[3]{\left(\begin{array}{r} #1 \\ #2 \\
#3\end{array}\right)}
\newcommand{\cvecthree}[3]{\left(\begin{array}{c} #1 \\ #2 \\
#3\end{array}\right)}
\newcommand{\cmattwo}[4]{\left(\begin{array}{cc} #1 & #2\\ #3
&#4\end{array}\right)}

\newcommand{\Matrix}[1]{\ensuremath{\left(\begin{array}{rrrrrrrrrrrrrrrrrr} #1 \end{array}\right)}}

\newcommand{\Matrixc}[1]{\ensuremath{\left(\begin{array}{cccccccccccc} #1 \end{array}\right)}}



\renewcommand{\labelenumi}{\theenumi)}
\newenvironment{enumeratea}%
{\begingroup
 \renewcommand{\theenumi}{\alph{enumi}}
 \renewcommand{\labelenumi}{(\theenumi)}
 \begin{enumerate}}
 {\end{enumerate}\endgroup}



\newcounter{help}
\renewcommand{\thehelp}{\thesection.\arabic{equation}}

%\newenvironment{equation*}%
%{\renewcommand\endequation{\eqno (\theequation)* $$}%
%   \begin{equation}}%
%   {\end{equation}\renewcommand\endequation{\eqno \@eqnnum
%$$\global\@ignoretrue}}

%\input{psfig.tex}

\author{Martin Golubitsky and Michael Dellnitz}

%\newenvironment{matlabEquation}%
%{\renewcommand\endequation{\eqno (\theequation*) $$}%
%   \begin{equation}}%
%   {\end{equation}\renewcommand\endequation{\eqno \@eqnnum
% $$\global\@ignoretrue}}

\newcommand{\soln}{\textbf{Solution:} }
\newcommand{\exercap}[1]{\centerline{Figure~\ref{#1}}}
\newcommand{\exercaptwo}[1]{\centerline{Figure~\ref{#1}a\hspace{2.1in}
Figure~\ref{#1}b}}
\newcommand{\exercapthree}[1]{\centerline{Figure~\ref{#1}a\hspace{1.2in}
Figure~\ref{#1}b\hspace{1.2in}Figure~\ref{#1}c}}
\newcommand{\para}{\hspace{0.4in}}

\renewenvironment{solution}{\suppress}{\endsuppress}

\ifxake
\newenvironment{matlabEquation}{\begin{equation}}{\end{equation}}
\else
\newenvironment{matlabEquation}%
{\let\oldtheequation\theequation\renewcommand{\theequation}{\oldtheequation*}\begin{equation}}%
  {\end{equation}\let\theequation\oldtheequation}
\fi

\makeatother


\title{MATLAB}

\begin{document}
\begin{abstract}
\end{abstract}
\maketitle


\label{S:1.2}

We shall use \Matlab to compute addition and scalar
multiplication of vectors in two and three dimensions.  This
will serve the purpose of introducing some basic \Matlab
commands.

\subsection*{Entering Vectors and Vector Operations}

Begin a \Matlab session.  We now discuss how to enter a vector
into \Matlabp.  The syntax is straightforward; to enter the row
vector $x=(1,2,1)$ type\footnote{\Matlab has several useful line
editing features.  We point out two here:
\begin{enumerate}
\item[(a)]	Horizontal arrow keys ($\rightarrow,\leftarrow$)
	move the cursor one space without deleting a character.
\item[(b)]	Vertical arrow keys ($\uparrow,\downarrow$)
	recall previous and next command lines.
\end{enumerate} }

\begin{verbatim}
x = [1 2 1]
\end{verbatim}
\index{\computer![1 2 1]} and \Matlab responds with
\begin{verbatim}
x =
     1     2     1
\end{verbatim}

Next we show how easy it is to perform addition and scalar
multiplication\index{scalar multiplication!in \protect\Matlab}
in \Matlabp.  Enter the row vector $y=(2,-1,1)$ by typing
\begin{verbatim}
y = [2 -1 1]
\end{verbatim}
and \Matlab responds with
\begin{verbatim}
y =
     2    -1     1
\end{verbatim}
To add the vectors $x$ and $y$, type
\begin{verbatim}
x + y
\end{verbatim}
and \Matlab responds with
\begin{verbatim}
ans =
     3     1     2
\end{verbatim}
This vector is easily checked to be the sum of the vectors $x$
and $y$.  Similarly, to perform a scalar multiplication, type
\begin{verbatim}
2*x
\end{verbatim}
which yields
\begin{verbatim}
ans =
     2     4     2
\end{verbatim}
\Matlab subtracts the vector $y$ from the vector $x$ in the natural
way.  Type
\begin{verbatim}
x - y
\end{verbatim}
to obtain
\begin{verbatim}
ans =
    -1     3     0
\end{verbatim}

We mention two points concerning the operations that
we have just performed in \Matlabp.
\begin{enumerate}
\item[(a)] When entering a vector or a number, \Matlab
automatically echoes what has been entered.  {\em This echoing
can be suppressed by appending a semicolon to the line.\/} For
example, type
\begin{verbatim}
z = [-1 2 3];
\end{verbatim}
\index{\computer!;} and \Matlab responds with a new line awaiting a new
command.  To see the contents of the vector $z$ just type {\tt
z} and \Matlab responds with
\begin{verbatim}
z =
    -1     2     3
\end{verbatim}

\item[(b)] \Matlab stores in a new vector the information obtained by
algebraic manipulation.  Type
\begin{verbatim}
a = 2*x - 3*y + 4*z;
\end{verbatim}
Now type {\tt a} to find
\begin{verbatim}
a =
    -8    15    11
\end{verbatim}
We see that \Matlab has created a new row vector $a$ with the
correct number of entries.
\end{enumerate}
\noindent Note:  In order to use the result of a calculation later in a 
\Matlab session, we need to name the result of that calculation.  To recall 
the calculation {\tt 2*x - 3*y + 4*z}, we needed to name that calculation, 
which we did by typing {\tt a = 2*x - 3*y + 4*z}.  Then we were able to 
recall the result just by typing {\tt a}.


We have seen that we enter a row $n$ vector into \Matlab by
surrounding a list of $n$ numbers separated by spaces with
square brackets.  For example, to enter the $5$-vector
$w=(1,3,5,7,9)$ just type
\begin{verbatim}
w = [1 3 5 7 9]
\end{verbatim}
Note that the addition of two vectors is only defined when the
vectors have the same number of entries.  Trying to add the
$3$-vector $x$ with the $5$-vector $w$ by typing {\tt x + w} in \Matlab
yields the warning:
\begin{verbatim}
??? Error using ==> +
Matrix dimensions must agree.
\end{verbatim}

In \Matlab new rows are indicated by typing {\tt ;}. For
example, to enter the column vector
\[
z=\left(\begin{array}{r} -1 \\ 2\\ 3 \end{array}\right),
\]
just type:
\begin{verbatim}
z = [-1; 2; 3]
\end{verbatim}
\index{\computer![1; 2; 3]} and \Matlab responds with
\begin{verbatim}
z =
    -1
     2
     3
\end{verbatim}
Note that \Matlab will not add a row vector and a column vector.
Try typing {\tt x + z}.

Individual entries of a vector can also be addressed.  For instance,
to display the first component of {\tt z} type {\tt z(1)}.

\subsection*{Entering Matrices}

Matrices\index{matrix} are entered into \Matlab row by row with
rows separated either by semicolons or by line returns.  To enter
the $2\times 3$ matrix
\[
A=\left(\begin{array}{ccc} 2 & 3 & 1 \\ 1 & 4 & 7
\end{array}\right),
\]
just type
\begin{verbatim}
A = [2 3 1; 1 4 7]
\end{verbatim}

\Matlab has very sophisticated methods for addressing the entries
of a matrix.  You can directly address individual entries,
individual rows, and individual columns.  To display the entry in
the $1^{st}$ row, $3^{rd}$ column of $A$, type {\tt A(1,3)}.  To
display the $2^{nd}$ column of $A$, type {\tt A(:,2)}; and to
display the $1^{st}$ row of $A$, type {\tt A(1,:)}.  For example,
to add the two rows of $A$ and store them in the vector $x$,
just type
\begin{verbatim}
x = A(1,:) + A(2,:)
\end{verbatim}
\index{\computer!:}

\Matlab has many operations involving matrices --- these will
be introduced later, as needed.

\EXER

\includeexercises




\end{document}
