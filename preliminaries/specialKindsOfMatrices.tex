\documentclass{ximera}

 

\usepackage{epsfig}

\graphicspath{
  {./}
  {figures/}
}

\usepackage{morewrites}
\makeatletter
\newcommand\subfile[1]{%
\renewcommand{\input}[1]{}%
\begingroup\skip@preamble\otherinput{#1}\endgroup\par\vspace{\topsep}
\let\input\otherinput}
\makeatother

\newcommand{\includeexercises}{\directlua{dofile("/home/jim/linearAlgebra/laode/exercises.lua")}}

%\newcounter{ccounter}
%\setcounter{ccounter}{1}
%\newcommand{\Chapter}[1]{\setcounter{chapter}{\arabic{ccounter}}\chapter{#1}\addtocounter{ccounter}{1}}

%\newcommand{\section}[1]{\section{#1}\setcounter{thm}{0}\setcounter{equation}{0}}

%\renewcommand{\theequation}{\arabic{chapter}.\arabic{section}.\arabic{equation}}
%\renewcommand{\thefigure}{\arabic{chapter}.\arabic{figure}}
%\renewcommand{\thetable}{\arabic{chapter}.\arabic{table}}

%\newcommand{\Sec}[2]{\section{#1}\markright{\arabic{ccounter}.\arabic{section}.#2}\setcounter{equation}{0}\setcounter{thm}{0}\setcounter{figure}{0}}

\newcommand{\Sec}[2]{\section{#1}}

\setcounter{secnumdepth}{2}
%\setcounter{secnumdepth}{1} 

%\newcounter{THM}
%\renewcommand{\theTHM}{\arabic{chapter}.\arabic{section}}

\newcommand{\trademark}{{R\!\!\!\!\!\bigcirc}}
%\newtheorem{exercise}{}

\newcommand{\dfield}{{\sf dfield9}}
\newcommand{\pplane}{{\sf pplane9}}

\newcommand{\EXER}{\section*{Exercises}}%\vspace*{0.2in}\hrule\small\setcounter{exercise}{0}}
\newcommand{\CEXER}{}%\vspace{0.08in}\begin{center}Computer Exercises\end{center}}
\newcommand{\TEXER}{} %\vspace{0.08in}\begin{center}Hand Exercises\end{center}}
\newcommand{\AEXER}{} %\vspace{0.08in}\begin{center}Hand Exercises\end{center}}

% BADBAD: \newcommand{\Bbb}{\bf}

\newcommand{\R}{\mbox{$\Bbb{R}$}}
\newcommand{\C}{\mbox{$\Bbb{C}$}}
\newcommand{\Z}{\mbox{$\Bbb{Z}$}}
\newcommand{\N}{\mbox{$\Bbb{N}$}}
\newcommand{\D}{\mbox{{\bf D}}}
\usepackage{amssymb}
%\newcommand{\qed}{\hfill\mbox{\raggedright$\square$} \vspace{1ex}}
%\newcommand{\proof}{\noindent {\bf Proof:} \hspace{0.1in}}

\newcommand{\setmin}{\;\mbox{--}\;}
\newcommand{\Matlab}{{M\small{AT\-LAB}} }
\newcommand{\Matlabp}{{M\small{AT\-LAB}}}
\newcommand{\computer}{\Matlab Instructions}
\newcommand{\half}{\mbox{$\frac{1}{2}$}}
\newcommand{\compose}{\raisebox{.15ex}{\mbox{{\scriptsize$\circ$}}}}
\newcommand{\AND}{\quad\mbox{and}\quad}
\newcommand{\vect}[2]{\left(\begin{array}{c} #1_1 \\ \vdots \\
 #1_{#2}\end{array}\right)}
\newcommand{\mattwo}[4]{\left(\begin{array}{rr} #1 & #2\\ #3
&#4\end{array}\right)}
\newcommand{\mattwoc}[4]{\left(\begin{array}{cc} #1 & #2\\ #3
&#4\end{array}\right)}
\newcommand{\vectwo}[2]{\left(\begin{array}{r} #1 \\ #2\end{array}\right)}
\newcommand{\vectwoc}[2]{\left(\begin{array}{c} #1 \\ #2\end{array}\right)}

\newcommand{\ignore}[1]{}


\newcommand{\inv}{^{-1}}
\newcommand{\CC}{{\cal C}}
\newcommand{\CCone}{\CC^1}
\newcommand{\Span}{{\rm span}}
\newcommand{\rank}{{\rm rank}}
\newcommand{\trace}{{\rm tr}}
\newcommand{\RE}{{\rm Re}}
\newcommand{\IM}{{\rm Im}}
\newcommand{\nulls}{{\rm null\;space}}

\newcommand{\dps}{\displaystyle}
\newcommand{\arraystart}{\renewcommand{\arraystretch}{1.8}}
\newcommand{\arrayfinish}{\renewcommand{\arraystretch}{1.2}}
\newcommand{\Start}[1]{\vspace{0.08in}\noindent {\bf Section~\ref{#1}}}
\newcommand{\exer}[1]{\noindent {\bf \ref{#1}}}
\newcommand{\ans}{}
\newcommand{\matthree}[9]{\left(\begin{array}{rrr} #1 & #2 & #3 \\ #4 & #5 & #6
\\ #7 & #8 & #9\end{array}\right)}
\newcommand{\cvectwo}[2]{\left(\begin{array}{c} #1 \\ #2\end{array}\right)}
\newcommand{\cmatthree}[9]{\left(\begin{array}{ccc} #1 & #2 & #3 \\ #4 & #5 &
#6 \\ #7 & #8 & #9\end{array}\right)}
\newcommand{\vecthree}[3]{\left(\begin{array}{r} #1 \\ #2 \\
#3\end{array}\right)}
\newcommand{\cvecthree}[3]{\left(\begin{array}{c} #1 \\ #2 \\
#3\end{array}\right)}
\newcommand{\cmattwo}[4]{\left(\begin{array}{cc} #1 & #2\\ #3
&#4\end{array}\right)}

\newcommand{\Matrix}[1]{\ensuremath{\left(\begin{array}{rrrrrrrrrrrrrrrrrr} #1 \end{array}\right)}}

\newcommand{\Matrixc}[1]{\ensuremath{\left(\begin{array}{cccccccccccc} #1 \end{array}\right)}}



\renewcommand{\labelenumi}{\theenumi)}
\newenvironment{enumeratea}%
{\begingroup
 \renewcommand{\theenumi}{\alph{enumi}}
 \renewcommand{\labelenumi}{(\theenumi)}
 \begin{enumerate}}
 {\end{enumerate}\endgroup}



\newcounter{help}
\renewcommand{\thehelp}{\thesection.\arabic{equation}}

%\newenvironment{equation*}%
%{\renewcommand\endequation{\eqno (\theequation)* $$}%
%   \begin{equation}}%
%   {\end{equation}\renewcommand\endequation{\eqno \@eqnnum
%$$\global\@ignoretrue}}

%\input{psfig.tex}

\author{Martin Golubitsky and Michael Dellnitz}

%\newenvironment{matlabEquation}%
%{\renewcommand\endequation{\eqno (\theequation*) $$}%
%   \begin{equation}}%
%   {\end{equation}\renewcommand\endequation{\eqno \@eqnnum
% $$\global\@ignoretrue}}

\newcommand{\soln}{\textbf{Solution:} }
\newcommand{\exercap}[1]{\centerline{Figure~\ref{#1}}}
\newcommand{\exercaptwo}[1]{\centerline{Figure~\ref{#1}a\hspace{2.1in}
Figure~\ref{#1}b}}
\newcommand{\exercapthree}[1]{\centerline{Figure~\ref{#1}a\hspace{1.2in}
Figure~\ref{#1}b\hspace{1.2in}Figure~\ref{#1}c}}
\newcommand{\para}{\hspace{0.4in}}

\renewenvironment{solution}{\suppress}{\endsuppress}

\ifxake
\newenvironment{matlabEquation}{\begin{equation}}{\end{equation}}
\else
\newenvironment{matlabEquation}%
{\let\oldtheequation\theequation\renewcommand{\theequation}{\oldtheequation*}\begin{equation}}%
  {\end{equation}\let\theequation\oldtheequation}
\fi

\makeatother


\title{Special Kinds of Matrices}

\begin{document}
\begin{abstract}
\end{abstract}
\maketitle


\label{S:1.3}

There are many matrices that have special forms and hence
have special names --- which we now list.

\begin{itemize}

\item A {\em square\/} matrix\index{matrix!square} is a matrix
with the same number of rows and columns; that is, a square
matrix is an $n\times n$ matrix.

\item A {\em diagonal\/} matrix\index{matrix!diagonal} is a
square matrix whose only
nonzero entries are along the main diagonal; that is, $a_{ij}=0$
if $i\neq j$.  The following is a $3\times 3$ diagonal matrix
\[
\left(\begin{array}{ccc} 1 & 0 & 0 \\ 0 & 2 & 0 \\ 0 & 0 & 3
\end{array} \right).
\]
There is a shorthand in \Matlab for entering diagonal matrices. To
enter this $3\times 3$ matrix, type {\tt diag([1 2 3])}\index{\computer!diag}.

\item The {\em identity\/} matrix\index{matrix!identity} is the
diagonal matrix all of whose diagonal entries equal $1$.  The
$n\times n$ identity matrix is denoted by $I_n$.  This identity
matrix is entered in \Matlab by typing {\tt eye(n)}\index{\computer!eye}.

\item A {\em zero\/} matrix \index{matrix!zero} is a matrix all
of whose entries are $0$. A zero matrix is denoted by $0$.  This
notation is ambiguous since there is a zero $m\times n$ matrix
for every $m$ and $n$.  Nevertheless, this ambiguity rarely
causes any difficulty.  In \Matlabp, to define an $m\times n$
matrix $A$ whose entries all equal $0$, just type
{\tt A = zeros(m,n)}\index{\computer!zeros}.
To define an $n\times n$ zero matrix $B$, type {\tt B = zeros(n)}.

\item The {\em transpose\/}\index{matrix!transpose} of an $m\times n$
matrix $A$ is the $n\times m$ matrix obtained from $A$ by
interchanging rows and columns.  Thus the transpose of the
$4\times 2$ matrix
\[
\left(\begin{array}{rr} 2 & 1 \\ -1 & 2 \\ 3 & -4 \\ 5 & 7
\end{array}\right)
\]
is the $2\times 4$ matrix
\[
\left(\begin{array}{rrrr} 2 & -1 & 3 & 5 \\ 1 & 2 & -4 & 7
\end{array}\right).
\]
Suppose that you enter this $4\times 2$ matrix into \Matlab by
typing
\begin{verbatim}
A = [2 1; -1 2; 3 -4; 5 7]
\end{verbatim}
The transpose of a matrix $A$ is denoted by $A^t$.  To compute
the transpose of $A$ in \Matlabp, just type {\tt A$'$}.
\index{\computer!'}

\item A {\em symmetric\/} matrix\index{matrix!symmetric} is a
square matrix whose entries are symmetric about the main
diagonal; that is $a_{ij}=a_{ji}$.  Note that a symmetric matrix
is a square matrix $A$ for which $A^t=A$.

\item An {\em upper triangular\/} matrix\index{matrix!upper
triangular} is a square matrix all of whose entries below the
main diagonal are $0$; that is, $a_{ij}=0$ if $i>j$.  A {\em
strictly upper triangular\/} matrix is an upper triangular
matrix whose diagonal entries are also equal to $0$.  Similar
definitions hold for {\em lower triangular\/}\index{matrix!lower triangular}
and {\em strictly lower triangular\/} matrices.  The following four 
$3\times 3$ matrices are examples of upper triangular, strictly upper 
triangular, lower triangular, and strictly lower triangular matrices:
\[
\left(\begin{array}{rrr} 1 & 2 & 3\\ 0 & 2 & 4\\ 0 & 0 & 6\end{array}\right)
\quad
\left(\begin{array}{rrr} 0 & 2 & 3\\ 0 & 0 & 4\\ 0 & 0 & 0\end{array}\right)
\quad
\left(\begin{array}{rrr} 7 & 0 & 0\\ 5 & 2 & 0\\ -4 & 1 & -3\end{array}\right)
\quad
\left(\begin{array}{rrr} 0 & 0 & 0\\ 5 & 0 & 0\\ 10 & 1 & 0\end{array}\right).
\]

\item A square matrix $A$ is {\em block diagonal\/}
\index{matrix!block diagonal} if
\[
A = \left(\begin{array}{cccc} B_1 & 0 & \cdots & 0 \\
0 & B_2 & \cdots & 0 \\ \vdots & \vdots & \ddots & \vdots \\
0 & 0 & \cdots & B_k \end{array} \right)
\]
where each $B_j$ is itself a square matrix. An example of a $5\times 5$
block diagonal matrix with one $2\times 2$ block and one $3\times 3$ block is:
\[
\left(\begin{array}{ccccc} 2 & 3 & 0 & 0 & 0\\ 4 & 1 & 0 & 0 & 0\\
0 & 0 & 1 & 2 & 3\\ 0 & 0 & 3 & 2 & 4\\ 0 & 0 & 1 & 1 & 5\end{array}\right).
\]


\end{itemize}

\EXER

\TEXER

\noindent In Exercises~\ref{c1.1.01a} -- \ref{c1.1.01e} decide whether or
not the given matrix is symmetric.
\begin{exercise} \label{c1.1.01a}
 $\mattwo{2}{1}{1}{5}$.
\end{exercise}
\begin{exercise} \label{c1.1.01b}
 $\mattwo{1}{1}{0}{-5}$.
\end{exercise}
\begin{exercise} \label{c1.1.01c}
 $(3)$.
\end{exercise}
\begin{exercise} \label{c1.1.01d}
 $\left( \begin{array}{rr}
 3 & 4 \\
 4 & 3 \\
 0 & 1 \end{array} \right)$.
\end{exercise}
\begin{exercise} \label{c1.1.01e}
 $\left( \begin{array}{rrr}
 3 & 4 & -1\\
 4 & 3 &  1\\
 -1 & 1 & 10\end{array} \right)$.
\end{exercise}

\noindent In Exercises~\ref{c1.1.02a} -- \ref{c1.1.02e} decide which of
the given matrices are upper triangular and which are strictly upper
triangular.

\begin{exercise} \label{c1.1.02a}
 $\mattwo{2}{0}{-1}{-2}$.
\end{exercise}
\begin{exercise} \label{c1.1.02b}
 $\mattwo{0}{4}{0}{0}$.
\end{exercise}
\begin{exercise} \label{c1.1.02c}
 $(2)$.
\end{exercise}
\begin{exercise} \label{c1.1.02d}
 $\left( \begin{array}{rr}
 3 & 2 \\
 0 & 1 \\
 0 & 0 \end{array} \right)$.
\end{exercise}
\begin{exercise} \label{c1.1.02e}
 $\left( \begin{array}{rrr}
 0 & 2 & -4\\
 0 & 7 & -2\\
 0 & 0 & 0\end{array} \right)$.
\end{exercise}


\noindent A general $2\times 2$ diagonal matrix has the form
$\mattwo{a}{0}{0}{b}$.  Thus the two unknown real numbers $a$ and $b$ are
needed to specify each $2\times 2$ diagonal matrix.  In
Exercises~\ref{c1.3.1a} -- \ref{c1.3.3c}, how many unknown real numbers
are needed to specify each of the given matrices:
\begin{exercise}  \label{c1.3.1a}
An upper triangular $2\times 2$ matrix?
\end{exercise}
\begin{exercise}  \label{c1.3.1b}
A symmetric $2\times 2$ matrix?
\end{exercise}
\begin{exercise}  \label{c1.3.2}
An $m\times n$ matrix?
\end{exercise}
\begin{exercise}  \label{c1.3.3a}
A diagonal $n\times n$ matrix?
\end{exercise}
\begin{exercise}  \label{c1.3.3b}
An upper triangular $n\times n$ matrix?   {\bf Hint:}  Recall the
summation formula:
\[
1 + 2 + \cdots + k = \frac{k(k+1)}{2}.
\]
\end{exercise}
\begin{exercise}  \label{c1.3.3c}
A symmetric $n\times n$ matrix? 
\end{exercise}

\noindent In each of Exercises~\ref{c1.3.4a} -- \ref{c1.3.4c} determine
whether the statement is {\em True\/} or {\em False\/}?
\begin{exercise} \label{c1.3.4a}
 Every symmetric, upper triangular matrix is diagonal.
\end{exercise}
\begin{exercise} \label{c1.3.4b}
 Every diagonal matrix is a multiple of the identity matrix.
\end{exercise}
\begin{exercise} \label{c1.3.4c}
 Every block diagonal matrix is symmetric.
\end{exercise}

\CEXER

\begin{exercise} \label{c1.3.5a}
Use \Matlab to compute $A^t$ when
\begin{equation} 
A = \left(\begin{array}{cccc} 
1 & 2 & 4 & 7 \\ 2 & 1 & 5 & 6 \\  4 & 6 & 2 & 1 
\end{array}\right)
\end{equation}
Use \Matlab to verify that $(A^t)^t=A$ by setting {\tt B=A'}, {\tt C=B'},
and checking that $C=A$.
\end{exercise}

\begin{exercise} \label{c1.3.5b}
Use \Matlab to compute $A^t$ when $A=(3)$ is a $1\times 1$ matrix.
\end{exercise}


\end{document}
