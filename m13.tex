\documentclass{ximera}

 

\usepackage{epsfig}

\graphicspath{
  {./}
  {figures/}
}

\usepackage{morewrites}
\makeatletter
\newcommand\subfile[1]{%
\renewcommand{\input}[1]{}%
\begingroup\skip@preamble\otherinput{#1}\endgroup\par\vspace{\topsep}
\let\input\otherinput}
\makeatother

\newcommand{\includeexercises}{\directlua{dofile("/home/jim/linearAlgebra/laode/exercises.lua")}}

%\newcounter{ccounter}
%\setcounter{ccounter}{1}
%\newcommand{\Chapter}[1]{\setcounter{chapter}{\arabic{ccounter}}\chapter{#1}\addtocounter{ccounter}{1}}

%\newcommand{\section}[1]{\section{#1}\setcounter{thm}{0}\setcounter{equation}{0}}

%\renewcommand{\theequation}{\arabic{chapter}.\arabic{section}.\arabic{equation}}
%\renewcommand{\thefigure}{\arabic{chapter}.\arabic{figure}}
%\renewcommand{\thetable}{\arabic{chapter}.\arabic{table}}

%\newcommand{\Sec}[2]{\section{#1}\markright{\arabic{ccounter}.\arabic{section}.#2}\setcounter{equation}{0}\setcounter{thm}{0}\setcounter{figure}{0}}

\newcommand{\Sec}[2]{\section{#1}}

\setcounter{secnumdepth}{2}
%\setcounter{secnumdepth}{1} 

%\newcounter{THM}
%\renewcommand{\theTHM}{\arabic{chapter}.\arabic{section}}

\newcommand{\trademark}{{R\!\!\!\!\!\bigcirc}}
%\newtheorem{exercise}{}

\newcommand{\dfield}{{\sf dfield9}}
\newcommand{\pplane}{{\sf pplane9}}

\newcommand{\EXER}{\section*{Exercises}}%\vspace*{0.2in}\hrule\small\setcounter{exercise}{0}}
\newcommand{\CEXER}{}%\vspace{0.08in}\begin{center}Computer Exercises\end{center}}
\newcommand{\TEXER}{} %\vspace{0.08in}\begin{center}Hand Exercises\end{center}}
\newcommand{\AEXER}{} %\vspace{0.08in}\begin{center}Hand Exercises\end{center}}

% BADBAD: \newcommand{\Bbb}{\bf}

\newcommand{\R}{\mbox{$\Bbb{R}$}}
\newcommand{\C}{\mbox{$\Bbb{C}$}}
\newcommand{\Z}{\mbox{$\Bbb{Z}$}}
\newcommand{\N}{\mbox{$\Bbb{N}$}}
\newcommand{\D}{\mbox{{\bf D}}}
\usepackage{amssymb}
%\newcommand{\qed}{\hfill\mbox{\raggedright$\square$} \vspace{1ex}}
%\newcommand{\proof}{\noindent {\bf Proof:} \hspace{0.1in}}

\newcommand{\setmin}{\;\mbox{--}\;}
\newcommand{\Matlab}{{M\small{AT\-LAB}} }
\newcommand{\Matlabp}{{M\small{AT\-LAB}}}
\newcommand{\computer}{\Matlab Instructions}
\newcommand{\half}{\mbox{$\frac{1}{2}$}}
\newcommand{\compose}{\raisebox{.15ex}{\mbox{{\scriptsize$\circ$}}}}
\newcommand{\AND}{\quad\mbox{and}\quad}
\newcommand{\vect}[2]{\left(\begin{array}{c} #1_1 \\ \vdots \\
 #1_{#2}\end{array}\right)}
\newcommand{\mattwo}[4]{\left(\begin{array}{rr} #1 & #2\\ #3
&#4\end{array}\right)}
\newcommand{\mattwoc}[4]{\left(\begin{array}{cc} #1 & #2\\ #3
&#4\end{array}\right)}
\newcommand{\vectwo}[2]{\left(\begin{array}{r} #1 \\ #2\end{array}\right)}
\newcommand{\vectwoc}[2]{\left(\begin{array}{c} #1 \\ #2\end{array}\right)}

\newcommand{\ignore}[1]{}


\newcommand{\inv}{^{-1}}
\newcommand{\CC}{{\cal C}}
\newcommand{\CCone}{\CC^1}
\newcommand{\Span}{{\rm span}}
\newcommand{\rank}{{\rm rank}}
\newcommand{\trace}{{\rm tr}}
\newcommand{\RE}{{\rm Re}}
\newcommand{\IM}{{\rm Im}}
\newcommand{\nulls}{{\rm null\;space}}

\newcommand{\dps}{\displaystyle}
\newcommand{\arraystart}{\renewcommand{\arraystretch}{1.8}}
\newcommand{\arrayfinish}{\renewcommand{\arraystretch}{1.2}}
\newcommand{\Start}[1]{\vspace{0.08in}\noindent {\bf Section~\ref{#1}}}
\newcommand{\exer}[1]{\noindent {\bf \ref{#1}}}
\newcommand{\ans}{}
\newcommand{\matthree}[9]{\left(\begin{array}{rrr} #1 & #2 & #3 \\ #4 & #5 & #6
\\ #7 & #8 & #9\end{array}\right)}
\newcommand{\cvectwo}[2]{\left(\begin{array}{c} #1 \\ #2\end{array}\right)}
\newcommand{\cmatthree}[9]{\left(\begin{array}{ccc} #1 & #2 & #3 \\ #4 & #5 &
#6 \\ #7 & #8 & #9\end{array}\right)}
\newcommand{\vecthree}[3]{\left(\begin{array}{r} #1 \\ #2 \\
#3\end{array}\right)}
\newcommand{\cvecthree}[3]{\left(\begin{array}{c} #1 \\ #2 \\
#3\end{array}\right)}
\newcommand{\cmattwo}[4]{\left(\begin{array}{cc} #1 & #2\\ #3
&#4\end{array}\right)}

\newcommand{\Matrix}[1]{\ensuremath{\left(\begin{array}{rrrrrrrrrrrrrrrrrr} #1 \end{array}\right)}}

\newcommand{\Matrixc}[1]{\ensuremath{\left(\begin{array}{cccccccccccc} #1 \end{array}\right)}}



\renewcommand{\labelenumi}{\theenumi)}
\newenvironment{enumeratea}%
{\begingroup
 \renewcommand{\theenumi}{\alph{enumi}}
 \renewcommand{\labelenumi}{(\theenumi)}
 \begin{enumerate}}
 {\end{enumerate}\endgroup}



\newcounter{help}
\renewcommand{\thehelp}{\thesection.\arabic{equation}}

%\newenvironment{equation*}%
%{\renewcommand\endequation{\eqno (\theequation)* $$}%
%   \begin{equation}}%
%   {\end{equation}\renewcommand\endequation{\eqno \@eqnnum
%$$\global\@ignoretrue}}

%\input{psfig.tex}

\author{Martin Golubitsky and Michael Dellnitz}

%\newenvironment{matlabEquation}%
%{\renewcommand\endequation{\eqno (\theequation*) $$}%
%   \begin{equation}}%
%   {\end{equation}\renewcommand\endequation{\eqno \@eqnnum
% $$\global\@ignoretrue}}

\newcommand{\soln}{\textbf{Solution:} }
\newcommand{\exercap}[1]{\centerline{Figure~\ref{#1}}}
\newcommand{\exercaptwo}[1]{\centerline{Figure~\ref{#1}a\hspace{2.1in}
Figure~\ref{#1}b}}
\newcommand{\exercapthree}[1]{\centerline{Figure~\ref{#1}a\hspace{1.2in}
Figure~\ref{#1}b\hspace{1.2in}Figure~\ref{#1}c}}
\newcommand{\para}{\hspace{0.4in}}

\renewenvironment{solution}{\suppress}{\endsuppress}

\ifxake
\newenvironment{matlabEquation}{\begin{equation}}{\end{equation}}
\else
\newenvironment{matlabEquation}%
{\let\oldtheequation\theequation\renewcommand{\theequation}{\oldtheequation*}\begin{equation}}%
  {\end{equation}\let\theequation\oldtheequation}
\fi

\makeatother


\title{m13.tex}

\begin{document}
\begin{abstract}
BADBAD
\end{abstract}
\maketitle

\chapter{Matrix Normal Forms}

\subsection*{Section~\protect{\ref{S:RDM}} Real Diagonalizable Matrices}
\rhead{S:RDM}{REAL DIAGONALIZABLE MATRICES}

\exer{c10.3.1}
(a) The eigenvalues of $A$ are $\lambda_1 = 3$ and $\lambda_2 = -3$,
with corresponding eigenvectors $v_1 = (1,1)^t$ and $v_2 = (1,-1)^t$,
respectively.

(b) Let
\[
S = (v_1|v_2) = \mattwo{1}{1}{1}{-1}.
\]
Then $D = S^{-1}AS = \mattwo{3}{0}{0}{-3}$ is a diagonal matrix.

\exer{c10.3.2}
The eigenvectors of $A$ are $v_1 = (1,1,-1)^t$ associated to eigenvalue
$\lambda_1 = 2$; $v_2 = (1,-1,-1)^t$ associated to eigenvalue
$\lambda_2 = -2$; and $v_3 = (1,-1,1)^t$ associated to eigenvalue
$\lambda_3 = -4$.  Find these vectors by solving $(A - \lambda I_3)v = 0$
for each eigenvalue $\lambda$.  The matrix $S$ such that $S^{-1}AS$ is
diagonal is
\[
S = (v_1|v_2|v_3) = \matthree{1}{1}{1}{1}{-1}{-1}{-1}{-1}{1}.
\]

\exer{c10.3.3}
The eigenvalues of $A$ are $\lambda_1 = 1$ and $\lambda_2 = -1$.  The
eigenvector associated to $\lambda_1$ is $v_1 = (1,1,1)^t$.  There are
two eigenvectors associated to $\lambda_2$: $v_2 = (1,0,0)^t$ and
$v_3 = (0,1,2)^t$.
\[
S = (v_1|v_2|v_3) = \matthree{1}{1}{0}{1}{0}{1}{1}{0}{2}.
\]

\exer{c10.3.4}
(a) Let $B = P^{-1}AP$ be a matrix similar to some invertible matrix $A$.
Then 
\[
B^{-1} = (P^{-1}AP)^{-1} = P^{-1}A^{-1}(P^{-1})^{-1} = P^{-1}A^{-1}P.
\]
Since $A^{-1}$ exists, $B^{-1}$ exists also.

(b) If $B = P^{-1}AP$, then $B^{-1} = (P^{-1}AP)^{-1} = P^{-1}A^{-1}P$.
Therefore,
\[
B + B^{-1} = P^{-1}AP + P^{-1}A^{-1}P = P^{-1}(A + A^{-1})P
\]
since matrix multiplication is associative.  Therefore, $A + A^{-1}$ is
similar to $B + B^{-1}$.

\exer{c10.3.5}
Let $A = P^{-1}BP$ for some invertible matrix $P$, and let
$D = S^{-1}AS$, where $D$ is a diagonal matrix.  Then
\[
D = S^{-1}AS = S^{-1}(P^{-1}BP)S = (S^{-1}P^{-1})B(PS) = 
(PS)^{-1}B(PS).
\]
Therefore, $B$ is also similar to $D$, so $B$ is real diagonalizable.

\exer{c10.3.6}
Let $S$ be a matrix such that $D = S^{-1}AS$ is a diagonal matrix.
Then
\[
S^{-1}(A + \alpha I_n)S = S^{-1}AS + S^{-1}(\alpha I_n)S =
D + \alpha I_n.
\]
The matrices $D$ and $\alpha I_n$ are both diagonal; so $D + \alpha I_n$
is also diagonal.  Therefore, $A + \alpha I_n$ is diagonalizable.


\exer{c10.3.6A}  We assume that $Av=\lambda v$ and that $AB=BA$.  It follows that 
$ABv=BAv=\lambda Bv$.  Therefore $Bv$ is an eigenvector of $A$ with eigenvalue
$\lambda$.  Since $\lambda$ has only one independent eigenvector, it follows that 
$v$ is a multiple of $v$; that is, there is a scalar $\mu$ such that $Bv=\mu v$.

\exer{c10.3.6B}  Suppose that $BA=AB$ and that the eigenvalues of $A$ are distinct. 
By Theorem~\ref{T:diagsimple} and Lemma~\ref{L:eigenv-diag}, 
there is a basis $v_1,\ldots,v_n$ of $\R^n$ consisting of eigenvectors of $A$.  By 
Exercise~\ref{c10.3.6A}, these vectors are also eigenvectors of $B$.  Let 
$S=(v_1|\cdots|v_n)$.  Then both matrices $S\inv AS$ and $S\inv BS$ are diagonal 
matrices.

\exer{c10.3.6C} Since $A$ is diagonalizable, there is an invertible matrix $S$ such 
that $S\inv AS$ is diagonal.  The diagonal entries of  $S\inv AS$ are the eigenvalues
of $A$; that is, the diagonal entries equal $\pm 1$.  Therefore, $(S\inv AS)^2=I_n$.
But $(S\inv AS)^2 = S\inv A^2 S$.   Therefore, $S\inv A^2 S=I_n$ which implies that
$A^2=I_n$.

\exer{c10.3.6D}  Since $A$ is diagonalizable, there is an invertible matrix $S$ such 
that $S\inv AS$ is diagonal.  The diagonal entries of  $S\inv AS$ are the eigenvalues
of $A$; that is, the diagonal entries equal $0$ and $1$.  Therefore, 
$(S\inv AS)^2=S\inv AS$.  But $(S\inv AS)^2 = S\inv A^2 S$.   Therefore, 
$S\inv A^2 S= S\inv AS$ which implies that $A^2=A$.

\exer{c10.3.7}
Verify that the eigenvalues of $C$ are real and distinct using the
\Matlab command {\tt eig(C)}, which yields:
\begin{verbatim}
ans =
   -4.0000
  -12.0000
   -8.0000
  -16.0000
\end{verbatim}
In order to find the matrix $S$, either use the {\tt null} command to
find the eigenvectors of $C$ individually, or type
{\tt [S,D] = eig(C)} to obtain the matrix $S$ and the diagonal matrix
$D = S^{-1}CS$:
\begin{verbatim}
S =
    0.5314   -0.5547    0.0000    0.4082
   -0.4871    0.5547   -0.4082   -0.8165
    0.6199   -0.5547    0.8165    0.4082
   -0.3100    0.2774   -0.4082    0.0000

D =
   -4.0000         0         0         0
         0  -12.0000         0         0
         0         0   -8.0000         0
         0         0         0  -16.0000
\end{verbatim}

\exer{c10.3.8a}
\ans Matrix $A$ is real diagonalizable.

\soln Compute the eigenvalues of $A$ using \Matlabp.  By
Theorem~\ref{T:diagsimple}, a matrix is real diagonalizable if it has
real distinct eigenvalues.  Thus, $A$ is real diagonalizable.

\exer{c10.3.8b}
\ans Matrix $B$ is not real diagonalizable.

\soln Compute the eigenvalues of $B$ using \Matlabp.  If a matrix is
real diagonalizable, it has real eigenvalues.  Matrix $B$ has complex
eigenvectors, and is therefore not real diagonalizable.


\subsection*{Section~\protect{\ref{S:CSE}} Complex Simple Eigenvalues}
\rhead{S:CSE}{COMPLEX SIMPLE EIGENVALUES}

\exer{c10.4.3} 
\ans
\[
T = \cmattwo{1 - i}{1 + i}{2i}{-2i}
\AND S = \mattwo{1}{-1}{0}{2}.
\]

\soln $T$ is the matrix whose columns are the complex eigenvectors
of $A$.  Multiplying $T^{-1}AT$ yields
\[
\cmattwo{2 + i}{0}{0}{2 - i},
\]
the matrix with the eigenvalues of $A$ along the main diagonal.  The
first column of the matrix $S$ contains the real part of the eigenvector
$(1 + i, -2i)^t$, and the second column contains the imaginary part of
this eigenvector. Multiplying $S^{-1}AS$ yields
\[
\cmattwo{2}{-1}{1}{2} = \mattwo{\sigma}{-\tau}{\tau}{\sigma}
\]
where $\sigma - i\tau$ is the eigenvalue $2 - i$ associated to
eigenvector $(1 + i, -2i)^t$.

\exer{c10.4.4}
\ans $T = \mattwoc{5}{5}{-1+3i}{-1-3i}$ and $S=\mattwoc{5}{-1}{0}{3}$ 

\soln The eigenvalues of $A$ are $1\pm 3i$ and the associated eigenvectors 
are $(5,-1\pm 3i)^t$.  The columns of $T$ are just the eigenvectors of $A$
and the real and imaginary parts of the eigenvectors are the columns of $S$. 


\exer{c10.4.rotate}
Matrix $A$ rotates a vector by $45^{\circ}$ counterclockwise, then expands
the vector by a factor of $\sqrt{2}$.  Let $v_1 = Av_0$, $v_2 =
Av_1$, and so on.  Note that $v_8 = 16v_0$.  That is, after eight
iterations, the vector points in its original direction and has length
$16|v_0|$.

\para Let $w_1 = Bv_0$ and $w_2 = Bw_1$.  Then, $w_8 = 16v_0
= v_8$.  The rotation and dilatation caused by matrix $B$ coincides
with that of matrix $A$ every 8 iterations, or $360^\circ$.  However,
$B$ does not cause a single constant rotation or dilatation on each
iteration, as does $A$.

\exer{c10.4.6}
(a) Enter the matrix in \Matlabp, then type {\tt [S,D] = eig(A)} to
find the eigenvectors 
S =
 
   0.8165             0.8165
  -0.4082 - 0.4083i  -0.4082 + 0.4083i

$v_1 = (0.8165, -0.4082 - 0.4083i)^t$ and
$v_2 = (0.8165, -0.4082 + 0.4083i)^t$ with eigenvalues
$\lambda_1 = 0.1559 + 0.9878i$ and $\lambda_2 = 0.1559 - 0.9878i$.

(b) The real block diagonal normal form of $A$ is
\[
R = \mattwo{\sigma}{-\tau}{\tau}{\sigma} = \mattwo{0.1559}{-0.9878}
{0.9878}{0.1559}
\]
where $\lambda_1 = \sigma + i\tau$.  Note that $|\lambda_j|=1$.  
This matrix rotates a vector through an angle of 
$\cos^{-1}(0.1559) \approx \sqrt{2}$, and does not alter its length.  

\newpage
(c) In this case, the matrix rotates and dilatates the vector so that
the endpoints of the iterated vectors lie on an ellipse.

\exer{c10.4.7a}
\ans The matrices are:
\begin{verbatim}
T =
   0.9690             0.0197 + 0.3253i   0.0197 - 0.3253i
   0.1840             0.0506 - 0.5592i   0.0506 + 0.5592i
   0.1647            -0.4757 - 0.5935i  -0.4757 + 0.5935i
S =
   0.9690             0.0197        0.3253   
   0.1840             0.0506       -0.5592 
   0.1647            -0.4757       -0.5935 
\end{verbatim}

\soln The matrix $T$ is the matrix whose columns are eigenvectors of $A$. 
Find $T$ in \Matlab by typing {\tt [T,D] = eig(A)}.  The second and third columns
of $S$ correspond to the real and imaginary parts of the second eigenvector
of $A$.  Verify these solutions by noting that
{\tt inv(T)*A*T} yields a diagonal matrix with the eigenvalues of $A$
along the diagonal, and that {\tt inv(S)*A*S} yields a block matrix
with blocks consisting of the real and imaginary parts of each complex
conjugate pair of eigenvalues.

\exer{c10.4.7b}
\ans The matrices are:
\begin{verbatim}
T =
  -0.1818 + 0.0422i  -0.1818 - 0.0422i   0.2143 - 0.0122i   0.2143 + 0.0122i
  -0.3649 - 0.0311i  -0.3649 + 0.0311i   0.3154 + 0.0613i   0.3154 - 0.0613i
   0.7291             0.7291            -0.6934            -0.6934
  -0.5465 + 0.0289i  -0.5465 - 0.0289i   0.6072 - 0.0347i   0.6072 + 0.0347i
S =
   -0.1818    0.0422   0.2143  -0.0122
   -0.3649   -0.0311   0.3154   0.0613
    0.7291       0    -0.6934      0
   -0.5465    0.0289   0.6072  -0.0347
\end{verbatim}

\soln The matrix $T$ is the matrix whose columns are eigenvectors of $A$. 
Find $T$ in \Matlab by typing {\tt [T,D] = eig(A)}.  The first two columns
of $S$ correspond to the real and imaginary parts of the first eigenvector
of $A$, and the last two columns contain the real and imaginary parts of
the third eigenvector.  Type 
\begin{verbatim}
S = [real(T(:,1)) imag(T(:,1)) real(T(:,3)) imag(T(:,3))]   
\end{verbatim}
in Matlab to compute $S$.  Verify these solutions by noting that
{\tt inv(T)*A*T} yields a diagonal matrix with the eigenvalues of $A$
along the diagonal, and that {\tt inv(S)*A*S} yields a block matrix
with blocks consisting of the real and imaginary parts of each complex
conjugate pair of eigenvalues.

\newpage
\exer{c10.4.7c}
\ans The matrices are:
\begin{verbatim}
T =
  Columns 1 through 4 
  -0.1933-0.2068i  -0.1933+0.2068i  -0.6791+0.5708i  -0.6791-0.5708i
  -0.0362+0.4192i  -0.0362-0.4192i   0.2735-0.3037i   0.2735+0.3037i
   0.4084+0.1620i   0.4084-0.1620i   0.0881+0.0243i   0.0881-0.0243i
  -0.0000-0.0000i  -0.0000+0.0000i  -0.0000+0.0000i  -0.0000-0.0000i
  -0.1933-0.2068i  -0.1933+0.2068i  -0.1321-0.0365i  -0.1321+0.0365i
   0.2657-0.6317i   0.2657+0.6317i   0.1321+0.0365i   0.1321-0.0365i
  Columns 5 through 6 
   0.4205-0.1238i   0.4205+0.1238i
   0.0855+0.2601i   0.0855-0.2601i
  -0.1639-0.1479i  -0.1639+0.1479i
  -0.5203+0.1710i  -0.5203-0.1710i
   0.4205-0.1238i   0.4205+0.1238i
  -0.4205+0.1238i  -0.4205-0.1238i
S =
   -0.1933   -0.2068   -0.6791    0.5708    0.4205   -0.1238
   -0.0362    0.4192    0.2735   -0.3037    0.0855    0.2601
    0.4084    0.1620    0.0881    0.0243   -0.1639   -0.1479
   -0.0000   -0.0000   -0.0000    0.0000   -0.5203    0.1710
   -0.1933   -0.2068   -0.1321   -0.0365    0.4205   -0.1238
    0.2657   -0.6317    0.1321    0.0365   -0.4205    0.1238
\end{verbatim}

\soln The matrix $T$ is the matrix whose columns are eigenvectors of $A$. 
Find $T$ in \Matlab by typing {\tt [T,D] = eig(A)}.  The first two columns
of $S$ correspond to the real and imaginary parts of the first eigenvector
of $A$, the third and fourth columns of $S$ contain the real and imaginary parts of 
the third eigenvector, and the last two columns contain the real and imaginary 
parts of the fifth eigenvector.  Verify these solutions by noting that
{\tt inv(T)*A*T} yields a diagonal matrix with the eigenvalues of $A$
along the diagonal, and that {\tt inv(S)*A*S} yields a block matrix
with blocks consisting of the real and imaginary parts of each complex
conjugate pair of eigenvalues.


\exer{c10.4.7d}
\ans The matrices are:
\begin{verbatim}
T =
   0.9602             0.4405 - 0.1340i   0.4405 + 0.1340i
  -0.0818             0.5397 - 0.0860i   0.5397 + 0.0860i
   0.2671             0.0569 + 0.6972i   0.0569 - 0.6972i
S =
    0.9602    0.4405   -0.1340
   -0.0818    0.5397   -0.0860
    0.2671    0.0569    0.6972
\end{verbatim}

\soln The matrix $T$ is the matrix whose columns are eigenvectors of $A$. 
Find $T$ in \Matlab by typing {\tt [T,D] = eig(A)}.  The second and third columns
of $S$ correspond to the real and imaginary parts of the second eigenvector
of $A$.  Verify these solutions by noting that
{\tt inv(T)*A*T} yields a diagonal matrix with the eigenvalues of $A$
along the diagonal, and that {\tt inv(S)*A*S} yields a block matrix
with blocks consisting of the real and imaginary parts of each complex
conjugate pair of eigenvalues.


\subsection*{Section~\protect{\ref{S:MGE}} Multiplicity and Generalized Eigenvectors}
\rhead{S:MGE}{MULTIPLICITY AND GENERALIZED EIGENVECTORS}

\exer{c10.5.1a} The eigenvalues of matrix $A$ are:
\begin{center}
\begin{tabular}{|c|c|c|}
\hline
eigenvalue & algebraic multiplicity & geometric multiplicity \\
\hline
$2$ & $1$ & $1$ \\
$3$ & $2$ & $1$ \\
$4$ & $1$ & $1$ \\
\hline
\end{tabular}
\end{center}


\exer{c10.5.1b} The eigenvalues of matrix $B$ are:
\begin{center}
\begin{tabular}{|c|c|c|}
\hline
eigenvalue & algebraic multiplicity & geometric multiplicity \\
\hline
$2$ & $2$ & $2$ \\
$3$ & $2$ & $1$ \\
\hline
\end{tabular}
\end{center}

\exer{c10.5.1c} The eigenvalues of matrix $C$ are:
\begin{center}
\begin{tabular}{|c|c|c|}
\hline
eigenvalue & algebraic multiplicity & geometric multiplicity \\
\hline
$-1$ & $3$ & $2$ \\
$1$ & $1$ & $1$ \\
\hline
\end{tabular}
\end{center}

\exer{c10.5.1d} The eigenvalues of matrix $D$ are:
\begin{center}
\begin{tabular}{|c|c|c|}
\hline
eigenvalue & algebraic multiplicity & geometric multiplicity \\
\hline
$2 + i$ & $1$ & $1$ \\
$2 - i$ & $1$ & $1$ \\
$2$ & $2$ & $1$ \\
\hline
\end{tabular}
\end{center}

\exer{c10.5.2a} \ans $v_1=(-1,1)$ and $v_2 = (0,1)$ is a basis.

\soln  The characteristic polynomial of $A$ is $\lambda^2-4\lambda+4=(\lambda-2)^2$.
Therefore, the eigenvalues of $A$ both equal $2$.  Find the eigenvectors of $A$ by 
solving $(A-2I_2)v=0$.  That is, solve
\[
\mattwo{-1}{-1}{1}{1}v = 0.
\]
$v_1=(-1,1)$ is the only independent solution.  Choose any vector for $v_2$ that is
independent of $v_1$.

\exer{c10.5.2b} \ans $v_1=(1,-1,-1)$, $v_2=(-2,0,1)$, $v_3=(0,1,1)$ is a basis.

\soln  Determine either by direct calculation of the characteristic polyomial of $B$ 
or by using \Matlab that the eiegenvalues of $B$ are $0,-1,-1$.  Let $v_1$ be the
eigenvector with $0$ eigenvalue; so $v_1=(1,-1,-1)$.  There is only one independent 
eigenvector associated with the eigenvalue $-1$ and that eigenvector is
$v_2=(-2,0,1)$.  Let $v_3$ be any generalized eigenvector associated with the
eigenvalue $-1$; one choice is $v_3=(0,1,1)$.  These eigenvectors can be found by
direct calculation or by using \Matlab.  When using \Matlab find $v_1$ by typing {\tt
null(B)}, $v_2$ by typing {\tt null(B+eye(3))}, and $v_3$ by typing 
{\tt null((B+eye(3))\^{}2)}

\exer{c10.5.2c} \ans $v_1=(9,1,-1)$, $v_2=(-2,0,1)$, $v_3=(9,1,-2)$ is a basis.

\soln  Determine either by direct calculation of the characteristic polyomial of $C$ 
or by using \Matlab that the eiegenvalues of $C$ are $-1,1,1$.  Let $v_1$ be the
eigenvector with $-1$ eigenvalue; so $v_1=(9,1,-1)$.  There is only one independent 
eigenvector associated with the eigenvalue $1$ and that eigenvector is
$v_2=(-2,0,1)$.  Let $v_3$ be any generalized eigenvector associated with the
eigenvalue $1$; one choice is $v_3=(9,1,-2)$.  These eigenvectors can be found by
direct calculation or by using \Matlab.  When using \Matlab find $v_1$ by typing {\tt
null(C+eye(3))}, $v_2$ by typing {\tt null(D-eye(3))}, and $v_3$ by typing 
{\tt null((C-eye(3))\^{}2)}

\exer{c10.5.2d} \ans $v_1=(1,-3,0)$, $v_2=(0,1,-2)$, $v_3=(1,0,0)$ is a basis.

\soln  Determine either by direct calculation of the characteristic polyomial of $D$ 
or by using \Matlab that the eiegenvalues of $C$ are $2,2,2$.  Let $v_1$ be the
eigenvector with eigenvalue $2$; so $v_1=(1,-3,0)$.  There is only one independent 
generalized eigenvector of index $2$ associated with the eigenvalue $2$ and that 
generalized eigenvector is $v_2=(0,1,-2)$.  Let $v_3$ be any generalized eigenvector 
of index $3$ associated with the eigenvalue $2$; one choice is $v_3=(1,0,0)$.  These 
eigenvectors can be found by direct calculation or by using \Matlab.  When using 
\Matlab find $v_1$ by typing {\tt null(C-2*eye(3))} and $v_2$ by typing 
{\tt null((D-eye(3))\^{}2)}.

\exer{c10.5.3A} \ans The eigenvalue $2$ has algebraic multiplicity $4$ and geometric
multiplicity $1$.

\soln Using \Matlab to find eigenvalues of high algebraic multiplicity is numerically
dangerous. Type {\tt eig(A)} and obtain
\begin{verbatim}
   2.0000 + 0.0006i
   2.0000 - 0.0006i
   1.9994          
   2.0006          
\end{verbatim}
Since the coefficients of $A$ are all integers, you might be suspicious of the answer
and guess that all of the eigenvalues of $A$ equal $2$.  Type {\tt null(A-2*eye(4))}
and obtain
\begin{verbatim}
ans =
   -0.4804
   -0.8006
   -0.1601
    0.3203
\end{verbatim}
dividing by {\tt ans(3)} yields the eigenvector $v_1=(3,5,1,-2)$.  To check whether
the eigenvalue $2$ has algebraic multiplicity greater than $1$, type 
{\tt null((A-eye(4))\^{}2)} and obtain
\begin{verbatim}
ans =
    0.5071         0
    0.8452         0
    0.1690         0
         0    1.0000
\end{verbatim}
Thus $v_2=(0,0,0,1)$ is a generalized eigenvector of $A$ with index $2$ and eigenvalue
$2$.  To find generalized eigenvectors of index $3$ type {\tt null((A-eye(4))\^{}3)} 
and obtain
\begin{verbatim}
ans =
   -0.9487         0         0
         0    1.0000         0
   -0.3162         0         0
         0         0    1.0000
\end{verbatim}
Thus, $v_3=(0,1,0,0)$ is a generalized eigenvector of index $3$.  Type 
{\tt null((A-eye(4))\^{}4)} and obtain
\begin{verbatim}
ans =
     1     0     0     0
     0     1     0     0
     0     0     1     0
     0     0     0     1
\end{verbatim}
to see that $2$ is an eigenvalue of $A$ of algebraic multiplicity $4$ and geometric
multiplicity $1$.  

\newpage
\exer{c10.5.3B} \ans The eigenvalue $-1$ has algebraic multiplicity $5$ and geometric
multiplicity $3$.

\soln Type {\tt eig(A)} and obtain
\begin{verbatim}
ans =
  -1.0000          
  -1.0000 + 0.0000i
  -1.0000 - 0.0000i
  -1.0000          
  -1.0000         
\end{verbatim}
Type {\tt null(B+eye(5))}
and obtain
\begin{verbatim}
ans =
    0.7701   -0.1043    0.0000
    0.6160   -0.0835    0.0000
   -0.0000   -0.0000   -1.0000
   -0.0966   -0.9443    0.0000
   -0.1349   -0.3008    0.0000
\end{verbatim}
We see that $-1$ is an eigenvalue of $B$ with geometric multiplicity $5$. 



\subsection*{Section~\protect{\ref{S:JNF}} The Jordan Normal Form Theorem}
\rhead{S:JNF}{THE JORDAN NORMAL FORM THEOREM}


\exer{c10.5.2}
\ans Two such matrices are:
\[
\left(\begin{array}{rrrr}
2 & 1 & 0 & 0 \\
0 & 2 & 0 & 0 \\
0 & 0 & 2 & 1 \\
0 & 0 & 0 & 2 \end{array}\right)
\AND
\left(\begin{array}{rrrr}
2 & 1 & 0 & 0 \\
0 & 2 & 1 & 0 \\
0 & 0 & 2 & 0 \\
0 & 0 & 0 & 2 \end{array}\right)
\]

\exer{c10.5.2A} \ans There are $10$ different Jordan normal form matrices.

\soln  There is $1$ matrix with six Jordan blocks.  There is $1$ matrix with five
Jordan blocks.  There are $2$ matrices with four Jordan blocks; one matrix has a
Jordan block of size three and three blocks of size one and the other matrix has two
Jordan blocks of size two and two of size one.  There are $2$ matrices with three
Jordan blocks; one has a block of size four and one has three blocks of size two.
There are $3$ matrices with two Jordan blocks; one matrix each with a block of
size $n$, where $n=1,2,3$.  There is $1$ Jordan matrix with one Jordan block. 
Altogether, there are $10$ different matrices.

\exer{c10.5.2B} \ans There are six different Jordan normal form matrices.

\soln  There are $3$ different Jordan matrices associated with the eigenvalue $4$ and
$2$ different Jordan matrices associated with the eigenvalue $-3$.  Altogether, there
are $3\cdot 2=6$ different Jordan matrices.

\newpage
\exer{c10.5.2C} \ans There are $6$ possible Jordan normal form matrices which are
determined by computing nullity$(A-2I_8)$ and nullity$(A-(1+i)I_8)$.

\soln  Since $A$ is real, the eigenvalues of $A$ are $2$ with multiplicity three; 
$1+i$ with multiplicity two; $1-i$ with multiplicity two; and $0$ with multiplicity 
$1$.  There are three possible Jordan blocks associated with the eigenvalues $2$
determined by the geometric multiplicity of the eigenvalue $2$ and two possible Jordan
blocks associated with the eigenvalues $1\pm i$ determined by the geometric
multiplicity of the eigenvalue $1+i$.  The geometric multiplicity of the
eigenvalue $\lambda$ is the dimension of the null space of $A-\lambda I_8$.  Thus 
there are six possible Jordan normal forms and they are determined by computing the
dimensions of $\nulls(A-2I_8)$ and $\nulls(A-(1+i)I_8)$.


\exer{c10.5.3a}
\ans The Jordan normal form of matrix $A$ is
\[
\cmattwo{\frac{3 + \sqrt{17}}{2}}{0}{0}{\frac{3 - \sqrt{17}}{2}}.
\]

\soln Matrix $A$ has two distinct real eigenvalues at
$\lambda = \frac{3 + \sqrt{17}}{2}$ and 
$\lambda = \frac{3 - \sqrt{17}}{2}$.

\exer{c10.5.3b}
\ans The Jordan normal form of matrix $B$ is
\[
\mattwo{-1}{1}{0}{-1}.
\]

\soln Matrix $B$ has one real eigenvalue at $\lambda = -1$ with one
linearly independent eigenvector.

\exer{c10.5.4}
\ans The Jordan normal form of matrix $C$ is
\[
\matthree{-1}{0}{0}{0}{2}{1}{0}{0}{2}.
\]

\soln Matrix $C$ has two real eigenvalues.  The eigenvalue at $-1$ has
algebraic multiplicity of $1$, and the eigenvalue at $2$ has algebraic
multiplicity of $2$, and only $1$ linearly independent eigenvector.


\exer{c10.5.4a} \ans The Jordan normal form of matrix $D$ is
\[
\matthree{-1}{1}{0}{0}{-1}{0}{0}{0}{1}.
\]

\soln Matrix $D$ has two real eigenvalues.  The eigenvalue at $-1$ has
algebraic multiplicity of $2$ and geometric multiplicity $1$, and the eigenvalue at 
$1$ has multiplicity of $1$.

\exer{c10.5.4b} \ans The Jordan normal form of matrix $E$ is
\[
\matthree{1}{1}{0}{0}{1}{1}{0}{0}{1}.
\]

\soln Matrix $E$ has one eigenvalue at $1$ of algebraic multiplicity of $3$ and 
geometric multiplicity $1$.

\exer{c10.5.4c} \ans The Jordan normal form of matrix $F$ is
\[
\matthree{2}{1}{0}{0}{2}{0}{0}{0}{1}.
\]

\soln Matrix $F$ has two real eigenvalues.  The eigenvalue at $2$ has
algebraic multiplicity of $2$ and geometric multiplicity $1$, and the eigenvalue at 
$1$ has multiplicity of $1$.

\exer{c10.5.5A} \ans $e^{tJ} =
\cmatthree{e^t}{0}{0}{0}{e^{-t}}{te^{-t}}{0}{0}{e^{-t}}$.

\soln The matrix $J$ is in block diagonal form so we can compute the matrix exponential 
by computing the matrix exponential of each block.  In particular, let 
\[
M = \mattwo{-1}{1}{0}{-1}.
\]
Then 
\[
e^{tM} = e^{-t}(I_2+tM) = e^{-t}\mattwo{1}{t}{0}{1}.
\]
Therefore,
\[
e^{tJ} = \mattwo{e^t}{0}{0}{e^{tM}} = 
\cmatthree{e^t}{0}{0}{0}{e^{-t}}{te^{-t}}{0}{0}{e^{-t}}.
\]

\exer{c10.5.5B} \ans $e^J = 
\cmattwo{e^{2t}\mattwo{1}{t}{0}{1}}{0}{0}{e^{3t}\cmatthree{1}{t}{\frac{1}{2}t^2}{0}{1}{t}{0}{0}{1}}$.

\soln  The matrix $J$ has block diagonal form 
\[
J = \mattwo{L}{0}{0}{M}
\]
where
\[
L = \mattwo{2}{1}{0}{2} \AND M = \matthree{3}{1}{0}{0}{3}{1}{0}{0}{3}.
\]
Therefore,
\[
e^{tJ} = \cmattwo{e^L}{0}{0}{e^M}
\]
where
\[
e^L = e^{2t}\mattwo{1}{t}{0}{1} \AND 
e^M = e^{3t}\cmatthree{1}{t}{\frac{1}{2}t^2}{0}{1}{t}{0}{0}{1}.
\]


\exer{c10.5.5}
(a) The linear map $N^2$ satisfies
\[
N^2e_1 = N^2e_2 = 0, \AND N^2e_j = e_{j - 2}
\mbox{ for } j = 3,\ldots,n.
\]
The linear map $N^3$ satisfies
\[
N^3e_1 = N^3e_2 = N^3e_3 = 0, \AND N^3e_j = e_{j - 3} \mbox{ for }
j = 4,\ldots,n.
\]
By induction, $N^ne_j = 0$ for $j = 1,\dots,n$.  So $N^n = 0$, and $N$
is nilpotent.

(b) Let $\lambda$ be an eigenvalue of matrix $N$.  Then, there
exists a nonzero vector $v$ such that $Nv = \lambda v$.  Therefore,
$N^nv = \lambda^nv$.  If $N$ is nilpotent, then $\lambda^nv = 0$.  Since
$v$ is nonzero, $\lambda^n = 0$, which implies $\lambda = 0$.

(c) Let $M = P^{-1}NP$ be a matrix similar to nilpotent $n \times n$
matrix $N$.  Then $M^n = (P^{-1}NP)^n = P^{-1}N^nP$.  Since $N$ is
nilpotent, $N^n = 0$.  So $M^n = 0$, and $M$ is also nilpotent.

(d) By the Jordan normal form theorem, $N$ is similar to some matrix $W$ which
is block diagonal and has the matrices $N_j$ along the diagonal, where each
$N_j$ is a Jordan block.  By (a) of this problem, $N_j^n = 0$ for each
$j$.  $W^n$ is block diagonal with blocks $N_j^n$.  Since $N_j^n = 0$ for all
$j$, $W^n = 0$.  Thus, $W$ is nilpotent, and, by (c) of this problem, $N$
is also nilpotent.

\exer{c10.5.5C}
The characteristic polynomial of a $3\times 3$ matrix $A$ has the form 
\[
p_A(\lambda) = -\lambda^3 + b_2\lambda^2 + b_1\lambda + b_0.
\]
The Cayley-Hamilton theorem states that 
\[
p_A(A) = -A^3 + b_2A^2 + b_1A + b_0I_3 = 0.
\]
Therefore
\[
A(-A^2 +b_2A +b_1I_3) = -b_0I_3.
\]
Since $p_A(\lambda) =\det(A-\lambda I_3)$, it follows that $b_0=p_A(0)=\det(A)$.
Since $A$ is invertible, $\det(A)\neq 0$; that is, $b_0\neq 0$.  Finally, 
\[
A\inv = -\frac{1}{b_0}(-A^2+b_2A+b_1I_3) = aI_3 + bA + cA^2,
\]
where $a = -b_1/b_0$, $b = -b_2/b_0$, and $c=1/b_0$.



\exer{E:jnfma}
(a) \ans The Jordan normal form of $A$ is
\[
J = \left(\begin{array}{rrrr}
3 & 0 & 0 & 0 \\
0 & 1 & 0 & 0 \\
0 & 0 & 0 & 0 \\
0 & 0 & 0 & -1 \end{array}\right).
\]

\soln
To find the Jordan normal form of matrix $A$, type {\tt eig(A)} to find
the eigenvalues.  Matrix $A$ has four distinct eigenvalues, so the
Jordan normal form is the diagonal matrix with these eigenvalues along
its diagonal.  

(b)\ans   The diagonalizing matrix is
\begin{verbatim}
S =
   -0.1387   -0.1543   -0.0000   -0.5774
    0.1387   -0.3086   -0.4082    0.0000
    0.1387    0.9258    0.8165    0.5774
   -0.9707   -0.1543    0.4082   -0.5774
\end{verbatim}

\soln The columns of matrix $S$ consist of the eigenvectors of $A$.


\exer{E:jnfmb}
(a) \ans The Jordan normal form of $A$ is
\[
J = \left(\begin{array}{rrrr}
2 & 0 & 0 & 0 \\
0 & -1 & 1 & 0 \\
0 & 0 & -1 & 1 \\
0 & 0 & 0 & -1 \end{array}\right).
\]

\soln Type {\tt eig(A)} to find the eigenvalues of matrix $A$.  Not all
eigenvalues are simple, so type {\tt null(A - lambda*eye(4))} for each
eigenvalue $\lambda$ to find the number of linearly independent eigenvectors
associated to it.  Matrix $A$ has a simple eigenvalue at $2$, and
an eigenvalue at $-1$ with algebraic multiplicity 3 and one linearly
independent eigenvector.

(b) \ans   The diagonalizing matrix is
\begin{verbatim}
S =
   -0.1387   -0.5902   -1.0165   -0.9837
    0.1387    0.0000   -0.5902    0.1640
    0.1387    0.5902    2.1970    0.0656
   -0.9707   -0.5902   -0.4263    0.0328
\end{verbatim}

\soln The first column of $S$ is $v_1$, the eigenvector associated with
eigenvalue $2$.  To find the other columns, note that the nullity of
$(A + I_4)$ is 1, the nullity of $(A + I_4)^2$ is 2, and the nullity of
$(A + I_4)^3$ is 3.  Then, select one vector from $\null((A + I_4)^3)$
and label it $v_{23}$.  Set $v_{22} = (A + I_4)v_{23}$ and
$v_{21} = (A + I_4)^2v_{23}$.  Then, $S = (v_1|v_{21}|v_{22}|v_{23})$.

\exer{E:jnfmc}
(a) \ans The Jordan normal form of $A$ is
\[
J = \left(\begin{array}{rrrr}
 0 & 2 &  0 &  0 \\
-2 & 0 &  0 &  0 \\
 0 & 0 & -2 &  1 \\
 0 & 0 & -1 & -2 \end{array}\right).
\]

\soln Type {\tt eig(A)} to find that matrix $A$ has distinct eigenvalues
at $\pm 2i$ and $2 \pm i$.

(b) \ans   The diagonalizing matrix is
\begin{verbatim}
T =
   -0.2118   -0.0456    0.2211    0.0060
   -0.8548   -0.2507    0.8762    0.1803
   -0.3555   -0.0988    0.3529    0.0669
   -0.1437   -0.0531    0.1440    0.0344
\end{verbatim}

\soln The columns of matrix $S$ consist of the eigenvectors of $A$.


\exer{E:jnfmd}
(a) \ans The Jordan normal form of $A$ is
\[
J = \left(\begin{array}{rrrr}
-2 & 1 & 0 & 0 \\
0 & -2 & 0 & 0 \\
0 & 0 & -2 & 1 \\
0 & 0 & 0 & -2 \end{array}\right).
\]

\soln
Type {\tt eig(A)} to find that $-2$ is the only eigenvalues of $A$.  Then 
type {\tt null(A + 2*eye(4))} to find the number of linearly
independent eigenvectors associated to eigenvalue $\lambda = -2$.  
The eigenvalue has algebraic multiplicity $4$ and geometric multiplicity
$2$.  We then find that the nullity of $(A + 2I_4)^2$ is $4$. 
Therefore, all generalized eigenvectors $v$ of $D$ are in the null space
of $A + 2I_4$.

(b) \ans   The diagonalizing matrix is
\begin{verbatim}
S =
     3     1     0     0
    12     0    -5     1
     5     0    -2     0
     2     0    -1     0
\end{verbatim}

\soln To find $S$, find that $A$ has two linearly independent
eigenvectors associated to $-2$, and that the null space
of $(A + 2I_4)$ is $\R^4$.  Then, select two vectors in $\R^4$, in this
case $v_{12} = (1,0,0,0)^t$ and $v_{22} = (0,1,0,0)^t$, and set
$v_{11} = (A + 2I_4)v_{12}$ and $v_{21} = (A + 2I_4)v_{22}$.  Then,
$S = (v_{11}|v_{12}|v_{21}|v_{22})$.

\exer{E:jnfme}
(a) \ans The Jordan normal form of $A$ is
\[
J = \left(\begin{array}{rrrr}
-1 &  1 &  0 & 0 \\
0  & -1 &  1 & 0 \\
0  &  0 & -1 & 0 \\
0  &  0 &  0 & 1 \end{array}\right).
\]

(b) \ans  The diagonalizing matrix is
\begin{verbatim}
S =
    -1     1     0     0
    -1     1     0     1
    -1     0     1     1
    -1     0     0     1
\end{verbatim}

\soln To find $S$, use \Matlab to see that $A$ has two eigenvalues: $-1$ of
algebraic multiplicity three and $2$ of multiplicity one.  The geometric 
multiplicity of $-1$ is one.  Just type {\tt null(A+eye(4))} to see that 
$(1,1,1,1)$ is the only eigenvector corresponding to this eigenvalue.  Choose 
{\tt v3} in the null space of $(A+I_4)^3$ but not in the null space of 
$(A+I_4)^2$.  For example let {\tt v3 = [0 0 1 0]'}.  Then set 
{\tt v2 = (A+eye(4))*v3} and {\tt v1 = (A+eye(4))*v2}.  Finally, set 
{\tt v4 = null(A-2*eye(4))} and {\tt S = [v1 v2 v3 v4]}.

\exer{E:jnfmf} (a) \ans The Jordan normal form of $A$ is
\[
J = \left(\begin{array}{rrrrr}
-1 &  1 &  0 & 0 &  0 \\
 0 & -1 &  1 & 0 &  0 \\
 0 &  0 & -1 & 0 &  0 \\
 0 &  0 &  0 & 2 & -1 \\
 0 &  0 &  0 & 1 &  2 \end{array}\right).
\]
(b) \ans
\begin{verbatim}
S =
    7.5    6.5      0    2   0
    7.5   -1.0    1.0    2   0
    7.5   -1.0   -6.5    2   0
    0.0    0.0    0.0    2   0
    0.0    0.0    0.0    1   1
\end{verbatim}

\soln  A \Matlab calculation shows that the eigenvalues of $A$ are $-1$ of 
algebraic multiplicity three and geometric multiplicity one and simple 
eigenvalues $2\pm i$.  Choose {\tt v3} in the null space of $(A+I_5)^3$ but not in the null space of 
$(A+I_5)^2$.  One choice is {\tt v3 = [0 1 -6.5 0 0]'}.  Let {\tt v2 = (A+eye(5))*v3}
and {\tt v1 = (A+eye(5))*v2}.  Next, set {\tt v4} equal to the real part of the
eigenvector associated to the eigenvalue $2+i$ and let {\tt v5} be the complex part of
that eigenvector.  Finally, set {\tt S = [v1 v2 v3 v4 v5]}.


\subsection*{Section~\protect{\ref{S:TransitionTheory}} *Appendix:Markov Matrix
Theory}
\rhead{S:TransitionTheory}{*APPENDIX:MARKOV MATRIX THEORY}

\exer{c10.6.1}
Let $\lambda$ be an eigenvalue of $A$, and let $v \neq 0$ be an eigenvector
associated to $\lambda$.  Then,
\[
0 = \lim_{k \to \infty}A^kv = \lim_{k \to \infty}\lambda^kv
= \left(\lim_{k \to \infty}\lambda^k\right)v.
\]
Since $v$ is nonzero,
$\dps \lim_{k \to \infty}\lambda^k = 0$.  This is true only when
$|\lambda| < 1$.



 
\end{document}
