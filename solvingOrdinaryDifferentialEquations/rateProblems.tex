\documentclass{ximera}

 

\usepackage{epsfig}

\graphicspath{
  {./}
  {figures/}
}

\usepackage{morewrites}
\makeatletter
\newcommand\subfile[1]{%
\renewcommand{\input}[1]{}%
\begingroup\skip@preamble\otherinput{#1}\endgroup\par\vspace{\topsep}
\let\input\otherinput}
\makeatother

\newcommand{\includeexercises}{\directlua{dofile("/home/jim/linearAlgebra/laode/exercises.lua")}}

%\newcounter{ccounter}
%\setcounter{ccounter}{1}
%\newcommand{\Chapter}[1]{\setcounter{chapter}{\arabic{ccounter}}\chapter{#1}\addtocounter{ccounter}{1}}

%\newcommand{\section}[1]{\section{#1}\setcounter{thm}{0}\setcounter{equation}{0}}

%\renewcommand{\theequation}{\arabic{chapter}.\arabic{section}.\arabic{equation}}
%\renewcommand{\thefigure}{\arabic{chapter}.\arabic{figure}}
%\renewcommand{\thetable}{\arabic{chapter}.\arabic{table}}

%\newcommand{\Sec}[2]{\section{#1}\markright{\arabic{ccounter}.\arabic{section}.#2}\setcounter{equation}{0}\setcounter{thm}{0}\setcounter{figure}{0}}

\newcommand{\Sec}[2]{\section{#1}}

\setcounter{secnumdepth}{2}
%\setcounter{secnumdepth}{1} 

%\newcounter{THM}
%\renewcommand{\theTHM}{\arabic{chapter}.\arabic{section}}

\newcommand{\trademark}{{R\!\!\!\!\!\bigcirc}}
%\newtheorem{exercise}{}

\newcommand{\dfield}{{\sf dfield9}}
\newcommand{\pplane}{{\sf pplane9}}

\newcommand{\EXER}{\section*{Exercises}}%\vspace*{0.2in}\hrule\small\setcounter{exercise}{0}}
\newcommand{\CEXER}{}%\vspace{0.08in}\begin{center}Computer Exercises\end{center}}
\newcommand{\TEXER}{} %\vspace{0.08in}\begin{center}Hand Exercises\end{center}}
\newcommand{\AEXER}{} %\vspace{0.08in}\begin{center}Hand Exercises\end{center}}

% BADBAD: \newcommand{\Bbb}{\bf}

\newcommand{\R}{\mbox{$\Bbb{R}$}}
\newcommand{\C}{\mbox{$\Bbb{C}$}}
\newcommand{\Z}{\mbox{$\Bbb{Z}$}}
\newcommand{\N}{\mbox{$\Bbb{N}$}}
\newcommand{\D}{\mbox{{\bf D}}}
\usepackage{amssymb}
%\newcommand{\qed}{\hfill\mbox{\raggedright$\square$} \vspace{1ex}}
%\newcommand{\proof}{\noindent {\bf Proof:} \hspace{0.1in}}

\newcommand{\setmin}{\;\mbox{--}\;}
\newcommand{\Matlab}{{M\small{AT\-LAB}} }
\newcommand{\Matlabp}{{M\small{AT\-LAB}}}
\newcommand{\computer}{\Matlab Instructions}
\newcommand{\half}{\mbox{$\frac{1}{2}$}}
\newcommand{\compose}{\raisebox{.15ex}{\mbox{{\scriptsize$\circ$}}}}
\newcommand{\AND}{\quad\mbox{and}\quad}
\newcommand{\vect}[2]{\left(\begin{array}{c} #1_1 \\ \vdots \\
 #1_{#2}\end{array}\right)}
\newcommand{\mattwo}[4]{\left(\begin{array}{rr} #1 & #2\\ #3
&#4\end{array}\right)}
\newcommand{\mattwoc}[4]{\left(\begin{array}{cc} #1 & #2\\ #3
&#4\end{array}\right)}
\newcommand{\vectwo}[2]{\left(\begin{array}{r} #1 \\ #2\end{array}\right)}
\newcommand{\vectwoc}[2]{\left(\begin{array}{c} #1 \\ #2\end{array}\right)}

\newcommand{\ignore}[1]{}


\newcommand{\inv}{^{-1}}
\newcommand{\CC}{{\cal C}}
\newcommand{\CCone}{\CC^1}
\newcommand{\Span}{{\rm span}}
\newcommand{\rank}{{\rm rank}}
\newcommand{\trace}{{\rm tr}}
\newcommand{\RE}{{\rm Re}}
\newcommand{\IM}{{\rm Im}}
\newcommand{\nulls}{{\rm null\;space}}

\newcommand{\dps}{\displaystyle}
\newcommand{\arraystart}{\renewcommand{\arraystretch}{1.8}}
\newcommand{\arrayfinish}{\renewcommand{\arraystretch}{1.2}}
\newcommand{\Start}[1]{\vspace{0.08in}\noindent {\bf Section~\ref{#1}}}
\newcommand{\exer}[1]{\noindent {\bf \ref{#1}}}
\newcommand{\ans}{}
\newcommand{\matthree}[9]{\left(\begin{array}{rrr} #1 & #2 & #3 \\ #4 & #5 & #6
\\ #7 & #8 & #9\end{array}\right)}
\newcommand{\cvectwo}[2]{\left(\begin{array}{c} #1 \\ #2\end{array}\right)}
\newcommand{\cmatthree}[9]{\left(\begin{array}{ccc} #1 & #2 & #3 \\ #4 & #5 &
#6 \\ #7 & #8 & #9\end{array}\right)}
\newcommand{\vecthree}[3]{\left(\begin{array}{r} #1 \\ #2 \\
#3\end{array}\right)}
\newcommand{\cvecthree}[3]{\left(\begin{array}{c} #1 \\ #2 \\
#3\end{array}\right)}
\newcommand{\cmattwo}[4]{\left(\begin{array}{cc} #1 & #2\\ #3
&#4\end{array}\right)}

\newcommand{\Matrix}[1]{\ensuremath{\left(\begin{array}{rrrrrrrrrrrrrrrrrr} #1 \end{array}\right)}}

\newcommand{\Matrixc}[1]{\ensuremath{\left(\begin{array}{cccccccccccc} #1 \end{array}\right)}}



\renewcommand{\labelenumi}{\theenumi)}
\newenvironment{enumeratea}%
{\begingroup
 \renewcommand{\theenumi}{\alph{enumi}}
 \renewcommand{\labelenumi}{(\theenumi)}
 \begin{enumerate}}
 {\end{enumerate}\endgroup}



\newcounter{help}
\renewcommand{\thehelp}{\thesection.\arabic{equation}}

%\newenvironment{equation*}%
%{\renewcommand\endequation{\eqno (\theequation)* $$}%
%   \begin{equation}}%
%   {\end{equation}\renewcommand\endequation{\eqno \@eqnnum
%$$\global\@ignoretrue}}

%\input{psfig.tex}

\author{Martin Golubitsky and Michael Dellnitz}

%\newenvironment{matlabEquation}%
%{\renewcommand\endequation{\eqno (\theequation*) $$}%
%   \begin{equation}}%
%   {\end{equation}\renewcommand\endequation{\eqno \@eqnnum
% $$\global\@ignoretrue}}

\newcommand{\soln}{\textbf{Solution:} }
\newcommand{\exercap}[1]{\centerline{Figure~\ref{#1}}}
\newcommand{\exercaptwo}[1]{\centerline{Figure~\ref{#1}a\hspace{2.1in}
Figure~\ref{#1}b}}
\newcommand{\exercapthree}[1]{\centerline{Figure~\ref{#1}a\hspace{1.2in}
Figure~\ref{#1}b\hspace{1.2in}Figure~\ref{#1}c}}
\newcommand{\para}{\hspace{0.4in}}

\renewenvironment{solution}{\suppress}{\endsuppress}

\ifxake
\newenvironment{matlabEquation}{\begin{equation}}{\end{equation}}
\else
\newenvironment{matlabEquation}%
{\let\oldtheequation\theequation\renewcommand{\theequation}{\oldtheequation*}\begin{equation}}%
  {\end{equation}\let\theequation\oldtheequation}
\fi

\makeatother


\title{*Rate Problems}

\begin{document}
\begin{abstract}
\end{abstract}
\maketitle

  \label{S:growthmodels}




%\subsection*{Some Examples of \protect\eqref{ivp1}}

Even though the homogeneous linear  differential equation \eqref{ivp1} is one of the
simplest differential equations, it still has some use in applications.
We present two here: compound interest and population dynamics.

\subsubsection*{Compound Interest} \index{compound interest}

Banks pay interest on an account in the following way.  At the
end of each day, the bank determines the interest rate $r_{day}$ for
that day, checks the principal $P$ in the account, and then
deposits an additional $r_{day}P$.  So the next day the principal in
this account is $(1+r_{day})P$.  Note that if $r$ denotes the
interest rate per year, then $r_{day} = r/365$.  Of course, a day
is just a convenient measure for elapsed time.  Before computers
were prevalent, banks paid interest yearly or quarterly or monthly
or, in a few cases, even weekly, depending on the particular bank rules.

Observe that the more frequently interest is paid, the more
money is earned. For example, if interest is paid only once at
the end of a year, then the money in the account at the end of
the year is $(1+r)P$, and the amount $rP$ is called {\em simple
interest\/}.  But if interest is paid twice a year, then the principal
at the end of six months will be $(1+\frac{r}{2})P$, and the principal
at the end of the year will be $(1+\frac{r}{2})^2P$.  Since
\[
\left(1+\frac{r}{2}\right)^2 = 1+r+\frac{1}{4}r^2 > 1+r,
\]
there is more money in the account at the end of the
year if the interest is compounded semiannually rather than
annually.  But how much is the difference and what is the
maximum earning potential?

While making the calculation in the previous paragraph, we
implicitly made a number of simplifying assumptions.  In
particular, we assumed
\begin{itemize}
\item	an initial principal $P_0$ is deposited in the bank on January 1,
\item	the money is not withdrawn for one year,
\item	no new money is deposited in that account during the year,
\item	the yearly interest rate $r$ remains constant throughout
	the year, and
\item	interest is added to the account $N$ times during the year.
\end{itemize}
In this {\em model\/}, simple interest corresponds to $N=1$, compound monthly
interest to $N=12$, and compound daily interest to $N=365$.

We first answer the question: How much money is in this account after
one year?  After one time unit of $\frac{1}{N}$ year, the amount of
money in the account is
\[
Q_1 = \left(1+\frac{r}{N}\right)P_0.
\]
The interest rate in each time period is $\frac{r}{N}$,
the yearly rate $r$ divided by the number of time periods $N$.  Here
we have used the assumption that the interest rate remains constant
throughout the year.  After two time units, the principal is:
\[
Q_2 = \left(1+\frac{r}{N}\right)Q_1 = \left(1+\frac{r}{N}\right)^2P_0,
\]
and at the end of the year (that is, after $N$ time periods)
\begin{equation} \label{compint}
Q_N = \left(1+\frac{r}{N}\right)^N P_0.
\end{equation}
Here we have used the assumption that money is neither deposited nor
withdrawn from our account.   Note that $Q_N$ is the amount of money
in the bank after {\bf one} year assuming that interest has been
compounded $N$ (equally spaced) times during that year, and the effective
interest rate when compounding $N$ times is:
\[
\left(1+\frac{r}{N}\right)^N - 1.
\]

For the curious, we can write a program in \Matlab to compute
\eqref{compint}.  Suppose we assume that the initial deposit $P_0=\$1,000$,
the simple interest rate is $6\%$ per year, and the interest payments
are made monthly. In \Matlab type
\begin{verbatim}
N  = 12;
P0 = 1000;
r  = 0.06;
QN = (1 + r/N)^N*P0
\end{verbatim}
The answer is $QN=\$1,061.68$, and the {\em effective\/}
interest rate for monthly payments is $6.16778\%$.  For daily
interest payments $N=365$, the answer is $QN=\$1,061.83$, and
the effective interest rate is $6.18313\%$.

To find the maximum effective interest, we ask the bank to compound interest
continuously; that is, we ask the bank to compute
\[
\lim_{N\to\infty} \left(1 + \frac{r}{N}\right)^N.
\]
We compute this limit using differential equations.  The concept of
continuous interest is rephrased as follows.  Let $P(t)$ be the
principal at time $t$, where $t$ is measured in units of years.
Suppose that we assume that interest is compounded $N$ times during
the year.  The length of time in each compounding period is
\[
\Delta t = \frac{1}{N},
\]
and the change in principal during that time period is
\[
\Delta P = \frac{r}{N} P = rP\Delta t.
\]
It follows that
\[
\frac{\Delta P}{\Delta t} = rP,
\]
and, on taking the limit $\Delta t \to 0$, we have the differential equation
\[
\frac{dP}{dt}(t) = rP(t).
\]

Since $P(0)=P_0$ the solution of the initial value problem given
in Theorem~\ref{T:singleeqn} shows that
\[
P(t) = P_0 e^{rt}.
\]
After one year ($t=1$) we find that
\[
P(1) =  e^r P_0.
\]
Note that
\[
P(1) = \lim_{N\to\infty} Q_N,
\]
and we have thus verified that
\[
 \lim_{N\to\infty} \left(1 + \frac{r}{N}\right)^N = e^r.
\]
Thus the maximum effective interest rate is $e^r-1$.  When $r=6\%$
the maximum effective interest rate is $6.18365\%$.


\subsubsection*{Population Dynamics}
\index{population dynamics}

To provide a second interpretation of the constant $\lambda$ in
\eqref{lin1}, we discuss a simplified model for population dynamics.
Let $p(t)$ be the size of a population of a certain species at
time $t$ and let $r$ be the rate at which the population $p$ is
changing at time $t$.  In general, $r$ depends on the time $t$
and is a complicated function of birth and death rates and of
immigration and emigration, as well as of other factors.
Indeed, the rate $r$ may well depend
on the size of the population itself.  (Overcrowding can be
modeled by assuming that the death rate increases with the size
of the population.) These population models assume that the
rate of change in the size of the population $dp/dt$ is given by
\begin{equation}  \label{pop_model}
        \frac{dp}{dt}(t) = r p(t),
\end{equation}
they just differ on the precise form of $r$.  In general, the
rate $r$ will depend on the size of the population $p$ as well
as the time $t$, that is, $r$ is a function $r(p,t)$.

The simplest population model\index{population model} --- which
we now assume --- is the
one in which $r$ is assumed to be constant.  Then equation
\eqref{pop_model} is identical to \eqref{lin1} after identifying $p$
with $x$ and $r$ with $\lambda$.  Hence we may interpret $r$ as
the growth rate for the population.  The form of the solution in
\eqref{soln1} shows that the size of a population grows
exponentially if $r>0$ and decays exponentially if $r<0$.

The mathematical description of this simplest population model
shows that the assumption of a constant growth rate leads to
exponential growth (or exponential decay).  Is this realistic?
Surely, no population will grow exponentially for all time, and
other factors, such as limited living space, have to be taken
into account.  On the other hand, exponential growth describes
well the growth in human population during much of human
history.  So this model, though surely oversimplified, gives
some insight into population growth.

\EXER

\includeexercises


\end{document}
