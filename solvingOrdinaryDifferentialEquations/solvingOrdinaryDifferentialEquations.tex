\documentclass{ximera}

 

\usepackage{epsfig}

\graphicspath{
  {./}
  {figures/}
}

\usepackage{morewrites}
\makeatletter
\newcommand\subfile[1]{%
\renewcommand{\input}[1]{}%
\begingroup\skip@preamble\otherinput{#1}\endgroup\par\vspace{\topsep}
\let\input\otherinput}
\makeatother

\newcommand{\includeexercises}{\directlua{dofile("/home/jim/linearAlgebra/laode/exercises.lua")}}

%\newcounter{ccounter}
%\setcounter{ccounter}{1}
%\newcommand{\Chapter}[1]{\setcounter{chapter}{\arabic{ccounter}}\chapter{#1}\addtocounter{ccounter}{1}}

%\newcommand{\section}[1]{\section{#1}\setcounter{thm}{0}\setcounter{equation}{0}}

%\renewcommand{\theequation}{\arabic{chapter}.\arabic{section}.\arabic{equation}}
%\renewcommand{\thefigure}{\arabic{chapter}.\arabic{figure}}
%\renewcommand{\thetable}{\arabic{chapter}.\arabic{table}}

%\newcommand{\Sec}[2]{\section{#1}\markright{\arabic{ccounter}.\arabic{section}.#2}\setcounter{equation}{0}\setcounter{thm}{0}\setcounter{figure}{0}}

\newcommand{\Sec}[2]{\section{#1}}

\setcounter{secnumdepth}{2}
%\setcounter{secnumdepth}{1} 

%\newcounter{THM}
%\renewcommand{\theTHM}{\arabic{chapter}.\arabic{section}}

\newcommand{\trademark}{{R\!\!\!\!\!\bigcirc}}
%\newtheorem{exercise}{}

\newcommand{\dfield}{{\sf dfield9}}
\newcommand{\pplane}{{\sf pplane9}}

\newcommand{\EXER}{\section*{Exercises}}%\vspace*{0.2in}\hrule\small\setcounter{exercise}{0}}
\newcommand{\CEXER}{}%\vspace{0.08in}\begin{center}Computer Exercises\end{center}}
\newcommand{\TEXER}{} %\vspace{0.08in}\begin{center}Hand Exercises\end{center}}
\newcommand{\AEXER}{} %\vspace{0.08in}\begin{center}Hand Exercises\end{center}}

% BADBAD: \newcommand{\Bbb}{\bf}

\newcommand{\R}{\mbox{$\Bbb{R}$}}
\newcommand{\C}{\mbox{$\Bbb{C}$}}
\newcommand{\Z}{\mbox{$\Bbb{Z}$}}
\newcommand{\N}{\mbox{$\Bbb{N}$}}
\newcommand{\D}{\mbox{{\bf D}}}
\usepackage{amssymb}
%\newcommand{\qed}{\hfill\mbox{\raggedright$\square$} \vspace{1ex}}
%\newcommand{\proof}{\noindent {\bf Proof:} \hspace{0.1in}}

\newcommand{\setmin}{\;\mbox{--}\;}
\newcommand{\Matlab}{{M\small{AT\-LAB}} }
\newcommand{\Matlabp}{{M\small{AT\-LAB}}}
\newcommand{\computer}{\Matlab Instructions}
\newcommand{\half}{\mbox{$\frac{1}{2}$}}
\newcommand{\compose}{\raisebox{.15ex}{\mbox{{\scriptsize$\circ$}}}}
\newcommand{\AND}{\quad\mbox{and}\quad}
\newcommand{\vect}[2]{\left(\begin{array}{c} #1_1 \\ \vdots \\
 #1_{#2}\end{array}\right)}
\newcommand{\mattwo}[4]{\left(\begin{array}{rr} #1 & #2\\ #3
&#4\end{array}\right)}
\newcommand{\mattwoc}[4]{\left(\begin{array}{cc} #1 & #2\\ #3
&#4\end{array}\right)}
\newcommand{\vectwo}[2]{\left(\begin{array}{r} #1 \\ #2\end{array}\right)}
\newcommand{\vectwoc}[2]{\left(\begin{array}{c} #1 \\ #2\end{array}\right)}

\newcommand{\ignore}[1]{}


\newcommand{\inv}{^{-1}}
\newcommand{\CC}{{\cal C}}
\newcommand{\CCone}{\CC^1}
\newcommand{\Span}{{\rm span}}
\newcommand{\rank}{{\rm rank}}
\newcommand{\trace}{{\rm tr}}
\newcommand{\RE}{{\rm Re}}
\newcommand{\IM}{{\rm Im}}
\newcommand{\nulls}{{\rm null\;space}}

\newcommand{\dps}{\displaystyle}
\newcommand{\arraystart}{\renewcommand{\arraystretch}{1.8}}
\newcommand{\arrayfinish}{\renewcommand{\arraystretch}{1.2}}
\newcommand{\Start}[1]{\vspace{0.08in}\noindent {\bf Section~\ref{#1}}}
\newcommand{\exer}[1]{\noindent {\bf \ref{#1}}}
\newcommand{\ans}{}
\newcommand{\matthree}[9]{\left(\begin{array}{rrr} #1 & #2 & #3 \\ #4 & #5 & #6
\\ #7 & #8 & #9\end{array}\right)}
\newcommand{\cvectwo}[2]{\left(\begin{array}{c} #1 \\ #2\end{array}\right)}
\newcommand{\cmatthree}[9]{\left(\begin{array}{ccc} #1 & #2 & #3 \\ #4 & #5 &
#6 \\ #7 & #8 & #9\end{array}\right)}
\newcommand{\vecthree}[3]{\left(\begin{array}{r} #1 \\ #2 \\
#3\end{array}\right)}
\newcommand{\cvecthree}[3]{\left(\begin{array}{c} #1 \\ #2 \\
#3\end{array}\right)}
\newcommand{\cmattwo}[4]{\left(\begin{array}{cc} #1 & #2\\ #3
&#4\end{array}\right)}

\newcommand{\Matrix}[1]{\ensuremath{\left(\begin{array}{rrrrrrrrrrrrrrrrrr} #1 \end{array}\right)}}

\newcommand{\Matrixc}[1]{\ensuremath{\left(\begin{array}{cccccccccccc} #1 \end{array}\right)}}



\renewcommand{\labelenumi}{\theenumi)}
\newenvironment{enumeratea}%
{\begingroup
 \renewcommand{\theenumi}{\alph{enumi}}
 \renewcommand{\labelenumi}{(\theenumi)}
 \begin{enumerate}}
 {\end{enumerate}\endgroup}



\newcounter{help}
\renewcommand{\thehelp}{\thesection.\arabic{equation}}

%\newenvironment{equation*}%
%{\renewcommand\endequation{\eqno (\theequation)* $$}%
%   \begin{equation}}%
%   {\end{equation}\renewcommand\endequation{\eqno \@eqnnum
%$$\global\@ignoretrue}}

%\input{psfig.tex}

\author{Martin Golubitsky and Michael Dellnitz}

%\newenvironment{matlabEquation}%
%{\renewcommand\endequation{\eqno (\theequation*) $$}%
%   \begin{equation}}%
%   {\end{equation}\renewcommand\endequation{\eqno \@eqnnum
% $$\global\@ignoretrue}}

\newcommand{\soln}{\textbf{Solution:} }
\newcommand{\exercap}[1]{\centerline{Figure~\ref{#1}}}
\newcommand{\exercaptwo}[1]{\centerline{Figure~\ref{#1}a\hspace{2.1in}
Figure~\ref{#1}b}}
\newcommand{\exercapthree}[1]{\centerline{Figure~\ref{#1}a\hspace{1.2in}
Figure~\ref{#1}b\hspace{1.2in}Figure~\ref{#1}c}}
\newcommand{\para}{\hspace{0.4in}}

\renewenvironment{solution}{\suppress}{\endsuppress}

\ifxake
\newenvironment{matlabEquation}{\begin{equation}}{\end{equation}}
\else
\newenvironment{matlabEquation}%
{\let\oldtheequation\theequation\renewcommand{\theequation}{\oldtheequation*}\begin{equation}}%
  {\end{equation}\let\theequation\oldtheequation}
\fi

\makeatother


\title{Solving Ordinary Differential Equations}

\begin{document}
\begin{abstract}
\end{abstract}
\maketitle


\label{chap:SolveOdes}

\normalsize

The study of linear systems of equations given in Chapter~\ref{lineq}
provides one motivation for the study of matrices and linear algebra.  
Linear constant coefficient systems of ordinary differential equations 
provide a second motivation for this study.  In this chapter we
show how the phase space geometry of systems of differential equations
motivates the idea of {\em eigendirections} (or invariant directions) and
{\em eigenvalues\/} (or growth rates).  

We begin this chapter with a discussion of the theory and application
of the simplest of linear differential equations, the linear growth equation,
$\dot{x}=\lambda x$.  In Section~\ref{S:growthmodels}, we solve the linear
growth equation and discuss the fact that solutions to differential equations
are functions; and we emphasize this point by using \Matlab to graph
solutions of $x$ as a function of $t$.  We also illustrate the applicability
of this very simple equation with a discussion of compound interest and
a simple population model.

In Section~\ref{S:3.2} we discuss two different ways to plot solutions of
differential equations: {\em time series\/} and {\em phase space\/} plots.
The first method just plots the graph of a solution as a function
of time $t$, as discussed in Section~\ref{S:growthmodels}, while the second
method is based on thinking of the differential equation as describing
how a point moves in space.  Both methods are important: time series are
typical ways of representing results of experiments and phase space plots
are central to a geometric understanding of solutions to differential
equations.  In Sections~\ref{S:3.2} and \ref{S:PSP&E} we introduce two \Matlab
programs {\sf dfield8} (written by John Polking)\index{Polking, John} and 
{\sf pline} that illustrate the two methods of plotting the output of a 
differential equation.

In the optional Section~\ref{sec:sov} we present one method for solving
differential equations analytically where $f(t,x)$, the right hand side in 
the ODE, is a product of a function of $x$ and a function of $t$. This method 
is called {\em separation of variables} and is based on integration theory
from calculus.  We will see that even these simple differential
equations may lead to solutions that are defined only implicitly and not in
closed form.

The next two sections introduce planar constant coefficient linear
differential equations.  In these sections we use the program {\sf pplane8}
(also written by John Polking) that solves numerically planar systems of
differential equations.  In Section~\ref{sec:UncoupledLS} we discuss 
uncoupled systems --- two independent one dimensional systems like those 
presented in Section~\ref{S:growthmodels} --- whose solution geometry in the 
plane is somewhat more complicated than might be expected.  In 
Section~\ref{s:3.5} we discuss coupled linear systems.  Here we
illustrate the existence and nonexistence of eigendirections.

In Section~\ref{S:IVP&E} we show how the initial value problems can be solved 
by building the solution --- through the use of superposition as discussed in 
Section~\ref{S:Superposition} --- from simpler solutions.  These simpler
solutions are ones generated from real eigenvalues and eigenvectors
--- when they exist.  In Section~\ref{S:evchp} we develop the theory of  
{\em eigenvalues\/} and {\em characteristic polynomials\/} of $2\times 2$ 
matrices.  (The corresponding theory for $n\times n$ matrices is developed in 
Chapter~\ref{C:D&E}.)

The method for solving planar constant coefficient linear differential 
equations with real eigenvalues is summarized in Section~\ref{S:IVPR}.  This 
method is based on the material of Sections~\ref{S:IVP&E} and \ref{S:evchp}.  
The complete discussion of the solutions of linear planar systems of 
differential equations is given in Chapter~\ref{Chap:Planar}.

The chapter ends with an optional discussion of {\em Markov chains\/} in 
Section~\ref{S:TransitionApplied}.  Markov chains give a method for 
analyzing branch processes where at each time unit several outcomes are 
possible, each with a given probability.

\end{document}
