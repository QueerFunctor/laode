\documentclass{ximera}

 

\usepackage{epsfig}

\graphicspath{
  {./}
  {figures/}
}

\usepackage{morewrites}
\makeatletter
\newcommand\subfile[1]{%
\renewcommand{\input}[1]{}%
\begingroup\skip@preamble\otherinput{#1}\endgroup\par\vspace{\topsep}
\let\input\otherinput}
\makeatother

\newcommand{\includeexercises}{\directlua{dofile("/home/jim/linearAlgebra/laode/exercises.lua")}}

%\newcounter{ccounter}
%\setcounter{ccounter}{1}
%\newcommand{\Chapter}[1]{\setcounter{chapter}{\arabic{ccounter}}\chapter{#1}\addtocounter{ccounter}{1}}

%\newcommand{\section}[1]{\section{#1}\setcounter{thm}{0}\setcounter{equation}{0}}

%\renewcommand{\theequation}{\arabic{chapter}.\arabic{section}.\arabic{equation}}
%\renewcommand{\thefigure}{\arabic{chapter}.\arabic{figure}}
%\renewcommand{\thetable}{\arabic{chapter}.\arabic{table}}

%\newcommand{\Sec}[2]{\section{#1}\markright{\arabic{ccounter}.\arabic{section}.#2}\setcounter{equation}{0}\setcounter{thm}{0}\setcounter{figure}{0}}

\newcommand{\Sec}[2]{\section{#1}}

\setcounter{secnumdepth}{2}
%\setcounter{secnumdepth}{1} 

%\newcounter{THM}
%\renewcommand{\theTHM}{\arabic{chapter}.\arabic{section}}

\newcommand{\trademark}{{R\!\!\!\!\!\bigcirc}}
%\newtheorem{exercise}{}

\newcommand{\dfield}{{\sf dfield9}}
\newcommand{\pplane}{{\sf pplane9}}

\newcommand{\EXER}{\section*{Exercises}}%\vspace*{0.2in}\hrule\small\setcounter{exercise}{0}}
\newcommand{\CEXER}{}%\vspace{0.08in}\begin{center}Computer Exercises\end{center}}
\newcommand{\TEXER}{} %\vspace{0.08in}\begin{center}Hand Exercises\end{center}}
\newcommand{\AEXER}{} %\vspace{0.08in}\begin{center}Hand Exercises\end{center}}

% BADBAD: \newcommand{\Bbb}{\bf}

\newcommand{\R}{\mbox{$\Bbb{R}$}}
\newcommand{\C}{\mbox{$\Bbb{C}$}}
\newcommand{\Z}{\mbox{$\Bbb{Z}$}}
\newcommand{\N}{\mbox{$\Bbb{N}$}}
\newcommand{\D}{\mbox{{\bf D}}}
\usepackage{amssymb}
%\newcommand{\qed}{\hfill\mbox{\raggedright$\square$} \vspace{1ex}}
%\newcommand{\proof}{\noindent {\bf Proof:} \hspace{0.1in}}

\newcommand{\setmin}{\;\mbox{--}\;}
\newcommand{\Matlab}{{M\small{AT\-LAB}} }
\newcommand{\Matlabp}{{M\small{AT\-LAB}}}
\newcommand{\computer}{\Matlab Instructions}
\newcommand{\half}{\mbox{$\frac{1}{2}$}}
\newcommand{\compose}{\raisebox{.15ex}{\mbox{{\scriptsize$\circ$}}}}
\newcommand{\AND}{\quad\mbox{and}\quad}
\newcommand{\vect}[2]{\left(\begin{array}{c} #1_1 \\ \vdots \\
 #1_{#2}\end{array}\right)}
\newcommand{\mattwo}[4]{\left(\begin{array}{rr} #1 & #2\\ #3
&#4\end{array}\right)}
\newcommand{\mattwoc}[4]{\left(\begin{array}{cc} #1 & #2\\ #3
&#4\end{array}\right)}
\newcommand{\vectwo}[2]{\left(\begin{array}{r} #1 \\ #2\end{array}\right)}
\newcommand{\vectwoc}[2]{\left(\begin{array}{c} #1 \\ #2\end{array}\right)}

\newcommand{\ignore}[1]{}


\newcommand{\inv}{^{-1}}
\newcommand{\CC}{{\cal C}}
\newcommand{\CCone}{\CC^1}
\newcommand{\Span}{{\rm span}}
\newcommand{\rank}{{\rm rank}}
\newcommand{\trace}{{\rm tr}}
\newcommand{\RE}{{\rm Re}}
\newcommand{\IM}{{\rm Im}}
\newcommand{\nulls}{{\rm null\;space}}

\newcommand{\dps}{\displaystyle}
\newcommand{\arraystart}{\renewcommand{\arraystretch}{1.8}}
\newcommand{\arrayfinish}{\renewcommand{\arraystretch}{1.2}}
\newcommand{\Start}[1]{\vspace{0.08in}\noindent {\bf Section~\ref{#1}}}
\newcommand{\exer}[1]{\noindent {\bf \ref{#1}}}
\newcommand{\ans}{}
\newcommand{\matthree}[9]{\left(\begin{array}{rrr} #1 & #2 & #3 \\ #4 & #5 & #6
\\ #7 & #8 & #9\end{array}\right)}
\newcommand{\cvectwo}[2]{\left(\begin{array}{c} #1 \\ #2\end{array}\right)}
\newcommand{\cmatthree}[9]{\left(\begin{array}{ccc} #1 & #2 & #3 \\ #4 & #5 &
#6 \\ #7 & #8 & #9\end{array}\right)}
\newcommand{\vecthree}[3]{\left(\begin{array}{r} #1 \\ #2 \\
#3\end{array}\right)}
\newcommand{\cvecthree}[3]{\left(\begin{array}{c} #1 \\ #2 \\
#3\end{array}\right)}
\newcommand{\cmattwo}[4]{\left(\begin{array}{cc} #1 & #2\\ #3
&#4\end{array}\right)}

\newcommand{\Matrix}[1]{\ensuremath{\left(\begin{array}{rrrrrrrrrrrrrrrrrr} #1 \end{array}\right)}}

\newcommand{\Matrixc}[1]{\ensuremath{\left(\begin{array}{cccccccccccc} #1 \end{array}\right)}}



\renewcommand{\labelenumi}{\theenumi)}
\newenvironment{enumeratea}%
{\begingroup
 \renewcommand{\theenumi}{\alph{enumi}}
 \renewcommand{\labelenumi}{(\theenumi)}
 \begin{enumerate}}
 {\end{enumerate}\endgroup}



\newcounter{help}
\renewcommand{\thehelp}{\thesection.\arabic{equation}}

%\newenvironment{equation*}%
%{\renewcommand\endequation{\eqno (\theequation)* $$}%
%   \begin{equation}}%
%   {\end{equation}\renewcommand\endequation{\eqno \@eqnnum
%$$\global\@ignoretrue}}

%\input{psfig.tex}

\author{Martin Golubitsky and Michael Dellnitz}

%\newenvironment{matlabEquation}%
%{\renewcommand\endequation{\eqno (\theequation*) $$}%
%   \begin{equation}}%
%   {\end{equation}\renewcommand\endequation{\eqno \@eqnnum
% $$\global\@ignoretrue}}

\newcommand{\soln}{\textbf{Solution:} }
\newcommand{\exercap}[1]{\centerline{Figure~\ref{#1}}}
\newcommand{\exercaptwo}[1]{\centerline{Figure~\ref{#1}a\hspace{2.1in}
Figure~\ref{#1}b}}
\newcommand{\exercapthree}[1]{\centerline{Figure~\ref{#1}a\hspace{1.2in}
Figure~\ref{#1}b\hspace{1.2in}Figure~\ref{#1}c}}
\newcommand{\para}{\hspace{0.4in}}

\renewenvironment{solution}{\suppress}{\endsuppress}

\ifxake
\newenvironment{matlabEquation}{\begin{equation}}{\end{equation}}
\else
\newenvironment{matlabEquation}%
{\let\oldtheequation\theequation\renewcommand{\theequation}{\oldtheequation*}\begin{equation}}%
  {\end{equation}\let\theequation\oldtheequation}
\fi

\makeatother


\title{The Initial Value Problem and Eigenvectors}

\begin{document}
\begin{abstract}
\end{abstract}
\maketitle


\label{S:IVP&E} \index{initial value problem}

The general {\em constant coefficient\/}
\index{system of differential equations!constant coefficient}
system of $n$ differential equations in $n$ unknown functions has the form
\renewcommand{\arraystretch}{1.8}
\begin{equation}\label{lingen}
\begin{array}{ccc}
\dps \frac{dx_1}{dt}(t) & = & c_{11}x_1(t) + \cdots + c_{1n}x_n(t) \\
\vdots  & \vdots & \vdots \\
\dps \frac{dx_n}{dt}(t) & = & c_{n1}x_1(t) + \cdots + c_{nn}x_n(t)
\end{array}
\end{equation}
\renewcommand{\arraystretch}{1.0}%
where the coefficients $c_{ij}\in \R$ are constants.  Suppose that 
\eqref{lingen} satisfies the initial conditions 
\[
x_1(0) = K_1, \ldots,  x_n(0) = K_n.
\]

Using matrix multiplication of a vector and matrix, we can rewrite these 
differential equations in a compact form.   Consider the $n\times n$ 
coefficient matrix
\[
C = \left(
\begin{array}{rrrr}
 c_{11} & c_{12} & \cdots & c_{1n} \\
 c_{21} & c_{22} & \cdots & c_{2n}  \\
 \vdots & \vdots &        & \vdots  \\
 c_{n1} & c_{n2} & \cdots & c_{nn}
\end{array}
\right)
\]
and the $n$ vectors of initial conditions and unknowns
\[
X_0=\left(
\begin{array}{ccc}
K_1 \\ \vdots  \\ K_n
\end{array}
\right) \AND
X=\left(
\begin{array}{ccc}
x_1 \\ \vdots  \\ x_n
\end{array}
\right).
\]
Then \eqref{lingen} has the compact form
\begin{equation}  \label{E:geneqn}
\begin{array}{rcl}
\dps\frac{dX}{dt} & = & CX\\
X(0) & = & X_0.  
\end{array}
\end{equation}


In Section~\ref{s:3.5}, we plotted the phase space picture 
of the planar system of differential equations
\begin{equation} \label{-13}
\vectwo{\dot{x}}{\dot{y}}
= C \vectwo{x(t)}{y(t)}
\end{equation}
where
\[
C = \mattwo{-1}{3}{3}{-1}.
\]
In those calculations we observed that there is a solution to
\eqref{-13} that stayed on the main diagonal for each moment in
time.  Note that a vector is on the main diagonal if it is a
scalar multiple of $\vectwo{1}{1}$.  Thus a solution that stays
on the main diagonal for all time $t$ must have the form
\begin{equation} \label{e:diagform}
\vectwo{x(t)}{y(t)} = u(t) \vectwo{1}{1}
\end{equation}
for some real-valued function $u(t)$.  When a function of form
\eqref{e:diagform} is a solution to \eqref{-13}, it satisfies:
\begin{align*}
  \dot{u}(t)\vectwo{1}{1} &= \vectwo{\dot{x}(t)}{\dot{y}(t)} 
                            = C \vectwo{x(t)}{y(t)} \\
                          &= C u(t)\vectwo{1}{1} = u(t) C \vectwo{1}{1}.
\end{align*}
A calculation shows that
\[
C \vectwo{1}{1} = 2 \vectwo{1}{1}.
\]
Hence
\[
\dot{u}(t) \vectwo{1}{1} =  2 u(t) \vectwo{1}{1}.
\]
It follows that the function $u(t)$ must satisfy the
differential equation
\[
\frac{du}{dt} = 2u.
\]
whose solutions are
\[
u(t) = \alpha e^{2t},
\]
for some scalar $\alpha$.

Similarly, we also saw in our \Matlab experiments that there was
a solution that for all time stayed on the anti-diagonal, the
line $y=-x$.  Such a solution must have the form
\[
\vectwo{x(t)}{y(t)} = v(t) \vectwo{1}{-1}.
\]
A similar calculation shows that $v(t)$ must satisfy the
differential equation
\[
\frac{dv}{dt} = -4v.
\]
Solutions to this equation all have the form
\[
v(t) = \beta e^{-4t},
\]
for some real constant $\beta$.

Thus, using matrix multiplication, we are able to prove
analytically that there are solutions to \eqref{-13} of exactly
the type suggested by our \Matlab experiments.  However, even
more is true and this extension is based on the principle of 
superposition that was introduced for algebraic equations in 
Section~\ref{S:Superposition}.  

\subsection*{Superposition in Linear Differential Equations}
\index{superposition}\index{differential equation!superposition}

Consider a general linear differential equation of the form
\begin{equation} \label{gen1}
\frac{dX}{dt} = CX,
\end{equation}
where $C$ is an $n\times n$ matrix.  Suppose that $Y(t)$ and
$Z(t)$ are solutions to \eqref{gen1} and $\alpha,\beta\in\R$ are
scalars.  Then $X(t)=\alpha Y(t)+\beta Z(t)$ is also a solution.
We verify this fact using the `linearity' of $d/dt$.  Calculate
\begin{eqnarray*}
\frac{d}{dt} X(t) & = &
\alpha \frac{dY}{dt}(t) + \beta \frac{dZ}{dt}(t) \\
 & = &\alpha CY(t) + \beta CZ(t)\\
 & = & C(\alpha Y(t) + \beta Z(t))\\
 & = & CX(t).
\end{eqnarray*}
So superposition is valid for solutions of linear differential equations.


\subsection*{Initial Value Problems}
\index{initial value problem}

Suppose that we wish to find a solution to
\eqref{-13} satisfying the initial conditions\index{initial condition}
\[
\left(\begin{array}{c} x(0) \\ y(0) \end{array}\right) =
\left(\begin{array}{c}1\\3\end{array}\right).
\]
Then we can use the principle of superposition to find this solution in 
closed form.  Superposition implies that for each pair of scalars 
$\alpha,\beta\in\R$, the functions
\begin{equation}  \label{e:solnODE}
\left(\begin{array}{c} x(t) \\ y(t) \end{array}\right) =
\alpha e^{2t}\left(\begin{array}{c}1\\1\end{array}\right) +
\beta e^{-4t}\left(\begin{array}{r} 1\\-1\end{array}\right),
\end{equation}
are solutions to \eqref{-13}.  Moreover, for a solution of this form 
\[
\left(\begin{array}{c} x(0) \\ y(0) \end{array}\right) =
\left(\begin{array}{c} \alpha+\beta \\ \alpha-\beta
\end{array}\right).
\]

Thus we can solve our prescribed initial value problem, if we can
solve the system of linear equations
\begin{eqnarray*}
\alpha + \beta = 1\ \\
\alpha - \beta = 3.
\end{eqnarray*}
This system is solved for $\alpha=2$ and $\beta=-1$. Thus
\[
\vectwo{x(t)}{y(t)} = 2e^{2t}\vectwo{1}{1} - e^{-4t}\vectwo{1}{-1}
\]
is the desired closed form solution.

\subsection*{Eigenvectors and Eigenvalues}

We emphasize that just knowing that there are two lines in the
plane that are invariant under the dynamics of the system of
linear differential equations is sufficient information to solve
these equations.  So it seems appropriate to ask the question:
When is there a line that is invariant under the dynamics of a
system of linear differential equations?  This question is
equivalent to asking:  When is there a nonzero vector $v$ and a
nonzero real-valued function $u(t)$ such that
\[
X(t) = u(t) v
\]
is a solution to \eqref{gen1}?

Suppose that $X(t)$ is a solution to the system of differential
equations $\dot{X}=CX$.  Then $u(t)$ and $v$ must satisfy
\begin{equation}  \label{E:diffdir}
\dot{u}(t)v = \frac{dX}{dt} = CX(t) = u(t) Cv.
\end{equation}
Since $u$ is nonzero, it follows that $v$ and $Cv$ must lie on the
same line through the origin.  Hence
\begin{equation}  \label{e:eigendef}
Cv = \lambda v,
\end{equation}
for some real number $\lambda$.

\begin{definition}  \label{D:eigenvalue1}
A nonzero vector $v$ satisfying \eqref{e:eigendef} is called an
{\em eigenvector\/} \index{eigenvector} of the matrix $C$, and
the number $\lambda$ is an {\em eigenvalue\/} \index{eigenvalue}
of the matrix $C$.
\end{definition}
Geometrically, the matrix $C$ maps an eigenvector onto a multiple
of itself --- that multiple is the eigenvalue.

Note that scalar multiples of eigenvectors are also eigenvectors.  More 
precisely:
\begin{lemma}  \label{L:e'vector}
Let $v$ be an eigenvector of the matrix $C$ with eigenvalue $\lambda$.   
Then $\alpha v$ is also an eigenvector of $C$ with eigenvalue $\lambda$ 
as long as $\alpha\neq 0$.
\end{lemma}

\begin{proof} By assumption, $Cv=\lambda v$ and $v$ is nonzero. Now calculate
\[
C(\alpha v) = \alpha Cv = \alpha\lambda v = \lambda(\alpha v).
\]
The lemma follows from the definition of eigenvector.  \end{proof}

It follows from \eqref{E:diffdir} and \eqref{e:eigendef} that if $v$ is
an eigenvector of $C$ with eigenvalue $\lambda$, then
\[
\frac{du}{dt} = \lambda u.
\]
Thus we have returned to our original linear differential
equation that has solutions
\[
u(t) = Ke^{\lambda t},
\]
for all constants $K$.

We have proved the following theorem.
\begin{theorem}  \label{T:eigensoln}
Let $v$ be an eigenvector of the $n\times n$ matrix $C$ with
eigenvalue $\lambda$.  Then
\[
X(t) = e^{\lambda t}v
\]
is a solution to the system of differential equations $\dot{X}=CX$.
\end{theorem}



Finding eigenvalues and eigenvectors from first principles --- even for 
$2\times 2$ matrices --- is not a simple task.  We end this section with 
a calculation illustrating that real eigenvalues need not exist.  In 
Section~\ref{S:evchp}, we present a natural method for computing  
eigenvalues (and eigenvectors) of $2\times2$ matrices.  We defer the 
discuss of how to find eigenvalues and eigenvectors of $n\times n$ matrices 
until Chapter~\ref{C:D&E}.


\subsubsection*{An Example of a Matrix with No Real Eigenvalues}

Not every matrix has {\em real\/} eigenvalues\index{eigenvalue!real} and
eigenvectors\index{eigenvector!real}.  Recall the linear system of differential 
equations $\dot{x}=Cx$ whose phase plane is pictured in Figure~\ref{pp_dsp2}.  
That phase plane showed no evidence of an invariant line and indeed there is 
none.  The matrix $C$ in that example was
\[
C=\mattwo{-1}{-2}{3}{-1}.
\]
We ask: Is there a value of $\lambda$ and a nonzero vector
$(x,y)$ such that
\begin{equation}  \label{E:eigexamp}
C\left(\begin{array}{c} x\\y\end{array}\right) =
\lambda  \left(\begin{array}{c} x\\y\end{array}\right)?
\end{equation}
Equation \eqref{E:eigexamp} implies that
\[
\left(\begin{array}{cc} -1-\lambda & -2 \\ 3 & -1-\lambda
\end{array}\right) \left(\begin{array}{c}
x\\y\end{array}\right) =0.
\]
If this matrix is row equivalent to the identity matrix, then
the only solution of the linear system is $x=y=0$.  To have a
nonzero solution, the matrix
\[
\left(\begin{array}{cc} -1-\lambda & -2 \\ 3 & -1-\lambda
\end{array}\right)
\]
must not be row equivalent to $I_2$.  Dividing the $1^{st}$ row by
$-(1+\lambda)$ leads to
\[
\left(\begin{array}{cc} 1 & \frac{2}{1+\lambda} \\ 3 & -1-\lambda
\end{array}\right).
\]
Subtracting $3$ times the $1^{st}$ row from the second produces
the matrix
\[
\left(\begin{array}{cc} 1 & \frac{2}{1+\lambda} \\ 0 &
-(1+\lambda) - \frac{6}{1+\lambda}
\end{array}\right).
\]
This matrix is not row equivalent to $I_2$ when the lower
right hand entry is zero; that is, when
\[
(1+\lambda) +\frac{6}{1+\lambda} = 0.
\]
That is, when
\[
(1+\lambda)^2 = -6,
\]
which is not possible for any real number $\lambda$.  This
example shows that the question of whether a given matrix has a
real eigenvalue and a real eigenvector --- and hence when the
associated system of differential equations has a line that is
invariant under the dynamics --- is a subtle question.

Questions concerning eigenvectors and eigenvalues are central to
much of the theory of linear algebra.  We discuss this
topic for $2\times 2$ matrices in Section~\ref{S:evchp} and
Chapter~\ref{Chap:Planar} and for general square matrices in
Chapters~\ref{C:D&E} and \ref{C:HDeigenvalues}.



\includeexercises


\end{document}

%%% Local Variables:
%%% mode: latex
%%% TeX-master: "../linearAlgebra"
%%% End:
