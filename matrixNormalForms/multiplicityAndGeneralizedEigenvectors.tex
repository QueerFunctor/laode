\documentclass{ximera}

 

\usepackage{epsfig}

\graphicspath{
  {./}
  {figures/}
}

\usepackage{morewrites}
\makeatletter
\newcommand\subfile[1]{%
\renewcommand{\input}[1]{}%
\begingroup\skip@preamble\otherinput{#1}\endgroup\par\vspace{\topsep}
\let\input\otherinput}
\makeatother

\newcommand{\includeexercises}{\directlua{dofile("/home/jim/linearAlgebra/laode/exercises.lua")}}

%\newcounter{ccounter}
%\setcounter{ccounter}{1}
%\newcommand{\Chapter}[1]{\setcounter{chapter}{\arabic{ccounter}}\chapter{#1}\addtocounter{ccounter}{1}}

%\newcommand{\section}[1]{\section{#1}\setcounter{thm}{0}\setcounter{equation}{0}}

%\renewcommand{\theequation}{\arabic{chapter}.\arabic{section}.\arabic{equation}}
%\renewcommand{\thefigure}{\arabic{chapter}.\arabic{figure}}
%\renewcommand{\thetable}{\arabic{chapter}.\arabic{table}}

%\newcommand{\Sec}[2]{\section{#1}\markright{\arabic{ccounter}.\arabic{section}.#2}\setcounter{equation}{0}\setcounter{thm}{0}\setcounter{figure}{0}}

\newcommand{\Sec}[2]{\section{#1}}

\setcounter{secnumdepth}{2}
%\setcounter{secnumdepth}{1} 

%\newcounter{THM}
%\renewcommand{\theTHM}{\arabic{chapter}.\arabic{section}}

\newcommand{\trademark}{{R\!\!\!\!\!\bigcirc}}
%\newtheorem{exercise}{}

\newcommand{\dfield}{{\sf dfield9}}
\newcommand{\pplane}{{\sf pplane9}}

\newcommand{\EXER}{\section*{Exercises}}%\vspace*{0.2in}\hrule\small\setcounter{exercise}{0}}
\newcommand{\CEXER}{}%\vspace{0.08in}\begin{center}Computer Exercises\end{center}}
\newcommand{\TEXER}{} %\vspace{0.08in}\begin{center}Hand Exercises\end{center}}
\newcommand{\AEXER}{} %\vspace{0.08in}\begin{center}Hand Exercises\end{center}}

% BADBAD: \newcommand{\Bbb}{\bf}

\newcommand{\R}{\mbox{$\Bbb{R}$}}
\newcommand{\C}{\mbox{$\Bbb{C}$}}
\newcommand{\Z}{\mbox{$\Bbb{Z}$}}
\newcommand{\N}{\mbox{$\Bbb{N}$}}
\newcommand{\D}{\mbox{{\bf D}}}
\usepackage{amssymb}
%\newcommand{\qed}{\hfill\mbox{\raggedright$\square$} \vspace{1ex}}
%\newcommand{\proof}{\noindent {\bf Proof:} \hspace{0.1in}}

\newcommand{\setmin}{\;\mbox{--}\;}
\newcommand{\Matlab}{{M\small{AT\-LAB}} }
\newcommand{\Matlabp}{{M\small{AT\-LAB}}}
\newcommand{\computer}{\Matlab Instructions}
\newcommand{\half}{\mbox{$\frac{1}{2}$}}
\newcommand{\compose}{\raisebox{.15ex}{\mbox{{\scriptsize$\circ$}}}}
\newcommand{\AND}{\quad\mbox{and}\quad}
\newcommand{\vect}[2]{\left(\begin{array}{c} #1_1 \\ \vdots \\
 #1_{#2}\end{array}\right)}
\newcommand{\mattwo}[4]{\left(\begin{array}{rr} #1 & #2\\ #3
&#4\end{array}\right)}
\newcommand{\mattwoc}[4]{\left(\begin{array}{cc} #1 & #2\\ #3
&#4\end{array}\right)}
\newcommand{\vectwo}[2]{\left(\begin{array}{r} #1 \\ #2\end{array}\right)}
\newcommand{\vectwoc}[2]{\left(\begin{array}{c} #1 \\ #2\end{array}\right)}

\newcommand{\ignore}[1]{}


\newcommand{\inv}{^{-1}}
\newcommand{\CC}{{\cal C}}
\newcommand{\CCone}{\CC^1}
\newcommand{\Span}{{\rm span}}
\newcommand{\rank}{{\rm rank}}
\newcommand{\trace}{{\rm tr}}
\newcommand{\RE}{{\rm Re}}
\newcommand{\IM}{{\rm Im}}
\newcommand{\nulls}{{\rm null\;space}}

\newcommand{\dps}{\displaystyle}
\newcommand{\arraystart}{\renewcommand{\arraystretch}{1.8}}
\newcommand{\arrayfinish}{\renewcommand{\arraystretch}{1.2}}
\newcommand{\Start}[1]{\vspace{0.08in}\noindent {\bf Section~\ref{#1}}}
\newcommand{\exer}[1]{\noindent {\bf \ref{#1}}}
\newcommand{\ans}{}
\newcommand{\matthree}[9]{\left(\begin{array}{rrr} #1 & #2 & #3 \\ #4 & #5 & #6
\\ #7 & #8 & #9\end{array}\right)}
\newcommand{\cvectwo}[2]{\left(\begin{array}{c} #1 \\ #2\end{array}\right)}
\newcommand{\cmatthree}[9]{\left(\begin{array}{ccc} #1 & #2 & #3 \\ #4 & #5 &
#6 \\ #7 & #8 & #9\end{array}\right)}
\newcommand{\vecthree}[3]{\left(\begin{array}{r} #1 \\ #2 \\
#3\end{array}\right)}
\newcommand{\cvecthree}[3]{\left(\begin{array}{c} #1 \\ #2 \\
#3\end{array}\right)}
\newcommand{\cmattwo}[4]{\left(\begin{array}{cc} #1 & #2\\ #3
&#4\end{array}\right)}

\newcommand{\Matrix}[1]{\ensuremath{\left(\begin{array}{rrrrrrrrrrrrrrrrrr} #1 \end{array}\right)}}

\newcommand{\Matrixc}[1]{\ensuremath{\left(\begin{array}{cccccccccccc} #1 \end{array}\right)}}



\renewcommand{\labelenumi}{\theenumi)}
\newenvironment{enumeratea}%
{\begingroup
 \renewcommand{\theenumi}{\alph{enumi}}
 \renewcommand{\labelenumi}{(\theenumi)}
 \begin{enumerate}}
 {\end{enumerate}\endgroup}



\newcounter{help}
\renewcommand{\thehelp}{\thesection.\arabic{equation}}

%\newenvironment{equation*}%
%{\renewcommand\endequation{\eqno (\theequation)* $$}%
%   \begin{equation}}%
%   {\end{equation}\renewcommand\endequation{\eqno \@eqnnum
%$$\global\@ignoretrue}}

%\input{psfig.tex}

\author{Martin Golubitsky and Michael Dellnitz}

%\newenvironment{matlabEquation}%
%{\renewcommand\endequation{\eqno (\theequation*) $$}%
%   \begin{equation}}%
%   {\end{equation}\renewcommand\endequation{\eqno \@eqnnum
% $$\global\@ignoretrue}}

\newcommand{\soln}{\textbf{Solution:} }
\newcommand{\exercap}[1]{\centerline{Figure~\ref{#1}}}
\newcommand{\exercaptwo}[1]{\centerline{Figure~\ref{#1}a\hspace{2.1in}
Figure~\ref{#1}b}}
\newcommand{\exercapthree}[1]{\centerline{Figure~\ref{#1}a\hspace{1.2in}
Figure~\ref{#1}b\hspace{1.2in}Figure~\ref{#1}c}}
\newcommand{\para}{\hspace{0.4in}}

\renewenvironment{solution}{\suppress}{\endsuppress}

\ifxake
\newenvironment{matlabEquation}{\begin{equation}}{\end{equation}}
\else
\newenvironment{matlabEquation}%
{\let\oldtheequation\theequation\renewcommand{\theequation}{\oldtheequation*}\begin{equation}}%
  {\end{equation}\let\theequation\oldtheequation}
\fi

\makeatother


\title{Multiplicity and Generalized Eigenvectors}

\begin{document}
\begin{abstract}
\end{abstract}
\maketitle

  \label{S:MGE}

The difficulty in generalizing the results in the previous two sections to
matrices with multiple eigenvalues stems from the fact that these matrices 
may not have enough (linearly independent) eigenvectors.  In this section we 
present the basic examples of matrices with a deficiency of eigenvectors, as
well as the definitions of algebraic and geometric multiplicity.  These 
matrices will be the building blocks of the Jordan normal form theorem --- 
the theorem that classifies all matrices up to similarity.

\subsubsection*{Deficiency in Eigenvectors for Real Eigenvalues}

An example of deficiency in eigenvectors is given by the following 
$n\times n$ matrix
\begin{equation}  \label{E:JnR}
J_n(\lambda_0)=\left(\begin{array}{cccccc} \lambda_0 & 1 & 0 & \cdots & 0 & 0\\
	0 & \lambda_0 & 1 & \cdots & 0 & 0 \\
	0 & 0 & \lambda_0  & \cdots & 0 & 0\\
	\vdots & \vdots & \vdots & \ddots & \vdots & \vdots\\
	0 & 0 & 0 & \cdots & \lambda_0 & 1 \\
	0 & 0 & 0 & \cdots & 0 & \lambda_0 \end{array}\right)
\end{equation}
where $\lambda_0\in\R$.  Note that $J_n(\lambda_0)$ has all diagonal 
entries equal to $\lambda_0$,
all superdiagonal entries equal to $1$, and all other entries equal 
to $0$. Since $J_n(\lambda_0)$ is upper triangular, all $n$ 
eigenvalues of $J_n(\lambda_0)$ are equal to $\lambda_0$.  However,
$J_n(\lambda_0)$ has only one linearly independent eigenvector.  To 
verify this assertion let 
\[
N = J_n(\lambda_0) - \lambda_0I_n.
\]
Then $v$ is an eigenvector of $J_n(\lambda_0)$ if and only if $Nv=0$.
Therefore, $J_n(\lambda_0)$ has a unique linearly independent 
eigenvector  if 
\begin{lemma}
{\rm nullity}$(N)=1$.
\end{lemma}\index{nullity}

\begin{proof}  In coordinates the equation $Nv=0$ is:
\[
\left(\begin{array}{cccccc} 0 & 1 & 0 & \cdots & 0 & 0\\
	0 & 0 & 1 & \cdots & 0 & 0 \\
	0 & 0 & 0  & \cdots & 0 & 0\\
	\vdots & \vdots & \vdots & \ddots & \vdots & \vdots\\
	0 & 0 & 0 & \cdots & 0 & 1 \\
	0 & 0 & 0 & \cdots & 0 & 0 \end{array}\right)
\left(\begin{array}{l} v_1 \\ v_2 \\ v_3 \\ \vdots \\ v_{n-1} \\v_n
	\end{array}\right) 
= \left(\begin{array}{l} v_2 \\ v_3 \\ v_4 \\ \vdots \\ v_n \\ 0
	\end{array}\right) = 0.
\]
Thus $v_2 = v_3 = \cdots v_n = 0$, and the solutions are all multiples 
of $e_1$.  Therefore, the nullity of $N$ is $1$.  \end{proof}

Note that we can express matrix multiplication by $N$ as
\begin{equation}  \label{e:Ndef}
\begin{array}{rcll} 
Ne_1 & = & 0 & \\
Ne_j & = & e_{j-1} & \quad j=2,\ldots,n .
\end{array}
\end{equation}
Note that \eqref{e:Ndef} implies that $N^n=0$.

The $n\times n$ matrix $N$ motivates the following definitions.  
\begin{definition} \label{D:multiplicities}
Let $\lambda_0$ be an eigenvalue of $A$.  The {\em algebraic multiplicity\/} 
\index{multiplicity!algebraic} of $\lambda_0$ is the number of times 
that $\lambda_0$ appears as a root of the characteristic polynomial 
$p_A(\lambda)$.  The {\em geometric multiplicity\/} 
\index{multiplicity!geometric} of $\lambda_0$ is the number of linearly 
independent eigenvectors of $A$ having eigenvalue equal to $\lambda_0$.
\end{definition}  
Abstractly, the geometric multiplicity is:
\[
{\rm nullity}(A-\lambda_0I_n).
\]

Our previous calculations show that the matrix $J_n(\lambda_0)$
has an eigenvalue $\lambda_0$ with algebraic multiplicity equal
to $n$ and geometric multiplicity equal to $1$.

\begin{lemma} 
The algebraic multiplicity of an eigenvalue is greater than 
or equal to its geometric multiplicity.
\end{lemma}

\begin{proof} For ease of notation we prove this lemma only for real 
eigenvalues, though the proof for complex eigenvalues is similar.  Let 
$A$ be an $n\times n$ matrix and let $\lambda_0$ be a real eigenvalue 
of $A$. Let $k$ be the geometric multiplicity of $\lambda_0$ and let 
$v_1,\ldots,v_k$ be $k$ linearly independent eigenvectors of $A$ with 
eigenvalue $\lambda_0$.   We can extend $\{v_1,\ldots,v_k\}$ to be a basis
${\cal V} = \{v_1,\ldots,v_n\}$ of $\R^n$.  In this basis, the matrix
of $A$ is 
\[
[A]_{\cal V} = \mattwo{\lambda_0 I_k}{(*)}{0}{B}.
\]
The matrices $A$ and $[A]_{\cal V}$ are similar matrices. Therefore,
they have the same 
characteristic polynomials\index{characteristic polynomial}
and the same eigenvalues 
with the same algebraic multiplicities.  It follows from 
Lemma~\ref{L:detblockdiag} that the characteristic polynomial of $A$ is:
\[
p_A(\lambda) = p_{[A]_{\cal V}}(\lambda) = (\lambda-\lambda_0)^kp_B(\lambda).
\]
Hence $\lambda_0$ appears as a root of $p_A(\lambda)$ at least 
$k$ times and the algebraic multiplicity of $\lambda_0$ 
is greater than or equal to $k$.   The same proof works when $\lambda_0$ 
is a complex eigenvalue --- but all vectors chosen must be complex rather
than real.   \end{proof}

\subsubsection*{Deficiency in Eigenvectors with Complex Eigenvalues}

An example of a real matrix with complex conjugate eigenvalues having 
geometric multiplicity less than algebraic multiplicity is the 
$2n \times 2n$ block matrix
\begin{equation} \label{E:JnC}
\widehat{J}_n(\lambda_0)=
\left(\begin{array}{cccccc} B & I_2 & 0 & \cdots & 0 & 0\\
	0 & B & I_2 & \cdots & 0 & 0 \\
	0 & 0 & B  & \cdots & 0 & 0\\
	\vdots & \vdots & \vdots & \ddots & \vdots & \vdots\\
	0 & 0 & 0 & \cdots & B & I_2 \\
	0 & 0 & 0 & \cdots & 0 & B \end{array}\right)
\end{equation}
where $\lambda_0=\sigma+i\tau$ and $B$ is the $2\times 2$ matrix
\[
B = \mattwo{\sigma}{-\tau}{\tau}{\sigma}.
\]

\begin{lemma}
Let $\lambda_0$ be a complex number.  Then the algebraic multiplicity
of the eigenvalue $\lambda_0$ in the $2n\times 2n$ matrix
$\widehat{J}_n(\lambda_0)$ is $n$ and the geometric multiplicity is $1$.
\end{lemma}

\begin{proof} We begin by showing that the eigenvalues of
$J=\widehat{J}_n(\lambda_0)$ are $\lambda_0$ and $\overline{\lambda_0}$, each 
with algebraic multiplicity $n$.  The characteristic polynomial of $J$ is 
$p_J(\lambda)=\det(J-\lambda I_{2n})$.  From Lemma~\ref{L:detblockdiag} of 
Chapter~\ref{C:D&E} and induction, we see that $p_J(\lambda)=p_B(\lambda)^n$. 
Since the eigenvalues of $B$ are $\lambda_0$ and $\overline{\lambda_0}$, we
have proved that the algebraic multiplicity of each of these eigenvalues in 
$J$ is $n$.
   
Next, we compute the eigenvectors of $J$.  Let $Jv=\lambda_0v$ 
and let $v=(v_1,\ldots,v_n)$ where each $v_j\in\C^2$.  Observe that 
$(J-\lambda_0I_{2n})v=0$ if and only if
\begin{eqnarray*}
Qv_1 + v_2 & = & 0 \\
& \vdots & \\
Qv_{n-1} + v_n & = & 0 \\
Qv_n & = & 0,
\end{eqnarray*}
where $Q=B-\lambda_0I_2$.
Using the fact that $\lambda_0=\sigma+i\tau$, it follows that 
\[
Q=B-\lambda_0I_2 = -\tau\mattwo{i}{1}{-1}{i}.
\]
Hence
\[
Q^2 = 2\tau^2i\mattwo{i}{1}{-1}{i}=-2\tau iQ.
\]
Thus
\[
0 = Q^2v_{n-1} + Qv_n
= -2\tau iQv_{n-1},
\]
from which it follows that $Qv_{n-1}+v_n = v_n = 0$.  Similarly, 
$v_2=\cdots=v_{n-1}=0$.  Since there is only one nonzero complex vector 
$v_1$ (up to a complex scalar multiple) satisfying
\[
Qv_1 = 0,
\]
it follows that the geometric multiplicity of $\lambda_0$ in the matrix
$\widehat{J}_n(\lambda_0)$ equals $1$.  \end{proof}

\begin{definition}  \label{D:jordanblock}
The real matrices $J_n(\lambda_0)$ when $\lambda_0\in\R$ and 
$\widehat{J}_n(\lambda_0)$ when $\lambda_0\in\C$ are {\em real Jordan
blocks\/}.  The matrices $J_n(\lambda_0)$ when $\lambda_0\in\C$ are (complex)
{\em Jordan blocks\/}.
\end{definition} \index{Jordan block}

\subsection*{Generalized Eigenvectors and Generalized Eigenspaces}

What happens when $n\times n$ matrices have fewer that $n$ linearly 
independent eigenvectors?  Answer: The matrices gain generalized 
eigenvectors.

\begin{definition}
A vector $v\in\C^n$ is a {\em generalized eigenvector\/} for the 
$n\times n$ matrix $A$ with eigenvalue $\lambda$ if
\begin{equation}  \label{e:geneig}
(A-\lambda I_n)^kv = 0
\end{equation}
for some positive integer $k$. The smallest integer $k$ for which
\eqref{e:geneig} is satisfied is called the {\em index\/}
\index{index} of the generalized eigenvector $v$.
\end{definition}  \index{eigenvector!generalized}
Note: Eigenvectors\index{eigenvector} are generalized eigenvectors with
index equal to $1$.

Let $\lambda_0$ be a real number and let $N=J_n(\lambda_0)-\lambda_0 I_n$.  
Recall that \eqref{e:Ndef} implies that $N^n=0$.  Hence every vector in $\R^n$ 
is a generalized eigenvector for the matrix $J_n(\lambda_0)$.  So 
$J_n(\lambda_0)$ provides a good example of a matrix whose lack of 
eigenvectors (there is only one independent eigenvector) is made up for by 
generalized eigenvectors (there are $n$ independent generalized eigenvectors).

Let $\lambda_0$ be an eigenvalue of the $n\times n$ matrix $A$ and let 
$A_0 = A-\lambda_0 I_n$.  For simplicity, assume that $\lambda_0$ is real.
Note that 
\begin{align*}
  \nulls(A_0) &\subset \nulls(A_0^2)\subset\cdots \\
  &\subset\nulls(A_0^k)\subset\cdots \\
  &\subset\R^n.
\end{align*}
Therefore, the dimensions of the null spaces are bounded above by $n$ and
there must be a smallest $k$ such that 
\[
\dim\nulls(A_0^k)=\dim\nulls(A_0^{k+1}).
\]
It follows that 
\begin{equation}  \label{E:nullsequal}
\nulls(A_0^k)=\nulls(A_0^{k+1}).
\end{equation}
\begin{lemma}  \label{L:Jordan}
Let $\lambda_0$ be a real eigenvalue of the $n\times n$ matrix $A$ and let 
$A_0 = A-\lambda_0 I_n$.  Let $k$ be the smallest integer for which 
\eqref{E:nullsequal} is valid.  Then 
\[
\nulls(A_0^k)=\nulls(A_0^{k+j})
\]
for every interger $j>0$.
\end{lemma}

\begin{proof} We can prove the lemma by induction on $j$ if we can show that 
\[
\nulls(A_0^{k+1})=\nulls(A_0^{k+2}).
\]
Since $\nulls(A_0^{k+1})\subset\nulls(A_0^{k+2})$, we need to show that
\[
\nulls(A_0^{k+2})\subset\nulls(A_0^{k+1}).
\]
Let $w\in\nulls(A_0^{k+2})$.  It follows that 
\[
A^{k+1}Aw = A^{k+2}w = 0;
\]
so $Aw\in\nulls(A_0^{k+1})=\nulls(A_0^k)$, by \eqref{E:nullsequal}.  Therefore,
\[
A^{k+1}w = A^k(Aw) = 0,
\]
which verifies that $w\in\nulls(A_0^{k+1})$.  \end{proof}

Let $V_{\lambda_0}$ be the set of all generalized eigenvectors of $A$ with 
eigenvalue $\lambda_0$.  Let $k$ be the smallest integer satisfying
\eqref{E:nullsequal}, then Lemma~\ref{L:Jordan} implies that 
\[
V_{\lambda_0}=\nulls(A_0^k)\subset \R^n
\]
is a subspace called the {\em generalized eigenspace\/}
\index{generalized eigenspace} of $A$ associated to the eigenvalue
$\lambda_0$.  It will follow from the Jordan normal form theorem (see
Theorem~\ref{T:Jordan}) that the dimension of $V_{\lambda_0}$ is the 
algebraic multiplicity of $\lambda_0$.


\subsubsection*{An Example of Generalized Eigenvectors}
\index{eigenvector!generalized}

Find the generalized eigenvectors of the $4\times 4$ matrix
\begin{matlabEquation}\label{MATLAB:1}
A=\left(\begin{array}{rrrr}
  -24 & -58 &  -2 &  -8\\
   15 &  35 &   1 &   4\\
    3 &   5 &   7 &   4\\
    3 &   6 &   0 &   6
\end{array}\right).
\end{matlabEquation}
and their indices.  When finding generalized eigenvectors of a matrix $A$, 
the first two steps are:
\begin{enumerate}
\item[(i)]  Find the eigenvalues of $A$.
\item[(ii)]  Find the eigenvectors of $A$.
\end{enumerate}
After entering $A$ into \Matlab by typing {\tt e13\_3\_6}, we type 
{\tt eig(A)} and find that all of the eigenvalues of $A$ equal $6$.  Without 
additional information, there could be 1,2,3 or 4 linearly independent 
eigenvectors of $A$ corresponding to the eigenvalue $6$.  In \Matlab we 
determine the number of linearly independent eigenvectors by typing 
{\tt null(A-6*eye(4))} and obtaining
\begin{verbatim}
ans =
    0.8892         0
   -0.4446    0.0000
   -0.0262    0.9701
   -0.1046   -0.2425
\end{verbatim}
 
We now know that (numerically) there are two linearly independent
eigenvectors.  The next step is find the number of independent generalized 
eigenvectors of index 2.\index{index}  To complete this calculation, we find 
a basis for the null space of $(A-6I_4)^2$ by typing 
{\tt null((A-6*eye(4))\^{ }2)} obtaining
\begin{verbatim}
ans =
     1     0     0     0
     0     1     0     0
     0     0     1     0
     0     0     0     1
\end{verbatim}
Thus, for this example, all generalized eigenvectors that are not
eigenvectors have index $2$.

\EXER

\TEXER

\noindent In Exercises~\ref{c10.5.1a} -- \ref{c10.5.1d} determine the 
eigenvalues and their geometric and algebraic multiplicities for the 
given matrix.
\begin{exercise} \label{c10.5.1a}
$A = \left(\begin{array}{cccc} 2 & 0 &  0 & 0\\ 0 & 3 & 1 & 0 \\
0 & 0 & 3 & 0 \\ 0 & 0 & 0 & 4 \end{array}\right)$.
\end{exercise}
\begin{exercise} \label{c10.5.1b}
$B = \left(\begin{array}{cccc} 2 & 0 &  0 & 0\\ 0 & 2 & 0 & 0 \\
0 & 0 & 3 & 1 \\ 0 & 0 & 0 & 3 \end{array}\right)$.
\end{exercise}
\begin{exercise} \label{c10.5.1c}
$C = \left(\begin{array}{rrrr} -1 & 1 &  0 & 0\\ 0 & -1 & 0 & 0 \\
0 & 0 & -1 & 0 \\ 0 & 0 & 0 & 1 \end{array}\right)$.
\end{exercise}
\begin{exercise} \label{c10.5.1d}
$D = \left(\begin{array}{rrrr} 2 & -1 &  0 & 0\\ 1 & 2 & 0 & 0 \\
0 & 0 & 2 & 1 \\ 0 & 0 & 0 & 2 \end{array}\right)$.
\end{exercise}

\noindent In Exercises~\ref{c10.5.2a} -- \ref{c10.5.2d} find a basis 
consisting of the eigenvectors for the given matrix supplemented by 
generalized eigenvectors.  Choose the generalized eigenvectors with 
lowest index possible.
\begin{exercise}  \label{c10.5.2a}
$A=\mattwo{1}{-1}{1}{3}$.
\end{exercise}
\begin{exercise} \label{c10.5.2b}
$B=\left(\begin{array}{rrr} -2 & 0 & -2 \\-1 & 1 & -2 \\ 0 & 1 & -1 \end{array}
\right)$.
\end{exercise}
\begin{exercise} \label{c10.5.2c}
$C=\left(\begin{array}{rrr} -6 & 31 & -14 \\-1 & 6 & -2 \\ 0 & 2 & 1\end{array}
\right)$.
\end{exercise}
\begin{exercise} \label{c10.5.2d}
$D=\left(\begin{array}{rrr} 5 & 1 & 0 \\-3 & 1 & 1 \\ -12 & -4 & 0\end{array}
\right)$.
\end{exercise}


\CEXER
\noindent In Exercises~\ref{c10.5.3A} -- \ref{c10.5.3B}, use \Matlab to find 
the eigenvalues and their algebraic and geometric multiplicities for the given 
matrix.
\begin{exercise} \label{c10.5.3A}
\begin{matlabEquation}\label{eigenvalue-exercise}
A=\left(\begin{array}{rrrr} 2 & 3 & -21 & -3 \\2 & 7 & -41 & -5 \\ 
0 & 1 & -5 & -1 \\ 0 & 0 & 4 & 4 \end{array}
\right).
\end{matlabEquation}
\end{exercise}
\begin{exercise} \label{c10.5.3B}
\begin{matlabEquation}\label{eigenvalue-exercise-2}
B=\left(\begin{array}{rrrrr} 179 & -230 & 0 & 10 & -30 \\
144 & -185 & 0 & 8 & -24 \\ 30 & -39 & -1 & 3 & -9 \\ 192 & -245 & 0 & 9 & -30 
\\ 40 & -51 & 0 & 2 & -7\end{array}\right).
\end{matlabEquation}
\end{exercise}


\end{document}

%%% Local Variables:
%%% mode: latex
%%% TeX-master: "../linearAlgebra"
%%% End:
