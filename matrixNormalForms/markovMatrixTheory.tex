\documentclass{ximera}

 

\usepackage{epsfig}

\graphicspath{
  {./}
  {figures/}
}

\usepackage{morewrites}
\makeatletter
\newcommand\subfile[1]{%
\renewcommand{\input}[1]{}%
\begingroup\skip@preamble\otherinput{#1}\endgroup\par\vspace{\topsep}
\let\input\otherinput}
\makeatother

\newcommand{\includeexercises}{\directlua{dofile("/home/jim/linearAlgebra/laode/exercises.lua")}}

%\newcounter{ccounter}
%\setcounter{ccounter}{1}
%\newcommand{\Chapter}[1]{\setcounter{chapter}{\arabic{ccounter}}\chapter{#1}\addtocounter{ccounter}{1}}

%\newcommand{\section}[1]{\section{#1}\setcounter{thm}{0}\setcounter{equation}{0}}

%\renewcommand{\theequation}{\arabic{chapter}.\arabic{section}.\arabic{equation}}
%\renewcommand{\thefigure}{\arabic{chapter}.\arabic{figure}}
%\renewcommand{\thetable}{\arabic{chapter}.\arabic{table}}

%\newcommand{\Sec}[2]{\section{#1}\markright{\arabic{ccounter}.\arabic{section}.#2}\setcounter{equation}{0}\setcounter{thm}{0}\setcounter{figure}{0}}

\newcommand{\Sec}[2]{\section{#1}}

\setcounter{secnumdepth}{2}
%\setcounter{secnumdepth}{1} 

%\newcounter{THM}
%\renewcommand{\theTHM}{\arabic{chapter}.\arabic{section}}

\newcommand{\trademark}{{R\!\!\!\!\!\bigcirc}}
%\newtheorem{exercise}{}

\newcommand{\dfield}{{\sf dfield9}}
\newcommand{\pplane}{{\sf pplane9}}

\newcommand{\EXER}{\section*{Exercises}}%\vspace*{0.2in}\hrule\small\setcounter{exercise}{0}}
\newcommand{\CEXER}{}%\vspace{0.08in}\begin{center}Computer Exercises\end{center}}
\newcommand{\TEXER}{} %\vspace{0.08in}\begin{center}Hand Exercises\end{center}}
\newcommand{\AEXER}{} %\vspace{0.08in}\begin{center}Hand Exercises\end{center}}

% BADBAD: \newcommand{\Bbb}{\bf}

\newcommand{\R}{\mbox{$\Bbb{R}$}}
\newcommand{\C}{\mbox{$\Bbb{C}$}}
\newcommand{\Z}{\mbox{$\Bbb{Z}$}}
\newcommand{\N}{\mbox{$\Bbb{N}$}}
\newcommand{\D}{\mbox{{\bf D}}}
\usepackage{amssymb}
%\newcommand{\qed}{\hfill\mbox{\raggedright$\square$} \vspace{1ex}}
%\newcommand{\proof}{\noindent {\bf Proof:} \hspace{0.1in}}

\newcommand{\setmin}{\;\mbox{--}\;}
\newcommand{\Matlab}{{M\small{AT\-LAB}} }
\newcommand{\Matlabp}{{M\small{AT\-LAB}}}
\newcommand{\computer}{\Matlab Instructions}
\newcommand{\half}{\mbox{$\frac{1}{2}$}}
\newcommand{\compose}{\raisebox{.15ex}{\mbox{{\scriptsize$\circ$}}}}
\newcommand{\AND}{\quad\mbox{and}\quad}
\newcommand{\vect}[2]{\left(\begin{array}{c} #1_1 \\ \vdots \\
 #1_{#2}\end{array}\right)}
\newcommand{\mattwo}[4]{\left(\begin{array}{rr} #1 & #2\\ #3
&#4\end{array}\right)}
\newcommand{\mattwoc}[4]{\left(\begin{array}{cc} #1 & #2\\ #3
&#4\end{array}\right)}
\newcommand{\vectwo}[2]{\left(\begin{array}{r} #1 \\ #2\end{array}\right)}
\newcommand{\vectwoc}[2]{\left(\begin{array}{c} #1 \\ #2\end{array}\right)}

\newcommand{\ignore}[1]{}


\newcommand{\inv}{^{-1}}
\newcommand{\CC}{{\cal C}}
\newcommand{\CCone}{\CC^1}
\newcommand{\Span}{{\rm span}}
\newcommand{\rank}{{\rm rank}}
\newcommand{\trace}{{\rm tr}}
\newcommand{\RE}{{\rm Re}}
\newcommand{\IM}{{\rm Im}}
\newcommand{\nulls}{{\rm null\;space}}

\newcommand{\dps}{\displaystyle}
\newcommand{\arraystart}{\renewcommand{\arraystretch}{1.8}}
\newcommand{\arrayfinish}{\renewcommand{\arraystretch}{1.2}}
\newcommand{\Start}[1]{\vspace{0.08in}\noindent {\bf Section~\ref{#1}}}
\newcommand{\exer}[1]{\noindent {\bf \ref{#1}}}
\newcommand{\ans}{}
\newcommand{\matthree}[9]{\left(\begin{array}{rrr} #1 & #2 & #3 \\ #4 & #5 & #6
\\ #7 & #8 & #9\end{array}\right)}
\newcommand{\cvectwo}[2]{\left(\begin{array}{c} #1 \\ #2\end{array}\right)}
\newcommand{\cmatthree}[9]{\left(\begin{array}{ccc} #1 & #2 & #3 \\ #4 & #5 &
#6 \\ #7 & #8 & #9\end{array}\right)}
\newcommand{\vecthree}[3]{\left(\begin{array}{r} #1 \\ #2 \\
#3\end{array}\right)}
\newcommand{\cvecthree}[3]{\left(\begin{array}{c} #1 \\ #2 \\
#3\end{array}\right)}
\newcommand{\cmattwo}[4]{\left(\begin{array}{cc} #1 & #2\\ #3
&#4\end{array}\right)}

\newcommand{\Matrix}[1]{\ensuremath{\left(\begin{array}{rrrrrrrrrrrrrrrrrr} #1 \end{array}\right)}}

\newcommand{\Matrixc}[1]{\ensuremath{\left(\begin{array}{cccccccccccc} #1 \end{array}\right)}}



\renewcommand{\labelenumi}{\theenumi)}
\newenvironment{enumeratea}%
{\begingroup
 \renewcommand{\theenumi}{\alph{enumi}}
 \renewcommand{\labelenumi}{(\theenumi)}
 \begin{enumerate}}
 {\end{enumerate}\endgroup}



\newcounter{help}
\renewcommand{\thehelp}{\thesection.\arabic{equation}}

%\newenvironment{equation*}%
%{\renewcommand\endequation{\eqno (\theequation)* $$}%
%   \begin{equation}}%
%   {\end{equation}\renewcommand\endequation{\eqno \@eqnnum
%$$\global\@ignoretrue}}

%\input{psfig.tex}

\author{Martin Golubitsky and Michael Dellnitz}

%\newenvironment{matlabEquation}%
%{\renewcommand\endequation{\eqno (\theequation*) $$}%
%   \begin{equation}}%
%   {\end{equation}\renewcommand\endequation{\eqno \@eqnnum
% $$\global\@ignoretrue}}

\newcommand{\soln}{\textbf{Solution:} }
\newcommand{\exercap}[1]{\centerline{Figure~\ref{#1}}}
\newcommand{\exercaptwo}[1]{\centerline{Figure~\ref{#1}a\hspace{2.1in}
Figure~\ref{#1}b}}
\newcommand{\exercapthree}[1]{\centerline{Figure~\ref{#1}a\hspace{1.2in}
Figure~\ref{#1}b\hspace{1.2in}Figure~\ref{#1}c}}
\newcommand{\para}{\hspace{0.4in}}

\renewenvironment{solution}{\suppress}{\endsuppress}

\ifxake
\newenvironment{matlabEquation}{\begin{equation}}{\end{equation}}
\else
\newenvironment{matlabEquation}%
{\let\oldtheequation\theequation\renewcommand{\theequation}{\oldtheequation*}\begin{equation}}%
  {\end{equation}\let\theequation\oldtheequation}
\fi

\makeatother


\title{Markov Matrix Theory}

\begin{document}
\begin{abstract}
\end{abstract}
\maketitle


\label{S:TransitionTheory} \index{matrix!Markov}\index{matrix!Markov}

In this appendix we use the Jordan normal form theorem to study the asymptotic
dynamics of transition matrices\index{matrix!transition} such as those of 
Markov chains\index{Markov chain} introduced in 
Section~\ref{S:TransitionApplied}.

The basic result is the following theorem.
\begin{theorem} \label{T:convergeto0}
Let $A$ be an $n\times n$ matrix and assume that all eigenvalues $\lambda$ of 
$A$ satisfy $|\lambda|<1$.  Then for every vector $v_0\in\R^n$
\begin{equation}  \label{E:convergeto0}
\lim_{k\to\infty} A^kv_0 = 0.
\end{equation}
\end{theorem}

\begin{proof}  Suppose that $A$ and $B$ are similar matrices; that is, $B=SAS\inv$
for some invertible matrix $S$.  Then $B^k = SA^kS\inv$ and for any vector 
$v_0\in\R^n$ \Ref{E:convergeto0} is valid if and only if
\[
\lim_{k\to\infty} B^kv_0 = 0.
\]
Thus, when proving this theorem, we may assume that $A$ is in Jordan normal
form.

Suppose that $A$ is in block diagonal form; that is, suppose
\[
A=\mattwo{C}{0}{0}{D},
\]
where $C$ is an $\ell\times\ell$ matrix and $D$ is a $(n-\ell)\times (n-\ell)$
matrix.  Then 
\[
A^k = \mattwo{C^k}{0}{0}{D^k}.
\]
So for every vector $v_0=(w_0,u_0)\in\R^\ell\times\R^{n-\ell}$ 
\Ref{E:convergeto0} is valid if and only if 
\[
\lim_{k\to\infty} C^kv_0 = 0 \AND \lim_{k\to\infty} D^kv_0 = 0.
\]
So, when proving this theorem, we may assume that $A$ is a Jordan block.

Consider the case of a simple Jordan block\index{Jordan block}.  
Suppose that $n=1$ and that 
$A=(\lambda)$ where $\lambda$ is either real or complex.  Then 
\[
A^kv_0 = \lambda^kv_0.
\]
It follows that \Ref{E:convergeto0} is valid precisely when $|\lambda|<1$.  
Next, suppose that $A$ is a nontrivial Jordan block.  For example, let 
\[
A=\mattwo{\lambda}{1}{0}{\lambda} = \lambda I_2+N
\]
where $N^2=0$.  It follows by induction that 
\[
A^kv_0 = \lambda^kv_0 + k\lambda^{k-1}Nv_0 = \lambda^kv_0 + 
k\lambda^k\frac{1}{\lambda}Nv_0.
\]
Thus \Ref{E:convergeto0} is valid precisely when $|\lambda|<1$.  The reason for this convergence is as follows.  The first term converges to $0$ as 
before but the second term is the product of three terms $k$, $\lambda^k$,
and $\frac{1}{\lambda}Nv_0$.  The first increases to infinity, the second
decreases to zero, and the third is constant independent of $k$.  In fact,
geometric decay\index{geometric decay} 
($\lambda^k$, when $|\lambda|<1$) always beats 
polynomial growth\index{polynomial growth}.  Indeed,
\begin{equation}  \label{E:PG}
\lim_{m\to\infty}m^j\lambda^m = 0
\end{equation}
for any integer $j$.  This fact can be proved using l'H\^{o}spital's rule 
and induction.

So we see that when $A$ has a nontrivial Jordan block, convergence is 
subtler than when $A$ has only simple Jordan blocks, as initially the 
vectors $Av_0$ grow in magnitude.  For example, suppose that $\lambda=0.75$ 
and $v_0=(1,0)^t$.  Then $A^8v_0 = (0.901,0.075)^t$ is the first vector in 
the sequence $A^kv_0$ whose norm is less than $1$; that is, $A^8v_0$ is the 
first vector in the sequence closer to the origin than $v_0$.  

It is also true that \Ref{E:convergeto0} is valid for any Jordan block 
$A$ and for all $v_0$ precisely when $|\lambda|<1$.  To verify this fact 
we use the binomial theorem\index{binomial theorem}.  
We can write a nontrivial Jordan block as
$\lambda I_n+N$ where $N^{k+1}=0$ for some integer $k$.  We just discussed 
the case $k=1$.  In this case
\begin{align*}
(\lambda I_n+N)^m &= \lambda^mI_n+m\lambda^{m-1}N + 
\left(\begin{array}{@{}c@{}} m\\2\end{array}\right)\lambda^{m-2}N^2
  +\cdots \\
  &\quad + 
\left(\begin{array}{@{}c@{}} m\\k \end{array}\right)\lambda^{m-k}N^k,
\end{align*}
where
\[
\left(\begin{array}{@{}c@{}} m \\ j\end{array}\right)
 = \frac{m!}{j!(m-j)!}=\frac{m(m-1)\cdots(m-j+1)}{j!}.
\]
To verify that 
\[
\lim_{m\to\infty}(\lambda I_n+N)^m = 0
\]
we need only verify that each term
\[
\lim_{m\to\infty}\left(\begin{array}{@{}c@{}} m\\j \end{array}\right)
\lambda^{m-j}N^j = 0
\]
Such terms are the product of three terms 
\[
m(m-1)\cdots(m-j+1) \AND \lambda^m \AND \frac{1}{j!\lambda^j}N^j.
\]
The first term has polynomial growth to infinity dominated by $m^j$, the 
second term decreases to zero geometrically, and the third term is 
constant independent of $m$. The desired convergence to zero follows from 
\Ref{E:PG}.  \end{proof}

\begin{definition}  The $n\times n$ matrix $A$ has a {\em dominant\/} eigenvalue 
$\lambda_0>0$ if $\lambda_0$ is a simple eigenvalue and all other eigenvalues
$\lambda$ of $A$ satisfy $|\lambda|<\lambda_0$.
\end{definition}\index{eigenvalue!dominant}


\begin{theorem}  \label{T:Markovdom}
Let $P$ be a Markov matrix. Then $1$ is a dominant eigenvalue of $P$.
\end{theorem}\index{matrix!Markov}\index{matrix!Markov}

\begin{proof}  Recall from Chapter~\ref{chap:matrices}, Definition~\ref{D:Markov} 
that a Markov matrix is a square matrix $P$ whose entries are nonnegative, 
whose rows sum to $1$, and for which a power $P^k$ that has all positive 
entries.  To prove this theorem we must show that all eigenvalues $\lambda$ 
of $P$ satisfy $|\lambda|\leq 1$ and that $1$ is a simple eigenvalue of $P$.

Let $\lambda$ be an eigenvalue of $P$ and let $v=(v_1,\ldots,v_n)^t$ be an 
eigenvector corresponding to the eigenvalue $\lambda$.  We prove that 
$|\lambda|\leq 1$.  Choose $j$ so that $|v_j|\ge|v_i|$ for all $i$.  Since 
$Pv=\lambda v$, we can equate the $j^{th}$ coordinates of both sides of 
this equality, obtaining
\[
p_{j1}v_1 + \cdots + p_{jn}v_n = \lambda v_j.
\]
Therefore,
\[
 |\lambda| |v_j| = |p_{j1}v_1 + \cdots + p_{jn}v_n| \leq 
p_{j1}|v_1| + \cdots + p_{jn}|v_n|,
\]
since the $p_{ij}$ are nonnegative.  It follows that 
\[
|\lambda| |v_j| \leq (p_{j1}+\cdots+p_{jn})|v_j| =|v_j|,
\]
since $|v_i|\le|v_j|$ and rows of $P$ sum to $1$.  Since $|v_j|>0$, it 
follows that $\lambda\leq 1$.

Next we show that $1$ is a simple eigenvalue of $P$.  Recall, or just 
calculate directly, that the vector $(1,\ldots,1)^t$ is an eigenvector of $P$ 
with eigenvalue $1$.  Now let $v=(v_1,\ldots,v_n)^t$ be an eigenvector of $P$ 
with eigenvalue $1$.  Let $Q=P^k$ so that all entries of $Q$ are positive. 
Observe that $v$ is an eigenvector of $Q$ with eigenvalue $1$, and hence that 
all rows of $Q$ also sum to $1$.

To show that $1$ is a simple eigenvalue of $Q$, and therefore of $P$, we must 
show that all coordinates of $v$ are equal.  Using the previous estimates 
(with $\lambda=1$), we obtain 
\begin{equation}  \label{E:ineqM}
|v_j|= |q_{j1}v_1 + \cdots + q_{jn}v_n| \leq  q_{j1}|v_1| + \cdots + 
q_{jn}|v_n| \leq |v_j|.
\end{equation}
Hence 
\[
|q_{j1}v_1 + \cdots + q_{jn}v_n| =  q_{j1}|v_1| + \cdots + q_{jn}|v_n|.
\]
This equality is valid only if all of the $v_i$ are nonnegative or all are 
nonpositive.  Without loss of generality, we assume that all $v_i\geq 0$.
It follows from \Ref{E:ineqM} that 
\[
v_j= q_{j1}v_1 + \cdots + q_{jn}v_n.
\]
Since $q_{ji}>0$, this inequality can hold only if all of the $v_i$ are
equal.  \end{proof}


\begin{theorem} \label{T:convergetoeig}
(a)  Let $Q$ be an $n\times n$ matrix with dominant eigenvalue 
$\lambda>0$ and associated eigenvector $v$.  Let $v_0$ be any vector in 
$\R^n$.  Then
\[
\lim_{k\to\infty}\frac{1}{\lambda^k}Q^kv_0 = cv,
\]
for some scalar $c$.

(b)  Let $P$ be a Markov matrix and $v_0$ a nonzero vector in $\R^n$
with all entries nonnegative.  Then 
\[
\lim_{k\to\infty}(P^t)^kv_0 = V
\]
where $V$ is the eigenvector of $P^t$ with eigenvalue $1$ such that the 
sum of the entries in $V$ is equal to the sum of the entries in $v_0$.
\end{theorem} \index{matrix!Markov}\index{matrix!Markov}

\begin{proof}  (a) After a similarity transformation, if needed, 
we can assume that $Q$ is in 
Jordan normal form.  More precisely, we can assume that 
\[
\frac{1}{\lambda}Q = \mattwo{1}{0}{0}{A}
\]
where $A$ is an $(n-1)\times (n-1)$ matrix with all eigenvalues $\mu$
satisfying $|\mu|<1$.  Suppose $v_0=(c_0,w_0)\in\R\times\R^{n-1}$.  It 
follows from Theorem~\ref{T:convergeto0} that 
\[
\lim_{k\to\infty}\frac{1}{\lambda^k}Q^kv_0 = 
\lim_{k\to\infty}(\frac{1}{\lambda}Q)^kv_0 =
\lim_{k\to\infty}\mattwo{c_0}{0}{0}{A^kw_0} = c_0e_1.
\]
Since $e_1$ is the eigenvector of $Q$ with eigenvalue $\lambda$ Part (a) 
is proved.

(b)   Theorem~\ref{T:Markovdom} states that a Markov matrix has a dominant 
eigenvalue equal to $1$.  The Jordan normal form theorem implies that the 
eigenvalues of $P^t$ are equal to the eigenvalues of $P$ with the same 
algebraic and geometric multiplicities.  It follows that $1$ is also a 
dominant eigenvalue of $P^t$.  It follows from Part (a) that
\[
\lim_{k\to\infty}(P^t)^kv_0 = cV
\]
for some scalar $c$.  But Theorem~\ref{T:Markov} in 
Chapter~\ref{chap:matrices} implies that the sum of the
entries in $v_0$ equals the sum of the entries in $cV$ which, by assumption
equals the sum of the entries in $V$.  Thus, $c=1$.   \end{proof}





\EXER

\TEXER

\begin{exercise} \label{c10.6.1}
Let $A$ be an $n\times n$ matrix.   Suppose that 
\[
\lim_{k\to\infty} A^kv_0 = 0.
\]
for every vector $v_0\in\R^n$.  Then the eigenvalues $\lambda$ of $A$ all
satisfy $|\lambda|<1$.
\end{exercise}






\end{document}

%%% Local Variables:
%%% mode: latex
%%% TeX-master: "../linearAlgebra"
%%% End:
