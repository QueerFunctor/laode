\documentclass{ximera}

 

\usepackage{epsfig}

\graphicspath{
  {./}
  {figures/}
}

\usepackage{morewrites}
\makeatletter
\newcommand\subfile[1]{%
\renewcommand{\input}[1]{}%
\begingroup\skip@preamble\otherinput{#1}\endgroup\par\vspace{\topsep}
\let\input\otherinput}
\makeatother

\newcommand{\includeexercises}{\directlua{dofile("/home/jim/linearAlgebra/laode/exercises.lua")}}

%\newcounter{ccounter}
%\setcounter{ccounter}{1}
%\newcommand{\Chapter}[1]{\setcounter{chapter}{\arabic{ccounter}}\chapter{#1}\addtocounter{ccounter}{1}}

%\newcommand{\section}[1]{\section{#1}\setcounter{thm}{0}\setcounter{equation}{0}}

%\renewcommand{\theequation}{\arabic{chapter}.\arabic{section}.\arabic{equation}}
%\renewcommand{\thefigure}{\arabic{chapter}.\arabic{figure}}
%\renewcommand{\thetable}{\arabic{chapter}.\arabic{table}}

%\newcommand{\Sec}[2]{\section{#1}\markright{\arabic{ccounter}.\arabic{section}.#2}\setcounter{equation}{0}\setcounter{thm}{0}\setcounter{figure}{0}}

\newcommand{\Sec}[2]{\section{#1}}

\setcounter{secnumdepth}{2}
%\setcounter{secnumdepth}{1} 

%\newcounter{THM}
%\renewcommand{\theTHM}{\arabic{chapter}.\arabic{section}}

\newcommand{\trademark}{{R\!\!\!\!\!\bigcirc}}
%\newtheorem{exercise}{}

\newcommand{\dfield}{{\sf dfield9}}
\newcommand{\pplane}{{\sf pplane9}}

\newcommand{\EXER}{\section*{Exercises}}%\vspace*{0.2in}\hrule\small\setcounter{exercise}{0}}
\newcommand{\CEXER}{}%\vspace{0.08in}\begin{center}Computer Exercises\end{center}}
\newcommand{\TEXER}{} %\vspace{0.08in}\begin{center}Hand Exercises\end{center}}
\newcommand{\AEXER}{} %\vspace{0.08in}\begin{center}Hand Exercises\end{center}}

% BADBAD: \newcommand{\Bbb}{\bf}

\newcommand{\R}{\mbox{$\Bbb{R}$}}
\newcommand{\C}{\mbox{$\Bbb{C}$}}
\newcommand{\Z}{\mbox{$\Bbb{Z}$}}
\newcommand{\N}{\mbox{$\Bbb{N}$}}
\newcommand{\D}{\mbox{{\bf D}}}
\usepackage{amssymb}
%\newcommand{\qed}{\hfill\mbox{\raggedright$\square$} \vspace{1ex}}
%\newcommand{\proof}{\noindent {\bf Proof:} \hspace{0.1in}}

\newcommand{\setmin}{\;\mbox{--}\;}
\newcommand{\Matlab}{{M\small{AT\-LAB}} }
\newcommand{\Matlabp}{{M\small{AT\-LAB}}}
\newcommand{\computer}{\Matlab Instructions}
\newcommand{\half}{\mbox{$\frac{1}{2}$}}
\newcommand{\compose}{\raisebox{.15ex}{\mbox{{\scriptsize$\circ$}}}}
\newcommand{\AND}{\quad\mbox{and}\quad}
\newcommand{\vect}[2]{\left(\begin{array}{c} #1_1 \\ \vdots \\
 #1_{#2}\end{array}\right)}
\newcommand{\mattwo}[4]{\left(\begin{array}{rr} #1 & #2\\ #3
&#4\end{array}\right)}
\newcommand{\mattwoc}[4]{\left(\begin{array}{cc} #1 & #2\\ #3
&#4\end{array}\right)}
\newcommand{\vectwo}[2]{\left(\begin{array}{r} #1 \\ #2\end{array}\right)}
\newcommand{\vectwoc}[2]{\left(\begin{array}{c} #1 \\ #2\end{array}\right)}

\newcommand{\ignore}[1]{}


\newcommand{\inv}{^{-1}}
\newcommand{\CC}{{\cal C}}
\newcommand{\CCone}{\CC^1}
\newcommand{\Span}{{\rm span}}
\newcommand{\rank}{{\rm rank}}
\newcommand{\trace}{{\rm tr}}
\newcommand{\RE}{{\rm Re}}
\newcommand{\IM}{{\rm Im}}
\newcommand{\nulls}{{\rm null\;space}}

\newcommand{\dps}{\displaystyle}
\newcommand{\arraystart}{\renewcommand{\arraystretch}{1.8}}
\newcommand{\arrayfinish}{\renewcommand{\arraystretch}{1.2}}
\newcommand{\Start}[1]{\vspace{0.08in}\noindent {\bf Section~\ref{#1}}}
\newcommand{\exer}[1]{\noindent {\bf \ref{#1}}}
\newcommand{\ans}{}
\newcommand{\matthree}[9]{\left(\begin{array}{rrr} #1 & #2 & #3 \\ #4 & #5 & #6
\\ #7 & #8 & #9\end{array}\right)}
\newcommand{\cvectwo}[2]{\left(\begin{array}{c} #1 \\ #2\end{array}\right)}
\newcommand{\cmatthree}[9]{\left(\begin{array}{ccc} #1 & #2 & #3 \\ #4 & #5 &
#6 \\ #7 & #8 & #9\end{array}\right)}
\newcommand{\vecthree}[3]{\left(\begin{array}{r} #1 \\ #2 \\
#3\end{array}\right)}
\newcommand{\cvecthree}[3]{\left(\begin{array}{c} #1 \\ #2 \\
#3\end{array}\right)}
\newcommand{\cmattwo}[4]{\left(\begin{array}{cc} #1 & #2\\ #3
&#4\end{array}\right)}

\newcommand{\Matrix}[1]{\ensuremath{\left(\begin{array}{rrrrrrrrrrrrrrrrrr} #1 \end{array}\right)}}

\newcommand{\Matrixc}[1]{\ensuremath{\left(\begin{array}{cccccccccccc} #1 \end{array}\right)}}



\renewcommand{\labelenumi}{\theenumi)}
\newenvironment{enumeratea}%
{\begingroup
 \renewcommand{\theenumi}{\alph{enumi}}
 \renewcommand{\labelenumi}{(\theenumi)}
 \begin{enumerate}}
 {\end{enumerate}\endgroup}



\newcounter{help}
\renewcommand{\thehelp}{\thesection.\arabic{equation}}

%\newenvironment{equation*}%
%{\renewcommand\endequation{\eqno (\theequation)* $$}%
%   \begin{equation}}%
%   {\end{equation}\renewcommand\endequation{\eqno \@eqnnum
%$$\global\@ignoretrue}}

%\input{psfig.tex}

\author{Martin Golubitsky and Michael Dellnitz}

%\newenvironment{matlabEquation}%
%{\renewcommand\endequation{\eqno (\theequation*) $$}%
%   \begin{equation}}%
%   {\end{equation}\renewcommand\endequation{\eqno \@eqnnum
% $$\global\@ignoretrue}}

\newcommand{\soln}{\textbf{Solution:} }
\newcommand{\exercap}[1]{\centerline{Figure~\ref{#1}}}
\newcommand{\exercaptwo}[1]{\centerline{Figure~\ref{#1}a\hspace{2.1in}
Figure~\ref{#1}b}}
\newcommand{\exercapthree}[1]{\centerline{Figure~\ref{#1}a\hspace{1.2in}
Figure~\ref{#1}b\hspace{1.2in}Figure~\ref{#1}c}}
\newcommand{\para}{\hspace{0.4in}}

\renewenvironment{solution}{\suppress}{\endsuppress}

\ifxake
\newenvironment{matlabEquation}{\begin{equation}}{\end{equation}}
\else
\newenvironment{matlabEquation}%
{\let\oldtheequation\theequation\renewcommand{\theequation}{\oldtheequation*}\begin{equation}}%
  {\end{equation}\let\theequation\oldtheequation}
\fi

\makeatother


\title{Simple Complex Eigenvalues}

\begin{document}
\begin{abstract}
\end{abstract}
\maketitle

  \label{S:CSE}

Theorem~\ref{T:diagsimple} states that a matrix $A$ with real
unequal eigenvalues may be diagonalized. It follows that 
in an appropriately chosen basis (the basis of eigenvectors), 
matrix multiplication\index{matrix!multiplication} by $A$ acts 
as multiplication by these real eigenvalues.  Moreover, geometrically, 
multiplication by $A$ stretches or contracts vectors in 
eigendirections (depending on whether the eigenvalue is greater
than or less than $1$ in absolute value).

The purpose of this section is to show that a similar kind of diagonalization 
is possible when the matrix has distinct complex eigenvalues. See
Theorem~\ref{T:diagsimplecpx} and Theorem~\ref{T:Complexdiag}.  We show 
that multiplication by a matrix with complex eigenvalues corresponds
to multiplication by complex numbers.  We also show that multiplication by 
complex eigenvalues correspond geometrically to rotations\index{rotation} 
as well as expansions\index{expansion} and contractions\index{contraction}.

\subsubsection*{The Geometry of Complex Eigenvalues: Rotations and
Dilatations}

Real $2\times 2$ matrices are the smallest real matrices where complex 
eigenvalues can possibly occur.  In Chapter~\ref{Chap:Planar}, 
Theorem~\ref{T:putinform}(b) we discussed the classification of such matrices 
up to similarity.  Recall that if the eigenvalues of a $2\times 2$ matrix $A$ 
are $\sigma\pm i\tau$, 
then $A$ is similar to the matrix 
\begin{equation} \label{E:normalfm2}
\mattwo{\sigma}{-\tau}{\tau}{\sigma}.
\end{equation}
Moreover, the basis in which $A$ has the form \eqref{E:normalfm2} is found
as follows.  Let $v=w_1+iw_2$ be the eigenvector of $A$ corresponding to 
the eigenvalue $\sigma-i\tau$.  Then $\{w_1,w_2\}$ is the desired 
basis.\index{basis}

Geometrically, multiplication of vectors in $\R^2$ by \eqref{E:normalfm2} is 
the same as a rotation followed by a dilatation\index{dilatation}.  More 
specifically, let $r=\sqrt{\sigma^2+\tau^2}$.  So the point $(\sigma,\tau)$ 
lies on the circle of radius $r$ about the origin, and there is an angle 
$\theta$ such that $(\sigma,\tau)=(r\cos\theta,r\sin\theta)$.  Now we can 
rewrite \eqref{E:normalfm2} as
\[
\mattwo{\sigma}{-\tau}{\tau}{\sigma}
= r\mattwo{\cos\theta}{-\sin{\theta}}{\sin\theta}{\cos\theta}=rR_{\theta},
\]
where $R_\theta$ is rotation counterclockwise through angle $\theta$.  From 
this discussion we see that geometrically complex eigenvalues are associated 
with rotations followed either by stretching ($r>1$) or contracting ($r<1$).

As an example, consider the matrix
\begin{equation}  \label{E:exampA}
A = \mattwo{2}{1}{-2}{0}.
\end{equation}
The characteristic polynomial\index{characteristic polynomial}
of $A$ is $p_A(\lambda)=\lambda^2-2\lambda+2$.
Thus the eigenvalues of $A$ are $1\pm i$, and $\sigma=1$ and $\tau=1$ for
this example.  An eigenvector associated to the eigenvalue $1-i$ is 
$v=(1,-1-i)^t=(1,-1)^t+i(0,-1)^t$. Therefore, 
\[
B = S\inv A S = \mattwo{1}{-1}{1}{1} \quad\mbox{where}\quad 
S = \mattwo{1}{0}{-1}{-1},
\]
as can be checked by direct calculation.  Moreover, we can rewrite
\[
B = \sqrt{2}\mattwo{\frac{\sqrt{2}}{2}}{-\frac{\sqrt{2}}{2}}
{\frac{\sqrt{2}}{2}}{\frac{\sqrt{2}}{2}} =
\sqrt{2}R_{\frac{\pi}{4}}.
\]
So, in an appropriately chosen coordinate system, multiplication by $A$ rotates
vectors counterclockwise by $45^\circ$ and then expands the result by a 
factor of $\sqrt{2}$.  See Exercise~\ref{c10.4.rotate}.

\subsubsection*{The Algebra of Complex Eigenvalues: Complex Multiplication}
\index{eigenvalue!complex}

We have shown that the normal form\index{normal form} \eqref{E:normalfm2} can 
be interpreted geometrically as a rotation followed by a dilatation.  There 
is a second algebraic interpretation of \eqref{E:normalfm2}, and this 
interpretation is based on multiplication by complex numbers.

Let $\lambda=\sigma+i\tau$ be a complex number and consider the matrix 
associated with complex multiplication, that is, the linear mapping
\begin{equation}  \label{e:cplxmap}
z\mapsto \lambda z
\end{equation}
on the complex plane.  By identifying real and imaginary parts, we can 
rewrite \eqref{e:cplxmap} as a real $2\times 2$ matrix in the following way. 
Let $z=x + iy$.  Then
\[
\lambda z = (\sigma+i\tau)(x+iy) = (\sigma x - \tau y) + 
i(\tau x + \sigma y).
\]
Now identify $z$ with the vector $(x,y)$; that is, the vector whose first
component is the real part of $z$ and whose second component is the 
imaginary part.  Using this identification the complex number $\lambda z$
is identified with the vector $(\sigma x - \tau y,\tau x + \sigma y)$. 
So, in real coordinates and in matrix form, \eqref{e:cplxmap} becomes
\[
\vectwo{x}{y}\mapsto \vectwo{\sigma x - \tau y}{\tau x + \sigma y} =
\mattwo{\sigma}{-\tau}{\tau}{\sigma} \vectwo{x}{y}.
\]
That is, the matrix corresponding to multiplication of $z=x+iy$ by the 
complex number $\lambda=\sigma+i\tau$ is the one that multiplies the 
vector $(x,y)^t$ by the normal form matrix \eqref{E:normalfm2}.

\subsubsection*{Direct Agreement Between the Two Interpretations of 
\protect\eqref{E:normalfm2}}   

We have shown that matrix multiplication by \eqref{E:normalfm2} may be thought 
of either algebraically as multiplication by a complex number (an eigenvalue)
or geometrically as a rotation followed by a dilatation.  We now show how to
go directly from the algebraic interpretation to the geometric interpretation.

Euler's formula\index{Euler's formula} (Chapter~\ref{Chap:Planar}, 
\eqref{E:Euler}) states that
\[
e^{i\theta} = \cos\theta + i\sin\theta
\]
for any real number $\theta$. It follows that we can write a 
complex number $\lambda=\sigma+i\tau$ in polar form as
\[
\lambda = re^{i\theta}
\]
where $r^2=\lambda\overline{\lambda}=\sigma^2+\tau^2$, $\sigma=r\cos\theta$,
and $\tau=r\sin\theta$.  

Now consider multiplication by $\lambda$ in polar form.  Write 
$z=se^{i\varphi}$ in polar form, and compute
\[
\lambda z = re^{i\theta}se^{i\varphi} = rse^{i(\varphi+\theta)}.
\]
It follows from polar form that multiplication of $z$ by 
$\lambda=re^{i\theta}$  rotates 
$z$ through an angle $\theta$ and dilates the result by the factor $r$.
Thus Euler's formula directly relates the geometry of 
rotations\index{rotation} and 
dilatations\index{dilatation} with the 
algebra of multiplication by a complex number.


\subsection*{Normal Form Matrices with Distinct Complex Eigenvalues}
\index{normal form}\index{eigenvalue!complex!distinct}

In the first parts of this section we have discussed a geometric and 
an algebraic approach to matrix multiplication by $2\times 2$ matrices with 
complex eigenvalues.   We now turn our attention to classifying $n\times n$ 
matrices that have distinct eigenvalues, whether these eigenvalues are real 
or complex.  We will see that there are two ways to frame this classification 
--- one algebraic (using complex numbers) and one geometric (using rotations 
and dilatations).

\subsubsection*{Algebraic Normal Forms: The Complex Case}

Let $A$ be an $n\times n$ matrix with real entries and $n$ distinct 
eigenvalues $\lambda_1,\ldots,\lambda_n$.  Let $v_j$ be an eigenvector 
associated with the eigenvalue $\lambda_j$.  By methods that are 
entirely analogous to those in Section~\ref{S:RDM} we can diagonalize 
the matrix $A$ over the complex numbers.  The resulting theorem is 
analogous to Theorem~\ref{T:diagsimple}.  

More precisely, the $n\times n$ matrix $A$ is {\em complex diagonalizable\/} 
\index{complex diagonalizable}
if there is a complex $n\times n$ matrix $T$ such that
\[
T\inv AT = \left(\begin{array}{cccc} \lambda_1 & 0 & \cdots & 0 \\
	0 & \lambda_2 & \cdots & 0\\
	\vdots & \vdots & \ddots & \vdots \\
	0 & 0 & \cdots & \lambda_n \end{array} \right).
\]

\begin{theorem}  \label{T:diagsimplecpx}
Let $A$ be an $n\times n$ matrix with $n$ distinct eigenvalues.  Then $A$ 
is complex diagonalizable.
\end{theorem}  \index{eigenvalue}

The proof of Theorem~\ref{T:diagsimplecpx} follows from a theoretical 
development virtually word for word the same as that used to prove 
Theorem~\ref{T:diagsimple} in Section~\ref{S:RDM}.  Beginning from the 
theory that we have developed so far, the difficulty in proving this 
theorem lies in the need to base the theory of linear algebra on complex 
scalars rather than real scalars.  We will not pursue that development here.

As in Theorem~\ref{T:diagsimple}, the proof of Theorem~\ref{T:diagsimplecpx} 
shows that the complex matrix $T$ is the matrix whose columns are
the eigenvectors $v_j$ of $A$; that is,
\[
T = (v_1|\cdots|v_n).
\] 

Finally, we mention that the computation of inverse\index{inverse}
matrices with complex entries is the same as that for matrices with real 
entries.  That is, row reduction\index{row!reduction}
of the $n\times 2n$ matrix $(T|I_n)$ leads, when $T$ is
invertible, to the matrix $(I_n|T\inv)$.

\subsubsection*{Two Examples}

As a first example, consider the normal form $2\times 2$ matrix 
\eqref{E:normalfm2} that has eigenvalues $\lambda$ and $\overline{\lambda}$, 
where $\lambda=\sigma+i\tau$.   Let 
\[
B = \mattwo{\sigma}{-\tau}{\tau}{\sigma} \AND 
C = \mattwo{\lambda}{0}{0}{\overline{\lambda}}.
\]
Since the eigenvalues of $B$ and $C$ are identical, 
Theorem~\ref{T:diagsimplecpx} implies that there is a $2\times 2$ complex 
matrix $T$ such that
\begin{equation}  \label{E:BCsimilar}
C = T\inv B T.
\end{equation}
Moreover, the columns of $T$ are the complex eigenvectors $v_1$ and $v_2$
associated to the eigenvalues $\lambda$ and $\overline{\lambda}$.

It can be checked that the eigenvectors of $B$ are $v_1=(1,-i)^t$ and 
$v_2=(1,i)^t$.  On setting 
\[
T = \mattwo{1}{1}{-i}{i},
\]
it is a straightforward calculation to verify that $C=T\inv BT$.  


As a second example, consider the matrix 
\begin{matlabEquation}\label{compute-eigenvalues}
A = \left(\begin{array}{rrr}     4  &   2   &  1\\
     2  &  -3  &   1\\  1 &   -1  &  -3 \end{array} \right).
\end{matlabEquation}
Using \Matlab we find the eigenvalues of $A$ by typing {\tt eig(A)}. 
They are:
\begin{verbatim}
ans =
   4.6432          
  -3.3216 + 0.9014i
  -3.3216 - 0.9014i
\end{verbatim}
We can diagonalize (over the complex numbers) using \Matlab --- indeed 
\Matlab is programmed to do these calculations over the complex numbers. 
Type {\tt [T,D] = eig(A)} and obtain
\begin{verbatim}
T =
   0.9604    -0.1299 + 0.1587i  -0.1299 - 0.1587i
   0.2632     0.0147 - 0.5809i   0.0147 + 0.5809i
   0.0912     0.7788 - 0.1173i   0.7788 + 0.1173i

D =
   4.6432          0                  0          
        0    -3.3216 + 0.9014i        0          
        0          0            -3.3216 - 0.9014i
\end{verbatim}
This calculation can be checked by typing {\tt inv(T)*A*T} to see that 
the diagonal matrix\index{matrix!diagonal} {\tt D} appears.  
One can also check that the columns of 
{\tt T} are eigenvectors of $A$.

Note that the development here does not depend on the matrix $A$ having 
real entries.  Indeed, this diagonalization can be completed using 
$n\times n$ matrices with complex entries --- and \Matlab can handle such 
calculations.


\subsubsection*{Geometric Normal Forms: Block Diagonalization}
\index{normal form!geometric}

There is a second normal form theorem based on the geometry of rotations 
and dilatations for real $n\times n$ matrices $A$.  In this normal form 
we determine all matrices $A$ that have distinct eigenvalues --- up to 
similarity by real $n\times n$ matrices $S$.  The normal form results in 
matrices that are block diagonal with either $1\times 1$ blocks or 
$2\times 2$ blocks of the form \eqref{E:normalfm2} on the diagonal.

A real $n\times n$ matrix is in 
{\em real block diagonal form\/}
\index{matrix!block diagonal!real}
if it is a block diagonal matrix
\begin{equation} \label{e:blockform}
\left(\begin{array}{cccc} B_1 & 0 & \cdots & 0 \\
	0 & B_2 & \cdots & 0\\
	\vdots & \vdots & \ddots & \vdots \\
	0 & 0 & \cdots & B_m \end{array} \right),
\end{equation}
where each $B_j$ is either a $1\times 1$ block 
\[
B_j = \lambda_j
\]
for some real number $\lambda_j$ or a $2\times 2$ block
\begin{equation} \label{e:2x2block}
B_j =   \mattwo{\sigma_j}{-\tau_j}{\tau_j}{\sigma_j}
\end{equation}
where $\sigma_j$ and $\tau_j\neq 0$ are real numbers.  A
matrix is {\em real block diagonalizable\/} if it is 
similar\index{similar} to a real block diagonal form matrix.

Note that the real eigenvalues of a real block diagonal form matrix 
are just the real numbers $\lambda_j$ that occur in the $1\times 1$
blocks.  The complex eigenvalues are the eigenvalues of the
$2\times 2$ blocks $B_j$ and are $\sigma_j\pm i\tau_j$.


\begin{theorem} \label{T:Complexdiag}
Every $n\times n$ matrix $A$ with $n$ distinct eigenvalues is 
real block diagonalizable.
\end{theorem}

We need two preliminary results.  
\begin{lemma} \label{L:indepcomplx}
Let $\lambda_1,\ldots,\lambda_q$ be distinct (possible complex) eigenvalues 
of an $n\times n$ matrix $A$.  Let $v_j$ be a (possibly complex)
eigenvector associated with the eigenvalue $\lambda_j$.  Then
$v_1,\ldots,v_q$ are linearly independent in the sense that if
\begin{equation}  \label{E:indepcomplx}
\alpha_1v_1 + \cdots + \alpha_qv_q = 0
\end{equation}
for (possibly complex) scalars $\alpha_j$, then $\alpha_j=0$ for all $j$.
\end{lemma}

\begin{proof}  The proof is identical in spirit with the proof of 
Lemma~\ref{L:simpleeigen}.  Proceed by induction on $q$.  When $q=1$ the
lemma is trivially valid, as $\alpha v=0$ for $v\neq 0$ implies that
$\alpha=0$, even when $\alpha\in\C$ and $v\in\C^n$.  

By induction assume the lemma is valid for $q-1$.  Now apply $A$ to 
\eqref{E:indepcomplx} obtaining
\[
\alpha_1\lambda_1v_1 + \cdots + \alpha_q\lambda_qv_q = 0.
\]
Subtract this identity from $\lambda_q$ times \eqref{E:indepcomplx}, and obtain
\[
\alpha_1(\lambda_1-\lambda_q)v_1 + \cdots +
\alpha_{q-1}(\lambda_{q-1}-\lambda_q)v_{q-1} = 0.
\]
By induction 
\[
\alpha_j(\lambda_j-\lambda_q) = 0
\]
for $j=1,\ldots, q-1$.  Since the $\lambda_j$ are distinct it follows that 
$\alpha_j=0$ for $j=1,\ldots, q-1$.  Hence \eqref{E:indepcomplx} implies that 
$\alpha_qv_q = 0$; since $v_q\neq 0$, $\alpha_q=0$.  \end{proof}

\begin{lemma}  \label{L:rlcmplx}
Let $\mu_1,\ldots,\mu_k$ be distinct real eigenvalues of an $n\times n$ 
matrix $A$ and let $\nu_1,\overline{\nu}_1\ldots,\nu_\ell,\overline{\nu}_\ell$
be distinct complex conjugate eigenvalues of $A$.  Let $v_j\in\R^n$ be
eigenvectors associated to $\mu_j$ and let $w_j=w_j^r+iw_j^i$ be eigenvectors
associated with the eigenvalues $\nu_j$.  Then the $k+2\ell$ vectors 
\[
v_1,\ldots,v_k,w_1^r,w_1^i,\ldots,w_\ell^r,w_\ell^i
\]
in $\R^n$ are linearly independent.
\end{lemma}

\begin{proof}  Let $w=w^r+iw^i$ be a vector in $\C^n$ and let $\beta^r$ and
$\beta^i$ be real scalars.  Then 
\begin{equation}  \label{E:realcmplx}
\beta^rw^r + \beta^iw^i = \beta w + \overline{\beta} \overline{w},
\end{equation}
where $\beta = \frac{1}{2}(\beta^r-i\beta^i)$.  Identity \eqref{E:realcmplx} 
is verified by direct calculation.

Suppose now that 
\begin{equation}  \label{E:rlcplxlc} 
\alpha_1v_1+\cdots+\alpha_kv_k + \beta_1^rw_1^r+\beta_1^iw_1^i + \cdots +
\beta_\ell^rw_\ell^r+\beta_\ell^iw_\ell^i = 0
\end{equation}
for real scalars $\alpha_j$, $\beta_j^r$ and $\beta_j^i$.  Using 
\eqref{E:realcmplx} we can rewrite \eqref{E:rlcplxlc} as
\[
\alpha_1v_1+\cdots+\alpha_kv_k + \beta_1w_1+\overline{\beta}_1\overline{w}_1 
+ \cdots + \beta_\ell w_\ell+\overline{\beta}_\ell\overline{w}_\ell = 0,
\]
where $\beta_j = \frac{1}{2}(\beta_j^r-i\beta_j^i)$.  Since the eigenvalues 
\[
\mu_1,\ldots,\mu_k,\nu_1,\overline{\nu}_1\ldots,\nu_\ell,\overline{\nu}_\ell
\]
are all distinct, we may apply Lemma~\ref{L:indepcomplx} to conclude that 
$\alpha_j=0$ and $\beta_j=0$.  It follows that $\beta_j^r=0$ and
$\beta_j^i=0$, as well, thus proving linear independence.  \end{proof}. 


\begin{proof}[Theorem~\ref{T:Complexdiag}]   Let $\mu_j$ for 
$j=1,\ldots,k$ be the real eigenvalues of $A$ and let 
$\nu_j,\overline{\nu}_j$ for $j=1,\ldots,\ell$ be the complex eigenvalues of 
$A$. Since the eigenvalues are all distinct, it follows that $k+2\ell=n$.

Let $v_j$ and $w_j=w_j^r+iw_j^i$ be eigenvectors associated with the 
eigenvalues $\mu_j$ and $\overline{\nu}_j$.  It follows from 
Lemma~\ref{L:rlcmplx} that the $n$ real vectors
\begin{equation}  \label{e:complexeigen}
v_1,\ldots,v_k,w_1^r,w_1^i,\ldots,w_\ell^r,w_\ell^i
\end{equation}
are linearly independent and hence form a basis for $\R^n$.

We now show that $A$ is real block diagonalizable.  Let $S$ be the $n\times
n$ matrix whose columns are the vectors in \eqref{e:complexeigen}.  Since
these vectors are linearly independent, $S$ is invertible.  We claim that 
$S\inv AS$ is real block diagonal.  This statement is verified by direct
calculation.

First, note that $Se_j=v_j$ for $j=1,\ldots,k$ and compute
\[
(S\inv AS)e_j = S\inv Av_j = \mu_j S\inv v_j = \mu_j e_j.
\]
It follows that the first $k$ columns of $S\inv AS$ are zero except for the 
diagonal entries, and those diagonal entries equal $\mu_1,\ldots,\mu_k$.

Second, note that $Se_{k+1}=w_1^r$ and $Se_{k+2}=w_1^i$.  Write the complex 
eigenvalues as
\[
\nu_j = \sigma_j+i\tau_j.
\]
Since 
$Aw_1 = \overline{\nu}_1w_1$, it follows that
\begin{eqnarray*}
Aw_1^r+iAw_1^i & = & (\sigma_1-i\tau_1)(w_1^r+iw_1^i)\\
& = & (\sigma_1w_1^r+\tau_1w_1^i) + i(-\tau_1w_1^r +\sigma_1w_1^i).
\end{eqnarray*}
Equating real and imaginary parts leads to 
\begin{equation}  \label{e:complexsimple}
\begin{array}{ccc} Aw_1^r & = & \sigma_1w_1^r+\tau_1w_1^i\\
Aw_1^i & = &   -\tau_1w_1^r+\sigma_1w_1^i. \end{array}
\end{equation}
Using \eqref{e:complexsimple}, compute
\begin{align*}
(S\inv AS)e_{k+1} &= S\inv Aw_1^r = S\inv(\sigma_1w_1^r+\tau_1w_1^i)  \\
&= \sigma_1e_{k+1} + \tau_1e_{k+2}.
\end{align*}
Similarly, 
\begin{align*}
(S\inv AS)e_{k+2} &= S\inv Aw_1^i = S\inv(-\tau_1w_1^r+\sigma_1w_1^i) \\
&= -\tau_1e_{k+1} + \sigma_1e_{k+2}.
\end{align*}
Thus, the $k^{th}$ and $(k+1)^{st}$ columns of $S\inv AS$ have the desired
diagonal block in the $k^{th}$ and $(k+1)^{st}$ rows, and have all other 
entries equal to zero.

The same calculation is valid for the complex eigenvalues 
$\nu_2,\ldots,\nu_\ell$.  Thus, $S\inv AS$ is real block diagonal, as 
claimed.   \end{proof}


\subsubsection*{\Matlab Calculations of Real Block Diagonal Form}
\index{real block diagonal form!in \protect\Matlab}

Let $C$ be the $4\times 4$ matrix
\begin{matlabEquation}\label{four-by-four-eigenvalues}
C =\left(\begin{array}{rrrr}  1 & 0 & 2 & 3 \\ 2 & 1 & 4 & 6\\
-1 & -5 & 1 & 3 \\ 1 & 4 & 7 & 10 \end{array}\right).
\end{matlabEquation}
Using \Matlab enter $C$ by typing {\tt e13\_2\_14} and find the 
eigenvalues of $C$ by typing {\tt eig(C)}\index{\computer!eig} 
to obtain
\begin{verbatim}
ans =
   0.5855 + 0.8861i
   0.5855 - 0.8861i
  -0.6399          
  12.4690    
\end{verbatim}
We see that $C$ has two real and two complex conjugate eigenvalues.
To find the complex eigenvectors associated with these eigenvalues,
type 
\begin{verbatim}
[T,D] = eig(C)
\end{verbatim}
\Matlab responds with
\begin{verbatim}
T =
  -0.0787 + 0.0899i  -0.0787 - 0.0899i   0.0464             0.2209          
   0.0772 + 0.2476i   0.0772 - 0.2476i   0.0362             0.4803          
  -0.5558 - 0.5945i  -0.5558 + 0.5945i  -0.8421            -0.0066          
   0.3549 + 0.3607i   0.3549 - 0.3607i   0.5361             0.8488          
D =
   0.5855 + 0.8861i        0                  0                  0          
        0             0.5855 - 0.8861i        0                  0          
        0                  0            -0.6399                  0          
        0                  0                  0            12.4690          
\end{verbatim}
The $4\times 4$ matrix $T$ has the eigenvectors of $C$ as columns.
The $j^{th}$ column is the eigenvector associated with the $j^{th}$
diagonal entry in the diagonal matrix $D$. 

To find the matrix $S$ that puts $C$ in real block diagonal form, we
need to take the real and imaginary parts of the eigenvectors 
corresponding to the complex eigenvalues and the real eigenvectors 
corresponding to the real eigenvalues.  In this case, type
\index{\computer!real}\index{\computer!imag}
\begin{verbatim}
S = [real(T(:,1)) imag(T(:,1)) T(:,3) T(:,4)]
\end{verbatim}
to obtain 
\begin{verbatim}
S =
   -0.0787    0.0899    0.0464    0.2209
    0.0772    0.2476    0.0362    0.4803
   -0.5558   -0.5945   -0.8421   -0.0066
    0.3549    0.3607    0.5361    0.8488
\end{verbatim} 
Note that the $1^{st}$ and $2^{nd}$ columns are the real and 
imaginary parts of the complex eigenvector.  Check that 
{\tt inv(S)*C*S} is the matrix in complex diagonal form
\begin{verbatim}
ans = 
    0.5855    0.8861    0.0000    0.0000
   -0.8861    0.5855    0.0000   -0.0000
    0.0000    0.0000   -0.6399    0.0000
   -0.0000   -0.0000   -0.0000   12.4690
\end{verbatim}
as proved in Theorem~\ref{T:Complexdiag}.



\EXER

\TEXER


\begin{exercise}  \label{c10.4.3}
Consider the $2\times 2$ matrix 
\[
A = \mattwo{3}{1}{-2}{1},
\]
whose eigenvalues are $2\pm i$ and whose associated eigenvectors are:
\[
\vectwoc{1-i}{2i} \AND \vectwoc{1+i}{-2i}
\]
Find a complex $2\times 2$ matrix $T$ such that $C=T\inv AT$ is complex
diagonal and a real $2\times 2$ matrix $S$ so that $B=S\inv AS$ is in real
block diagonal form.
\end{exercise}

\begin{exercise}  \label{c10.4.4}
Let 
\[
A=\mattwo{2}{5}{-2}{0}.
\]
Find a complex $2\times 2$ matrix $T$ such that $T\inv AT$ is complex
diagonal and a real $2\times 2$ matrix $S$ so that $S\inv AS$ is in real
block diagonal form.
\end{exercise}



\CEXER

\begin{exercise} \label{c10.4.rotate}
Use {\sf map}\index{\computer!map} to verify that 
the normal form matrices \eqref{E:normalfm2} are 
just rotations followed by dilatations.  In particular, use {\sf map} to 
study the normal form matrix
\[
A = \mattwo{1}{-1}{1}{1}.
\]
Then compare your results with the similar matrix
\[
B = \mattwo{2}{1}{-2}{0}.
\]
\end{exercise}

\begin{exercise} \label{c10.4.6}
Consider the $2\times 2$ matrix
\[
A = \mattwo{-0.8318}{-1.9755}{0.9878}{1.1437}.
\]
\begin{itemize}
\item[(a)]  Use \Matlab to find the complex conjugate eigenvalues and 
eigenvectors of $A$.
\item[(b)]  Find the real block diagonal normal form of $A$ and describe
geometrically the motion of this normal form on the plane.
\item[(c)]  Using {\sf map} describe geometrically how $A$ maps vectors in 
the plane to vectors in the plane. 
\end{itemize}
\end{exercise}


\noindent In Exercises~\ref{c10.4.7a} -- \ref{c10.4.7d} find a square real 
matrix $S$ so that $S\inv AS$ is in real block diagonal form and a complex 
square matrix $T$ so that $T\inv AT$ is in complex diagonal form.
\begin{exercise} \label{c10.4.7a}
\begin{matlabEquation}\label{block-diagonal-exercise}
A = \left(\begin{array}{rrr}
    1 &     2 &     4 \\
    2 &    -4 &    -5\\
    1 &    10 &   -15
\end{array}\right).
\end{matlabEquation}
\end{exercise}
\begin{exercise} \label{c10.4.7b}
\begin{matlabEquation}\label{simple-eigenvalue-exercise}
A = \left(\begin{array}{rrrr}
  -15.1220 &  12.2195 &  13.6098 &  14.9268 \\
  -28.7805 &  21.8049 &  25.9024 &  28.7317 \\
   60.1951 & -44.9512 & -53.9756 & -60.6829 \\
  -44.5122 &  37.1220 &  43.5610 &  47.2927
\end{array}\right).
\end{matlabEquation}
\end{exercise} 
\begin{exercise} \label{c10.4.7c}
\begin{matlabEquation}\label{simple-eigenvalue-exercise-2}
A = \left(\begin{array}{rrrrrr}
    2.2125 &    5.1750 &    8.4250 &   15.0000 &   19.2500 &    0.5125 \\
   -1.9500 &   -3.9000 &   -6.5000 &   -7.4000 &  -12.0000 &   -2.9500\\
    2.2250 &    3.9500 &    6.0500 &    0.9000 &    1.5000 &    1.0250\\
   -0.2000 &   -0.4000 &         0 &    0.1000 &         0 &   -0.2000\\
   -0.7875 &   -0.8250 &   -1.5750 &    1.0000 &    2.2500 &    0.5125\\
    1.7875 &    2.8250 &    4.5750 &         0 &    4.7500 &    5.4875
\end{array}\right).
\end{matlabEquation}
\end{exercise}
\begin{exercise} \label{c10.4.7d}
\begin{matlabEquation}\label{simple-eigenvalue-exercise-3}
A = \left(\begin{array}{rrr}
   -12 &    15 &     0\\
     1 &     5 &     2\\
    -5 &     1 &     5
\end{array}\right).
\end{matlabEquation}
\end{exercise}



\end{document}

%%% Local Variables:
%%% mode: latex
%%% TeX-master: "../linearAlgebra"
%%% End:
