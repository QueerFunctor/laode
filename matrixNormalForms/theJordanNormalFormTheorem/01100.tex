\documentclass{ximera}
 

\usepackage{epsfig}

\graphicspath{
  {./}
  {figures/}
}

\usepackage{morewrites}
\makeatletter
\newcommand\subfile[1]{%
\renewcommand{\input}[1]{}%
\begingroup\skip@preamble\otherinput{#1}\endgroup\par\vspace{\topsep}
\let\input\otherinput}
\makeatother

\newcommand{\includeexercises}{\directlua{dofile("/home/jim/linearAlgebra/laode/exercises.lua")}}

%\newcounter{ccounter}
%\setcounter{ccounter}{1}
%\newcommand{\Chapter}[1]{\setcounter{chapter}{\arabic{ccounter}}\chapter{#1}\addtocounter{ccounter}{1}}

%\newcommand{\section}[1]{\section{#1}\setcounter{thm}{0}\setcounter{equation}{0}}

%\renewcommand{\theequation}{\arabic{chapter}.\arabic{section}.\arabic{equation}}
%\renewcommand{\thefigure}{\arabic{chapter}.\arabic{figure}}
%\renewcommand{\thetable}{\arabic{chapter}.\arabic{table}}

%\newcommand{\Sec}[2]{\section{#1}\markright{\arabic{ccounter}.\arabic{section}.#2}\setcounter{equation}{0}\setcounter{thm}{0}\setcounter{figure}{0}}

\newcommand{\Sec}[2]{\section{#1}}

\setcounter{secnumdepth}{2}
%\setcounter{secnumdepth}{1} 

%\newcounter{THM}
%\renewcommand{\theTHM}{\arabic{chapter}.\arabic{section}}

\newcommand{\trademark}{{R\!\!\!\!\!\bigcirc}}
%\newtheorem{exercise}{}

\newcommand{\dfield}{{\sf dfield9}}
\newcommand{\pplane}{{\sf pplane9}}

\newcommand{\EXER}{\section*{Exercises}}%\vspace*{0.2in}\hrule\small\setcounter{exercise}{0}}
\newcommand{\CEXER}{}%\vspace{0.08in}\begin{center}Computer Exercises\end{center}}
\newcommand{\TEXER}{} %\vspace{0.08in}\begin{center}Hand Exercises\end{center}}
\newcommand{\AEXER}{} %\vspace{0.08in}\begin{center}Hand Exercises\end{center}}

% BADBAD: \newcommand{\Bbb}{\bf}

\newcommand{\R}{\mbox{$\Bbb{R}$}}
\newcommand{\C}{\mbox{$\Bbb{C}$}}
\newcommand{\Z}{\mbox{$\Bbb{Z}$}}
\newcommand{\N}{\mbox{$\Bbb{N}$}}
\newcommand{\D}{\mbox{{\bf D}}}
\usepackage{amssymb}
%\newcommand{\qed}{\hfill\mbox{\raggedright$\square$} \vspace{1ex}}
%\newcommand{\proof}{\noindent {\bf Proof:} \hspace{0.1in}}

\newcommand{\setmin}{\;\mbox{--}\;}
\newcommand{\Matlab}{{M\small{AT\-LAB}} }
\newcommand{\Matlabp}{{M\small{AT\-LAB}}}
\newcommand{\computer}{\Matlab Instructions}
\newcommand{\half}{\mbox{$\frac{1}{2}$}}
\newcommand{\compose}{\raisebox{.15ex}{\mbox{{\scriptsize$\circ$}}}}
\newcommand{\AND}{\quad\mbox{and}\quad}
\newcommand{\vect}[2]{\left(\begin{array}{c} #1_1 \\ \vdots \\
 #1_{#2}\end{array}\right)}
\newcommand{\mattwo}[4]{\left(\begin{array}{rr} #1 & #2\\ #3
&#4\end{array}\right)}
\newcommand{\mattwoc}[4]{\left(\begin{array}{cc} #1 & #2\\ #3
&#4\end{array}\right)}
\newcommand{\vectwo}[2]{\left(\begin{array}{r} #1 \\ #2\end{array}\right)}
\newcommand{\vectwoc}[2]{\left(\begin{array}{c} #1 \\ #2\end{array}\right)}

\newcommand{\ignore}[1]{}


\newcommand{\inv}{^{-1}}
\newcommand{\CC}{{\cal C}}
\newcommand{\CCone}{\CC^1}
\newcommand{\Span}{{\rm span}}
\newcommand{\rank}{{\rm rank}}
\newcommand{\trace}{{\rm tr}}
\newcommand{\RE}{{\rm Re}}
\newcommand{\IM}{{\rm Im}}
\newcommand{\nulls}{{\rm null\;space}}

\newcommand{\dps}{\displaystyle}
\newcommand{\arraystart}{\renewcommand{\arraystretch}{1.8}}
\newcommand{\arrayfinish}{\renewcommand{\arraystretch}{1.2}}
\newcommand{\Start}[1]{\vspace{0.08in}\noindent {\bf Section~\ref{#1}}}
\newcommand{\exer}[1]{\noindent {\bf \ref{#1}}}
\newcommand{\ans}{}
\newcommand{\matthree}[9]{\left(\begin{array}{rrr} #1 & #2 & #3 \\ #4 & #5 & #6
\\ #7 & #8 & #9\end{array}\right)}
\newcommand{\cvectwo}[2]{\left(\begin{array}{c} #1 \\ #2\end{array}\right)}
\newcommand{\cmatthree}[9]{\left(\begin{array}{ccc} #1 & #2 & #3 \\ #4 & #5 &
#6 \\ #7 & #8 & #9\end{array}\right)}
\newcommand{\vecthree}[3]{\left(\begin{array}{r} #1 \\ #2 \\
#3\end{array}\right)}
\newcommand{\cvecthree}[3]{\left(\begin{array}{c} #1 \\ #2 \\
#3\end{array}\right)}
\newcommand{\cmattwo}[4]{\left(\begin{array}{cc} #1 & #2\\ #3
&#4\end{array}\right)}

\newcommand{\Matrix}[1]{\ensuremath{\left(\begin{array}{rrrrrrrrrrrrrrrrrr} #1 \end{array}\right)}}

\newcommand{\Matrixc}[1]{\ensuremath{\left(\begin{array}{cccccccccccc} #1 \end{array}\right)}}



\renewcommand{\labelenumi}{\theenumi)}
\newenvironment{enumeratea}%
{\begingroup
 \renewcommand{\theenumi}{\alph{enumi}}
 \renewcommand{\labelenumi}{(\theenumi)}
 \begin{enumerate}}
 {\end{enumerate}\endgroup}



\newcounter{help}
\renewcommand{\thehelp}{\thesection.\arabic{equation}}

%\newenvironment{equation*}%
%{\renewcommand\endequation{\eqno (\theequation)* $$}%
%   \begin{equation}}%
%   {\end{equation}\renewcommand\endequation{\eqno \@eqnnum
%$$\global\@ignoretrue}}

%\input{psfig.tex}

\author{Martin Golubitsky and Michael Dellnitz}

%\newenvironment{matlabEquation}%
%{\renewcommand\endequation{\eqno (\theequation*) $$}%
%   \begin{equation}}%
%   {\end{equation}\renewcommand\endequation{\eqno \@eqnnum
% $$\global\@ignoretrue}}

\newcommand{\soln}{\textbf{Solution:} }
\newcommand{\exercap}[1]{\centerline{Figure~\ref{#1}}}
\newcommand{\exercaptwo}[1]{\centerline{Figure~\ref{#1}a\hspace{2.1in}
Figure~\ref{#1}b}}
\newcommand{\exercapthree}[1]{\centerline{Figure~\ref{#1}a\hspace{1.2in}
Figure~\ref{#1}b\hspace{1.2in}Figure~\ref{#1}c}}
\newcommand{\para}{\hspace{0.4in}}

\renewenvironment{solution}{\suppress}{\endsuppress}

\ifxake
\newenvironment{matlabEquation}{\begin{equation}}{\end{equation}}
\else
\newenvironment{matlabEquation}%
{\let\oldtheequation\theequation\renewcommand{\theequation}{\oldtheequation*}\begin{equation}}%
  {\end{equation}\let\theequation\oldtheequation}
\fi

\makeatother

\begin{document}



\noindent In Exercises~\ref{E:jnfma} -- \ref{E:jnfme}, (a) determine the real
Jordan normal form for the given matrix $A$, and (b) find the matrix $S$ so 
that $S\inv AS$ is in real Jordan normal form.  
\begin{computerExercise}  \label{E:jnfma}
\begin{matlabEquation}\label{jordan-form-exercise}
A = \left(\begin{array}{rrrr} -3 & -4 & -2 & 0\\
-9 & -39 & -16 & -7\\ 18 & 64 & 27 & 10 \\ 15 & 86 & 34 & 18
\end{array}\right). 
\end{matlabEquation}

\begin{solution}

(a) \ans The Jordan normal form of $A$ is
\[
J = \left(\begin{array}{rrrr}
3 & 0 & 0 & 0 \\
0 & 1 & 0 & 0 \\
0 & 0 & 0 & 0 \\
0 & 0 & 0 & -1 \end{array}\right).
\]

\soln
To find the Jordan normal form of matrix $A$, type {\tt eig(A)} to find
the eigenvalues.  Matrix $A$ has four distinct eigenvalues, so the
Jordan normal form is the diagonal matrix with these eigenvalues along
its diagonal.  

(b)\ans   The diagonalizing matrix is
\begin{verbatim}
S =
   -0.1387   -0.1543   -0.0000   -0.5774
    0.1387   -0.3086   -0.4082    0.0000
    0.1387    0.9258    0.8165    0.5774
   -0.9707   -0.1543    0.4082   -0.5774
\end{verbatim}

\soln The columns of matrix $S$ consist of the eigenvectors of $A$.


\end{solution}
\end{computerExercise}
\begin{computerExercise}  \label{E:jnfmb}
\begin{matlabEquation}\label{jordan-form-exercise-2}
A =\left(\begin{array}{rrrr} 9 & 45 & 18 & 8\\
0 & -4 & -1 & -1\\ -16 & -69 & -29 & -12 \\ 25 & 123 & 49 & 23
\end{array}\right). 
\end{matlabEquation}

\begin{solution}

(a) \ans The Jordan normal form of $A$ is
\[
J = \left(\begin{array}{rrrr}
2 & 0 & 0 & 0 \\
0 & -1 & 1 & 0 \\
0 & 0 & -1 & 1 \\
0 & 0 & 0 & -1 \end{array}\right).
\]

\soln Type {\tt eig(A)} to find the eigenvalues of matrix $A$.  Not all
eigenvalues are simple, so type {\tt null(A - lambda*eye(4))} for each
eigenvalue $\lambda$ to find the number of linearly independent eigenvectors
associated to it.  Matrix $A$ has a simple eigenvalue at $2$, and
an eigenvalue at $-1$ with algebraic multiplicity 3 and one linearly
independent eigenvector.

(b) \ans   The diagonalizing matrix is
\begin{verbatim}
S =
   -0.1387   -0.5902   -1.0165   -0.9837
    0.1387    0.0000   -0.5902    0.1640
    0.1387    0.5902    2.1970    0.0656
   -0.9707   -0.5902   -0.4263    0.0328
\end{verbatim}

\soln The first column of $S$ is $v_1$, the eigenvector associated with
eigenvalue $2$.  To find the other columns, note that the nullity of
$(A + I_4)$ is 1, the nullity of $(A + I_4)^2$ is 2, and the nullity of
$(A + I_4)^3$ is 3.  Then, select one vector from $\null((A + I_4)^3)$
and label it $v_{23}$.  Set $v_{22} = (A + I_4)v_{23}$ and
$v_{21} = (A + I_4)^2v_{23}$.  Then, $S = (v_1|v_{21}|v_{22}|v_{23})$.

\end{solution}
\end{computerExercise}
\begin{computerExercise}  \label{E:jnfmc}
\begin{matlabEquation}\label{jordan-form-exercise-3}
A = \left(\begin{array}{rrrr} -5 & -13 & 17 & 42\\
-10 & -57 & 66 & 187\\ -4 & -23 & 26 & 77 \\ -1 & -9 & 9 & 32
\end{array}\right).  
\end{matlabEquation}

\begin{solution}

(a) \ans The Jordan normal form of $A$ is
\[
J = \left(\begin{array}{rrrr}
 0 & 2 &  0 &  0 \\
-2 & 0 &  0 &  0 \\
 0 & 0 & -2 &  1 \\
 0 & 0 & -1 & -2 \end{array}\right).
\]

\soln Type {\tt eig(A)} to find that matrix $A$ has distinct eigenvalues
at $\pm 2i$ and $2 \pm i$.

(b) \ans   The diagonalizing matrix is
\begin{verbatim}
T =
   -0.2118   -0.0456    0.2211    0.0060
   -0.8548   -0.2507    0.8762    0.1803
   -0.3555   -0.0988    0.3529    0.0669
   -0.1437   -0.0531    0.1440    0.0344
\end{verbatim}

\soln The columns of matrix $S$ consist of the eigenvectors of $A$.


\end{solution}
\end{computerExercise}
\begin{computerExercise}  \label{E:jnfmd}
\begin{matlabEquation}\label{jordan-form-exercise-4}
A = \left(\begin{array}{rrrr} 1 & 0 & -9 & 18 \\
12 & -7 & -26 & 77\\ 5 & -2 & -13 & 32 \\ 2 & -1 & -4 & 11
\end{array}\right). 
\end{matlabEquation}

\begin{solution}

(a) \ans The Jordan normal form of $A$ is
\[
J = \left(\begin{array}{rrrr}
-2 & 1 & 0 & 0 \\
0 & -2 & 0 & 0 \\
0 & 0 & -2 & 1 \\
0 & 0 & 0 & -2 \end{array}\right).
\]

\soln
Type {\tt eig(A)} to find that $-2$ is the only eigenvalues of $A$.  Then 
type {\tt null(A + 2*eye(4))} to find the number of linearly
independent eigenvectors associated to eigenvalue $\lambda = -2$.  
The eigenvalue has algebraic multiplicity $4$ and geometric multiplicity
$2$.  We then find that the nullity of $(A + 2I_4)^2$ is $4$. 
Therefore, all generalized eigenvectors $v$ of $D$ are in the null space
of $A + 2I_4$.

(b) \ans   The diagonalizing matrix is
\begin{verbatim}
S =
     3     1     0     0
    12     0    -5     1
     5     0    -2     0
     2     0    -1     0
\end{verbatim}

\soln To find $S$, find that $A$ has two linearly independent
eigenvectors associated to $-2$, and that the null space
of $(A + 2I_4)$ is $\R^4$.  Then, select two vectors in $\R^4$, in this
case $v_{12} = (1,0,0,0)^t$ and $v_{22} = (0,1,0,0)^t$, and set
$v_{11} = (A + 2I_4)v_{12}$ and $v_{21} = (A + 2I_4)v_{22}$.  Then,
$S = (v_{11}|v_{12}|v_{21}|v_{22})$.

\end{solution}
\end{computerExercise}
\begin{computerExercise} \label{E:jnfme}
\begin{matlabEquation}\label{jordan-form-exercise-5}
A = \left(\begin{array}{rrrr} 
    -1  &  -1  &   1   &  0\\
    -3  &   1  &   1   &  0\\
    -3  &   2  &  -1   &  1\\
    -3  &   2  &   0   &  0
 \end{array}\right). 
\end{matlabEquation}

\begin{solution}

(a) \ans The Jordan normal form of $A$ is
\[
J = \left(\begin{array}{rrrr}
-1 &  1 &  0 & 0 \\
0  & -1 &  1 & 0 \\
0  &  0 & -1 & 0 \\
0  &  0 &  0 & 1 \end{array}\right).
\]

(b) \ans  The diagonalizing matrix is
\begin{verbatim}
S =
    -1     1     0     0
    -1     1     0     1
    -1     0     1     1
    -1     0     0     1
\end{verbatim}

\soln To find $S$, use \Matlab to see that $A$ has two eigenvalues: $-1$ of
algebraic multiplicity three and $2$ of multiplicity one.  The geometric 
multiplicity of $-1$ is one.  Just type {\tt null(A+eye(4))} to see that 
$(1,1,1,1)$ is the only eigenvector corresponding to this eigenvalue.  Choose 
{\tt v3} in the null space of $(A+I_4)^3$ but not in the null space of 
$(A+I_4)^2$.  For example let {\tt v3 = [0 0 1 0]'}.  Then set 
{\tt v2 = (A+eye(4))*v3} and {\tt v1 = (A+eye(4))*v2}.  Finally, set 
{\tt v4 = null(A-2*eye(4))} and {\tt S = [v1 v2 v3 v4]}.

\end{solution}
\end{computerExercise}
\end{document}
