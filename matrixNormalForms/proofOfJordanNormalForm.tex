\documentclass{ximera}

 

\usepackage{epsfig}

\graphicspath{
  {./}
  {figures/}
}

\usepackage{morewrites}
\makeatletter
\newcommand\subfile[1]{%
\renewcommand{\input}[1]{}%
\begingroup\skip@preamble\otherinput{#1}\endgroup\par\vspace{\topsep}
\let\input\otherinput}
\makeatother

\newcommand{\includeexercises}{\directlua{dofile("/home/jim/linearAlgebra/laode/exercises.lua")}}

%\newcounter{ccounter}
%\setcounter{ccounter}{1}
%\newcommand{\Chapter}[1]{\setcounter{chapter}{\arabic{ccounter}}\chapter{#1}\addtocounter{ccounter}{1}}

%\newcommand{\section}[1]{\section{#1}\setcounter{thm}{0}\setcounter{equation}{0}}

%\renewcommand{\theequation}{\arabic{chapter}.\arabic{section}.\arabic{equation}}
%\renewcommand{\thefigure}{\arabic{chapter}.\arabic{figure}}
%\renewcommand{\thetable}{\arabic{chapter}.\arabic{table}}

%\newcommand{\Sec}[2]{\section{#1}\markright{\arabic{ccounter}.\arabic{section}.#2}\setcounter{equation}{0}\setcounter{thm}{0}\setcounter{figure}{0}}

\newcommand{\Sec}[2]{\section{#1}}

\setcounter{secnumdepth}{2}
%\setcounter{secnumdepth}{1} 

%\newcounter{THM}
%\renewcommand{\theTHM}{\arabic{chapter}.\arabic{section}}

\newcommand{\trademark}{{R\!\!\!\!\!\bigcirc}}
%\newtheorem{exercise}{}

\newcommand{\dfield}{{\sf dfield9}}
\newcommand{\pplane}{{\sf pplane9}}

\newcommand{\EXER}{\section*{Exercises}}%\vspace*{0.2in}\hrule\small\setcounter{exercise}{0}}
\newcommand{\CEXER}{}%\vspace{0.08in}\begin{center}Computer Exercises\end{center}}
\newcommand{\TEXER}{} %\vspace{0.08in}\begin{center}Hand Exercises\end{center}}
\newcommand{\AEXER}{} %\vspace{0.08in}\begin{center}Hand Exercises\end{center}}

% BADBAD: \newcommand{\Bbb}{\bf}

\newcommand{\R}{\mbox{$\Bbb{R}$}}
\newcommand{\C}{\mbox{$\Bbb{C}$}}
\newcommand{\Z}{\mbox{$\Bbb{Z}$}}
\newcommand{\N}{\mbox{$\Bbb{N}$}}
\newcommand{\D}{\mbox{{\bf D}}}
\usepackage{amssymb}
%\newcommand{\qed}{\hfill\mbox{\raggedright$\square$} \vspace{1ex}}
%\newcommand{\proof}{\noindent {\bf Proof:} \hspace{0.1in}}

\newcommand{\setmin}{\;\mbox{--}\;}
\newcommand{\Matlab}{{M\small{AT\-LAB}} }
\newcommand{\Matlabp}{{M\small{AT\-LAB}}}
\newcommand{\computer}{\Matlab Instructions}
\newcommand{\half}{\mbox{$\frac{1}{2}$}}
\newcommand{\compose}{\raisebox{.15ex}{\mbox{{\scriptsize$\circ$}}}}
\newcommand{\AND}{\quad\mbox{and}\quad}
\newcommand{\vect}[2]{\left(\begin{array}{c} #1_1 \\ \vdots \\
 #1_{#2}\end{array}\right)}
\newcommand{\mattwo}[4]{\left(\begin{array}{rr} #1 & #2\\ #3
&#4\end{array}\right)}
\newcommand{\mattwoc}[4]{\left(\begin{array}{cc} #1 & #2\\ #3
&#4\end{array}\right)}
\newcommand{\vectwo}[2]{\left(\begin{array}{r} #1 \\ #2\end{array}\right)}
\newcommand{\vectwoc}[2]{\left(\begin{array}{c} #1 \\ #2\end{array}\right)}

\newcommand{\ignore}[1]{}


\newcommand{\inv}{^{-1}}
\newcommand{\CC}{{\cal C}}
\newcommand{\CCone}{\CC^1}
\newcommand{\Span}{{\rm span}}
\newcommand{\rank}{{\rm rank}}
\newcommand{\trace}{{\rm tr}}
\newcommand{\RE}{{\rm Re}}
\newcommand{\IM}{{\rm Im}}
\newcommand{\nulls}{{\rm null\;space}}

\newcommand{\dps}{\displaystyle}
\newcommand{\arraystart}{\renewcommand{\arraystretch}{1.8}}
\newcommand{\arrayfinish}{\renewcommand{\arraystretch}{1.2}}
\newcommand{\Start}[1]{\vspace{0.08in}\noindent {\bf Section~\ref{#1}}}
\newcommand{\exer}[1]{\noindent {\bf \ref{#1}}}
\newcommand{\ans}{}
\newcommand{\matthree}[9]{\left(\begin{array}{rrr} #1 & #2 & #3 \\ #4 & #5 & #6
\\ #7 & #8 & #9\end{array}\right)}
\newcommand{\cvectwo}[2]{\left(\begin{array}{c} #1 \\ #2\end{array}\right)}
\newcommand{\cmatthree}[9]{\left(\begin{array}{ccc} #1 & #2 & #3 \\ #4 & #5 &
#6 \\ #7 & #8 & #9\end{array}\right)}
\newcommand{\vecthree}[3]{\left(\begin{array}{r} #1 \\ #2 \\
#3\end{array}\right)}
\newcommand{\cvecthree}[3]{\left(\begin{array}{c} #1 \\ #2 \\
#3\end{array}\right)}
\newcommand{\cmattwo}[4]{\left(\begin{array}{cc} #1 & #2\\ #3
&#4\end{array}\right)}

\newcommand{\Matrix}[1]{\ensuremath{\left(\begin{array}{rrrrrrrrrrrrrrrrrr} #1 \end{array}\right)}}

\newcommand{\Matrixc}[1]{\ensuremath{\left(\begin{array}{cccccccccccc} #1 \end{array}\right)}}



\renewcommand{\labelenumi}{\theenumi)}
\newenvironment{enumeratea}%
{\begingroup
 \renewcommand{\theenumi}{\alph{enumi}}
 \renewcommand{\labelenumi}{(\theenumi)}
 \begin{enumerate}}
 {\end{enumerate}\endgroup}



\newcounter{help}
\renewcommand{\thehelp}{\thesection.\arabic{equation}}

%\newenvironment{equation*}%
%{\renewcommand\endequation{\eqno (\theequation)* $$}%
%   \begin{equation}}%
%   {\end{equation}\renewcommand\endequation{\eqno \@eqnnum
%$$\global\@ignoretrue}}

%\input{psfig.tex}

\author{Martin Golubitsky and Michael Dellnitz}

%\newenvironment{matlabEquation}%
%{\renewcommand\endequation{\eqno (\theequation*) $$}%
%   \begin{equation}}%
%   {\end{equation}\renewcommand\endequation{\eqno \@eqnnum
% $$\global\@ignoretrue}}

\newcommand{\soln}{\textbf{Solution:} }
\newcommand{\exercap}[1]{\centerline{Figure~\ref{#1}}}
\newcommand{\exercaptwo}[1]{\centerline{Figure~\ref{#1}a\hspace{2.1in}
Figure~\ref{#1}b}}
\newcommand{\exercapthree}[1]{\centerline{Figure~\ref{#1}a\hspace{1.2in}
Figure~\ref{#1}b\hspace{1.2in}Figure~\ref{#1}c}}
\newcommand{\para}{\hspace{0.4in}}

\renewenvironment{solution}{\suppress}{\endsuppress}

\ifxake
\newenvironment{matlabEquation}{\begin{equation}}{\end{equation}}
\else
\newenvironment{matlabEquation}%
{\let\oldtheequation\theequation\renewcommand{\theequation}{\oldtheequation*}\begin{equation}}%
  {\end{equation}\let\theequation\oldtheequation}
\fi

\makeatother


\title{Proof of Jordan Normal Form}

\begin{document}
\begin{abstract}
\end{abstract}
\maketitle


\label{S:Jordan} \index{Jordan normal form}

We prove the Jordan normal form theorem under the assumption that the 
eigenvalues of $A$ are all real.  The proof for matrices having both real and 
complex eigenvalues proceeds along similar lines.

Let $A$ be an $n\times n$ matrix, let $\lambda_1,\ldots,\lambda_s$ be the
distinct eigenvalues of $A$, and let $A_j = A-\lambda_jI_n$.

\begin{lemma}  \label{L:commute}
The linear mappings $A_i$ and $A_j$ commute.
\end{lemma}

\begin{proof} Just compute
\[
A_iA_j = (A-\lambda_iI_n)(A-\lambda_jI_n)= A^2-\lambda_iA-\lambda_jA+
\lambda_i\lambda_jI_n,
\]
and
\[
A_jA_i = (A-\lambda_jI_n)(A-\lambda_iI_n)= A^2-\lambda_jA-\lambda_iA+
\lambda_j\lambda_iI_n.
\]
So $A_iA_j=A_jA_i$, as claimed.  \end{proof}

Let $V_j$ be the generalized eigenspace corresponding to eigenvalue 
$\lambda_j$. 

\begin{lemma}  \label{L:Ajinvertible}
$A_i:V_j\to V_j$ is invertible when $i\neq j$.
\end{lemma}

\begin{proof}  Recall from Lemma~\ref{L:Jordan} that $V_j=\nulls(A_j^k)$ for some 
$k\ge 1$.  Suppose that $v\in V_j$.  We first verify that $A_iv$ is also in 
$V_j$.  Using Lemma~\ref{L:commute}, just compute 
\[
A_j^kA_iv = A_iA_j^kv = A_i0 = 0.
\]
Therefore, $A_iv\in\nulls(A_j^k)=V_j$.
 
Let $B$ be the linear mapping $A_i|V_j$.  It follows from
Chapter~\ref{C:LMCC}, Theorem~\ref{T:nsr} that
\[
{\rm nullity}(B) +\dim{\rm range}(B) = \dim(V_j).
\]
Now $w\in\nulls(B)$ if $w\in V_j$ and $A_iw=0$.  Since
$A_iw = (A-\lambda_iI_n)w = 0$, it follows that $Aw = \lambda_iw$.  Hence 
\[
A_jw = (A-\lambda_jI_n)w = (\lambda_i-\lambda_j)w
\]
and
\[
A_j^kw = (\lambda_i-\lambda_j)^kw.
\]
Since $\lambda_i\neq\lambda_j$, it follows that $A_j^kw=0$ only when $w=0$.
Hence the nullity of $B$ is zero.  We conclude that
\[
\dim{\rm range}(B) = \dim(V_j).
\]
Thus, $B$ is invertible, since the domain and range of $B$ are the same
space.  \end{proof}

\begin{lemma}  \label{L:independentVj}
Nonzero vectors taken from different generalized eigenspaces $V_j$ are 
linearly independent.  More precisely, if $w_j\in V_j$ and 
\[
w = w_1 + \cdots + w_s = 0,
\]
then $w_j=0$.  
\end{lemma}

\begin{proof} Let $V_j=\nulls(A_j^{k_j})$ for some integer $k_j$.  Let
$C=A_2^{k_2}\compose\cdots\compose A_s^{k_s}$. Then 
\[
0 = Cw = Cw_1,
\]
since $A_j^{k_j}w_j=0$ for $j=2,\ldots,s$.   But Lemma~\ref{L:Ajinvertible} 
implies that $C|V_1$ is invertible.  Therefore, $w_1=0$.  Similarly, all of 
the remaining $w_j$ have to vanish.  \end{proof}

\begin{lemma}  \label{L:spanVj}
Every vector in $\R^n$ is a linear combination of vectors in the generalized 
eigenspaces $V_j$.
\end{lemma}

\begin{proof}  Let $W$ be the subspace
of $\R^n$ consisting of all vectors of the form $z_1+\cdots +z_s$ where 
$z_j\in V_j$.  We need to verify that $W=\R^n$.  Suppose that $W$ is a 
proper subspace.  Then choose a basis $w_1,\ldots,w_t$ of $W$ and extend
this set to a basis ${\cal W}$ of $\R^n$.  In this basis the matrix
$[A]_{\cal W}$ has block form, that is,
\[
[A]_{\cal W} = \mattwo{A_{11}}{A_{12}}{0}{A_{22}},
\]
where $A_{22}$ is an $(n-t)\times(n-t)$ matrix.  The eigenvalues of $A_{22}$ 
are eigenvalues of $A$.  Since all of the distinct eigenvalues and 
eigenvectors of $A$ are accounted for in $W$ (that is, in $A_{11}$), we have 
a contradiction.  So $W=\R^n$, as claimed.  \end{proof}

\begin{lemma}  \label{L:basisunion}
Let ${\cal V}_j$ be a basis for the generalized eigenspaces $V_j$ and let 
${\cal V}$ be the union of the sets ${\cal V}_j$.  Then ${\cal V}$ is a basis
for $\R^n$.
\end{lemma}

\begin{proof}  We first show that the vectors in ${\cal V}$ span $\R^n$.  It follows 
from Lemma~\ref{L:spanVj} that every vector in $\R^n$ is a linear combination
of vectors in $V_j$.  But each vector in $V_j$ is a linear combination of
vectors in ${\cal V}_j$.  Hence, the vectors in ${\cal V}$ span $\R^n$.

Second, we show that the vectors in ${\cal V}$ are linearly independent. 
Suppose that a linear combination of vectors in ${\cal V}$ sums to zero.  
We can write this sum as 
\[
w_1 + \cdots + w_s = 0,
\]
where $w_j$ is the linear combination of vectors in ${\cal V}_j$. 
Lemma~\ref{L:independentVj} implies that each $w_j=0$.  Since ${\cal V}_j$ is
a basis for $V_j$, it follows that the coefficients in the linear
combinations $w_j$ must all be zero.  Hence, the vectors in ${\cal V}$ are 
linearly independent.

Finally, it follows from Theorem~\ref{basis=span+indep} of 
Chapter~\ref{C:vectorspaces} that ${\cal V}$ is a basis.  \end{proof}

\begin{lemma} \label{L:diagVj}
In the basis ${\cal V}$ of $\R^n$ guaranteed by Lemma~\ref{L:basisunion}, the 
matrix $[A]_{\cal V}$ is block diagonal, that is,
\[
[A]_{\cal V} = \left(\begin{array}{ccc} A_{11} & 0 & 0  \\ 0 & \ddots & 0 \\
0 & 0 & A_{ss} \end{array}\right),
\]
where all of the eigenvalues of $A_{jj}$ equal $\lambda_j$.
\end{lemma}

\begin{proof}  It follows from Lemma~\ref{L:commute} that $A:V_j\to V_j$.  Suppose
that $v_j\in {\cal V}_j$.   Then $Av_j$ is in $V_j$ and $Av_j$ is a linear
combination of vectors in ${\cal V}_j$.   The block diagonalization of 
$[A]_{\cal V}$ follows.  Since $V_j=\nulls(A_j^{k_j})$, it follows that all
eigenvalues of $A_{jj}$ equal $\lambda_j$.   \end{proof}

Lemma~\ref{L:diagVj} implies that to prove the Jordan normal form theorem, 
we must find a basis in which the matrix $A_{jj}$ is in Jordan normal form.  
So, without loss of generality, we may assume that all eigenvalues of $A$ 
equal $\lambda_0$, and then find a basis in which $A$ is in Jordan normal 
form.  Moreover, we can replace $A$ by the matrix $A-\lambda_0I_n$, a
matrix all of whose eigenvalues are zero.  So, without loss of generality, we 
assume that $A$ is an $n\times n$ matrix all of whose eigenvalues are zero.  
We now sketch the remainder of the proof of Theorem~\ref{T:Jordan}.

Let $k$ be the smallest integer such that $\R^n = \nulls(A^k)$ and let 
\[
s=\dim\nulls(A^k)-\dim\nulls(A^{k-1})>0.
\]
Let $z_1,\ldots,z_{n-s}$ be 
a basis for $\nulls(A^{k-1})$ and extend this set to a basis for 
$\nulls(A^k)$ by adjoining the linearly independent vectors $w_1,\ldots,w_s$.  
Let 
\[
W_k=\Span\{w_1,\ldots,w_s\}.
\]
It follows that $W_k\cap\nulls(A^{k-1})=\{0\}$.  

We claim that the $ks$ vectors ${\cal W}=\{w_{j\ell}=A^\ell(w_j)\}$ where 
$0\le\ell\le {k-1}$ and $1\le j\le s$ are linearly independent.  We can write 
any linear combination of the vectors in ${\cal W}$ as $y_k+\cdots+y_1$, 
where $y_j\in A^{k-j}(W_k)$.  Suppose that 
\[
y_k+\cdots+y_1=0.
\]
Then $A^{k-1}(y_k+\cdots+y_1)= A^{k-1}y_k=0$.  Therefore, $y_k$ is in $W_k$ 
and in $\nulls(A^{k-1})$.  Hence, $y_k=0$.  Similarly, 
$A^{k-2}(y_{k-1}+\cdots+y_1)= A^{k-2}y_{k-1}=0$.  But $y_{k-1}=A\hat{y}_k$ 
where $\hat{y}_k\in W_k$ and $\hat{y}_k\in\nulls(A^{k-1})$.  Hence, 
$\hat{y}_k=0$ and $y_{k-1}=0$.  Similarly, all of the $y_j=0$.  It follows 
from $y_j=0$ that a linear combination of the vectors 
$A^{k-j}(w_1),\ldots,A^{k-j}(w_s)$ is zero; that is
\[
0 = \beta_1A^{k-j}(w_1) + \cdots + \beta_sA^{k-j}(w_s) =
A^{k-j}(\beta_1w_1+\cdots+\beta_sw_s).
\]
Applying  $A^{j-1}$ to this expression, we see that 
\[
\beta_1w_1+\cdots+\beta_sw_s
\]
is in $W_k$ and in the $\nulls(A^{k-1})$.  Hence, 
\[
\beta_1w_1+\cdots+\beta_sw_s = 0.
\]
Since the $w_j$ are linearly independent, each $\beta_j=0$, thus verifying 
the claim.

Next, we find the largest integer $m$ so that 
\[
t=\dim\nulls(A^m)-\dim\nulls(A^{m-1})>0.
\]
Proceed as above.  Choose a basis for $\nulls(A^{m-1})$ and extend to a basis 
for $\nulls(A^m)$ by adjoining the vectors $x_1,\ldots,x_t$.  Adjoin the $mt$ 
vectors $A^\ell x_j$ to the set ${\cal V}$ and verify that these vectors are 
all linearly independent.  And repeat the process.  Eventually, we arrive at 
a basis for $\R^n=\nulls(A^k)$.  

In this basis the matrix $[A]_{\cal V}$ is block diagonal; indeed, each of 
the blocks is a Jordan block, since 
\[
A(w_{j\ell}) = \left\{\begin{array}{cl}w_{j(\ell-1)} & 0<\ell\le k-1\\ 0 & 
\ell=1 \end{array}\right. .
\]
Note the resemblance with \Ref{e:Ndef}.


\end{document}
