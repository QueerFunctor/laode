\documentclass{ximera}

 

\usepackage{epsfig}

\graphicspath{
  {./}
  {figures/}
}

\usepackage{morewrites}
\makeatletter
\newcommand\subfile[1]{%
\renewcommand{\input}[1]{}%
\begingroup\skip@preamble\otherinput{#1}\endgroup\par\vspace{\topsep}
\let\input\otherinput}
\makeatother

\newcommand{\includeexercises}{\directlua{dofile("/home/jim/linearAlgebra/laode/exercises.lua")}}

%\newcounter{ccounter}
%\setcounter{ccounter}{1}
%\newcommand{\Chapter}[1]{\setcounter{chapter}{\arabic{ccounter}}\chapter{#1}\addtocounter{ccounter}{1}}

%\newcommand{\section}[1]{\section{#1}\setcounter{thm}{0}\setcounter{equation}{0}}

%\renewcommand{\theequation}{\arabic{chapter}.\arabic{section}.\arabic{equation}}
%\renewcommand{\thefigure}{\arabic{chapter}.\arabic{figure}}
%\renewcommand{\thetable}{\arabic{chapter}.\arabic{table}}

%\newcommand{\Sec}[2]{\section{#1}\markright{\arabic{ccounter}.\arabic{section}.#2}\setcounter{equation}{0}\setcounter{thm}{0}\setcounter{figure}{0}}

\newcommand{\Sec}[2]{\section{#1}}

\setcounter{secnumdepth}{2}
%\setcounter{secnumdepth}{1} 

%\newcounter{THM}
%\renewcommand{\theTHM}{\arabic{chapter}.\arabic{section}}

\newcommand{\trademark}{{R\!\!\!\!\!\bigcirc}}
%\newtheorem{exercise}{}

\newcommand{\dfield}{{\sf dfield9}}
\newcommand{\pplane}{{\sf pplane9}}

\newcommand{\EXER}{\section*{Exercises}}%\vspace*{0.2in}\hrule\small\setcounter{exercise}{0}}
\newcommand{\CEXER}{}%\vspace{0.08in}\begin{center}Computer Exercises\end{center}}
\newcommand{\TEXER}{} %\vspace{0.08in}\begin{center}Hand Exercises\end{center}}
\newcommand{\AEXER}{} %\vspace{0.08in}\begin{center}Hand Exercises\end{center}}

% BADBAD: \newcommand{\Bbb}{\bf}

\newcommand{\R}{\mbox{$\Bbb{R}$}}
\newcommand{\C}{\mbox{$\Bbb{C}$}}
\newcommand{\Z}{\mbox{$\Bbb{Z}$}}
\newcommand{\N}{\mbox{$\Bbb{N}$}}
\newcommand{\D}{\mbox{{\bf D}}}
\usepackage{amssymb}
%\newcommand{\qed}{\hfill\mbox{\raggedright$\square$} \vspace{1ex}}
%\newcommand{\proof}{\noindent {\bf Proof:} \hspace{0.1in}}

\newcommand{\setmin}{\;\mbox{--}\;}
\newcommand{\Matlab}{{M\small{AT\-LAB}} }
\newcommand{\Matlabp}{{M\small{AT\-LAB}}}
\newcommand{\computer}{\Matlab Instructions}
\newcommand{\half}{\mbox{$\frac{1}{2}$}}
\newcommand{\compose}{\raisebox{.15ex}{\mbox{{\scriptsize$\circ$}}}}
\newcommand{\AND}{\quad\mbox{and}\quad}
\newcommand{\vect}[2]{\left(\begin{array}{c} #1_1 \\ \vdots \\
 #1_{#2}\end{array}\right)}
\newcommand{\mattwo}[4]{\left(\begin{array}{rr} #1 & #2\\ #3
&#4\end{array}\right)}
\newcommand{\mattwoc}[4]{\left(\begin{array}{cc} #1 & #2\\ #3
&#4\end{array}\right)}
\newcommand{\vectwo}[2]{\left(\begin{array}{r} #1 \\ #2\end{array}\right)}
\newcommand{\vectwoc}[2]{\left(\begin{array}{c} #1 \\ #2\end{array}\right)}

\newcommand{\ignore}[1]{}


\newcommand{\inv}{^{-1}}
\newcommand{\CC}{{\cal C}}
\newcommand{\CCone}{\CC^1}
\newcommand{\Span}{{\rm span}}
\newcommand{\rank}{{\rm rank}}
\newcommand{\trace}{{\rm tr}}
\newcommand{\RE}{{\rm Re}}
\newcommand{\IM}{{\rm Im}}
\newcommand{\nulls}{{\rm null\;space}}

\newcommand{\dps}{\displaystyle}
\newcommand{\arraystart}{\renewcommand{\arraystretch}{1.8}}
\newcommand{\arrayfinish}{\renewcommand{\arraystretch}{1.2}}
\newcommand{\Start}[1]{\vspace{0.08in}\noindent {\bf Section~\ref{#1}}}
\newcommand{\exer}[1]{\noindent {\bf \ref{#1}}}
\newcommand{\ans}{}
\newcommand{\matthree}[9]{\left(\begin{array}{rrr} #1 & #2 & #3 \\ #4 & #5 & #6
\\ #7 & #8 & #9\end{array}\right)}
\newcommand{\cvectwo}[2]{\left(\begin{array}{c} #1 \\ #2\end{array}\right)}
\newcommand{\cmatthree}[9]{\left(\begin{array}{ccc} #1 & #2 & #3 \\ #4 & #5 &
#6 \\ #7 & #8 & #9\end{array}\right)}
\newcommand{\vecthree}[3]{\left(\begin{array}{r} #1 \\ #2 \\
#3\end{array}\right)}
\newcommand{\cvecthree}[3]{\left(\begin{array}{c} #1 \\ #2 \\
#3\end{array}\right)}
\newcommand{\cmattwo}[4]{\left(\begin{array}{cc} #1 & #2\\ #3
&#4\end{array}\right)}

\newcommand{\Matrix}[1]{\ensuremath{\left(\begin{array}{rrrrrrrrrrrrrrrrrr} #1 \end{array}\right)}}

\newcommand{\Matrixc}[1]{\ensuremath{\left(\begin{array}{cccccccccccc} #1 \end{array}\right)}}



\renewcommand{\labelenumi}{\theenumi)}
\newenvironment{enumeratea}%
{\begingroup
 \renewcommand{\theenumi}{\alph{enumi}}
 \renewcommand{\labelenumi}{(\theenumi)}
 \begin{enumerate}}
 {\end{enumerate}\endgroup}



\newcounter{help}
\renewcommand{\thehelp}{\thesection.\arabic{equation}}

%\newenvironment{equation*}%
%{\renewcommand\endequation{\eqno (\theequation)* $$}%
%   \begin{equation}}%
%   {\end{equation}\renewcommand\endequation{\eqno \@eqnnum
%$$\global\@ignoretrue}}

%\input{psfig.tex}

\author{Martin Golubitsky and Michael Dellnitz}

%\newenvironment{matlabEquation}%
%{\renewcommand\endequation{\eqno (\theequation*) $$}%
%   \begin{equation}}%
%   {\end{equation}\renewcommand\endequation{\eqno \@eqnnum
% $$\global\@ignoretrue}}

\newcommand{\soln}{\textbf{Solution:} }
\newcommand{\exercap}[1]{\centerline{Figure~\ref{#1}}}
\newcommand{\exercaptwo}[1]{\centerline{Figure~\ref{#1}a\hspace{2.1in}
Figure~\ref{#1}b}}
\newcommand{\exercapthree}[1]{\centerline{Figure~\ref{#1}a\hspace{1.2in}
Figure~\ref{#1}b\hspace{1.2in}Figure~\ref{#1}c}}
\newcommand{\para}{\hspace{0.4in}}

\renewenvironment{solution}{\suppress}{\endsuppress}

\ifxake
\newenvironment{matlabEquation}{\begin{equation}}{\end{equation}}
\else
\newenvironment{matlabEquation}%
{\let\oldtheequation\theequation\renewcommand{\theequation}{\oldtheequation*}\begin{equation}}%
  {\end{equation}\let\theequation\oldtheequation}
\fi

\makeatother


\title{Linear Differential Operators}

\begin{document}
\begin{abstract}
\end{abstract}
\maketitle

  
\label{S:LDO}

In Section~\ref{sec:2norderinhom} we describe a powerful method for 
solving certain second order linear differential equations.  To describe 
this method, it is convenient to introduce the notion of {\em linear 
differential operators}\index{linear!differential operator}.

We denote by $D$ the simplest differential 
operator\index{differential operator}, that is,
\[
D=\frac{d}{dt}.
\]
From differential calculus we know that $D$ acts linearly on 
(differentiable) functions, that is,
\begin{eqnarray*}
D(x(t)+y(t)) & = & Dx(t) + Dy(t) \\
D(cx(t)) & = & cDx(t),
\end{eqnarray*}
where $c\in\R$.  Thus we say that $D$ is a linear differential operator.

Higher order derivatives can be written in terms of $D$, that is,
\[
\frac{d^2x}{dt^2} = \frac{d}{dt}\left(\frac{dx}{dt}\right) = D(Dx) = D^2x,
\]
where $D^2$ is just the composition of $D$ with itself.  Similarly, 
\[
\frac{d^nx}{dt^n} = D^nx.
\]
It follows that $D^2,\ldots,D^n$ are all compositions of linear operators
and therefore each is linear.  We can even form a polynomial in $D$ by taking
linear combinations of the $D^k$.  For example,
\begin{equation}  \label{E:pdo}
D^4 - 3D^3 + D^2 -5D + 10 
\end{equation}
is a differential operator\index{differential operator}.  
We use the following polynomial notation to 
denote these operators.  Let $q(\lambda)$ be the polynomial
\[
q(\lambda) = \lambda^4 -3\lambda^3 + \lambda^2 - 5\lambda +10.
\]
Then we denote the linear operator in \Ref{E:pdo} by $q(D)$.  For example,
\begin{eqnarray*}
q(D)(\sin(2t)) & = & 16\sin(2t) + 24\cos(2t) - 4\sin(2t) - 10\cos(2t) + 
10\sin(2t) \\ & = & 22\sin(2t) + 14\cos(2t).
\end{eqnarray*}


With this notation in mind, we can reformulate much of the discussion of 
higher order equations in terms of linear differential operators.  Begin by 
rewriting the homogeneous equation \Ref{eq:nconsthom} 
\[
\frac{d^nx}{dt^n}+a_{n-1}\frac{d^{n-1}x}{dt^{n-1}}+\cdots+
a_1\frac{dx}{dt}+a_0x = (D^n+a_{n-1}D^{n-1}+\cdots+a_1D+a_0)x = 0.
\]
as
\begin{equation}  \label{E:opern2}
p(D)x = 0,
\end{equation}
where $p(\lambda)$ is the characteristic 
polynomial\index{characteristic polynomial!of higher order ODE} 
of \Ref{eq:nconsthom}.

We think of the {\em differential 
operator\/}\index{differential operator} $p(D)$ as operating on 
functions (that are sufficiently differentiable). 
\begin{lemma}  \label{L:p(D)linear}
The differential operator $p(D)$ is linear, that is,
\begin{eqnarray*}
p(D)(x+y) & = & p(D)x + p(D)y \\
p(D)(cx) & = & cp(D)x,
\end{eqnarray*}
for all sufficiently differentiable functions $x$ and $y$ and all scalars $c$.
\end{lemma}
The proof is left as an exercise.  See Exercise~\ref{c12.3.lem}.

Using the linearity of these differential operators allows us to reformulate
certain aspects of Section~\ref{sec:HighOrder} in this new language.

\begin{enumerate}
\item[(a)]  Solutions to the homogeneous equation \Ref{E:opern2} are just 
functions in the null space\index{null space!of differential operator} 
of $p(D)$.  
\item[(b)]  Using operator notation we can simplify \Ref{e:elam} as
\begin{equation}  \label{e:elam2}
p(D)\left(e^{\lambda t}\right) = p(\lambda)e^{\lambda t}.
\end{equation}
It follows from \Ref{e:elam2} that the functions $e^{\lambda t}$ are 
eigenvectors\index{eigenvector!of differential operator} 
of the operator $p(D)$ with eigenvalue $p(\lambda)$.  
Usually the functions $e^{\lambda t}$ are called 
{\em eigenfunctions}\index{eigenfunction}.
Perversely, we follow convention and reserve the term eigenvalue just 
for those $\lambda$ that are roots of the characteristic polynomial, that 
is, those values of $\lambda$ for which $p(\lambda)=0$.
\item[(c)]  We can rewrite the 
inhomogeneous\index{inhomogeneous} equation as:
\[
p(D)x = g.
\]
Showing that the inhomogeneous equation is solvable is
equivalent to showing that the 
function $g(t)$ is in the range\index{range!of differential operator} 
of the operator $p(D)$.
\end{enumerate}

\subsection*{Superposition and the Inhomogeneous Equation}
\index{principle of superposition}

\begin{lemma}  \label{L:inhsup}
Let $p(\lambda)$ be a polynomial, let $g_1(t),g_2(t)$ be continuous 
functions, and let $\alpha_1,\alpha_2$ be scalars.  We can find a particular 
solution $x_p(t)$ to the inhomogeneous differential equation
\[
p(D)x = \alpha_1g_1(t) + \alpha_2g_2(t)
\]
by first finding solutions $x_j(t)$ to $p(D)x_j = g_j$ and then setting 
\[
x_p(t) = \alpha_1x_1(t) + \alpha_2x_2(t).
\]
\end{lemma}

\begin{proof}  The proof follows directly from the linearity of $p(D)$.  Just 
compute
\[
p(D)(\alpha_1x_1 + \alpha_2x_2) = \alpha_1p(D)x_1 + 
\alpha_2p(D)x_2 = \alpha_1g_1 + \alpha_2g_2.
\]
Thus, the particular solution is a superposition of the solutions $x_1$ and
$x_2$.  \end{proof}

\subsection*{The Method of Elimination}

In Section~\ref{sec:HighOrder} we showed how to solve an $n^{th}$ order 
constant coefficient linear differential equation by converting that equation 
to a constant coefficient first order system of differential equations.  We 
now show that the process is reversible --- we can solve a first order 
system of $n$ equations by finding solutions to an associated $n^{th}$ order 
equation.  This procedure is called the {\em method of elimination\/}.
\index{method of elimination}  We first discuss this method abstractly using 
the language of differential operators and then discuss the pragmatic 
implementation of the method.

\begin{thm}  \label{T:Elimination}
Let $A$ be an $n\times n$ matrix and let $p_A(\lambda)$ be the characteristic 
polynomial of $A$.  Let $X(t)=(x_1(t),\ldots,x_n(t))^t$ be a solution to the 
system of ODEs 
\[
\frac{dX}{dt} = AX.
\]
Then each coordinate function $x_j(t)$ satisfies the $n^{th}$ order 
differential equation
\begin{equation}  \label{E:Elimination}
p_A(D)x_j = 0.
\end{equation}
\end{thm}

\begin{proof}  The proof of this theorem follows from the Cayley-Hamilton 
theorem, as follows.   Rewrite the differential equation using operator 
notation as  
\[
DX = AX,
\]
where $DX$ indicates differentiation of the vector $X(t)$ by $\frac{d}{dt}$ 
and $AX$ indicates multiplication of the vector $X(t)$ by the matrix $A$.
Since the coefficients of the matrix $A$ are constants (independent of $t$), 
it follows that $DAX=ADX$.  Hence
\[
D^2X = D(AX) = A(DX) = A^2X.
\]
Hence $D^kX = A^kX$ by induction, and $p(D)X = p(A)X$ for any polynomial $p(\lambda)$ by linearity.  The Cayley-Hamilton theorem (Theorem~\ref{T:CH} 
of Chapter~\ref{C:HDeigenvalues}) states that $p_A(A)=0$.  Hence $p_A(D)X=0$.  
So, in coordinates, $p_A(D)x_j=0$. \end{proof}

\subsubsection*{Implementation of the Method of Elimination}

Consider the first order system of differential equations
\begin{equation}  \label{E:Elim1}
\begin{array}{rcl}
\dot{x} & = & 2x-3y\\
\dot{y} & = & -5x+4y. 
\end{array}
\end{equation}
We can eliminate $y$ from the second equation in \Ref{E:Elim1} by solving for 
$y$ in the first equation, differentiating, and substituting, as follows:
\begin{equation} \label{E:Elim1a}
\begin{array}{rcl}
y & = & \frac{1}{3}(2x-\dot{x})\\
\dot{y} & = & \frac{1}{3}(2\dot{x}-\ddot{x})
\end{array}
\end{equation}
On substituting \Ref{E:Elim1a} into the second equation in \Ref{E:Elim1}, we
find
\[
\frac{1}{3}(2\dot{x}-\ddot{x}) = -5x + \frac{4}{3}(2x-\dot{x}).
\]
Simplification leads to the differential equation
\[
\ddot{x} - 6\dot{x} - 7x = 0.
\]
Since the characteristic polynomial of the coefficient matrix of \Ref{E:Elim1}
is $\lambda^2-6\lambda-7=(\lambda-7)(\lambda+1)$, this equation is the one 
predicted by Theorem~\ref{T:Elimination}.

Since the roots of the characteristic polynomial are $7$ and $-1$, it follows 
that $x(t)$ has the form
\[
x(t)  =  \alpha e^{7t} + \beta e^{-t}.
\]
We can solve for $y(t)$ using \Ref{E:Elim1a} and obtain
\[
y(t) = \frac{1}{3}(2x-\dot{x}) = -\frac{5}{3} \alpha e^{7t} + \beta e^{-t}.
\]
Thus the general solution to the first order system is:
\[
\vectwo{x(t)}{y(t)} = \frac{1}{3}\alpha e^{7t}\vectwo{3}{-5} + 
\beta e^{-t}\vectwo{1}{1}.
\]
Note that the vectors $(3,5)^t$ and $(1,1)^t$ are eigenvectors of the coefficient matrix of the system \Ref{E:Elim1}.

Note that the second half of the method of elimination --- in the previous 
example where we back substituted for the function $y(t)$ --- does not always 
work.  For example, if the system of differential equations decouples, then 
the second half of the method will fail.   See Exercise~\ref{c12.3.2} for an 
example. 

Thus the method of elimination provides another alternative to computing 
solutions to first order systems of differential equations; however, this 
procedure is not carried out easily for systems of more than two or three 
equations.


\EXER

\TEXER

\noindent In Exercises~\ref{c12.3.3a} -- \ref{c12.3.3d} apply the given
linear differential operator $p(D)$ to the functions: \\
(a) $\sin(3t)$ and  (b) $t^2 e^{-t}$.
\begin{exercise} \label{c12.3.3a}
$p(D) = D+1$.
\end{exercise}
\begin{exercise} \label{c12.3.3b}
$p(D) = D^2-D$.
\end{exercise}
\begin{exercise} \label{c12.3.3c}
$p(D) = D^4-2D^2+5$.
\end{exercise}
\begin{exercise} \label{c12.3.3d}
$p(D) = 2D^3-D^2+10D-1$.
\end{exercise}


\noindent In Exercises~\ref{c12.3.1a} -- \ref{c12.3.1f} write the 
homogeneous differential equation $p(D)x=0$ using $d/dt$ notation,
factor the characteristic polynomial, and then find the general solution.
\begin{exercise}  \label{c12.3.1a}
$p(D) = D + 1$.
\end{exercise}
\begin{exercise}  \label{c12.3.1b}
$p(D) = -D^2 - 5$.
\end{exercise}

\begin{exercise}  \label{c12.3.1f}
$p(D) = -4D^5 +4D^4 -8D^2$.
\end{exercise}

\begin{exercise}  \label{c12.3.2}
Attempt to use the method of elimination to solve the following system
\[
\begin{array}{rcl}
\dot{x} & = & 2x + z\\
\dot{y} & = & 3y + z\\
\dot{z} & = & z.
\end{array}
\]
Use the characteristic equation \Ref{E:Elimination} to solve for $x(t)$ and 
then attempt to use the system of differential equations to solve for $y(t)$ 
and $z(t)$.  What goes wrong?
\end{exercise} 

\begin{exercise}  \label{c12.3.lem}
Prove Lemma~\ref{L:p(D)linear}.
\end{exercise}

\noindent In Exercises~\ref{c12.3.4a} -- \ref{c12.3.4d} apply the
method of elimination to find the general solution of the given
system of differential equations.
\begin{exercise} \label{c12.3.4a}
$\begin{array}{rcl}
\dot{x} & = & y\\
\dot{y} & = & 2x+y
\end{array}$
\end{exercise}
\begin{exercise} \label{c12.3.4b}
$\begin{array}{rcl}
\dot{x} & = & x-y\\
\dot{y} & = & x+y
\end{array}$
\end{exercise}
\begin{exercise} \label{c12.3.4c}
$\begin{array}{rcl}
\dot{x} & = & x+y\\
\dot{y} & = & 4x-2y
\end{array}$
\end{exercise}
\begin{exercise} \label{c12.3.4d}
$\begin{array}{rcl}
\dot{x} & = & 2x-y\\
\dot{y} & = & y
\end{array}$
\end{exercise}

\CEXER

\noindent In Exercises~\ref{c12.3.1c} -- \ref{c12.3.1e} write the 
homogeneous differential equation $p(D)x=0$ using $d/dt$ notation,
use \Matlab to find the roots of the characteristic polynomial, and 
then find the general solution.
\begin{exercise}  \label{c12.3.1c}
$p(D) = 8D^2 - 4D +8$.
\end{exercise}
\begin{exercise}  \label{c12.3.1d}
$p(D) = D^4 + 4D^2 - 6D$.
\end{exercise}
\begin{exercise}  \label{c12.3.1e}
$p(D) = D^3 -6$.
\end{exercise}

\end{document}
