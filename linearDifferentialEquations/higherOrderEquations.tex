\documentclass{ximera}

 

\usepackage{epsfig}

\graphicspath{
  {./}
  {figures/}
}

\usepackage{morewrites}
\makeatletter
\newcommand\subfile[1]{%
\renewcommand{\input}[1]{}%
\begingroup\skip@preamble\otherinput{#1}\endgroup\par\vspace{\topsep}
\let\input\otherinput}
\makeatother

\newcommand{\includeexercises}{\directlua{dofile("/home/jim/linearAlgebra/laode/exercises.lua")}}

%\newcounter{ccounter}
%\setcounter{ccounter}{1}
%\newcommand{\Chapter}[1]{\setcounter{chapter}{\arabic{ccounter}}\chapter{#1}\addtocounter{ccounter}{1}}

%\newcommand{\section}[1]{\section{#1}\setcounter{thm}{0}\setcounter{equation}{0}}

%\renewcommand{\theequation}{\arabic{chapter}.\arabic{section}.\arabic{equation}}
%\renewcommand{\thefigure}{\arabic{chapter}.\arabic{figure}}
%\renewcommand{\thetable}{\arabic{chapter}.\arabic{table}}

%\newcommand{\Sec}[2]{\section{#1}\markright{\arabic{ccounter}.\arabic{section}.#2}\setcounter{equation}{0}\setcounter{thm}{0}\setcounter{figure}{0}}

\newcommand{\Sec}[2]{\section{#1}}

\setcounter{secnumdepth}{2}
%\setcounter{secnumdepth}{1} 

%\newcounter{THM}
%\renewcommand{\theTHM}{\arabic{chapter}.\arabic{section}}

\newcommand{\trademark}{{R\!\!\!\!\!\bigcirc}}
%\newtheorem{exercise}{}

\newcommand{\dfield}{{\sf dfield9}}
\newcommand{\pplane}{{\sf pplane9}}

\newcommand{\EXER}{\section*{Exercises}}%\vspace*{0.2in}\hrule\small\setcounter{exercise}{0}}
\newcommand{\CEXER}{}%\vspace{0.08in}\begin{center}Computer Exercises\end{center}}
\newcommand{\TEXER}{} %\vspace{0.08in}\begin{center}Hand Exercises\end{center}}
\newcommand{\AEXER}{} %\vspace{0.08in}\begin{center}Hand Exercises\end{center}}

% BADBAD: \newcommand{\Bbb}{\bf}

\newcommand{\R}{\mbox{$\Bbb{R}$}}
\newcommand{\C}{\mbox{$\Bbb{C}$}}
\newcommand{\Z}{\mbox{$\Bbb{Z}$}}
\newcommand{\N}{\mbox{$\Bbb{N}$}}
\newcommand{\D}{\mbox{{\bf D}}}
\usepackage{amssymb}
%\newcommand{\qed}{\hfill\mbox{\raggedright$\square$} \vspace{1ex}}
%\newcommand{\proof}{\noindent {\bf Proof:} \hspace{0.1in}}

\newcommand{\setmin}{\;\mbox{--}\;}
\newcommand{\Matlab}{{M\small{AT\-LAB}} }
\newcommand{\Matlabp}{{M\small{AT\-LAB}}}
\newcommand{\computer}{\Matlab Instructions}
\newcommand{\half}{\mbox{$\frac{1}{2}$}}
\newcommand{\compose}{\raisebox{.15ex}{\mbox{{\scriptsize$\circ$}}}}
\newcommand{\AND}{\quad\mbox{and}\quad}
\newcommand{\vect}[2]{\left(\begin{array}{c} #1_1 \\ \vdots \\
 #1_{#2}\end{array}\right)}
\newcommand{\mattwo}[4]{\left(\begin{array}{rr} #1 & #2\\ #3
&#4\end{array}\right)}
\newcommand{\mattwoc}[4]{\left(\begin{array}{cc} #1 & #2\\ #3
&#4\end{array}\right)}
\newcommand{\vectwo}[2]{\left(\begin{array}{r} #1 \\ #2\end{array}\right)}
\newcommand{\vectwoc}[2]{\left(\begin{array}{c} #1 \\ #2\end{array}\right)}

\newcommand{\ignore}[1]{}


\newcommand{\inv}{^{-1}}
\newcommand{\CC}{{\cal C}}
\newcommand{\CCone}{\CC^1}
\newcommand{\Span}{{\rm span}}
\newcommand{\rank}{{\rm rank}}
\newcommand{\trace}{{\rm tr}}
\newcommand{\RE}{{\rm Re}}
\newcommand{\IM}{{\rm Im}}
\newcommand{\nulls}{{\rm null\;space}}

\newcommand{\dps}{\displaystyle}
\newcommand{\arraystart}{\renewcommand{\arraystretch}{1.8}}
\newcommand{\arrayfinish}{\renewcommand{\arraystretch}{1.2}}
\newcommand{\Start}[1]{\vspace{0.08in}\noindent {\bf Section~\ref{#1}}}
\newcommand{\exer}[1]{\noindent {\bf \ref{#1}}}
\newcommand{\ans}{}
\newcommand{\matthree}[9]{\left(\begin{array}{rrr} #1 & #2 & #3 \\ #4 & #5 & #6
\\ #7 & #8 & #9\end{array}\right)}
\newcommand{\cvectwo}[2]{\left(\begin{array}{c} #1 \\ #2\end{array}\right)}
\newcommand{\cmatthree}[9]{\left(\begin{array}{ccc} #1 & #2 & #3 \\ #4 & #5 &
#6 \\ #7 & #8 & #9\end{array}\right)}
\newcommand{\vecthree}[3]{\left(\begin{array}{r} #1 \\ #2 \\
#3\end{array}\right)}
\newcommand{\cvecthree}[3]{\left(\begin{array}{c} #1 \\ #2 \\
#3\end{array}\right)}
\newcommand{\cmattwo}[4]{\left(\begin{array}{cc} #1 & #2\\ #3
&#4\end{array}\right)}

\newcommand{\Matrix}[1]{\ensuremath{\left(\begin{array}{rrrrrrrrrrrrrrrrrr} #1 \end{array}\right)}}

\newcommand{\Matrixc}[1]{\ensuremath{\left(\begin{array}{cccccccccccc} #1 \end{array}\right)}}



\renewcommand{\labelenumi}{\theenumi)}
\newenvironment{enumeratea}%
{\begingroup
 \renewcommand{\theenumi}{\alph{enumi}}
 \renewcommand{\labelenumi}{(\theenumi)}
 \begin{enumerate}}
 {\end{enumerate}\endgroup}



\newcounter{help}
\renewcommand{\thehelp}{\thesection.\arabic{equation}}

%\newenvironment{equation*}%
%{\renewcommand\endequation{\eqno (\theequation)* $$}%
%   \begin{equation}}%
%   {\end{equation}\renewcommand\endequation{\eqno \@eqnnum
%$$\global\@ignoretrue}}

%\input{psfig.tex}

\author{Martin Golubitsky and Michael Dellnitz}

%\newenvironment{matlabEquation}%
%{\renewcommand\endequation{\eqno (\theequation*) $$}%
%   \begin{equation}}%
%   {\end{equation}\renewcommand\endequation{\eqno \@eqnnum
% $$\global\@ignoretrue}}

\newcommand{\soln}{\textbf{Solution:} }
\newcommand{\exercap}[1]{\centerline{Figure~\ref{#1}}}
\newcommand{\exercaptwo}[1]{\centerline{Figure~\ref{#1}a\hspace{2.1in}
Figure~\ref{#1}b}}
\newcommand{\exercapthree}[1]{\centerline{Figure~\ref{#1}a\hspace{1.2in}
Figure~\ref{#1}b\hspace{1.2in}Figure~\ref{#1}c}}
\newcommand{\para}{\hspace{0.4in}}

\renewenvironment{solution}{\suppress}{\endsuppress}

\ifxake
\newenvironment{matlabEquation}{\begin{equation}}{\end{equation}}
\else
\newenvironment{matlabEquation}%
{\let\oldtheequation\theequation\renewcommand{\theequation}{\oldtheequation*}\begin{equation}}%
  {\end{equation}\let\theequation\oldtheequation}
\fi

\makeatother


\title{Higher Order Equations}

\begin{document}
\begin{abstract}
\end{abstract}
\maketitle


\label{sec:HighOrder}

The {\em order\/} of an ordinary differential equation is the highest 
order of differentiation that appears in the equation.  For instance, 
the differential equation  \index{order}
\[
\frac{d^3x}{dt^3} + 2\frac{dx}{dt}+ 5tx = \sin t
\]
has order three. 

In this section we begin the study of solutions to differential equations 
of the form
\begin{equation}  \label{eq:nconst}
a_n\frac{d^nx}{dt^n} + \cdots + a_1\frac{dx}{dt}+a_0x = g(t),
\end{equation}
where $a_j\in\R$ are constants with $a_n\neq 0$ and $g(t)$ is a 
continuous function. These equations are $n^{th}$ order and linear. 
The constants $a_j$ are called the 
{\em coefficients\/}\index{coefficients of a differential equation} of the 
differential equation \Ref{eq:nconst}.  After dividing \Ref{eq:nconst} by 
$a_n$, we may assume that $a_n=1$.  

In analogy to systems of linear equations, we call \Ref{eq:nconst} 
{\em homogeneous\/} if $g(t)=0$ for all $t$.  Otherwise the equation is 
{\em inhomogeneous}.   In particular, the principle of superposition is 
valid for the homogeneous equations.  \index{homogeneous}  
\index{inhomogeneous}

\subsection*{Reduction of Equations to First Order Systems}

In principle, we can find solutions to \Ref{eq:nconst} by finding 
solutions of a corresponding 
linear system of first order 
ODEs\index{system of differential equations!linear} with 
constant coefficients.  We have previously discussed this reduction in
the special case of second order equations to planar systems in 
Section~\ref{S:SOE}.  Let
\begin{equation}   \label{eq:trick}
y_1(t)=x(t),\quad y_2(t)=\frac{dx}{dt}(t),\quad
\ldots, \quad  y_n(t)=\frac{d^{n-1}x}{dt^{n-1}}(t).
\end{equation}
With these functions, \Ref{eq:nconst} (with $a_n=1$) can be rewritten as
\[
\frac{dy_n}{dt} + a_{n-1} y_n + \cdots + a_1 y_2 + a_0 y_1 = g(t).
\]
Hence, \Ref{eq:nconst} is equivalent to
\arraystart
\[
\begin{array}{rcl}
\dps \frac{dy_j}{dt}&=&y_{j+1} \hspace{1.6in} (j=1,\ldots,n-1)\\
\dps \frac{dy_n}{dt}&=& g(t) - a_0y_1 - \cdots - a_{n-1} y_n,
\end{array}
\]
\arrayfinish
which is a linear constant coefficient system of ODEs.  More explicitly,
define the $n\times n$ coefficient matrix $Q$ by 
\begin{equation}   \label{eq:coefma}
Q = \left(\begin{array}{ccccc}
0 & 1 & 0 & \cdots & 0\\
0 & 0 & 1 & \cdots & 0\\
\vdots & \vdots & \vdots & & \vdots\\
0 & 0 & 0 & \cdots & 1\\
-a_0 & -a_1 &  -a_2 & \cdots & -a_{n-1}
\end{array}\right),
\end{equation}
and introduce the vectors 
\[
Y(t) = (y_1(t),\ldots,y_n(t))^t  \AND
G(t) = (0,\ldots,0,g(t))^t.  
\]
Then we arrive at the system of ODEs
\begin{equation}  \label{eq:highrw}
\frac{dY}{dt} = QY+G(t).
\end{equation}
Let us summarize.
\begin{prop}   \label{prop:ho1}
The function $x(t)$ is a solution to the order $n$ ODE \Ref{eq:nconst}  
if and only if the vector of functions $Y(t)=(y_1(t),\ldots,y_n(t))^t$ is 
a solution of the first order system \Ref{eq:highrw}, where 
\[
y_1(t) = x(t) \AND y_{j+1}(t)=\dps \frac{d^jx}{dt^j}(t) 
\]
for $j=1,2,\ldots,n-1$.
\end{prop}\index{order}

\subsection*{The Homogeneous Equation}
\index{homogeneous}

Now consider specifically the case when \Ref{eq:nconst} is homogeneous.  
As before, we assume that $a_n=1$, and solve
\begin{equation}  \label{eq:nconsthom}
\frac{d^nx}{dt^n} + a_{n-1}\frac{d^{n-1}x}{dt^{n-1}} +
\cdots + a_1\frac{dx}{dt}+a_0x = 0.
\end{equation}
Proposition~\ref{prop:ho1} implies that each solution
of \Ref{eq:nconsthom} is the first component in a solution
to the $n\times n$ system 
\begin{equation}  \label{eq:highrwh}
\frac{dY}{dt} = QY,
\end{equation}
where $Q$ is the matrix defined in \Ref{eq:coefma}.  From the discussion 
in Section~\ref{sec:LinHomSys}, it follows that we can find all solutions 
to \Ref{eq:nconsthom} if we know the eigenvalues and the 
Jordan block\index{Jordan block} 
structure of $Q$.  In particular, the first component of solutions to 
\Ref{eq:highrwh} is always just a linear combination of the 
functions\index{linear!combination!of functions} 
$t^je^{\lambda t}$ where $\lambda$ is an eigenvalue of $Q$.  In the 
abstract, the only question that remains is which powers $t^j$ can 
actually occur.  Certainly, $j$ must be less than the multiplicity of the 
eigenvalue $\lambda$.  It also follows from Proposition~\ref{prop:ho1}
that there are precisely $n$ linearly independent functions that are 
solutions to \Ref{eq:nconsthom}.

We can use this abstract information to find a shortcut for solving 
\Ref{eq:nconsthom}.  Suppose that $x(t)=e^{\lambda t}$.  Then we can 
compute the left hand side of \Ref{eq:nconsthom} with this $x(t)$ 
obtaining:
\begin{equation}  \label{e:elam}
e^{\lambda t}\left(\lambda^n + a_{n-1}\lambda^{n-1}+\cdots+a_1\lambda
+a_0\right).
\end{equation}
Thus, $x(t)$ is a solution to \Ref{eq:nconsthom} precisely when $\lambda$
is a root of 
\begin{equation} \label{E:charpoly}
p(\lambda) = \lambda^n + a_{n-1}\lambda^{n-1}+\cdots+a_1\lambda +a_0.
\end{equation}
We call $p(\lambda)$ the {\em characteristic polynomial\/}
\index{characteristic polynomial!of higher order ODE} of the $n^{th}$ 
order equation \Ref{eq:nconsthom}.  The roots of the characteristic polynomial
are called {\em eigenvalues\/}.  \index{eigenvalue!of higher order ODE} 
This terminology is explained in Section~\ref{S:LDO}.
 
It follows immediately that there is one solution associated to 
each eigenvalue of the characteristic polynomial $p(\lambda)$ in 
\Ref{E:charpoly}.  If the eigenvalues $\lambda=\sigma\pm i\tau$ are complex, 
then there are two solutions to \Ref{eq:nconsthom}, namely,
$e^{\sigma t}\cos(\tau t)$ and $e^{\sigma t}\sin(\tau t)$.
\begin{thm} \label{T:hoe}
Let $\lambda_1,\ldots,\lambda_k$ be the distinct eigenvalues of the 
characteristic 
polynomial\index{characteristic polynomial!of higher order ODE} 
$p(\lambda)$ of \Ref{E:charpoly} with multiplicities 
$m_1,\ldots,m_k$.  Then 
\[
\{t^je^{\lambda_\ell}: 1\leq\ell\leq k, 0\leq j<m_\ell\}
\]
is a basis of solutions\index{basis!of solutions} to \Ref{eq:nconsthom}.
\end{thm} 

\begin{proof}  The previous calculations show that if $\lambda$ is an eigenvalue 
of $p(\lambda)$, then $e^{\lambda t}$ is a solution to \Ref{eq:nconsthom}.  
The difficulty in proving this theorem is in showing that $t^je^{\lambda t}$ 
is also a solution to \Ref{eq:nconsthom} for any $j<m$, where $m$ is the 
multiplicity of the eigenvalue $\lambda$.  We prove this theorem in two steps.
First, we show in Lemma~\ref{lem:ho1} that the eigenvalues of $p$ are just the 
eigenvalues of the characteristic polynomial of the coefficient matrix $Q$ in 
\Ref{eq:coefma}.  Second, we show in Lemma~\ref{lem:ho2} that every eigenvalue 
of $Q$ is associated to precisely one Jordan block and that
every eigenvector of $Q$ has a nonvanishing first component. 
Then the result follows from Theorem~\ref{T:JBsoln}.  \end{proof}

\begin{lemma}  \label{lem:ho1}
The characteristic polynomial\index{characteristic polynomial} of 
the matrix $Q$ in \Ref{eq:coefma} is 
\[
p_Q(\lambda) = (-1)^np(\lambda).
\]
\end{lemma}

\begin{proof} We prove the lemma by induction.
For $n=1$ the matrix $Q$ is just $(-a_0)$.  Hence
\[
p_Q(\lambda)=\det(Q-\lambda I_1) = -a_0-\lambda = -(\lambda+a_0)
\]
as desired.

Now suppose that the result is valid for all matrices of order $n-1$.  
Then we may expand the determinant\index{determinant} by cofactors
\index{cofactor} along the $1^{st}$ column (see \Ref{e:inductdetc} in 
Chapter~\ref{C:D&E}) and obtain 
{\small 
\[
p_Q(\lambda)  =  -\lambda \det\left(\begin{array}{cccc}
-\lambda & 1 &  \cdots & 0\\
0 & -\lambda &  \cdots & 0\\
\vdots & \vdots & & \vdots\\
0 & 0 & \cdots & 1\\
-a_1 & -a_2 & \cdots & -\lambda-a_{n-1}
\end{array}\right)
+(-1)^na_0\det\left(\begin{array}{ccccc}
1 & 0  & \cdots & 0 & 0\\
-\lambda &   1  & \cdots & 0 & 0\\
\vdots & \vdots & & \vdots & \vdots \\
0 & 0 & \cdots &  1  & 0\\
0 & 0 & \cdots & -\lambda & 1\\
\end{array}\right).
\]
}
By induction, the first determinant is 
\[
(-1)^{n-1}(\lambda^{n-1}+a_{n-1}\lambda^{n-2}+
\cdots+a_2\lambda+a_1),
\]
and by direct calculation the second determinant is $1$.  Therefore,
\begin{eqnarray*}
p_Q(\lambda) & = &
-\lambda (-1)^{n-1}(\lambda^{n-1}+a_{n-1}\lambda^{n-2}+
\cdots+a_2\lambda+a_1) +(-1)^n a_0 \\
& = & (-1)^n(\lambda^n+a_{n-1}\lambda^{n-1}+\cdots+a_1\lambda+a_0).
\end{eqnarray*}
This proves the lemma.   \end{proof}

\begin{lemma}   \label{lem:ho2}
Let $\lambda$ be an eigenvalue of $Q$.  Then, up to scaling, there is a 
unique eigenvector $v$ corresponding to the eigenvalue $\lambda$ given by
\[
v = \left(\begin{array}{c}
1\\
\lambda\\
\lambda^2\\
\vdots\\
\lambda^{n-1}
\end{array}\right).
\]
In particular, the geometric 
multiplicity\index{multiplicity!geometric} of each eigenvalue of $Q$ is 
one, and there is exactly one Jordan block\index{Jordan block} 
of $Q$ associated to the eigenvalue $\lambda$.
\end{lemma}

\begin{proof} Let $\lambda$ be an eigenvalue of $Q$ and let 
$v=(v_1,\ldots,v_n)^t$ be a
corresponding eigenvector.  Writing $Qv = \lambda v$
in coordinates we obtain
\[
\left(\begin{array}{c}
v_2\\
v_3\\
\vdots\\
v_n\\
-\sum_{j=0}^{n-1} a_j v_{j+1}
\end{array}\right)=
\left(\begin{array}{c}
\lambda v_1\\
\lambda v_2\\
\vdots\\
\lambda v_{n-1}\\
\lambda v_n
\end{array}\right).
\]
Note that if $v_1=0$ then $v_2=v_3=\cdots =v_n=0$, which 
contradicts the fact that $v$ is an eigenvector.  Hence $v_1\neq 0$ 
and we may rescale $v$ so that $v_1=1$.  It follows that 
\[
\begin{array}{rclrl}
v_2 & = & \lambda v_1 & = & \lambda \\
v_3 & = & \lambda v_2 & = & \lambda^2 \\
& & \vdots & & \\ 
v_n & = & \lambda v_{n-1} & = & \lambda^{n-1}.
\end{array}
\]
as desired.  \end{proof}

We restate the results in Theorem~\ref{T:hoe} in real form.
\begin{thm}   \label{thm:HOgen}
Let $\lambda$ be an eigenvalue of \Ref{E:charpoly} with multiplicity $m$.
\begin{itemize}
\item[(a)] If $\lambda$ is real\index{eigenvalue!real}, then 
$t^je^{\lambda t}$ is a solution 
of \Ref{eq:nconsthom} for $0\leq j < m$.  
\item[(b)] If $\lambda=\sigma+i\tau$ is complex\index{eigenvalue!complex}, 
then $t^je^{\sigma t}\cos(\tau t)$ and $t^je^{\sigma t}\sin(\tau t)$ are 
solutions of \Ref{eq:nconsthom} for $0\leq j < m$.
\item[(c)] Every solution of \Ref{eq:nconsthom} is a linear combination of
the functions\index{linear!combination!of functions} constructed 
in (a) and (b).
\end{itemize}
\end{thm}

\subsubsection*{Examples of Fourth Order}

Consider the linear homogeneous ordinary differential equation
\begin{equation}  \label{eq:soex3}
\frac{d^4x}{dt^4} + 2\frac{d^2x}{dt^2} + x= 0.
\end{equation}
This type of equation arises if one considers small deformations of
an elastic beam under pressure where the pressure is acting at both
ends of the beam in the direction of the beam.  In this case $x(t)$
represents the resulting deformation of the beam at the spatial point $t$.

First, we find all eigenvalues of the characteristic polynomial by solving
\[
\lambda^4 + 2\lambda^2 + 1 = (\lambda^2+1)^2 = 0.
\]
This fourth order polynomial has two roots namely $\pm i$, each of
algebraic multiplicity two.  By Theorem~\ref{thm:HOgen}, the general solution 
of \Ref{eq:soex3} has the form
\[
x(t) = c_1 \cos(t)+ c_2 \sin(t) +c_3 t\cos(t)+ c_4 t\sin(t).
\]
For instance, if we consider a beam which is hinged at both of its ends ---
say at $t=0$ and $t=\pi$ --- then the corresponding solution has to satisfy
\[
x(0)=x(\pi)=0 \AND \frac{d^2x}{dt^2}(0)=\frac{d^2x}{dt^2}(\pi)=0.
\]
These conditions on the solution $x(t)$ lead to a homogeneous system of four
linear equations in the four unknowns $c_1,c_2,c_3,c_4$.  This system can
easily be solved by hand, and it follows that $c_1$, $c_3$ and $c_4$ have
to vanish, whereas $c_2$ is arbitrary.  Thus, under these conditions a
small deformation of the beam is given by $x(t)=c_2 \sin(t)$.

In general, however, we can not expect to find the roots of a fourth order
characteristic polynomial by inspection.  For example, suppose we add a first
order term to \Ref{eq:soex3} obtaining the differential equation
\begin{equation}  \label{eq:soex3a}
\frac{d^4x}{dt^4} + 2\frac{d^2x}{dt^2} - \frac{dx}{dt} + x= 0.
\end{equation}
The characteristic polynomial of \Ref{eq:soex3a} is
\[
p(\lambda) = \lambda^4 + 2\lambda^2 - \lambda + 1 .
\]
We can use \Matlab to find the roots of $p(\lambda)$, as follows. 
Polynomials are entered into \Matlab by entering the coefficients in an array
from highest order to lowest.  So we enter $p(\lambda)$ into \Matlab by
typing
\begin{verbatim}
p = [1 0 2 -1 1];
\end{verbatim}
To find the roots of $p(\lambda)$ just type the command {\tt roots(p)}.
\Matlab responds with 
\begin{verbatim}
ans =
  -0.3438 + 1.3584i
  -0.3438 - 1.3584i
   0.3438 + 0.6254i
   0.3438 - 0.6254i
\end{verbatim}
By Theorem~\ref{thm:HOgen}, the general solution of \Ref{eq:soex3a} has the 
form
\[
\begin{array}{rcl}
x(t) & = & (c_1 \cos(1.3584t)+ c_2 \sin(1.3584t))e^{-0.3438t} \\
 & & + (c_3 \cos(0.6254t)+ c_4 \sin(0.6254t))e^{0.3438t}.\end{array}
\]
Note that the typical solution to the fourth order differential equation 
\Ref{eq:soex3a} contains sine and cosine terms with two different 
frequencies.  This is not surprising since the characteristic polynomial 
has two distinct complex conjugate pairs of roots.  See Section~\ref{S:NLD}.


\subsubsection*{The Initial Value Problem for Higher Order Equations}
\index{initial value problem!for higher order ODEs}

For motivation, recall the introductory mechanical examples.  The motion 
of mass points are described by second order ODEs.  To determine a 
{\em specific\/} motion we need to specify the position of the point 
together with its velocity at a certain time $t_0$.  In other words, to 
obtain existence and uniqueness of solutions to second order equations, 
\index{existence of solutions}\index{uniqueness of solutions}
both
\[
x(t_0)\AND \frac{dx}{dt}(t_0)
\]
have to be specified.  

\arraystart
\begin{thm}
The initial value problem
\begin{equation}
\begin{array}{c}
\dps \frac{d^nx}{dt^n} + a_{n-1}\frac{d^{n-1}x}{dt^{n-1}} +
\cdots + a_1\frac{dx}{dt}+a_0x = 0\\
\dps \frac{d^{n-1}x}{dt^{n-1}}(t_0)=x_{n-1},\ldots,\frac{dx}{dt}(t_0)=x_1,\quad
x(t_0)=x_0.
\end{array}
\end{equation}
has a unique solution $x(t)$.
\end{thm}
\arrayfinish

\begin{proof} The result is an immediate consequence of Proposition~\ref{prop:ho1},
since the corresponding solution of the system of linear differential
equations exists and is unique (see Theorem~\ref{T:linODEsoln} in 
Chapter~\ref{Chap:Planar}).
\end{proof}

\subsection*{Inhomogeneous Higher Order Equations}
\index{inhomogeneous}

Linearity implies that to find the 
general solution\index{general solution} the inhomogeneous
equation \Ref{eq:nconst} we need to find one solution to that equation 
and all solutions to the corresponding homogeneous equation 
\Ref{eq:nconsthom}.  Sections~\ref{sec:2norderinhom} and \ref{S:resonance} 
are devoted to finding specific solutions to inhomogeneous equations of 
various types using the method of undetermined coefficients.  Our study of
inhomogeneous equations will continue in Chapter~\ref{C:LT} where we use 
Laplace transforms to study discontinuous forcing and in Section~\ref{S:RO}  
where we discuss the method of reduction of order.



\EXER

\TEXER


\noindent In Exercises~\ref{c12.2.a} -- \ref{c12.2.d} rewrite the given
higher order equation as a first order system.
\begin{exercise} \label{c12.2.a}
$\ddot{x}+5\dot{x}+x = 0$
\end{exercise}
\begin{exercise} \label{c12.2.b}
$\ddot{x}-10x = 0$
\end{exercise}
\begin{exercise} \label{c12.2.c}
$\frac{d^3x}{dt^3}-4\frac{d^2x}{dt^2}-10x = 0$
\end{exercise}
\begin{exercise} \label{c12.2.d}
$\frac{d^4x}{dt^4}+2\frac{d^2x}{dt^2}-3\frac{dx}{dt}-2x = 0$
\end{exercise}

\noindent In Exercises~\ref{c12.2.1} -- \ref{c12.2.5} find the general 
solution to the given homogeneous linear differential equation.
\begin{exercise} \label{c12.2.1}
$\ddot{x}-3\dot{x}-4x = 0$
\end{exercise}
\begin{exercise} \label{c12.2.2}
$\ddot{x}+2\dot{x}+2x = 0$
\end{exercise}
\begin{exercise} \label{c12.2.3}
$\ddot{x}-6\dot{x}+9x = 0$
\end{exercise}
\begin{exercise} \label{c12.2.4}
$\frac{d^3x}{dt^3} -3\frac{d^2x}{dt^2}+3\frac{dx}{dt}-x=0$
\end{exercise}
\begin{exercise} \label{c12.2.5}
$\frac{d^3x}{dt^3}+4\frac{dx}{dt} = 0$
\end{exercise}

\begin{exercise} \label{c12.2.a6}
Find a solution to the differential equation 
\[
\ddot{x} -3\dot{x} + 2x = 0
\]
satisfying $x(0)=0$ and $x(\ln 2) = 4$.
\end{exercise}




\CEXER

\noindent In Exercises~\ref{c12.2.6} -- \ref{c12.2.11}, use \Matlab to find 
the roots of the characteristic polynomial for the given homogeneous linear 
differential equation, and then find the general solution to that equation.
\begin{exercise} \label{c12.2.6}
$\frac{d^3x}{dt^3} -6\frac{d^2x}{dt^2}+11\frac{dx}{dt}-6x=0$
\end{exercise}
\begin{exercise} \label{c12.2.7}
$\frac{d^3x}{dt^3} + \frac{d^2x}{dt^2}+ 10x=0$
\end{exercise}
\begin{exercise} \label{c12.2.8}
$\frac{d^3x}{dt^3} +\frac{d^2x}{dt^2}-4\frac{dx}{dt}-4x=0$
\end{exercise}
\begin{exercise} \label{c12.2.9}
$\frac{d^4x}{dt^4}-2\frac{d^3x}{dt^3}+\frac{d^2x}{dt^2}+8\frac{dx}{dt}-20x=0$
\end{exercise}
\begin{exercise} \label{c12.2.10}
$\frac{d^4x}{dt^4}-3\frac{d^2x}{dt^2}-4x=0$
\end{exercise}
\begin{exercise} \label{c12.2.11}
$\frac{d^4x}{dt^4}-4\frac{d^3x}{dt^3}+14\frac{d^2x}{dt^2}-4\frac{dx}{dt}+13x=0$
\end{exercise}





\end{document}
