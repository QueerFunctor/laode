\documentclass{ximera}
 

\usepackage{epsfig}

\graphicspath{
  {./}
  {figures/}
}

\usepackage{morewrites}
\makeatletter
\newcommand\subfile[1]{%
\renewcommand{\input}[1]{}%
\begingroup\skip@preamble\otherinput{#1}\endgroup\par\vspace{\topsep}
\let\input\otherinput}
\makeatother

\newcommand{\includeexercises}{\directlua{dofile("/home/jim/linearAlgebra/laode/exercises.lua")}}

%\newcounter{ccounter}
%\setcounter{ccounter}{1}
%\newcommand{\Chapter}[1]{\setcounter{chapter}{\arabic{ccounter}}\chapter{#1}\addtocounter{ccounter}{1}}

%\newcommand{\section}[1]{\section{#1}\setcounter{thm}{0}\setcounter{equation}{0}}

%\renewcommand{\theequation}{\arabic{chapter}.\arabic{section}.\arabic{equation}}
%\renewcommand{\thefigure}{\arabic{chapter}.\arabic{figure}}
%\renewcommand{\thetable}{\arabic{chapter}.\arabic{table}}

%\newcommand{\Sec}[2]{\section{#1}\markright{\arabic{ccounter}.\arabic{section}.#2}\setcounter{equation}{0}\setcounter{thm}{0}\setcounter{figure}{0}}

\newcommand{\Sec}[2]{\section{#1}}

\setcounter{secnumdepth}{2}
%\setcounter{secnumdepth}{1} 

%\newcounter{THM}
%\renewcommand{\theTHM}{\arabic{chapter}.\arabic{section}}

\newcommand{\trademark}{{R\!\!\!\!\!\bigcirc}}
%\newtheorem{exercise}{}

\newcommand{\dfield}{{\sf dfield9}}
\newcommand{\pplane}{{\sf pplane9}}

\newcommand{\EXER}{\section*{Exercises}}%\vspace*{0.2in}\hrule\small\setcounter{exercise}{0}}
\newcommand{\CEXER}{}%\vspace{0.08in}\begin{center}Computer Exercises\end{center}}
\newcommand{\TEXER}{} %\vspace{0.08in}\begin{center}Hand Exercises\end{center}}
\newcommand{\AEXER}{} %\vspace{0.08in}\begin{center}Hand Exercises\end{center}}

% BADBAD: \newcommand{\Bbb}{\bf}

\newcommand{\R}{\mbox{$\Bbb{R}$}}
\newcommand{\C}{\mbox{$\Bbb{C}$}}
\newcommand{\Z}{\mbox{$\Bbb{Z}$}}
\newcommand{\N}{\mbox{$\Bbb{N}$}}
\newcommand{\D}{\mbox{{\bf D}}}
\usepackage{amssymb}
%\newcommand{\qed}{\hfill\mbox{\raggedright$\square$} \vspace{1ex}}
%\newcommand{\proof}{\noindent {\bf Proof:} \hspace{0.1in}}

\newcommand{\setmin}{\;\mbox{--}\;}
\newcommand{\Matlab}{{M\small{AT\-LAB}} }
\newcommand{\Matlabp}{{M\small{AT\-LAB}}}
\newcommand{\computer}{\Matlab Instructions}
\newcommand{\half}{\mbox{$\frac{1}{2}$}}
\newcommand{\compose}{\raisebox{.15ex}{\mbox{{\scriptsize$\circ$}}}}
\newcommand{\AND}{\quad\mbox{and}\quad}
\newcommand{\vect}[2]{\left(\begin{array}{c} #1_1 \\ \vdots \\
 #1_{#2}\end{array}\right)}
\newcommand{\mattwo}[4]{\left(\begin{array}{rr} #1 & #2\\ #3
&#4\end{array}\right)}
\newcommand{\mattwoc}[4]{\left(\begin{array}{cc} #1 & #2\\ #3
&#4\end{array}\right)}
\newcommand{\vectwo}[2]{\left(\begin{array}{r} #1 \\ #2\end{array}\right)}
\newcommand{\vectwoc}[2]{\left(\begin{array}{c} #1 \\ #2\end{array}\right)}

\newcommand{\ignore}[1]{}


\newcommand{\inv}{^{-1}}
\newcommand{\CC}{{\cal C}}
\newcommand{\CCone}{\CC^1}
\newcommand{\Span}{{\rm span}}
\newcommand{\rank}{{\rm rank}}
\newcommand{\trace}{{\rm tr}}
\newcommand{\RE}{{\rm Re}}
\newcommand{\IM}{{\rm Im}}
\newcommand{\nulls}{{\rm null\;space}}

\newcommand{\dps}{\displaystyle}
\newcommand{\arraystart}{\renewcommand{\arraystretch}{1.8}}
\newcommand{\arrayfinish}{\renewcommand{\arraystretch}{1.2}}
\newcommand{\Start}[1]{\vspace{0.08in}\noindent {\bf Section~\ref{#1}}}
\newcommand{\exer}[1]{\noindent {\bf \ref{#1}}}
\newcommand{\ans}{}
\newcommand{\matthree}[9]{\left(\begin{array}{rrr} #1 & #2 & #3 \\ #4 & #5 & #6
\\ #7 & #8 & #9\end{array}\right)}
\newcommand{\cvectwo}[2]{\left(\begin{array}{c} #1 \\ #2\end{array}\right)}
\newcommand{\cmatthree}[9]{\left(\begin{array}{ccc} #1 & #2 & #3 \\ #4 & #5 &
#6 \\ #7 & #8 & #9\end{array}\right)}
\newcommand{\vecthree}[3]{\left(\begin{array}{r} #1 \\ #2 \\
#3\end{array}\right)}
\newcommand{\cvecthree}[3]{\left(\begin{array}{c} #1 \\ #2 \\
#3\end{array}\right)}
\newcommand{\cmattwo}[4]{\left(\begin{array}{cc} #1 & #2\\ #3
&#4\end{array}\right)}

\newcommand{\Matrix}[1]{\ensuremath{\left(\begin{array}{rrrrrrrrrrrrrrrrrr} #1 \end{array}\right)}}

\newcommand{\Matrixc}[1]{\ensuremath{\left(\begin{array}{cccccccccccc} #1 \end{array}\right)}}



\renewcommand{\labelenumi}{\theenumi)}
\newenvironment{enumeratea}%
{\begingroup
 \renewcommand{\theenumi}{\alph{enumi}}
 \renewcommand{\labelenumi}{(\theenumi)}
 \begin{enumerate}}
 {\end{enumerate}\endgroup}



\newcounter{help}
\renewcommand{\thehelp}{\thesection.\arabic{equation}}

%\newenvironment{equation*}%
%{\renewcommand\endequation{\eqno (\theequation)* $$}%
%   \begin{equation}}%
%   {\end{equation}\renewcommand\endequation{\eqno \@eqnnum
%$$\global\@ignoretrue}}

%\input{psfig.tex}

\author{Martin Golubitsky and Michael Dellnitz}

%\newenvironment{matlabEquation}%
%{\renewcommand\endequation{\eqno (\theequation*) $$}%
%   \begin{equation}}%
%   {\end{equation}\renewcommand\endequation{\eqno \@eqnnum
% $$\global\@ignoretrue}}

\newcommand{\soln}{\textbf{Solution:} }
\newcommand{\exercap}[1]{\centerline{Figure~\ref{#1}}}
\newcommand{\exercaptwo}[1]{\centerline{Figure~\ref{#1}a\hspace{2.1in}
Figure~\ref{#1}b}}
\newcommand{\exercapthree}[1]{\centerline{Figure~\ref{#1}a\hspace{1.2in}
Figure~\ref{#1}b\hspace{1.2in}Figure~\ref{#1}c}}
\newcommand{\para}{\hspace{0.4in}}

\renewenvironment{solution}{\suppress}{\endsuppress}

\ifxake
\newenvironment{matlabEquation}{\begin{equation}}{\end{equation}}
\else
\newenvironment{matlabEquation}%
{\let\oldtheequation\theequation\renewcommand{\theequation}{\oldtheequation*}\begin{equation}}%
  {\end{equation}\let\theequation\oldtheequation}
\fi

\makeatother

\begin{document}

\noindent  In Exercises~\ref{c12.1.8a} -- \ref{c12.1.8d} use
Theorem~\ref{T:etA} and \Matlab to compute $e^{tA}$ for the given matrix.
\begin{computerExercise} \label{c12.1.8a}
$A = \mattwo{0}{1}{0}{0}$.

\begin{solution}
\ans $e^{tA} = \mattwo{1}{t}{0}{1}$.

\soln The matrix $A = \mattwo{0}{1}{0}{0}$ has an eigenvalue $0$ with
multiplicity two.  Theorem~\ref{T:etA} states that 
\[
e^{tA} = a_1(A)P_1(A)\sum_{j=0}^1t^jA^j= a_1(A)P_1(A)(I_2+tA).
\]
In this case $P_1(\lambda)=1=a_1(\lambda)$.   Therefore,
\[
e^{tA} = I_2+tA = \mattwo{1}{t}{0}{1}.
\]


\end{solution}
\end{computerExercise}
\begin{computerExercise} \label{c12.1.8b}
$A = \mattwo{-1}{2}{0}{-1}$.

\begin{solution}
\ans $e^{tA} = e^{-t}\mattwo{1}{2t}{0}{1}$.

\soln The matrix $A = \mattwo{-1}{2}{0}{-1}$ has an eigenvalue $-1$ with
multiplicity two.  Theorem~\ref{T:etA} states that 
\[
e^{tA} = a_1(A)P_1(A)\sum_{j=0}^1e^{-t}t^j(A+I_2)^j=
a_1(A)P_1(A)e^{-t}(I_2+t(A+I_2)).
\]
In this case $P_1(\lambda)=1=a_1(\lambda)$.   Therefore,
\[
e^{tA} = e^{-t}(I_2+t(A+I_2)) = e^{-t}\mattwo{1}{2t}{0}{1}.
\]


\end{solution}
\end{computerExercise}
\begin{computerExercise} \label{c12.1.8c}
$A = \left(\begin{array}{rrr} 1 & 1 & 2\\ 1 & 1 & 0 \\ 0 & 0 & 0
\end{array}\right)$.

\begin{solution}
\ans $e^{tA}= \dps\frac{1}{4}\left(\begin{array}{rrc} 
1 & -1 & -2+2t\\ -1 & 1 & -2t \\ 0 & 0 & 2\end{array}\right) +e^{2t}\left(\begin{array}{rrr} 2 & 2 & 2\\  2 & 2 & 2 \\ 0 & 0 & 0 
\end{array}\right)$.

\soln The matrix 
$A = \left(\begin{array}{rrr} 1 & 1 & 2\\ 1 & 1 & 0 \\ 0 & 0 & 0
\end{array}\right)$ has eigenvalues $0$ with multiplicity two and $2$ 
with multiplicity one. Theorem~\ref{T:etA} states that 
\begin{eqnarray*}
e^{tA} & = & a_1(A)P_1(A)\sum_{j=0}^1t^jA^j + e^{2t}a_2(A)P_2(A)\\
 & = & a_1(A)P_1(A)(I_3+tA) + e^{2t}a_2(A)P_2(A).
\end{eqnarray*}
The characteristic polynomial of $A$ is $p_A(\lambda) =(2-\lambda)\lambda^2$.
From \eqref{e:Pj} we see that
\[
P_1(\lambda) = 2-\lambda \AND P_2(\lambda) = \lambda^2.
\]
We recall from \eqref{e:1/p} that 
\[
\frac{1}{p_A(\lambda)} = \frac{a_1(\lambda)}{2-\lambda} + 
\frac{a_2(\lambda)}{\lambda^2} = \frac{\frac{1}{4}}{2-\lambda}
+ \frac{\frac{1}{2}+\frac{1}{4}\lambda}{\lambda^2}.
\]
Therefore
\[
a_1(\lambda) = \frac{1}{4} \AND a_2(\lambda) = \frac{1}{2}+\frac{1}{4}\lambda,
\]
and
\begin{eqnarray*}
e^{tA} & = & \frac{1}{4}(2I_3-A)(I_3+tA) + 
e^{2t}\left(\frac{1}{2}I_3+\frac{1}{4}A\right)A^2\\
& = & \frac{1}{4}\left((2I_3-A)+t(2A-A^2)+e^{2t}(2A^2+A^3)\right).
\end{eqnarray*}
Hence
\[
e^{tA}= \frac{1}{4}\left(\left(\begin{array}{rrr} 1 & -1 & -2\\ -1 & 1 & 0 
\\ 0 & 0 & 2\end{array}\right)+t\left(\begin{array}{rrr} 0 & 0 & 2\\ 0 & 0 & -2
\\ 0 & 0 & 0 \end{array}\right)+e^{2t}\left(\begin{array}{rrr} 8 & 8 & 8\\ 
8 & 8 & 8 \\ 0 & 0 & 0\end{array}\right)\right).
\]


\end{solution}
\end{computerExercise}
\begin{computerExercise} \label{c12.1.8d}
$A = \left(\begin{array}{rrr} -1 & 2 & -6\\ 0 & 2 & 0 \\ 0 & -1 & 2
\end{array}\right)$.

\begin{solution}
\ans $e^{tA}= \dps\frac{1}{9}e^{2t}\left(\begin{array}{rcr} 
0 & -2-6t & 6\\ 0 & -3 & 0\\ 0 & 1+3t & -3\end{array}\right) 
+e^{-t}\left(\begin{array}{rrr} -6 & 0 & -12\\  0 & 0 & 0 \\ 0 & 0 & 0 
\end{array}\right)$.


\soln The matrix 
$A = \left(\begin{array}{rrr} -1 & 2 & -6\\ 0 & 2 & 0 \\ 0 & -1 & 2
\end{array}\right)$ has eigenvalues $2$ with multiplicity two and $-1$ 
with multiplicity one. Theorem~\ref{T:etA} states that 
\begin{eqnarray*}
e^{tA} & = & a_1(A)P_1(A)e^{2t}\sum_{j=0}^1t^j(A-2I_3)^j+e^{-t}a_2(A)P_2(A)\\
 & = & a_1(A)P_1(A)e^{2t}(I_3+t(A-2I_3)) + e^{-t}a_2(A)P_2(A).
\end{eqnarray*}
The characteristic polynomial of $A$ is 
$p_A(\lambda) =(2-\lambda)^2(-1-\lambda)$.  From \eqref{e:Pj} we see that
\[
P_1(\lambda) = -1-\lambda \AND P_2(\lambda) = (2-\lambda)^2.
\]
We recall from \eqref{e:1/p} that 
\[
\frac{1}{p_A(\lambda)} = \frac{a_1(\lambda)}{-1-\lambda} + 
\frac{a_2(\lambda)}{(\lambda-2)^2} = \frac{\frac{1}{9}}{-1-\lambda}
+ \frac{-\frac{5}{9}+\frac{1}{9}\lambda}{(\lambda-2)^2}.
\]
Therefore
\[
a_1(\lambda) = \frac{1}{9} \AND a_2(\lambda) = -\frac{5}{9}+\frac{1}{9}\lambda,
\]
and
\begin{eqnarray*}
e^{tA} & = & \frac{1}{9}\left((-I_3-A)e^{2t}(I_3+t(A-2I_3)) + 
e^{-t}(A-5I_3)(A-2I_3)^2\right)\\
& = & 
\frac{1}{9}\left(e^{2t}((-I_3-A)+t(-A^2+A+2I_3)+e^{-t}(A-5I_3)(A-2I_3)^2\right).
\end{eqnarray*}
Hence
\[
e^{tA}= \frac{1}{9}\left(e^{2t}\left(\left(\begin{array}{rrr} 0 & -2 & 6\\ 
0 & -3 & 0\\ 0 & 1 & -3\end{array}\right)+t\left(\begin{array}{rrr} 0& -6 & 0\\ 
0 & 0 & 0\\ 0 & 3 & 0 \end{array}\right)\right)+e^{-t}\left(\begin{array}{rrr} 
-54 & 0 & -108\\ 0 & 0 & 0 \\ 0 & 0 & 0\end{array}\right)\right).
\]




\end{solution}
\end{computerExercise} 
\end{document}
