\documentclass{ximera}
 

\usepackage{epsfig}

\graphicspath{
  {./}
  {figures/}
}

\usepackage{morewrites}
\makeatletter
\newcommand\subfile[1]{%
\renewcommand{\input}[1]{}%
\begingroup\skip@preamble\otherinput{#1}\endgroup\par\vspace{\topsep}
\let\input\otherinput}
\makeatother

\newcommand{\includeexercises}{\directlua{dofile("/home/jim/linearAlgebra/laode/exercises.lua")}}

%\newcounter{ccounter}
%\setcounter{ccounter}{1}
%\newcommand{\Chapter}[1]{\setcounter{chapter}{\arabic{ccounter}}\chapter{#1}\addtocounter{ccounter}{1}}

%\newcommand{\section}[1]{\section{#1}\setcounter{thm}{0}\setcounter{equation}{0}}

%\renewcommand{\theequation}{\arabic{chapter}.\arabic{section}.\arabic{equation}}
%\renewcommand{\thefigure}{\arabic{chapter}.\arabic{figure}}
%\renewcommand{\thetable}{\arabic{chapter}.\arabic{table}}

%\newcommand{\Sec}[2]{\section{#1}\markright{\arabic{ccounter}.\arabic{section}.#2}\setcounter{equation}{0}\setcounter{thm}{0}\setcounter{figure}{0}}

\newcommand{\Sec}[2]{\section{#1}}

\setcounter{secnumdepth}{2}
%\setcounter{secnumdepth}{1} 

%\newcounter{THM}
%\renewcommand{\theTHM}{\arabic{chapter}.\arabic{section}}

\newcommand{\trademark}{{R\!\!\!\!\!\bigcirc}}
%\newtheorem{exercise}{}

\newcommand{\dfield}{{\sf dfield9}}
\newcommand{\pplane}{{\sf pplane9}}

\newcommand{\EXER}{\section*{Exercises}}%\vspace*{0.2in}\hrule\small\setcounter{exercise}{0}}
\newcommand{\CEXER}{}%\vspace{0.08in}\begin{center}Computer Exercises\end{center}}
\newcommand{\TEXER}{} %\vspace{0.08in}\begin{center}Hand Exercises\end{center}}
\newcommand{\AEXER}{} %\vspace{0.08in}\begin{center}Hand Exercises\end{center}}

% BADBAD: \newcommand{\Bbb}{\bf}

\newcommand{\R}{\mbox{$\Bbb{R}$}}
\newcommand{\C}{\mbox{$\Bbb{C}$}}
\newcommand{\Z}{\mbox{$\Bbb{Z}$}}
\newcommand{\N}{\mbox{$\Bbb{N}$}}
\newcommand{\D}{\mbox{{\bf D}}}
\usepackage{amssymb}
%\newcommand{\qed}{\hfill\mbox{\raggedright$\square$} \vspace{1ex}}
%\newcommand{\proof}{\noindent {\bf Proof:} \hspace{0.1in}}

\newcommand{\setmin}{\;\mbox{--}\;}
\newcommand{\Matlab}{{M\small{AT\-LAB}} }
\newcommand{\Matlabp}{{M\small{AT\-LAB}}}
\newcommand{\computer}{\Matlab Instructions}
\newcommand{\half}{\mbox{$\frac{1}{2}$}}
\newcommand{\compose}{\raisebox{.15ex}{\mbox{{\scriptsize$\circ$}}}}
\newcommand{\AND}{\quad\mbox{and}\quad}
\newcommand{\vect}[2]{\left(\begin{array}{c} #1_1 \\ \vdots \\
 #1_{#2}\end{array}\right)}
\newcommand{\mattwo}[4]{\left(\begin{array}{rr} #1 & #2\\ #3
&#4\end{array}\right)}
\newcommand{\mattwoc}[4]{\left(\begin{array}{cc} #1 & #2\\ #3
&#4\end{array}\right)}
\newcommand{\vectwo}[2]{\left(\begin{array}{r} #1 \\ #2\end{array}\right)}
\newcommand{\vectwoc}[2]{\left(\begin{array}{c} #1 \\ #2\end{array}\right)}

\newcommand{\ignore}[1]{}


\newcommand{\inv}{^{-1}}
\newcommand{\CC}{{\cal C}}
\newcommand{\CCone}{\CC^1}
\newcommand{\Span}{{\rm span}}
\newcommand{\rank}{{\rm rank}}
\newcommand{\trace}{{\rm tr}}
\newcommand{\RE}{{\rm Re}}
\newcommand{\IM}{{\rm Im}}
\newcommand{\nulls}{{\rm null\;space}}

\newcommand{\dps}{\displaystyle}
\newcommand{\arraystart}{\renewcommand{\arraystretch}{1.8}}
\newcommand{\arrayfinish}{\renewcommand{\arraystretch}{1.2}}
\newcommand{\Start}[1]{\vspace{0.08in}\noindent {\bf Section~\ref{#1}}}
\newcommand{\exer}[1]{\noindent {\bf \ref{#1}}}
\newcommand{\ans}{}
\newcommand{\matthree}[9]{\left(\begin{array}{rrr} #1 & #2 & #3 \\ #4 & #5 & #6
\\ #7 & #8 & #9\end{array}\right)}
\newcommand{\cvectwo}[2]{\left(\begin{array}{c} #1 \\ #2\end{array}\right)}
\newcommand{\cmatthree}[9]{\left(\begin{array}{ccc} #1 & #2 & #3 \\ #4 & #5 &
#6 \\ #7 & #8 & #9\end{array}\right)}
\newcommand{\vecthree}[3]{\left(\begin{array}{r} #1 \\ #2 \\
#3\end{array}\right)}
\newcommand{\cvecthree}[3]{\left(\begin{array}{c} #1 \\ #2 \\
#3\end{array}\right)}
\newcommand{\cmattwo}[4]{\left(\begin{array}{cc} #1 & #2\\ #3
&#4\end{array}\right)}

\newcommand{\Matrix}[1]{\ensuremath{\left(\begin{array}{rrrrrrrrrrrrrrrrrr} #1 \end{array}\right)}}

\newcommand{\Matrixc}[1]{\ensuremath{\left(\begin{array}{cccccccccccc} #1 \end{array}\right)}}



\renewcommand{\labelenumi}{\theenumi)}
\newenvironment{enumeratea}%
{\begingroup
 \renewcommand{\theenumi}{\alph{enumi}}
 \renewcommand{\labelenumi}{(\theenumi)}
 \begin{enumerate}}
 {\end{enumerate}\endgroup}



\newcounter{help}
\renewcommand{\thehelp}{\thesection.\arabic{equation}}

%\newenvironment{equation*}%
%{\renewcommand\endequation{\eqno (\theequation)* $$}%
%   \begin{equation}}%
%   {\end{equation}\renewcommand\endequation{\eqno \@eqnnum
%$$\global\@ignoretrue}}

%\input{psfig.tex}

\author{Martin Golubitsky and Michael Dellnitz}

%\newenvironment{matlabEquation}%
%{\renewcommand\endequation{\eqno (\theequation*) $$}%
%   \begin{equation}}%
%   {\end{equation}\renewcommand\endequation{\eqno \@eqnnum
% $$\global\@ignoretrue}}

\newcommand{\soln}{\textbf{Solution:} }
\newcommand{\exercap}[1]{\centerline{Figure~\ref{#1}}}
\newcommand{\exercaptwo}[1]{\centerline{Figure~\ref{#1}a\hspace{2.1in}
Figure~\ref{#1}b}}
\newcommand{\exercapthree}[1]{\centerline{Figure~\ref{#1}a\hspace{1.2in}
Figure~\ref{#1}b\hspace{1.2in}Figure~\ref{#1}c}}
\newcommand{\para}{\hspace{0.4in}}

\renewenvironment{solution}{\suppress}{\endsuppress}

\ifxake
\newenvironment{matlabEquation}{\begin{equation}}{\end{equation}}
\else
\newenvironment{matlabEquation}%
{\let\oldtheequation\theequation\renewcommand{\theequation}{\oldtheequation*}\begin{equation}}%
  {\end{equation}\let\theequation\oldtheequation}
\fi

\makeatother

\begin{document}




\noindent  In Exercises~\ref{c12.1.10a} -- \ref{c12.1.10b}  solve the 
initial value problem for the given system of ODE $\dot{X}=AX$ with 
initial condition $X(0)=X_0$.
\begin{computerExercise}  \label{c12.1.10a}
$A=\left(\begin{array}{rrr} 
     1   &  2  &  -1 \\
    -1   &  1  &   1 \\
     2   &  2  &  -2
\end{array}\right) \AND 
X_0=\left(\begin{array}{r} 1 \\ 2 \\ -1  \end{array}\right)$.

\begin{solution}
\ans The solution to the given initial value problem is
\[
X(t) = -e^{-t}\vecthree{4}{-1}{6} +
3\vecthree{1}{0}{1} + e^t\vecthree{2}{1}{2}
= \cvecthree{-4e^{-t} + 3 + 2e^t}{e^{-t} + e^t}{-6e^{-t} + 3 + 2e^t}.
\]

\soln The eigenvalues of $A$ are $\lambda_1 = -1$, $\lambda_2 = 0$, and
$\lambda_3 = 1$, with associated eigenvectors $w_1 = (4,-1,6)^t$,
$w_2 = (1,0,1)^t$, and $w_3 = (2,1,2)^t$.  So,
\[
\begin{array}{c}
X_1(t) = e^{\lambda_1 t}w_1 = e^{-t}\vecthree{4}{-1}{6}, \quad
X_2(t) = e^{\lambda_2 t}w_2 = \vecthree{1}{0}{1}, \\
X_3(t) = e^{\lambda_3 t}w_3 = e^t\vecthree{2}{1}{2}
\end{array}
\]
are linearly independent solutions.  The general solution is a linear
combination of the $X_j(t)$, and the solution to the initial value
problem $X_0 = X(0)$ satisfies
\[
\vecthree{1}{2}{-1} = X(0) = \alpha_1w_1 + \alpha_2w_2 + \alpha_3w_3
= \alpha_1\vecthree{4}{-1}{6} + \alpha_2\vecthree{1}{0}{1}
+ \alpha_3\vecthree{2}{1}{2}.
\]
Solve this linear system to obtain $\alpha_1 = -1$,
$\alpha_2 = 3$, and $\alpha_3 = 1$.


\end{solution}
\end{computerExercise}
\begin{computerExercise}  \label{c12.1.10b}
$A=\left(\begin{array}{rrr}
    -3   &  0   &  2\\
     2   & -1   & -1\\
    -4   &  0   &  3
\end{array}\right) \AND 
X_0=\left(\begin{array}{r} 0 \\ 1 \\ 1  \end{array}\right)$.

\begin{solution}
\ans The solution to the given initial value problem is
\[
X(t) = e^t\vecthree{1}{0}{2} + e^{-t}\vecthree{0}{1}{0} -
e^{-t}\vecthree{1}{t}{1} =
\cvecthree{e^t - e^{-t}}{(1 - t)e^{-t}}{2e^t - e^{-t}}.
\]

\soln The matrix $A$ has an eigenvalue at $\lambda_1 = 1$ with associated
eigenvector $w_1 = (1,0,2)^t$ and a double eigenvalue at $\lambda_2 = -1$
with eigenvector $w_2 = (0,1,0)^t$.  Find a generalized eigenvector at
$\lambda_2$ by solving the equation $(A - \lambda_2I_3)w_3 = w_2$, obtaining
$w_3 = (1,0,1)^t$.  Thus,
\[
\begin{array}{c}
X_1(t) = e^{\lambda_1 t}w_1 = e^{t}\vecthree{1}{0}{2}, \quad
X_2(t) = e^{\lambda_2 t}w_2 = e^{-t}\vecthree{0}{1}{0}, \\
X_3(t) = e^{\lambda_2 t}(w_2t + w_3) = e^{-t}\vecthree{1}{t}{1}
\end{array}
\]
are linearly independent solutions.  The general solution is a linear
combination of the $X_j(t)$, and the solution to the initial value
problem $X_0 = X(0)$ satisfies
\[
\vecthree{0}{1}{1} = X(0) = \alpha_1w_1 + \alpha_2w_2 + \alpha_3w_3
= \alpha_1\vecthree{1}{0}{2} + \alpha_2\vecthree{0}{1}{0}
+ \alpha_3\vecthree{1}{0}{1}.
\]
Solve this linear system to obtain $\alpha_1 = 1$,
$\alpha_2 = 1$, and $\alpha_3 = -1$.

\end{solution}
\end{computerExercise}
\end{document}
