\documentclass{ximera}

 

\usepackage{epsfig}

\graphicspath{
  {./}
  {figures/}
}

\usepackage{morewrites}
\makeatletter
\newcommand\subfile[1]{%
\renewcommand{\input}[1]{}%
\begingroup\skip@preamble\otherinput{#1}\endgroup\par\vspace{\topsep}
\let\input\otherinput}
\makeatother

\newcommand{\includeexercises}{\directlua{dofile("/home/jim/linearAlgebra/laode/exercises.lua")}}

%\newcounter{ccounter}
%\setcounter{ccounter}{1}
%\newcommand{\Chapter}[1]{\setcounter{chapter}{\arabic{ccounter}}\chapter{#1}\addtocounter{ccounter}{1}}

%\newcommand{\section}[1]{\section{#1}\setcounter{thm}{0}\setcounter{equation}{0}}

%\renewcommand{\theequation}{\arabic{chapter}.\arabic{section}.\arabic{equation}}
%\renewcommand{\thefigure}{\arabic{chapter}.\arabic{figure}}
%\renewcommand{\thetable}{\arabic{chapter}.\arabic{table}}

%\newcommand{\Sec}[2]{\section{#1}\markright{\arabic{ccounter}.\arabic{section}.#2}\setcounter{equation}{0}\setcounter{thm}{0}\setcounter{figure}{0}}

\newcommand{\Sec}[2]{\section{#1}}

\setcounter{secnumdepth}{2}
%\setcounter{secnumdepth}{1} 

%\newcounter{THM}
%\renewcommand{\theTHM}{\arabic{chapter}.\arabic{section}}

\newcommand{\trademark}{{R\!\!\!\!\!\bigcirc}}
%\newtheorem{exercise}{}

\newcommand{\dfield}{{\sf dfield9}}
\newcommand{\pplane}{{\sf pplane9}}

\newcommand{\EXER}{\section*{Exercises}}%\vspace*{0.2in}\hrule\small\setcounter{exercise}{0}}
\newcommand{\CEXER}{}%\vspace{0.08in}\begin{center}Computer Exercises\end{center}}
\newcommand{\TEXER}{} %\vspace{0.08in}\begin{center}Hand Exercises\end{center}}
\newcommand{\AEXER}{} %\vspace{0.08in}\begin{center}Hand Exercises\end{center}}

% BADBAD: \newcommand{\Bbb}{\bf}

\newcommand{\R}{\mbox{$\Bbb{R}$}}
\newcommand{\C}{\mbox{$\Bbb{C}$}}
\newcommand{\Z}{\mbox{$\Bbb{Z}$}}
\newcommand{\N}{\mbox{$\Bbb{N}$}}
\newcommand{\D}{\mbox{{\bf D}}}
\usepackage{amssymb}
%\newcommand{\qed}{\hfill\mbox{\raggedright$\square$} \vspace{1ex}}
%\newcommand{\proof}{\noindent {\bf Proof:} \hspace{0.1in}}

\newcommand{\setmin}{\;\mbox{--}\;}
\newcommand{\Matlab}{{M\small{AT\-LAB}} }
\newcommand{\Matlabp}{{M\small{AT\-LAB}}}
\newcommand{\computer}{\Matlab Instructions}
\newcommand{\half}{\mbox{$\frac{1}{2}$}}
\newcommand{\compose}{\raisebox{.15ex}{\mbox{{\scriptsize$\circ$}}}}
\newcommand{\AND}{\quad\mbox{and}\quad}
\newcommand{\vect}[2]{\left(\begin{array}{c} #1_1 \\ \vdots \\
 #1_{#2}\end{array}\right)}
\newcommand{\mattwo}[4]{\left(\begin{array}{rr} #1 & #2\\ #3
&#4\end{array}\right)}
\newcommand{\mattwoc}[4]{\left(\begin{array}{cc} #1 & #2\\ #3
&#4\end{array}\right)}
\newcommand{\vectwo}[2]{\left(\begin{array}{r} #1 \\ #2\end{array}\right)}
\newcommand{\vectwoc}[2]{\left(\begin{array}{c} #1 \\ #2\end{array}\right)}

\newcommand{\ignore}[1]{}


\newcommand{\inv}{^{-1}}
\newcommand{\CC}{{\cal C}}
\newcommand{\CCone}{\CC^1}
\newcommand{\Span}{{\rm span}}
\newcommand{\rank}{{\rm rank}}
\newcommand{\trace}{{\rm tr}}
\newcommand{\RE}{{\rm Re}}
\newcommand{\IM}{{\rm Im}}
\newcommand{\nulls}{{\rm null\;space}}

\newcommand{\dps}{\displaystyle}
\newcommand{\arraystart}{\renewcommand{\arraystretch}{1.8}}
\newcommand{\arrayfinish}{\renewcommand{\arraystretch}{1.2}}
\newcommand{\Start}[1]{\vspace{0.08in}\noindent {\bf Section~\ref{#1}}}
\newcommand{\exer}[1]{\noindent {\bf \ref{#1}}}
\newcommand{\ans}{}
\newcommand{\matthree}[9]{\left(\begin{array}{rrr} #1 & #2 & #3 \\ #4 & #5 & #6
\\ #7 & #8 & #9\end{array}\right)}
\newcommand{\cvectwo}[2]{\left(\begin{array}{c} #1 \\ #2\end{array}\right)}
\newcommand{\cmatthree}[9]{\left(\begin{array}{ccc} #1 & #2 & #3 \\ #4 & #5 &
#6 \\ #7 & #8 & #9\end{array}\right)}
\newcommand{\vecthree}[3]{\left(\begin{array}{r} #1 \\ #2 \\
#3\end{array}\right)}
\newcommand{\cvecthree}[3]{\left(\begin{array}{c} #1 \\ #2 \\
#3\end{array}\right)}
\newcommand{\cmattwo}[4]{\left(\begin{array}{cc} #1 & #2\\ #3
&#4\end{array}\right)}

\newcommand{\Matrix}[1]{\ensuremath{\left(\begin{array}{rrrrrrrrrrrrrrrrrr} #1 \end{array}\right)}}

\newcommand{\Matrixc}[1]{\ensuremath{\left(\begin{array}{cccccccccccc} #1 \end{array}\right)}}



\renewcommand{\labelenumi}{\theenumi)}
\newenvironment{enumeratea}%
{\begingroup
 \renewcommand{\theenumi}{\alph{enumi}}
 \renewcommand{\labelenumi}{(\theenumi)}
 \begin{enumerate}}
 {\end{enumerate}\endgroup}



\newcounter{help}
\renewcommand{\thehelp}{\thesection.\arabic{equation}}

%\newenvironment{equation*}%
%{\renewcommand\endequation{\eqno (\theequation)* $$}%
%   \begin{equation}}%
%   {\end{equation}\renewcommand\endequation{\eqno \@eqnnum
%$$\global\@ignoretrue}}

%\input{psfig.tex}

\author{Martin Golubitsky and Michael Dellnitz}

%\newenvironment{matlabEquation}%
%{\renewcommand\endequation{\eqno (\theequation*) $$}%
%   \begin{equation}}%
%   {\end{equation}\renewcommand\endequation{\eqno \@eqnnum
% $$\global\@ignoretrue}}

\newcommand{\soln}{\textbf{Solution:} }
\newcommand{\exercap}[1]{\centerline{Figure~\ref{#1}}}
\newcommand{\exercaptwo}[1]{\centerline{Figure~\ref{#1}a\hspace{2.1in}
Figure~\ref{#1}b}}
\newcommand{\exercapthree}[1]{\centerline{Figure~\ref{#1}a\hspace{1.2in}
Figure~\ref{#1}b\hspace{1.2in}Figure~\ref{#1}c}}
\newcommand{\para}{\hspace{0.4in}}

\renewenvironment{solution}{\suppress}{\endsuppress}

\ifxake
\newenvironment{matlabEquation}{\begin{equation}}{\end{equation}}
\else
\newenvironment{matlabEquation}%
{\let\oldtheequation\theequation\renewcommand{\theequation}{\oldtheequation*}\begin{equation}}%
  {\end{equation}\let\theequation\oldtheequation}
\fi

\makeatother


\title{Undetermined Coefficients}

\begin{document}
\begin{abstract}
\end{abstract}
\maketitle


\label{sec:2norderinhom}

In this section we find solutions to inhomogeneous linear differential 
equations, such as the second order equation
\begin{equation}  \label{e:inhom1}
\ddot{x} + b\dot{x} + ax = g(t),
\end{equation}
where $g(t)$ is thought of as a forcing term\index{forcing!term}.   
To find all solutions to 
the inhomogeneous\index{inhomogeneous} 
equation \eqref{e:inhom1}, we need to find just one solution
to \eqref{e:inhom1} and then add to that solution all solutions to the 
homogeneous equation --- which we know how to solve.  If the forcing term 
$g(t)$ is sufficiently nice, then there is an elegant way to solve 
\eqref{e:inhom1} called the method of 
{\em undetermined coefficients}\index{undetermined coefficients}.


\subsubsection*{An Illustrative Example}

Consider the differential equation
\begin{equation}  \label{eq:undetcoeffex}
\ddot{x} + 3\dot{x}+2x = t.
\end{equation}
To solve \eqref{eq:undetcoeffex} we must find one solution to the inhomogeneous
equation and add to that particular solution\index{particular solution} the 
general solution\index{general solution} of the 
homogeneous equation.  

The general solution to the homogeneous equation is easily found using the
techniques of Section~\ref{sec:HighOrder}.  That is, the characteristic 
polynomial of the homogeneous equation is 
\[
p(\lambda) = \lambda^2 + 3\lambda + 2 = (\lambda+2)(\lambda+1).
\]
So the roots of $p(\lambda)$ are $-2$ and $-1$.  It follows that the general
solution to the homogeneous equation is:
\[
x_h(t)= \alpha_1e^{-2t} + \alpha_2e^{-t}.
\]

Therefore, to solve \eqref{eq:undetcoeffex} in general we must find just one 
solution to \eqref{eq:undetcoeffex}.  Can we guess the answer?  The answer is
yes in this case.  Since differentiation just lowers the degree of a 
polynomial, we can guess that there is a particular solution $x(t)$ to 
\eqref{eq:undetcoeffex} that is a polynomial of degree one, that is, $x(t)$ 
has the form 
\[
y(t)=d_1t+d_2
\]
for constants $d_1$ and $d_2$.  If we substitute $y(t)$ 
into the left hand side of \eqref{eq:undetcoeffex}, we obtain
\[
\left(\frac{d^2}{dt^2}+3\frac{d}{dt}+2\right)y(t) = 
0+3d_1+2(d_1t+d_2) = 2d_1t + (3d_1+2d_2).
\]
Since we want the result of this differentiation to be $t$, we must choose 
$d_1$ and $d_2$ to solve the linear equations
\[
2d_1 = 1 \AND 3d_1+2d_2=0.
\]
The solution to this linear system is $d_1=\frac{1}{2}$ and 
$d_2=-\frac{3}{4}$.   Therefore, we get a 
particular solution\index{particular solution} to 
\eqref{eq:undetcoeffex}, namely, $x_p(t)=\frac{1}{2}t-\frac{3}{4}$.  
It follows that the general solution to \eqref{eq:undetcoeffex} is:
\[
x(t) = x_h(t)+x_p(t) = \alpha_1e^{-2t} + \alpha_2e^{-t} + 
\frac{1}{2}t-\frac{3}{4}.
\]

\subsubsection*{Why did the guess work?}

What lies at the heart of undetermined coefficients is having a method 
for choosing a subspace of functions in which a particular solution resides.  
We call this subspace the {\em trial space}\index{trial!space}.  In example 
\eqref{eq:undetcoeffex} the trial space is the two dimensional subspace
\[
d_1t + d_2.
\]

The idea behind finding a trial subspace is the elimination of the 
inhomogeneity in \eqref{eq:undetcoeffex} using the fact that $g(t)=t$ is 
itself a solution to some homogeneous differential equation.  In example 
\eqref{eq:undetcoeffex}, $g(t)=t$ satisfies the differential equation
\[
\frac{d^2y}{dt^2} = 0.
\]
It follows that any solution $x$ of \eqref{eq:undetcoeffex} has to satisfy
\begin{equation}  \label{e:undetc2}
0 = \frac{d^2}{dt^2}t=\frac{d^2}{dt^2}(\ddot{x} + 3\dot{x}+2x) = 
\frac{d^4x}{dt^4} + 3\frac{d^3x}{dt^3} + 2\frac{d^2x}{dt^2}
\end{equation}
The characteristic 
polynomial\index{characteristic polynomial!of higher order ODE}
of the homogeneous equation 
\eqref{e:undetc2} is: 
\[
\lambda^4 + 3\lambda^3 + 2\lambda^2 = \lambda^2(\lambda+1)(\lambda+2),
\]
and its zeros are
\[
\lambda_1=\lambda_2=0,\quad \lambda_3=-1,\quad \lambda_4 = -2.
\]
Hence, the general solution\index{general solution} of \eqref{e:undetc2} is
\[
x(t) = c_1 + c_2 t + c_3 e^{-t} + c_4 e^{-2t}.
\]

Since we want to find a particular solution\index{particular solution} 
of the inhomogeneous equation, 
we need not consider terms that are solutions of the homogeneous equation. 
That is, we can set $c_3=c_4=0$ and try to find a solution of the form
\[
x(t) = c_1 + c_2 t,
\]
which explains more precisely why our guess of a trial space worked.


\subsection*{The Method of Undetermined Coefficients}

The method used in the previous example works for many differential 
equations.  We use the notation for 
linear differential operators\index{linear!differential operator} 
developed in Section~\ref{S:LDO} to discuss how the previous example 
generalizes to a large family of equations.  In fact, we can find a 
particular solution of the $n^{th}$ order inhomogeneous differential equation 
\begin{equation}  \label{eq:nconst2}
p(D)x = g,
\end{equation}
where $g(t)$ is sufficiently differentiable and 
\[
p(D) = \frac{d^n}{dt^n} + a_{n-1}\frac{d^{n-1}}{dt^{n-1}} + \cdots + 
a_1\frac{d}{dt}+a_0,
\]
as follows.  We divide the process into three steps.

\paragraph{Step 1. Find an annihilator\index{annihilator} of $g(t)$.} 
Find a linear differential operator 
\[
q(D) = D^k + b_{k-1}D^{k-1} + \cdots + b_1D+b_0
\]
such that 
\begin{equation}  \label{eq:undetcoeffb}
q(D)g = 0.
\end{equation}
This differential operator is called the {\em annihilator\/} of $g$. 

\noindent {\bf Remark:}  When $g(t)$ is a linear combination of functions
it is often simpler to solve a separate equation for each function in the 
linear combination, as discussed in Lemma~\ref{L:inhsup}.

It follows that if $x(t)$ is a solution to the 
inhomogeneous\index{inhomogeneous} equation \eqref{eq:nconst2}, then $x(t)$ is 
also a solution to the homogeneous\index{homogeneous} equation 
\begin{equation}  \label{E:prodode}
q(D)p(D)x = q(D)g = 0.
\end{equation}
Note that the roots of the 
characteristic 
polynomial\index{characteristic polynomial!of higher order ODE} 
$pq$ for \eqref{E:prodode}
are just the union of the roots of $p$ and the roots of $q$.  

We could take the trial space\index{trial!space} to 
be the space of solutions to \eqref{E:prodode};
but, in general, that space is too large, as it contains all solutions to the 
homogeneous equation $p(D)x=0$.

\paragraph{Step 2: Find the trial space.} 
Compute the general solution to \eqref{E:prodode} and set to zero 
coefficients of solutions of 
the original homogeneous equation $p(D)x=0$, obtaining a subspace of trial 
functions\index{trial!functions}
\[
y(t)=c_1 y_1(t) + \cdots + c_k y_k(t).
\]
Note that if $p$ and $q$ have no roots in common, then the trial space is 
precisely the general solution of equation \eqref{eq:undetcoeffb}.  If $p$ 
and $q$ have common roots, then the situation is more complicated.  See 
\eqref{E:exer4} for an example.

\paragraph{Step 3: Find the particular solution.} 
\index{particular solution}
Substitute the trial function $y$ into \eqref{eq:nconst2} and find constants 
$c_1,\ldots,c_k$ so that $y$ is a particular solution to \eqref{eq:nconst2}.

\subsubsection*{When Undetermined Coefficients Works}

In fact, it is not always possible to satisfy Step 1.  In Step 1 we may not 
be able to find a constant coefficient homogeneous linear differential 
equation that has $g(t)$ as a solution.  However, Theorem~\ref{thm:HOgen} 
shows that all functions $g(t)$ that are 
linear combination of the functions\index{linear!combination!of functions}
\[
t^je^{\lambda t},\quad t^je^{\sigma t}\sin(\tau t),\quad 
t^je^{\sigma t}\cos(\tau t),
\]
for $j=0,1,\ldots$, are solutions to some homogeneous linear differential 
equation (perhaps of high order).  So we can use 
undetermined coefficients\index{undetermined coefficients} to 
find particular solutions for a large class of possible forcing terms $g$.  
And when this method can be used, it is relatively straightforward to implement.


\subsubsection*{A Second Example}

Consider the differential equation
\begin{equation}  \label{e:undet1}
\ddot x + 2\dot x + 2x = \cos(3t).
\end{equation}
The characteristic polynomial of \eqref{e:undet1} is 
$p(\lambda)=\lambda^2+2\lambda+2$; the associated eigenvalues are 
$\lambda=-1\pm i$.

\paragraph{Step 1.} The function $g(t)=\cos(3t)$ is a solution to the 
differential equation
\begin{equation}  \label{e:undet2}
\ddot y + 9y=0.
\end{equation}
So the differential operator $q(D)=D^2+9$ is an 
annihilator\index{annihilator} of $g(t)=\cos(3t)$. 

\paragraph{Step 2.} The roots of $q(\lambda)$ are $\pm 3i$ and they are 
distinct from those of $p(\lambda)$.  Hence the 
general solution\index{general solution} of \eqref{e:undet2}, 
\begin{equation}   \label{E:undet2}
y(t) = c_1 \cos(3t)+ c_2 \sin(3t),
\end{equation}
is the trial space in which to look for a 
particular solution\index{particular solution} of \eqref{e:undet1}.

\paragraph{Step 3.} Substituting \eqref{E:undet2} into \eqref{e:undet1} yields
\begin{eqnarray*}
-9(c_1\cos(3t)+c_2 \sin(3t)) +6(c_2\cos(3t)-c_1\sin(3t)) & & \\
+2(c_1 \cos(3t)+ c_2 \sin(3t)) &  = & \cos(3t).
\end{eqnarray*}
That is,
\[
(-7c_1+6c_2)\cos(3t)+(-6c_1-7c_2)\sin(3t) = \cos(3t).
\]
Hence we have found a particular solution when the coefficients $c_1$ and 
$c_2$ satisfy 
\begin{eqnarray*}
-7c_1 +6c_2 & = & 1\\
-6c_1 -7c_2 & = & 0.
\end{eqnarray*}
The solution of this system is 
\[
c_1 = -\frac{7}{85},\quad c_2 = \frac{6}{85}.
\]
Thus, 
\[
x_p(t) = \frac{1}{85}(6\sin(3t)-7\cos(3t))
\]
is a particular solution of \eqref{e:undet1}.

\subsubsection*{A Third Example Using Superposition}
\index{principle of superposition}

Find a particular solution of the differential equation
\begin{equation}  \label{e:underdet3}
\frac{d^3x}{dt^3} -\frac{dx}{dt} + 2x = e^{-2t}\sin t + 1.
\end{equation}
The characteristic polynomial of the homogeneous equation associated with 
\eqref{e:underdet3} is $p(\lambda)=\lambda^3-\lambda+2$ and the roots are 
(approximately) $-1.52, 0.76 \pm 0.86i$.  This (numerical) information is
found using \Matlab by typing 
{\tt roots([1 0 -1 2])}\index{\computer!roots}, obtaining
\begin{verbatim}
ans =
  -1.5214       
   0.7607 + 0.8579i     
   0.7607 - 0.8579i
\end{verbatim}

Using Lemma~\ref{L:inhsup} we can find a particular solution to 
\eqref{e:underdet3} by adding together solutions to 
\begin{eqnarray}
\frac{d^3x_1}{dt^3} -\frac{dx_1}{dt} + 2x_1 & = & 1 \label{e:underdet3-1}\\
\frac{d^3x_2}{dt^3} -\frac{dx_2}{dt} + 2x_2 & = & e^{-2t}\sin t 
\label{e:underdet3-2}
\end{eqnarray}

The solution $x_1$ to \eqref{e:underdet3-1} is found by inspection --- the 
annihilator\index{annihilator} of $g_1(t)=1$ is just $D$, 
and the trial space\index{trial!space} consists of 
the constants.  By inspection, the answer is 
\[
x_1(t) = \frac{1}{2}.
\]

To solve \eqref{e:underdet3-2} for $x_2$ we proceed with the three steps 
associated with undetermined coefficients.

\paragraph{Step 1.} The right hand side of \eqref{e:underdet3-2}, 
$g_2(t)=e^{-2t}\sin t$, is a solution of the linear differential equation 
whose characteristic polynomial has roots $-2\pm i$.  Thus, an 
annihilator\index{annihilator} for $g_2$ is:
\begin{equation}  \label{eq:undetcoeffex2}
\frac{d^2y}{dt^2} +4\frac{dy}{dt} +5y = 0.
\end{equation}
To verify this point, observe that the characteristic 
polynomial\index{characteristic polynomial!of higher order ODE} of  
\eqref{eq:undetcoeffex2} is
\[
q(\lambda) = \lambda^2 + 4\lambda +5,
\]
which has roots $-2\pm i$.  

\paragraph{Step 2.} Since the roots of $p$ and $q$ are disjoint sets, the 
trial space\index{trial!space} is the general solution of 
\eqref{eq:undetcoeffex2}, namely,  
\[
y(t) = c_1e^{-2t}\cos t + c_2 e^{-2t}\sin t.
\]
We look for a particular solution to \eqref{e:underdet3-2} in this function 
subspace.

\paragraph{Step 3.} We compute
\begin{eqnarray*}
Dy   & = & e^{-2t}\Big((-2c_1+  c_2)\cos t - (  c_1+2c_2)\sin t\Big)\\
D^2y & = & e^{-2t}\Big(( 3c_1- 4c_2)\cos t + ( 4c_1+3c_2)\sin t\Big)\\
D^3y & = & e^{-2t}\Big((-2c_1+11c_2)\cos t - (11c_1+2c_2)\sin t\Big).
\end{eqnarray*}
Substituting into \eqref{e:underdet3-2} leads to 
\[
e^{-2t}\Big( (2c_1+10c_2)\cos t +(-10c_1+2c_2)\sin t\Big) = 
e^{-2t}\sin t.
\]
Hence $c_1$, $c_2$ satisfy
\begin{eqnarray*}
  2c_1 +10c_2 & = & 0\\
-10c_1 + 2c_2 & = & 1.
\end{eqnarray*}
The solution is $c_1=-\frac{5}{52}$ and $c_2=\frac{1}{52}$, and
the particular solution\index{particular solution} to \eqref{e:underdet3} 
is
\[
x_p(t) = \frac{1}{2} - \frac{5}{52}e^{-2t}\cos t + \frac{1}{52}e^{-2t}\sin t.
\]


\subsubsection*{An Example where $p$ and $q$ have Common Roots}

Consider the first order differential equation
\begin{equation}  \label{E:exer4}
\frac{dx}{dt} - x = e^t.
\end{equation}
The characteristic polynomial of the homogeneous equation is 
$p(\lambda)=\lambda-1$ whose root is $\lambda=1$.  An 
annihilator\index{annihilator} of 
$g(t)=e^t$ is $q(D)=D-1$.  Hence $q(\lambda)$ also has the root $\lambda=1$.
Thus to solve this differential equation by 
undetermined coefficients\index{undetermined coefficients} we
apply $q(D)$ to both sides of the equation, obtaining
\[
q(D) \left(\frac{dx}{dt} - x\right) = \ddot{x} -2\dot{x} +x =0.
\]
Since $\lambda=1$ is a double root for the characteristic polynomial of this 
equation, the general solution is:
\[
x(t) = c_1e^t + c_2te^t.
\]
Setting to zero the solution to the original homogeneous equation (that is, 
setting $c_1=0$), we find that the trial space\index{trial!space}
for the inhomogeneous equation 
\eqref{E:exer4} is
\[
y(t) = c_2te^t.
\] 
Substituting $y(t)$ into \eqref{E:exer4} yields
\[
c_2(e^t + te^t) - c_2te^t = e^t.
\] 
It follows that $c_2=1$ and that $x_p(t) = te^t$ is a particular solution. 
The general solution to \eqref{E:exer4} is:
\[
x(t) = \alpha e^t + te^t.
\]


\EXER

\TEXER

\noindent In Exercises~\ref{c12.4.1a} -- \ref{c12.4.1d} find annihilators 
for each of the given functions.
\begin{exercise}  \label{c12.4.1a}
$g(t) = e^{4t}\cos(5t)$.

\begin{solution}
\ans The equation
\[
\frac{d^2x}{dt^2} - 8\frac{dx}{dt} + 41x = 0
\]
is an annihilator for $g(t)$.

\soln The function $g(t) = e^{4t}\cos(5t)$ is a solution to any differential
equation which has $\lambda_1 = 4 \pm 5i$ as an eigenvalue.  The
simplest characteristic polynomial with $\lambda_1$ as a root is
$\lambda^2 - 8\lambda + 41$.

\end{solution}
\end{exercise}
\begin{exercise}  \label{c12.4.1b}
$g(t) = e^t-1$.

\begin{solution}
\ans The equation
\[
\frac{d^2x}{dt^2} - \frac{dx}{dt} = 0
\]
is an annihilator for $g(t)$.

\soln The function $g(t) = e^t - 1$ is a solution to any differential equation
which has $\lambda_1 = 1$ and $\lambda_2 = 0$ as eigenvalues.  The
simplest characteristic polynomial with $\lambda_1$ and $\lambda_2$ as
roots is $\lambda^2 - \lambda$.

\end{solution}
\end{exercise}
\begin{exercise}  \label{c12.4.1c}
$g(t) = t^2e^{2t}$.

\begin{solution}
\ans The equation
\[
\frac{d^3x}{dt^3} - 6\frac{d^2x}{dt^2} + 12\frac{dx}{dt} - 8x = 0
\]
is an annihilator for $g(t)$.

\soln The function $g(t) = t^2e^{2t}$ is a solution to any differential
equation which has $\lambda_1 = 2$ as a triple eigenvalue.  The simplest
characteristic polynomial with $\lambda_1$ as a triple root is
$(\lambda - 2)^3 = \lambda^3 - 6\lambda^2 + 12\lambda - 8$.

\end{solution}
\end{exercise}
\begin{exercise}  \label{c12.4.1d}
$g(t) = t\cos(5t)+5\cos(t)$.

\begin{solution}
\ans The equation
\[
\frac{d^6x}{dt^6} + 51\frac{d^4x}{dt^4} + 675\frac{d^2x}{dt^2} + 625x = 0
\]
is an annihilator for $g(t)$.

\soln The function $g(t) = t\cos(5t) + 5\cos t$ is a solution to any
differential equation which has $\lambda_1 = \pm i$ as an eigenvalue and
$\lambda_2 = \pm 5i$ as a double eigenvalue.  The simplest characteristic
polynomial with these roots is
$(\lambda^2 + 1)(\lambda^2 + 25)^2 = \lambda^6 + 51\lambda^4 + 675\lambda^2
+ 625$.

\end{solution}
\end{exercise}

\noindent In Exercises~\ref{c12.4.1} -- \ref{c12.4.7} use the method of 
undetermined coefficients to find particular solutions to the given 
differential equations.
\begin{exercise}  \label{c12.4.1}
$(D^2-3D+2)x = \sin(t)$.

\begin{solution}
\ans One solution to the differential equation is
\[
x(t) = \frac{1}{3}\cos t + \frac{1}{6}\sin t.
\]

\soln
\paragraph{Step 1.} First, find an annihilator for $g(t) = \sin t$.  Since
$\sin t$ is a solution to any homogeneous equation with eigenvalue
$\lambda = \pm i$, the differential equation $q(D) = D^2 + 1$ is an
annihilator.

\paragraph{Step 2.} Next, find the trial space of solutions for the
differential equation.  The homogeneous equation $\ddot{x} - 3\dot{x}
+ 2 = 0$ has eigenvalues $2$ and $3$.  These eigenvalues are distinct
from $\lambda$, so the trial space is the space of solutions to $q(D)x
= 0$, which is
\[
y(t) = c_1\cos t + c_2\sin t.
\]
\paragraph{Step 3.} Now substitute $y(t)$ into the differential equation:
\[
(D^2 - 3D + 2)(y(t)) = -(c_1\sin t + c_2\cos t) + 3(c_1\sin t - c_2\cos t)
+ 2(c_1\cos t + c_2\sin t) = \sin t.
\]
Solve this system for $c_1$ and $c_2$ to find that $y(t)$ is a solution to
the differential equation when $c_1 = \frac{1}{3}$ and $c_2 = \frac{1}{6}$.

\end{solution}
\end{exercise}
\begin{exercise}  \label{c12.4.2}
$\ddot{x}+2\dot{x}+x = t + e^t$.

\begin{solution}
\ans One solution to the differential equation is
\[
x(t) = \frac{1}{4}e^t + t - 2.
\]

\soln Let $g(t) = t$ and $h(t) = e^t$.  If $p(D) = D^2 + 2D + 1$, then the
sum of solutions to $p(D)x = g(t)$ and $p(D)x = h(t)$ is a solution to
$p(D)x = g(t) + h(t)$.  First find a solution to $p(D)x = g(t)$:

\paragraph{Step 1.} Since $t$ is a solution to any homogeneous equation
with a double eigenvalue at $\lambda = 0$, the differential equation
$q_1(D) = D^2$ is an annihilator.

\paragraph{Step 2.} The homogeneous equation $p(D)x = 0$ has a double
eigenvalue at $-1$.  This eigenvalue is distinct from $\lambda$, so
the trial space is the space of solutions to $q_1(D)x = 0$, which is
\[
y(t) = c_1 + c_2t.
\]
\paragraph{Step 3.} Now substitute $y(t)$ into $p(D)x = g(t)$:
\[
(D^2 + D2 + 1)(y(t)) = 2c_2 + (c_1 + c_2)t = g(t) = t.
\]
Solve this system for $c_1$ and $c_2$ to find that $y(t)$ is a solution to
the differential equation when $c_1 = -2$ and $c_2 = 1$.

\para Now find a solution for $p(D)x = h(t)$:

\paragraph{Step 1.} In this case, since $e^t$ is a solution to any
homogeneous equation with an eigenvalue at $\mu = 1$, the differential
equation $q_2(D) = D - 1$ is an annihilator.

\paragraph{Step 2.} The eigenvalue $\mu$ is distinct from the eigenvalues
of $p(D)$, so the trial space is the space of solutions to $q_2(D)x =
0$, which is
\[
z(t) = d_1e^t.
\]
\paragraph{Step 3.} Now substitute $z(t)$ into $p(D)x = h(t)$:
\[
(D^2 + D2 + 1)(z(t)) = d_1e^t + 2d_1e^t + d_1e^t = h(t) = e^t.
\]
Solve this system for $d_1$ to find that $z(t)$ is a solution to
the differential equation when $d_1 = \frac{1}{4}$.  Now, $x(t) = y(t) + z(t)$
is a solution to the differential equation $p(D)x = t + e^t$.


\end{solution}
\end{exercise}
\begin{exercise}  \label{c12.4.3}
$(D^3+6D^2+9D+4)x = e^{-t}$.

\begin{solution}
\ans One solution to the differential equation is
\[
x(t) = \frac{1}{6}t^2e^{-t}.
\]

\soln Let $p(D) = D^3 + 6D^2 + 9D + 4$, and let $g(t) = e^{-t}$.
\paragraph{Step 1.} An annihilator for $g(t)$ is $q(D) = D + 1$, since
$g(t)$ is a solution to any homogeneous equation with an eigenvalue at
$\mu = -1$.

\paragraph{Step 2.} The equation $p(D)x = 0$ has an eigenvalue at
$\lambda_1 = -4$, and a double eigenvalue at $\lambda_2 = -1$.  Since
$\mu = \lambda_2$, we find the trial space by applying $q(D)$ to both
sides of $p(D)x = g(t)$:
\[
q(D)(D^3 + 6D^2 + 9D + 4) = D^4 + 7D^3 + 15D^2 + 13D + 4 = 0.
\]
The general solution to this differential equation is
\[
x(t) = c_1e^{-4t} + c_2e^{-t} + c_3te^{-t} + c_4t^2e^{-t}.
\]
Since any solution to the homogeneous equation is equal to zero, we can set
$c_1 = c_2 = c_3 = 0$.  So the trial space is
\[
y(t) = c_4t^2e^{-t}.
\]
\paragraph{Step 3.} Substitute $y(t)$ into $p(D)x = g(t)$:
\[
p(D)(y(t)) = \frac{d^3x}{dt^3}(c_4t^2e^{-t}) +
6\frac{d^2x}{dt^2}(c_4t^2e^{-t}) + 9\frac{dx}{dt}(c_4t^2e^{-t}) +
4(c_4t^2e^{-t}) = 6c_4e^{-t} =  e^{-t}.
\]
Solve this system for $c_4$ to find that $y(t)$ is a solution to the
differential equation when $c_4 = \frac{1}{6}$.

\end{solution}
\end{exercise}
\begin{exercise}  \label{c12.4.4}
$(D^2+D-2)x = 3te^t$.

\begin{solution}
\ans One solution to the differential equation is
\[
x(t) = \frac{3}{4}t^2e^t - \frac{1}{2}te^t.
\]
\soln Let $p(D) = D^2 + D - 2$ and let $g(t) = 3te^t$.

\paragraph{Step 1.} An annihilator for $g(t)$ is
\[
q(D) = (D - 1)^2 = D^2 - 2D + 1,
\]
since $g(t)$ is a solution to any homogeneous differential equation
with a double eigenvalue at $\mu = 1$.

\paragraph{Step 2.} The equation $p(D)x = 0$ has eigenvalues at
$\lambda_1 = -2$ and $\lambda_2 = 1$.  Since $\mu = \lambda_2$, we
find the trial space by applying $q(D)$ to both sides of $p(D)x =
g(t)$:
\[
q(D)(D^2 + D - 2) = D^4 - D^3 - 3D^2 + 5D - 2 = 0.
\]
The general solution to this differential equation is
\[
x(t) = c_1e^{-2t} + c_2e^t + c_3te^t + c_4t^2e^t.
\]
Since any solution to the homogeneous equation is equal to zero, we can set
$c_1 = c_2 = 0$.  So the trial space is
\[
y(t) = c_3te^t + c_4t^2e^t.
\]
\paragraph{Step 3.} Substitute $y(t)$ into $p(D)x = g(t)$, obtaining
\[
c_3(3e^t) + c_4(2e^t + 4te^t) = 3te^t.
\]
Solve this system for $c_3$ and $c_4$ to find that $y(t)$ is a solution
to the differential equation when $c_3 = -\frac{1}{2}$ and
$c_4 = \frac{3}{4}$.

\end{solution}
\end{exercise}
\begin{exercise}  \label{c12.4.5}
$\ddot{x}+x = t\sin t$.

\begin{solution}
\ans One solution to the differential equation is
\[
x(t) = \frac{1}{4}(t\sin t- t^2\cos t).
\]

\soln Let $p(D) = D^2 + 1$ and let $g(t) = t\sin t$.
\paragraph{Step 1.} An annihilator for $g(t)$ is
\[
q(D) = (D^2 + 1)^2 = D^4 + 2D + 1,
\]
since $g(t)$ is a solution to any homogeneous differential equation
with a double eigenvalue at $\mu = \pm i$.

\paragraph{Step 2.} The equation $p(D)x = 0$ has eigenvalue $\lambda =
\pm i$.  Since $\mu = \lambda$, we find the trial space by applying
$q(D)$ to both sides of $p(D)x = g(t)$:
\[
q(D)(D^2 + 1) = D^6 + 2D^4 + 3D^2 + 1 = 0.
\]
The general solution to this differential equation is
\[
x(t) = c_1\cos t + c_2\sin t + c_3t\cos t + c_4t\sin t + c_5t^2\cos t
+ c_6t^2\sin t.
\]
Since any solution to the homogeneous equation is equal to zero, we can
set $c_1 = c_2 = 0$.  So the trial space is
\[
y(t) = c_3t\cos t + c_4t\sin t + c_5t^2\cos t + c_6t^2\sin t.
\]
\paragraph{Step 3.} Substitute $y(t)$ into $p(D)x = g(t)$, obtaining
\[
c_3(-2\sin t) + c_4(2\cos t) + c_5(2\cos t - 4t\sin t) +
c_6(2\sin t + 4t\cos t) = t\sin t.
\]
Solve this system for the scalars $c_j$ to find that $y(t)$ is a solution
to the differential equation when $c_3 = 0$, $c_4 = \frac{1}{4}$,
$c_5 = -\frac{1}{4}$, and $c_6 = 0$.

\end{solution}
\end{exercise}
\begin{exercise}  \label{c12.4.6}
$(D^3+D)x = 6t^2+\sin t$.

\begin{solution}
\ans One solution to the differential equation is
\[
x(t) = 2t^3 - 12t - \frac{1}{2}t\sin t.
\]

\soln Let $p(D) = D^3 + D$ and let $g(t) = 6t^2$ and $h(t) = \sin t$.  Then
the sum of solutions to $p(D)x = g(t)$ and $p(D)x = h(t)$ is a solution to
$p(D)x = g(t) + h(t)$.  So, first find a solution for $p(D)x = g(t)$:
\paragraph{Step 1.} An annihilator for $g(t)$ is
\[
q_1(D) = D^3,
\]
since $g(t)$ is a solution to any homogeneous differential equation
with a triple eigenvalue at $\mu_1 = 0$.

\paragraph{Step 2.} The equation $p(D)x = 0$ has eigenvalues
$\lambda_1 = 0$ and $\lambda_2 = \pm i$.  Since $\lambda_1 = \mu_1$,
we find the trial space by applying $q_1(D)$ to both sides of $p(D)x =
g(t)$:
\[
q_1(D)(D^3 + D) = D^6 + D^4 = 0.
\]
The general solution to this differential equation is
\[
x(t) = c_1 + c_2t + c_3t^2 + c_4t^3 + c_5\cos t + c_6\sin t.
\]
Since any solution to the homogeneous equation is equal to zero, we can
set $c_1 = c_5 = c_6 = 0$.  So the trial space is
\[
y(t) = c_2t + c_3t^2 + c_4t^3.
\]
\paragraph{Step 3.} Substitute $y(t)$ into $p(D)x = g(t)$, obtaining
\[
c_2 + c_3(3t^2) + c_4(t^2 + 6) = 6t^2.
\]
Solve this system for the scalars $c_j$ to find that $y(t)$ is a solution
to $p(D)x = g(t)$ when $c_2 = -12$, $c_3 = 0$, and $c_4 = 2$.

\para Now find a solution for $p(D)x = h(t)$:

\paragraph{Step 1.} In this case, $q_2(D) = D^2 + 1$ is an annihilator,
since $h(t)$ is a solution for any homogeneous differential equation
with an eigenvalue at $\mu_2 = \pm i$.

\paragraph{Step 2.} Since $\lambda_2 = \mu_2$, we find the trial space
by applying $q_2(D)$ to both sides of $p(D)x = h(t)$:
\[
q_2(D)(D^3 + D) = D^5 + 2D^3 + D = 0.
\]
The general solution to this differential equation is
\[
x(t) = d_1 + d_2\cos t + d_3\sin t + d_4t\cos t + d_5t\sin t.
\]
Since any solution to the homogeneous equation is equal to zero, we can
set $d_1 = d_2 = d_3 = 0$.  So the trial space is
\[
z(t) = d_4t\cos t + d_5t\sin t.
\]
\paragraph{Step 3.} Substitute $z(t)$ into $p(D)x = h(t)$, obtaining
\[
d_4(-4\cos t) + d_5(-2\sin t) = \sin t.
\]
Solve this system for $d_4$ and $d_5$ to find that $z(t)$ is a solution
to $p(D)x = h(t)$ when $d_4 = 0$ and $d_5 = -\frac{1}{2}$.  So $x(t) =
y(t) + z(t)$ is a solution to $(D^3 + D)x = 6t^2 + \sin t$.

\end{solution}
\end{exercise}
\begin{exercise}  \label{c12.4.7}
$(D^2+2D+2)x = 8e^{-t}\sin t$.

\begin{solution}
\ans One solution to the differential equation is
\[
x(t) = -4te^{-t}\cos t.
\]

\soln Let $p(D) = D^2 + 2D + 2$ and let $g(t) = 8e^{-t}\sin t$.
\paragraph{Step 1.} An annihilator for $g(t)$ is
\[
q(D) = D^2 + 2D + 2,
\]
since $g(t)$ is a solution to any homogeneous differential equation
with an eigenvalue at $\mu = -1 \pm i$.

\paragraph{Step 2.} The equation $p(D)x = 0$ has eigenvalue
$\lambda = -1 \pm i$.  Since $\mu = \lambda$, we find the trial space
by applying $q(D)$ to both sides of $p(D)x = g(t)$:
\[
q(D)(D^2 + 2D + 1) = D^4 + 4D^3 + 8D^2 + 8D + 4	= 0.
\]
The general solution to this differential equation is
\[
x(t) = c_1e^{-t}\cos t + c_2e^{-t}\sin t + c_3te^{-t}\cos t +
c_4te^{-t}\sin t.
\]
Since any solution to the homogeneous equation is equal to zero, we can
set $c_1 = c_2 = 0$.  So the trial space is
\[
y(t) = c_3te^{-t}\cos t + c_4te^{-t}\sin t.
\]
\paragraph{Step 3.} Substitute $y(t)$ into $p(D)x = g(t)$, obtaining
\[
c_3(-2e^{-t}\sin t) + c_4(2e^{-t}\cos t) = 8e^{-t}\sin t.
\]
Solve this system for $c_3$ and $c_4$ to find that $y(t)$ is a solution
to the differential equation when $c_3 = -4$ and $c_4 = 0$.

\end{solution}
\end{exercise}

\noindent In Exercises~\ref{c12.4.8}  -- \ref{c12.4.11} solve the given 
initial value problems.
\begin{exercise}  \label{c12.4.8}
$\ddot{x}-x = 1$ where $x(0)=(Dx)(0)=0$.

\begin{solution}
\ans The solution to the initial value problem is
\[
x(t) = \frac{1}{2}e^t + \frac{1}{2}e^{-t} - 1.
\]

\soln Let $p(D) = D^2 - 1$ and let $g(t) = 1$.  To find the general solution
to the inhomogeneous equation $p(D)x = g(t)$, first use the method of
undetermined coefficients to find one solution to the equation, then
add that solution to the general solution of the homogeneous system.

\paragraph{Step 1.} Since $g(t)$ is a solution to any homogeneous system
with an eigenvalue at $\mu = 0$, $q(D) = D$ is an annihilator for
$g(t)$.

\paragraph{Step 2.} The homogeneous equation $p(D)x = 0$ has eigenvalues
$\lambda_1 = 1$ and $\lambda_2 = -1$.  These eigenvalues are distinct,
so the trial space of solutions is the general solution to $q(D)x =
0$, which is
\[
y(t) = c_1.
\]
\paragraph{Step 3.} Substitute $y(t)$ into $p(D)x = g(t)$, obtaining
\[
-c_1 = 1.
\]
Thus, $y(t)$ is a solution to the differential equation when $c_1 = -1$.

\para The general solution to $p(D)x = 0$ is
\[
z(t) = \alpha_1e^t + \alpha_2e^{-t}.
\]
Thus, the general solution to the inhomogeneous system is
\[
x(t) = y(t) + z(t) = \alpha_1e^t + \alpha_2e^{-t} - 1.
\]
Now, substitute the initial conditions into $x(t)$, obtaining
\[
\begin{array}{rcl}
0 & = & x(0) = \alpha_1 + \alpha_2 - 1 \\
0 & = & (Dx)(0) = \alpha_1 - \alpha_2.
\end{array}
\]
Solve this system for $\alpha_1$ and $\alpha_2$, obtaining
$\alpha_1 = \frac{1}{2}$ and $\alpha_2 = \frac{1}{2}$.

\end{solution}
\end{exercise}
\begin{exercise}  \label{c12.4.9}
$(D^3-8)x = e^{2t}$ where $x(0)=(Dx)(0)=(D^2x)(0)=0$.

\begin{solution}
\ans The solution to the initial value problem is
\[
x(t) = \frac{1}{72}(-3e^{2t} + 6te^{2t} + 3e^{-t}\cos(\sqrt{3}t)
+ \sqrt{3}e^{-t}\sin(\sqrt{3}t)).
\]

\soln Let $p(D) = D^3 - 8$ and let $g(t) = e^{2t}$.  To find the general
solution to the inhomogeneous equation $p(D)x = g(t)$, first use the
method of undetermined coefficients to find one solution to the
equation, then add that solution to the general solution of the
homogeneous system.

\paragraph{Step 1.} Since $g(t)$ is a solution to any homogeneous
system with an eigenvalue at $\mu = 2$, $q(D) = D - 2$ is an annihilator
for $g(t)$.

\paragraph{Step 2.} The homogeneous equation $p(D)x = 0$ has eigenvalues
$\lambda_1 = 2$ and $\lambda)2 = -1 \pm \sqrt{3}$.  Since $\lambda_1 =
\mu$, we find the trial space by applying $q(D)$ to both sides of
$p(D)x = g(t)$:
\[
q(D)(D^3 - 8) = D^4 - 2D^3 - 8D + 16 = 0.
\]
The general solution to this equation is
\[
x(t) = c_1e^{2t} + c_2te^{2t} + c_3e^{-t}\cos(\sqrt{3}t)
+ c_4e^{-t}\sin(\sqrt{3}t).
\]
Since any solution to the homogeneous equation is equal to zero, we can
set $c_1 = c_3 = c_4 = 0$.  Thus, the trial space is
\[
y(t) = c_2te^{2t}.
\]
\paragraph{Step 3.} Substitute $y(t)$ into $p(D)x = g(t)$, obtaining
\[
12c_2e^{2t} = e^2t.
\]
So $y(t)$ is a solution to the differential equation when $c_2 = \frac{1}{12}$.

\para The general solution to $p(D)x = 0$ is
\[
z(t) = c_1e^{2t} + c_3e^{-t}\cos(\sqrt{3}t) + c_4e^{-t}\sin(\sqrt{3}t).
\]
Thus the general solution to the inhomogeneous system is
\[
x(t) = y(t) + z(t) = c_1e^{2t} + \frac{1}{12}te^{2t}
+ c_3e^{-t}\cos(\sqrt{3}t) + c_4e^{-t}\sin(\sqrt{3}t).
\]
Substitute the initial conditions into $x(t)$, obtaining
\[
\begin{array}{rcl}
0 & = & x(0) = c_1 + c_3 \\
0 & = & (Dx)(0) = 2c_1 + \frac{1}{12} - c_3 + \sqrt{3}c_4 \\
0 & = & (D^2x)(0) = 4c_1 + \frac{1}{3} - 2c_3 - 2\sqrt{3}c_4.
\end{array}
\]
Solve this system to obtain $c_1 = -\frac{1}{24}$, $c_3 = \frac{1}{24}$,
and $c_4 = \frac{\sqrt{3}}{72}$.

\end{solution}
\end{exercise}
\begin{exercise}  \label{c12.4.10}
$(D^2+D-6)x = 0$ where $x(0)=(Dx)(0)=0$.

\begin{solution}
\ans The solution to the initial value problem is
\[
x(t) = 0.
\]
\soln Let $p(D) = D^2 + D - 6$.  Then $p(D)x = 0$, $x(0) = 0$, and
$(Dx)(0) = 0$.  So, by observation, $x(t) = 0$ is a solution to the
initial value problem.


\end{solution}
\end{exercise}
\begin{exercise}  \label{c12.4.11}
$(D^2-1)x = 1$ where $x(1)=(Dx)(1)=0$.

\begin{solution}
\ans
\[
x(t) = \frac{1}{2}e^{t - 1} + \frac{1}{2}e^{-t + 1} - 1.
\]

\soln The inhomogeneous system is identical to the system in
Exercise~\ref{c12.4.8}.  By the same procedure used in that exercise,
the general solution is
\[
x(t) = \alpha_1e^t + \alpha_2e^{-t} - 1.
\]
Substitute the initial conditions into $x(t)$, obtaining
\[
\begin{array}{rcl}
0 & = & x(1) = e\alpha_1 + e^{-1}\alpha_2 - 1 \\
0 & = & (Dx)(1) = e\alpha_1 - e^{-1}\alpha_2.
\end{array}
\]
Solve this system to find $\alpha_1 = \frac{1}{2}e^{-1}$ and
$\alpha_2 = \frac{1}{2}e$.




\end{solution}
\end{exercise}


\end{document}
