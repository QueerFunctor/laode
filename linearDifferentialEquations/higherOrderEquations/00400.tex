\documentclass{ximera}
 

\usepackage{epsfig}

\graphicspath{
  {./}
  {figures/}
}

\usepackage{morewrites}
\makeatletter
\newcommand\subfile[1]{%
\renewcommand{\input}[1]{}%
\begingroup\skip@preamble\otherinput{#1}\endgroup\par\vspace{\topsep}
\let\input\otherinput}
\makeatother

\newcommand{\includeexercises}{\directlua{dofile("/home/jim/linearAlgebra/laode/exercises.lua")}}

%\newcounter{ccounter}
%\setcounter{ccounter}{1}
%\newcommand{\Chapter}[1]{\setcounter{chapter}{\arabic{ccounter}}\chapter{#1}\addtocounter{ccounter}{1}}

%\newcommand{\section}[1]{\section{#1}\setcounter{thm}{0}\setcounter{equation}{0}}

%\renewcommand{\theequation}{\arabic{chapter}.\arabic{section}.\arabic{equation}}
%\renewcommand{\thefigure}{\arabic{chapter}.\arabic{figure}}
%\renewcommand{\thetable}{\arabic{chapter}.\arabic{table}}

%\newcommand{\Sec}[2]{\section{#1}\markright{\arabic{ccounter}.\arabic{section}.#2}\setcounter{equation}{0}\setcounter{thm}{0}\setcounter{figure}{0}}

\newcommand{\Sec}[2]{\section{#1}}

\setcounter{secnumdepth}{2}
%\setcounter{secnumdepth}{1} 

%\newcounter{THM}
%\renewcommand{\theTHM}{\arabic{chapter}.\arabic{section}}

\newcommand{\trademark}{{R\!\!\!\!\!\bigcirc}}
%\newtheorem{exercise}{}

\newcommand{\dfield}{{\sf dfield9}}
\newcommand{\pplane}{{\sf pplane9}}

\newcommand{\EXER}{\section*{Exercises}}%\vspace*{0.2in}\hrule\small\setcounter{exercise}{0}}
\newcommand{\CEXER}{}%\vspace{0.08in}\begin{center}Computer Exercises\end{center}}
\newcommand{\TEXER}{} %\vspace{0.08in}\begin{center}Hand Exercises\end{center}}
\newcommand{\AEXER}{} %\vspace{0.08in}\begin{center}Hand Exercises\end{center}}

% BADBAD: \newcommand{\Bbb}{\bf}

\newcommand{\R}{\mbox{$\Bbb{R}$}}
\newcommand{\C}{\mbox{$\Bbb{C}$}}
\newcommand{\Z}{\mbox{$\Bbb{Z}$}}
\newcommand{\N}{\mbox{$\Bbb{N}$}}
\newcommand{\D}{\mbox{{\bf D}}}
\usepackage{amssymb}
%\newcommand{\qed}{\hfill\mbox{\raggedright$\square$} \vspace{1ex}}
%\newcommand{\proof}{\noindent {\bf Proof:} \hspace{0.1in}}

\newcommand{\setmin}{\;\mbox{--}\;}
\newcommand{\Matlab}{{M\small{AT\-LAB}} }
\newcommand{\Matlabp}{{M\small{AT\-LAB}}}
\newcommand{\computer}{\Matlab Instructions}
\newcommand{\half}{\mbox{$\frac{1}{2}$}}
\newcommand{\compose}{\raisebox{.15ex}{\mbox{{\scriptsize$\circ$}}}}
\newcommand{\AND}{\quad\mbox{and}\quad}
\newcommand{\vect}[2]{\left(\begin{array}{c} #1_1 \\ \vdots \\
 #1_{#2}\end{array}\right)}
\newcommand{\mattwo}[4]{\left(\begin{array}{rr} #1 & #2\\ #3
&#4\end{array}\right)}
\newcommand{\mattwoc}[4]{\left(\begin{array}{cc} #1 & #2\\ #3
&#4\end{array}\right)}
\newcommand{\vectwo}[2]{\left(\begin{array}{r} #1 \\ #2\end{array}\right)}
\newcommand{\vectwoc}[2]{\left(\begin{array}{c} #1 \\ #2\end{array}\right)}

\newcommand{\ignore}[1]{}


\newcommand{\inv}{^{-1}}
\newcommand{\CC}{{\cal C}}
\newcommand{\CCone}{\CC^1}
\newcommand{\Span}{{\rm span}}
\newcommand{\rank}{{\rm rank}}
\newcommand{\trace}{{\rm tr}}
\newcommand{\RE}{{\rm Re}}
\newcommand{\IM}{{\rm Im}}
\newcommand{\nulls}{{\rm null\;space}}

\newcommand{\dps}{\displaystyle}
\newcommand{\arraystart}{\renewcommand{\arraystretch}{1.8}}
\newcommand{\arrayfinish}{\renewcommand{\arraystretch}{1.2}}
\newcommand{\Start}[1]{\vspace{0.08in}\noindent {\bf Section~\ref{#1}}}
\newcommand{\exer}[1]{\noindent {\bf \ref{#1}}}
\newcommand{\ans}{}
\newcommand{\matthree}[9]{\left(\begin{array}{rrr} #1 & #2 & #3 \\ #4 & #5 & #6
\\ #7 & #8 & #9\end{array}\right)}
\newcommand{\cvectwo}[2]{\left(\begin{array}{c} #1 \\ #2\end{array}\right)}
\newcommand{\cmatthree}[9]{\left(\begin{array}{ccc} #1 & #2 & #3 \\ #4 & #5 &
#6 \\ #7 & #8 & #9\end{array}\right)}
\newcommand{\vecthree}[3]{\left(\begin{array}{r} #1 \\ #2 \\
#3\end{array}\right)}
\newcommand{\cvecthree}[3]{\left(\begin{array}{c} #1 \\ #2 \\
#3\end{array}\right)}
\newcommand{\cmattwo}[4]{\left(\begin{array}{cc} #1 & #2\\ #3
&#4\end{array}\right)}

\newcommand{\Matrix}[1]{\ensuremath{\left(\begin{array}{rrrrrrrrrrrrrrrrrr} #1 \end{array}\right)}}

\newcommand{\Matrixc}[1]{\ensuremath{\left(\begin{array}{cccccccccccc} #1 \end{array}\right)}}



\renewcommand{\labelenumi}{\theenumi)}
\newenvironment{enumeratea}%
{\begingroup
 \renewcommand{\theenumi}{\alph{enumi}}
 \renewcommand{\labelenumi}{(\theenumi)}
 \begin{enumerate}}
 {\end{enumerate}\endgroup}



\newcounter{help}
\renewcommand{\thehelp}{\thesection.\arabic{equation}}

%\newenvironment{equation*}%
%{\renewcommand\endequation{\eqno (\theequation)* $$}%
%   \begin{equation}}%
%   {\end{equation}\renewcommand\endequation{\eqno \@eqnnum
%$$\global\@ignoretrue}}

%\input{psfig.tex}

\author{Martin Golubitsky and Michael Dellnitz}

%\newenvironment{matlabEquation}%
%{\renewcommand\endequation{\eqno (\theequation*) $$}%
%   \begin{equation}}%
%   {\end{equation}\renewcommand\endequation{\eqno \@eqnnum
% $$\global\@ignoretrue}}

\newcommand{\soln}{\textbf{Solution:} }
\newcommand{\exercap}[1]{\centerline{Figure~\ref{#1}}}
\newcommand{\exercaptwo}[1]{\centerline{Figure~\ref{#1}a\hspace{2.1in}
Figure~\ref{#1}b}}
\newcommand{\exercapthree}[1]{\centerline{Figure~\ref{#1}a\hspace{1.2in}
Figure~\ref{#1}b\hspace{1.2in}Figure~\ref{#1}c}}
\newcommand{\para}{\hspace{0.4in}}

\renewenvironment{solution}{\suppress}{\endsuppress}

\ifxake
\newenvironment{matlabEquation}{\begin{equation}}{\end{equation}}
\else
\newenvironment{matlabEquation}%
{\let\oldtheequation\theequation\renewcommand{\theequation}{\oldtheequation*}\begin{equation}}%
  {\end{equation}\let\theequation\oldtheequation}
\fi

\makeatother

\begin{document}






\noindent In Exercises~\ref{c12.2.6} -- \ref{c12.2.11}, use \Matlab to find 
the roots of the characteristic polynomial for the given homogeneous linear 
differential equation, and then find the general solution to that equation.
\begin{computerExercise} \label{c12.2.6}
$\frac{d^3x}{dt^3} -6\frac{d^2x}{dt^2}+11\frac{dx}{dt}-6x=0$

\begin{solution}
\ans The general solution to the differential equation is
\[
x(t) = \alpha_1e^{3t} + \alpha_2e^{2t} + \alpha_3e^t.
\]

\soln Using \Matlab, find that the characteristic polynomial has real
eigenvalues at $\lambda_1 = 3$, $\lambda_2 = 2$, and $\lambda_3 = 1$.  So
all solutions are linear combinations of $e^{\lambda_1 t} = e^{3t}$,
$e^{\lambda_2 t} = e^{2t}$, and $e^{\lambda_3 t} = e^t$.

\end{solution}
\end{computerExercise}
\begin{computerExercise} \label{c12.2.7}
$\frac{d^3x}{dt^3} + \frac{d^2x}{dt^2}+ 10x=0$

\begin{solution}
\ans The general solution to the differential equation is
\[
x(t) = \alpha_1e^{0.7723t} + \alpha_2e^{0.7723t}\cos(1.8258t) +
\alpha_3e^{0.7723t}\sin(1.8258t).
\]

\soln The characteristic polynomial is $\lambda^3+\lambda^2+10$.  Using 
\Matlab, find that the roots of the characteristic polynomial are 
$\lambda_1 \approx  -2.5445$ and  complex conjugate eigenvalues 
$\lambda_2 \approx 0.7723 \pm 1.8258i$.  So, if $\lambda_2 = \sigma + i\tau$,
then all solutions are linear combinations of $e^{\lambda_1 t}=e^{0.7723t}$,
$e^{\sigma t}\cos(\tau t) = e^{0.7723t}\cos(1.8258t)$, and
$e^{\sigma t}\sin(\tau t) = e^{0.7723t}\sin(1.8258t)$.


\end{solution}
\end{computerExercise}
\begin{computerExercise} \label{c12.2.8}
$\frac{d^3x}{dt^3} +\frac{d^2x}{dt^2}-4\frac{dx}{dt}-4x=0$

\begin{solution}
\ans The general solution to the differential equation is
\[
x(t) = \alpha_1e^{2t} + \alpha_2e^{-t} + \alpha_3e^{-2t}.
\]

\soln Using \Matlab, find that the characteristic polynomial has real
eigenvalues at $\lambda_1 = 2$, $\lambda_2 = -1$, and $\lambda_3 = -2$.  So
all solutions are linear combinations of $e^{\lambda_1 t} = e^{2t}$,
$e^{\lambda_2 t} = e^{-t}$, and $e^{\lambda_3 t} = e^{-2t}$.

\end{solution}
\end{computerExercise}
\begin{computerExercise} \label{c12.2.9}
$\frac{d^4x}{dt^4}-2\frac{d^3x}{dt^3}+\frac{d^2x}{dt^2}+8\frac{dx}{dt}-20x=0$

\begin{solution}
\ans The general solution to the differential equation is
\[
x(t) = \alpha_1e^{2t} + \alpha_2e^{-2t} + \alpha_3e^t\cos(2t) +
\alpha_4e^t\sin(2t).
\]

\soln Using \Matlab, find that the characteristic polynomial has real
eigenvalues at $\lambda_1 = 2$ and $\lambda_2 = -2$, and complex
conjugate eigenvalues at $\lambda_3 = 1 \pm 2i$.  So, if $\lambda_3 =
\sigma + i\tau$, then all solutions are linear combinations of
$e^{\lambda_1 t} = e^{2t}$, $e{\lambda_2 t} = e^{-2t}$,
$e^{\sigma t}\cos(\tau t) = e^t\cos(2t)$, and
$e^{\sigma t}\sin(\tau t) = e^t\sin(2t)$.

\end{solution}
\end{computerExercise}
\begin{computerExercise} \label{c12.2.10}
$\frac{d^4x}{dt^4}-3\frac{d^2x}{dt^2}-4x=0$

\begin{solution}
\ans The general solution to the differential equation is
\[
x(t) = \alpha_1e^{2t} + \alpha_2e^{-2t} + \alpha_3\cos t +
\alpha_4\sin t.
\]

\soln Using \Matlab, find that the characteristic polynomial has real
eigenvalues at $\lambda_1 = 2$ and $\lambda_2 = -2$, and complex
conjugate eigenvalues at $\lambda_3 = \pm i$.  So, if $\lambda_3 =
\sigma + i\tau$, then all solutions are linear combinations of
$e^{\lambda_1 t} = e^{2t}$, $e{\lambda_2 t} = e^{-2t}$,
$e^{\sigma t}\cos(\tau t) = \cos t$, and $e^{\sigma t}\sin(\tau t) = \sin t$.

\end{solution}
\end{computerExercise}
\begin{computerExercise} \label{c12.2.11}
$\frac{d^4x}{dt^4}-4\frac{d^3x}{dt^3}+14\frac{d^2x}{dt^2}-4\frac{dx}{dt}+13x=0$

\begin{solution}
\ans The general solution to the differential equation is
\[
x(t) = \alpha_1e^{2t}\cos(3t) + \alpha_2e^{2t}\sin(3t) +
\alpha_3\cos t + \alpha_4\sin t.
\]

\soln Using \Matlab, find that the characteristic polynomial has complex
conjugate eigenvalues at $\lambda_1 = 2 \pm 3i$ and $\lambda_2 = \pm i$.
So, if $\lambda_j = \sigma_j + i\tau_j$, then all solutions are linear
combinations of $e^{\sigma_1 t}\cos(\tau_1 t) = e^{2t}\cos(3t)$,
$e^{\sigma_1 t}\sin(\tau_1 t) = e^{2t}\sin(3t)$,
$e^{\sigma_2 t}\cos(\tau_2 t) = \cos t$, and
$e^{\sigma_2 t}\sin(\tau_2 t) = \sin t$.





\end{solution}
\end{computerExercise}
\end{document}
