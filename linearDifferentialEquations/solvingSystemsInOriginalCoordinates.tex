\documentclass{ximera}

 

\usepackage{epsfig}

\graphicspath{
  {./}
  {figures/}
}

\usepackage{morewrites}
\makeatletter
\newcommand\subfile[1]{%
\renewcommand{\input}[1]{}%
\begingroup\skip@preamble\otherinput{#1}\endgroup\par\vspace{\topsep}
\let\input\otherinput}
\makeatother

\newcommand{\includeexercises}{\directlua{dofile("/home/jim/linearAlgebra/laode/exercises.lua")}}

%\newcounter{ccounter}
%\setcounter{ccounter}{1}
%\newcommand{\Chapter}[1]{\setcounter{chapter}{\arabic{ccounter}}\chapter{#1}\addtocounter{ccounter}{1}}

%\newcommand{\section}[1]{\section{#1}\setcounter{thm}{0}\setcounter{equation}{0}}

%\renewcommand{\theequation}{\arabic{chapter}.\arabic{section}.\arabic{equation}}
%\renewcommand{\thefigure}{\arabic{chapter}.\arabic{figure}}
%\renewcommand{\thetable}{\arabic{chapter}.\arabic{table}}

%\newcommand{\Sec}[2]{\section{#1}\markright{\arabic{ccounter}.\arabic{section}.#2}\setcounter{equation}{0}\setcounter{thm}{0}\setcounter{figure}{0}}

\newcommand{\Sec}[2]{\section{#1}}

\setcounter{secnumdepth}{2}
%\setcounter{secnumdepth}{1} 

%\newcounter{THM}
%\renewcommand{\theTHM}{\arabic{chapter}.\arabic{section}}

\newcommand{\trademark}{{R\!\!\!\!\!\bigcirc}}
%\newtheorem{exercise}{}

\newcommand{\dfield}{{\sf dfield9}}
\newcommand{\pplane}{{\sf pplane9}}

\newcommand{\EXER}{\section*{Exercises}}%\vspace*{0.2in}\hrule\small\setcounter{exercise}{0}}
\newcommand{\CEXER}{}%\vspace{0.08in}\begin{center}Computer Exercises\end{center}}
\newcommand{\TEXER}{} %\vspace{0.08in}\begin{center}Hand Exercises\end{center}}
\newcommand{\AEXER}{} %\vspace{0.08in}\begin{center}Hand Exercises\end{center}}

% BADBAD: \newcommand{\Bbb}{\bf}

\newcommand{\R}{\mbox{$\Bbb{R}$}}
\newcommand{\C}{\mbox{$\Bbb{C}$}}
\newcommand{\Z}{\mbox{$\Bbb{Z}$}}
\newcommand{\N}{\mbox{$\Bbb{N}$}}
\newcommand{\D}{\mbox{{\bf D}}}
\usepackage{amssymb}
%\newcommand{\qed}{\hfill\mbox{\raggedright$\square$} \vspace{1ex}}
%\newcommand{\proof}{\noindent {\bf Proof:} \hspace{0.1in}}

\newcommand{\setmin}{\;\mbox{--}\;}
\newcommand{\Matlab}{{M\small{AT\-LAB}} }
\newcommand{\Matlabp}{{M\small{AT\-LAB}}}
\newcommand{\computer}{\Matlab Instructions}
\newcommand{\half}{\mbox{$\frac{1}{2}$}}
\newcommand{\compose}{\raisebox{.15ex}{\mbox{{\scriptsize$\circ$}}}}
\newcommand{\AND}{\quad\mbox{and}\quad}
\newcommand{\vect}[2]{\left(\begin{array}{c} #1_1 \\ \vdots \\
 #1_{#2}\end{array}\right)}
\newcommand{\mattwo}[4]{\left(\begin{array}{rr} #1 & #2\\ #3
&#4\end{array}\right)}
\newcommand{\mattwoc}[4]{\left(\begin{array}{cc} #1 & #2\\ #3
&#4\end{array}\right)}
\newcommand{\vectwo}[2]{\left(\begin{array}{r} #1 \\ #2\end{array}\right)}
\newcommand{\vectwoc}[2]{\left(\begin{array}{c} #1 \\ #2\end{array}\right)}

\newcommand{\ignore}[1]{}


\newcommand{\inv}{^{-1}}
\newcommand{\CC}{{\cal C}}
\newcommand{\CCone}{\CC^1}
\newcommand{\Span}{{\rm span}}
\newcommand{\rank}{{\rm rank}}
\newcommand{\trace}{{\rm tr}}
\newcommand{\RE}{{\rm Re}}
\newcommand{\IM}{{\rm Im}}
\newcommand{\nulls}{{\rm null\;space}}

\newcommand{\dps}{\displaystyle}
\newcommand{\arraystart}{\renewcommand{\arraystretch}{1.8}}
\newcommand{\arrayfinish}{\renewcommand{\arraystretch}{1.2}}
\newcommand{\Start}[1]{\vspace{0.08in}\noindent {\bf Section~\ref{#1}}}
\newcommand{\exer}[1]{\noindent {\bf \ref{#1}}}
\newcommand{\ans}{}
\newcommand{\matthree}[9]{\left(\begin{array}{rrr} #1 & #2 & #3 \\ #4 & #5 & #6
\\ #7 & #8 & #9\end{array}\right)}
\newcommand{\cvectwo}[2]{\left(\begin{array}{c} #1 \\ #2\end{array}\right)}
\newcommand{\cmatthree}[9]{\left(\begin{array}{ccc} #1 & #2 & #3 \\ #4 & #5 &
#6 \\ #7 & #8 & #9\end{array}\right)}
\newcommand{\vecthree}[3]{\left(\begin{array}{r} #1 \\ #2 \\
#3\end{array}\right)}
\newcommand{\cvecthree}[3]{\left(\begin{array}{c} #1 \\ #2 \\
#3\end{array}\right)}
\newcommand{\cmattwo}[4]{\left(\begin{array}{cc} #1 & #2\\ #3
&#4\end{array}\right)}

\newcommand{\Matrix}[1]{\ensuremath{\left(\begin{array}{rrrrrrrrrrrrrrrrrr} #1 \end{array}\right)}}

\newcommand{\Matrixc}[1]{\ensuremath{\left(\begin{array}{cccccccccccc} #1 \end{array}\right)}}



\renewcommand{\labelenumi}{\theenumi)}
\newenvironment{enumeratea}%
{\begingroup
 \renewcommand{\theenumi}{\alph{enumi}}
 \renewcommand{\labelenumi}{(\theenumi)}
 \begin{enumerate}}
 {\end{enumerate}\endgroup}



\newcounter{help}
\renewcommand{\thehelp}{\thesection.\arabic{equation}}

%\newenvironment{equation*}%
%{\renewcommand\endequation{\eqno (\theequation)* $$}%
%   \begin{equation}}%
%   {\end{equation}\renewcommand\endequation{\eqno \@eqnnum
%$$\global\@ignoretrue}}

%\input{psfig.tex}

\author{Martin Golubitsky and Michael Dellnitz}

%\newenvironment{matlabEquation}%
%{\renewcommand\endequation{\eqno (\theequation*) $$}%
%   \begin{equation}}%
%   {\end{equation}\renewcommand\endequation{\eqno \@eqnnum
% $$\global\@ignoretrue}}

\newcommand{\soln}{\textbf{Solution:} }
\newcommand{\exercap}[1]{\centerline{Figure~\ref{#1}}}
\newcommand{\exercaptwo}[1]{\centerline{Figure~\ref{#1}a\hspace{2.1in}
Figure~\ref{#1}b}}
\newcommand{\exercapthree}[1]{\centerline{Figure~\ref{#1}a\hspace{1.2in}
Figure~\ref{#1}b\hspace{1.2in}Figure~\ref{#1}c}}
\newcommand{\para}{\hspace{0.4in}}

\renewenvironment{solution}{\suppress}{\endsuppress}

\ifxake
\newenvironment{matlabEquation}{\begin{equation}}{\end{equation}}
\else
\newenvironment{matlabEquation}%
{\let\oldtheequation\theequation\renewcommand{\theequation}{\oldtheequation*}\begin{equation}}%
  {\end{equation}\let\theequation\oldtheequation}
\fi

\makeatother


\title{Solving Systems in Original Coordinates}

\begin{document}
\begin{abstract}
\end{abstract}
\maketitle


\label{S:SEOC}

In Section~\ref{sec:LinHomSys} we discussed one method for solving systems of 
first order constant coefficient linear differential equations.   We saw that 
such systems can be solved by putting the coefficient matrix $A$ in Jordan 
normal form and then computing the exponential of the Jordan normal form 
matrix.  

In this section we describe a second and a third approach to solving linear 
systems; the second method is based on finding the generalized eigenvectors 
that put $A$ into Jordan normal form\index{Jordan normal form} and then
computing solutions directly using this information while the third method is 
based on deriving a formula for the exponential\index{matrix!exponential} 
$e^{tA}$ in original coordinates.  The advantage of the second method is
that it is not necessary to perform the similarity that transforms the matrix 
$A$ to Jordan normal form and the advantage of the third method is that it is 
not necessary even to compute the eigenvectors of $A$.  Be forewarned,
however, that all of these methods require substantial calculations.

Let $A$ be an $n\times n$ matrix.  All methods for solving the system 
\begin{equation} \label{dotX=AX}
\dot{X}=AX
\end{equation}
begin by finding the eigenvalues\index{eigenvalue} of $A$.  This can be done 
either analytically (sometimes) or numerically (using \Matlabp). Then the  
methods diverge.  In the first and second methods, we need to find the 
eigenvectors and, if need be, the generalized eigenvectors
\index{eigenvector!generalized} of $A$; while in the third method, we need to 
perform tedious calculations involving partial fractions and matrix 
multiplications.  With either method, the calculations simplify enormously 
when the eigenvalues are simple.  Indeed, this simplification also occurs in 
the Jordan normal form method of Section~\ref{sec:LinHomSys}.

\subsection*{A Method Based on Eigenvectors}
 
In this method we find a basis for solutions of \eqref{dotX=AX}.  First, we 
review the simpler case when there is a basis of eigenvectors and then we 
consider (part of) the case when there is a deficiency of eigenvectors.

\subsubsection*{A Complete Set of Eigenvectors}

The simplest case in solving \eqref{dotX=AX} occurs when $A$ has a basis of 
eigenvectors $v_1,\ldots,v_n$ corresponding to the (not necessarily distinct) 
eigenvalues $\lambda_1,\ldots,\lambda_n$.  We showed how to find a basis for
the solutions of \eqref{dotX=AX} in Section~\ref{S:TDM} but review the results
here.  Each eigenvector $v_j$ generates the solution 
$X_j(t)=e^{\lambda_j t}v_j$ to \eqref{dotX=AX} and the general solution is:
\begin{equation} \label{E:gensolns}
X(t) = \alpha_1 X_1(t) + \cdots + \alpha_nX_n(t),
\end{equation}
where the scalars $\alpha_j$ are real when $\lambda_j$ is real and complex
when the $\lambda_j$ is not real.  

The initial value problem\index{initial value problem} is then solved 
by finding scalars $\alpha_j$ so that
\begin{equation}  \label{E:gensolnsic}
X_0 = X(0) =\alpha_1v_1 + \cdots + \alpha_nv_n.
\end{equation}
The solution of \eqref{E:gensolnsic} is a well understood linear algebra 
problem.   

\subsubsection*{Complex Eigenvalues}

The only complication occurs when some of the eigenvalues are complex. 
If $\lambda$ is a complex eigenvalue of the real matrix $A$, so is 
$\overline{\lambda}$.  If $v$ is a complex eigenvector corresponding to 
$\lambda$, then $\overline{v}$ is the complex eigenvector corresponding to 
$\overline{\lambda}$.  With this choice of eigenvector, $\overline{\alpha}$ 
is the scalar corresponding to the eigenvector $\overline{v}$ where $\alpha$
is the scalar corresponding to the eigenvector $v$.

More precisely, let $\lambda = \sigma+i\tau$ be an eigenvalue of $A$ and 
let $v=u+iw$ be a corresponding eigenvector.  We claim that 
\begin{equation}  \label{eq:reimsol}
X_1(t) = e^{\sigma t}(\cos(\tau t)u - \sin(\tau t)w)\quad
\mbox{and}\quad
X_2(t) = e^{\sigma t}(\sin(\tau t)u + \cos(\tau t)w)
\end{equation}
are solutions of the homogeneous equation \eqref{dotX=AX}.  Verifying that
$X_1$ and $X_2$ are solutions proceeds as follows.  The solutions 
corresponding to the eigenvalue $\lambda$ are 
\[
\alpha e^{\lambda t}v + \overline{\alpha} e^{\overline{\lambda}t}\overline{v}
= 2\RE\left(\alpha e^{\lambda t}v\right)
\]
for all complex scalars $\alpha$.  If we set $\alpha=\frac{1}{2}$, then 
using Euler's formula, we obtain the solution 
\[
\RE\left(e^{\lambda t}v\right) = e^{\sigma t}\RE\left((\cos(\tau t)+i\sin(\tau t))(u + iw)\right)=X_1(t)
\]
Similarly, setting $\alpha=-\frac{1}{2}i$ leads to the solution $X_2(t)$.  


\subsubsection*{Two Examples with a Complete Set of Eigenvectors}

Next we consider two examples.  The first has distinct eigenvalues, some 
of which are complex, while the second has real eigenvalues one of which is 
multiple.

\noindent (a)  Find all the solutions of the linear system of ODEs
\arraystart
\begin{matlabEquation}  \label{eq:exsyslinc1}
\left(\begin{array}{c}
\dps\frac{dx_1}{dt} \\ \dps\frac{dx_2}{dt} \\ \dps\frac{dx_3}{dt}
\end{array}\right)
=
\left(\begin{array}{rrr}
     0  &  3  &  1\\
     4  &  1  & -1\\
     2  &  7  & -5
\end{array}\right)
\left(\begin{array}{c}
x_1 \\ x_2 \\ x_3
\end{array}\right).
\end{matlabEquation}
\arrayfinish
We load the coefficient matrix $A$ of \eqref{eq:exsyslinc1} by typing 
{\tt e15\_1\_5} and compute its eigenvalues and eigenvectors using the 
command 
\begin{verbatim}
[V D] = eig(A)
\end{verbatim}\index{\computer!eig}
This leads to
\begin{verbatim}
V =
  -0.5774             0.3757 + 0.0411i   0.3757 - 0.0411i
  -0.5774            -0.3757 - 0.0411i  -0.3757 + 0.0411i
  -0.5774            -0.4579 + 0.7103i  -0.4579 - 0.7103i

D =
   4.0000                  0                  0
        0            -4.0000 + 2.0000i        0
        0                  0            -4.0000 - 2.0000i
\end{verbatim}
The matrix $A$ has the three distinct eigenvalues $\rho=4$,
$\lambda=\sigma+i\tau = -4+2i$ and $\overline\lambda=\sigma-i\tau
= -4-2i$.  We can use these data to find all the solutions of 
\eqref{eq:exsyslinc1}.  Every solution of \eqref{eq:exsyslinc1} is a linear 
combination of the functions
\begin{eqnarray*}
X_1(t) & = & e^{4t}\left(\begin{array}{r} 1\\1\\1\end{array}\right),\\
\widehat{X}_2(t) & = & e^{-4t}\left[
\cos(2t)\left(\begin{array}{r}0.3757\\-0.3757\\-0.4579\end{array}\right)
-\sin(2t)\left(\begin{array}{r} 0.0411\\-0.0411\\0.7103\end{array}\right)
\right],\\
\widehat{X}_3(t) & = & e^{-4t}\left[
\sin(2t)\left(\begin{array}{r}0.3757\\-0.3757\\-0.4579\end{array}\right)
+\cos(2t)\left(\begin{array}{r} 0.0411\\-0.0411\\0.7103\end{array}\right)
\right].
\end{eqnarray*}


In fact, it is possible to obtain a simpler form for these solutions by 
using different eigenvectors associated with the eigenvalues $-4\pm 2i$.  
Rescaling {\tt V(:,2)} by {\tt V(:,2)/V(1,2)} yields the answer
\begin{verbatim}
ans =
   1.0000
  -1.0000 - 0.0000i
  -1.0000 + 2.0000i
\end{verbatim}
and therefore $(1,-1,-1+2i)^t$ is also an eigenvector belonging to the
eigenvalue $-4+2i$.  Using this eigenvector the two solutions in 
\eqref{eq:reimsol} take the form
\[
X_2(t) = e^{-4t}\left(\begin{array}{c}
\cos(2t)\\ -\cos(2t)\\ -\cos(2t)-2\sin(2t)\end{array}\right),\quad
X_3(t) = e^{-4t}\left(\begin{array}{c}
\sin(2t)\\ -\sin(2t)\\ -\sin(2t)+2\cos(2t)\end{array}\right).
\]

Suppose that we wish to find the solution satisfying the initial 
condition\index{initial condition} $X_0=(1,2,3)^t$.  Then we need 
to find scalars $\alpha_1$,
$\alpha_2$, and $\alpha_3$ that solve the system of linear equations
\[
X_0 = \alpha_1X_1(0) + \alpha_2X_2(0) + \alpha_3X_3(0).
\]
In coordinates, this linear system is:
\[
\left(\begin{array}{c} 1 \\ 2 \\ 3\end{array}\right) =
\left(\begin{array}{rrr} 1 & 1 & 0 \\ 1 & -1 & 0 \\ 1 & -1 & 2 \end{array}
\right)\left(\begin{array}{c} \alpha_1 \\ \alpha_2 \\ \alpha_3
\end{array}\right).
\] 
This linear system is easily solved by hand to obtain 
$\alpha_1=\frac{3}{2}$, $\alpha_2=-\frac{1}{2}$ and $\alpha_3=\frac{1}{2}$.
The closed form solution to this initial value problem is then 
\begin{equation}  \label{E:ecp}
X(t) = \frac{1}{2}\left(\begin{array}{c} 
3e^{4t} + e^{-4t}(\sin(2t)-\cos(2t)) \\
3e^{4t} + e^{-4t}(-\sin(2t)+\cos(2t))\\
3e^{4t} + e^{-4t}(\sin(2t)+3\cos(2t))
\end{array}\right).
\end{equation}


\noindent (b) As a second example, consider the system $\dot{X} = AX$ where
\begin{matlabEquation} \label{eq:exsyslin1}
A = 
\left(\begin{array}{rrr}
-2 & -2 & -4 \\
 0 & 0 & 4 \\
 0 & 2 & 2
\end{array}\right)
\end{matlabEquation}
By inspection, we see that the three eigenvalues of $A$ are $-2$ and the two
eigenvalues of the matrix $\mattwo{0}{4}{2}{2}$.  A quick calculation shows
that the eigenvalues of this $2\times 2$ matrix are $-2$ and $4$.  So the 
eigenvalues of $A$ are
\[
\lambda_1=\lambda_2=-2 \AND \lambda_3=4.
\]
There are three linearly independent eigenvectors corresponding to these 
eigenvalues: 
\[
v_1=\left(\begin{array}{r}
1 \\ 0 \\ 0
\end{array}\right),\qquad
v_2=\left(\begin{array}{r}
0 \\ -2 \\ 1
\end{array}\right),\qquad
v_3=\left(\begin{array}{r}
1 \\ -1 \\ -1
\end{array}\right).
\]
Therefore, every solution of the equation $\dot X=AX$ has the form
\[
X(t)=\alpha_1e^{-2t}\left(\begin{array}{r}
1 \\ 0 \\ 0
\end{array}\right) + \alpha_2e^{-2t}\left(\begin{array}{r}
0 \\ -2 \\ 1\end{array}\right) + \alpha_3e^{4t}\left(\begin{array}{r}
1 \\ -1 \\ -1
\end{array}\right)
\] 
with appropriate constants $\alpha_1,\alpha_2,\alpha_3$.


\subsubsection*{A Deficiency in Eigenvectors}

When working with Jordan normal forms\index{Jordan normal form} we have to 
find bases consisting of generalized eigenvectors.  As we know, this is a 
difficult problem.  We describe a method here for finding closed form 
solutions to linear systems when the eigenvalues are real and there is 
exactly one Jordan block\index{Jordan block} associated to each eigenvalue.  
This assumption is equivalent to assuming that the null space of 
$A-\lambda I_n$ being one dimensional, and under this assumption the 
computation of generalized eigenvalues is a tractable linear algebra problem 
(as we saw in Section~\ref{S:JNF}).

Suppose that $\lambda$ is a real eigenvalue of $A$ with algebraic
multiplicity $k$ and with one linearly independent eigenvector $w$.   Suppose 
that there exist linearly independent vectors $w_j$ ($j=1,\ldots,k$) such that
\begin{equation}  \label{eq:genvec}
\begin{array}{rcl}
(A-\lambda I_n)w_1 & = & 0 \\
(A-\lambda I_n)w_2 & = & w_1 \\
& \vdots & \\
(A-\lambda I_n)w_k & = & w_{k-1}.
\end{array}
\end{equation}
We use this information to find $k$ linearly independent solutions to
$\dot{X}=AX$.  See Theorem~\ref{T:JBsoln}.

The theory of Section~\ref{sec:LinHomSys} tells us that solutions of
\eqref{eq:linsys} have the form
\[
X(t)=e^{\lambda t}(\alpha_1(t)w_1+\alpha_2(t)w_2+\cdots+\alpha_k(t)w_k),
\]
where each $\alpha_j(t)$ is a polynomial of degree at most $k-1$.  See 
Lemma~\ref{R:pdeg} in Section~\ref{sec:LinHomSys}.  Using the product rule, 
compute
\[
\dps\frac{dX}{dt} = e^{\lambda t}
\left(\lambda (\alpha_1w_1+\cdots+\alpha_kw_k)+
\frac{d\alpha_1}{dt}w_1+\cdots+\frac{d\alpha_k}{dt}w_k\right).
\]
On the other hand, from \eqref{eq:genvec}
\begin{eqnarray*}
AX(t) & = & e^{\lambda t}(\alpha_1\lambda w_1+\alpha_2(w_1+\lambda w_2)
+\cdots+\alpha_k(w_{k-1}+\lambda w_k))\\
& = & e^{\lambda t}(\lambda (\alpha_1w_1+\cdots+\alpha_kw_k) +
\alpha_2w_1 + \cdots+ \alpha_kw_{k-1}).
\end{eqnarray*}
It follows that $X(t)$ is a solution if and only if
\begin{equation}  \label{E:alphas}
\frac{d\alpha_1}{dt}w_1+\cdots+
\frac{d\alpha_k}{dt}w_k= \alpha_2w_1 + \cdots+ \alpha_kw_{k-1}.
\end{equation}
Since $w_1,\ldots,w_k$ are linearly independent vectors, \eqref{E:alphas} is 
equivalent to
\begin{eqnarray*}
\frac{d\alpha_1}{dt} & = & \alpha_2    \\
                     & \vdots &        \\
\frac{d\alpha_{k-1}}{dt} & = & \alpha_k \\
\frac{d\alpha_k}{dt}  & = & 0.
\end{eqnarray*}

Next, choose $j$ such that $1\le j\le k$.  By setting
\[ 
\alpha_{j+1} =  \cdots = \alpha_k = 0
\]
and 
\[
\alpha_j(t)=1, \quad \alpha_{j-1}(t)=t, \quad \alpha_{j-2}(t)=\frac{t^2}{2},
\quad \ldots, \quad\alpha_1(t)=\frac{t^{j-1}}{(j-1)!}.
\]
Then the function 
\begin{equation}  \label{E:Xj}
X_j(t)=e^{\lambda t} \left(\frac{t^{j-1}}{(j-1)!}w_1+\frac{t^{j-2}}{(j-2)!}w_2
+\cdots+tw_{j-1}+w_j\right)
\end{equation}
is a solution of \eqref{eq:linsys}.   Moreover, since $X_j(0)=w_j$, the 
solutions $X_1(t),\ldots,X_k(t)$ are linearly independent.  We have proved:
\begin{theorem}  \label{T:JBsoln}
Let $w_1,\ldots,w_k$ be a set of linearly independent generalized
eigenvectors vectors satisfying \eqref{eq:genvec}. Then the functions 
$X_1(t),\ldots,X_k(t)$ in \eqref{E:Xj} are a linearly independent set of 
solutions\index{linearly!independent solutions} to $\dot{X}=AX$.
\end{theorem}

For example, when $k=3$, we have found three linearly independent solutions:
\begin{eqnarray*}
X_1(t)&=&e^{\lambda t} w_1\\
X_2(t)&=&e^{\lambda t} (tw_1+w_2)\\
X_3(t)&=&e^{\lambda t} \left(\frac{t^2}{2}w_1+tw_2+w_3\right).
\end{eqnarray*}


\subsubsection*{An Example}

Consider the system $\dot{X}=AX$, where 
\begin{matlabEquation}    \label{eq:exsyslin2}
A =
\left(\begin{array}{rrr}
     6  &  6  &  6\\
     5  & 11  & -1\\
     1  & -5  &  7
\end{array}\right).
\end{matlabEquation}
The eigenvalues of $A$ (using {\tt eig(A)})\index{\computer!eig} are
\[
\lambda_1=0\quad \mbox{and}\quad \lambda_2=\lambda_3=12.
\]
In this case (using {\tt [V,D] = eig(A)}) we find that there are just two
linearly independent eigenvectors
\[
v_1=\left(\begin{array}{r}
2 \\ -1 \\ -1
\end{array}\right),\quad
v_2=\left(\begin{array}{r}
0 \\ 1 \\ -1
\end{array}\right).
\]
It follows that the algebraic multiplicity\index{multiplicity!algebraic}
of $\lambda_2=12$
is two but the geometric multiplicity\index{multiplicity!geometric}
is just one.  Typing
\begin{verbatim}
null((A-12*eye(3))^2)
\end{verbatim}\index{\computer!null}\index{\computer!eye}
shows us that $w_2=(2,1,1)^t$ is a generalized eigenvector of $A$.
Set $w_1=(A-12I_3)w_2=(0,8,-8)^t$.  Then all solutions of \eqref{eq:exsyslin2}
are linear combinations of
\[
X_1(t)=\left(\begin{array}{r}
2 \\ -1 \\ -1
\end{array}\right), \quad
X_2(t)=e^{12t}\left(\begin{array}{r}
0 \\ 1 \\ -1
\end{array}\right), \quad
X_3(t)=e^{12t}\left[
t\left(\begin{array}{r}
0 \\ 8 \\ -8
\end{array}\right)
+\left(\begin{array}{r}
2 \\ 1 \\ 1
\end{array}\right)\right].
\]



\subsection*{The Direct Computation of the Matrix Exponential $e^{tA}$}

Explicitly solving a linear system of $n$ ODEs when the coefficient 
matrix $A$ has fewer than $n$ linearly independent eigenvectors is 
a difficult problem.  Several methods have been developed to find 
closed form solutions when eigenvector deficiencies exist; all of these 
methods require many calculations and are difficult to carry out using 
only hand calculation.  We now describe one of these methods --- the 
direct computation of the exponential $e^{tA}$ by a method based on the 
Cayley-Hamilton theorem\index{Cayley Hamilton theorem} 
(Theorem~\ref{T:CH} of 
Chapter~\ref{C:HDeigenvalues}), which was proved as a corollary to the 
Jordan normal form theorem.

\subsubsection*{A Method Based on Cayley-Hamilton and Partial Fractions}
\index{partial fractions}

Let $\lambda_1,\ldots,\lambda_k$ be the distinct eigenvalues of the 
$n\times n$ matrix $A$.  Let $m_j$ be the algebraic multiplicity of the 
eigenvalue $\lambda_j$.  Therefore, the characteristic polynomial
\index{characteristic polynomial} of $A$ is 
\[
p_A(\lambda) = (\lambda_1-\lambda)^{m_1}\cdots(\lambda_k-\lambda)^{m_k}.
\]

We begin with a preliminary discussion of $p_A$ based on partial fractions.
In particular, we claim that 
\begin{equation}  \label{e:1/p}
\frac{1}{p_A(\lambda)} = \frac{a_1(\lambda)}{(\lambda_1-\lambda)^{m_1}} +\cdots
	+ \frac{a_k(\lambda)}{(\lambda_k-\lambda)^{m_k}}
\end{equation}
where $a_j(\lambda)$ is a polynomial with $\deg(a_j)\leq m_j-1$.  Partial 
fractions state that $1/p_A(\lambda)$ can be written as sums of expressions 
of the form
\[
\frac{c_{j1}}{(\lambda_j-\lambda)} + \cdots + \frac{c_{jm_j}}{(\lambda_j-\lambda)^{m_j}},
\]
for scalars $c_{ji}$.  Putting these terms over a common denominator proves 
\eqref{e:1/p}.

Define the polynomials 
\begin{equation}  \label{e:Pj}
P_j(\lambda) = \frac{p_A(\lambda)}{(\lambda_j-\lambda)^{m_j}}.
\end{equation}
For instance, if 
$p_A(\lambda)=(\lambda_1-\lambda)(\lambda_2-\lambda)^2(\lambda_3-\lambda)$ 
with $\lambda_1 = -2$, $\lambda_2 = 5$ and $\lambda_3 = 4$ then
\begin{eqnarray*}
P_1(\lambda) &=& \frac{p_A(\lambda)}{-2-\lambda}=(5-\lambda)^2(4-\lambda)\\
P_2(\lambda) &=& \frac{p_A(\lambda)}{(5-\lambda)^2}=(-2-\lambda)(4-\lambda)\\
P_3(\lambda) &=& \frac{p_A(\lambda)}{4-\lambda}=(-2-\lambda)(5-\lambda)^2.
\end{eqnarray*}
We can now state one method for computing $e^{tA}$.
\begin{theorem} \label{T:etA}
Let $A$ be an $n\times n$ matrix with distinct eigenvalues 
$\lambda_1,\ldots,\lambda_k$ and algebraic multiplicities $m_1,\ldots,m_k$.  
Then
\[
e^{tA} = \sum_{\ell=1}^k \left[e^{\lambda_\ell t}a_\ell(A)P_\ell(A)
\sum_{j=0}^{m_\ell-1}\frac{1}{j!}t^j(A-\lambda_\ell I_n)^j\right].
\]
\end{theorem}\index{matrix!exponential}\index{multiplicity}
\index{distinct eigenvalues}

Note that this matrix exponential consists of functions that are linear 
combinations of $t^je^{\lambda_\ell t}$ where $j\leq m_\ell-1$.  These
are just the type of terms that appeared in our discussion of solutions 
of equations in Jordan normal form in Section~\ref{sec:LinHomSys}.

\begin{proof} Multiplying \eqref{e:1/p} by $p_A(\lambda)$ yields the identity
\begin{equation}  \label{e:p=aP}
1 = a_1(\lambda)P_1(\lambda) + \cdots + a_k(\lambda)P_k(\lambda).
\end{equation}
Identity \eqref{e:p=aP} is valid for every number $\lambda$.  This is 
possible only if, for each $j>0$, the sum of all coefficients of terms 
with $\lambda^j$ is zero.  Therefore, we can substitute $A$ into 
\eqref{e:p=aP} and obtain
\begin{equation}  \label{e:p=AP}
I_n = a_1(A)P_1(A) + \cdots + a_k(A)P_k(A).
\end{equation}

We now compute 
\[
e^{tA}   =  e^{\lambda_\ell t}e^{t(A-\lambda_\ell I_n)} =  
e^{\lambda_\ell t}\sum_{j=0}^\infty\frac{1}{j!}t^j(A-\lambda_\ell I_n)^j.
\]
Multiplying this identity by $a_\ell(A)P_\ell(A)$ yields
\[
a_\ell(A)P_\ell(A)e^{tA} = e^{\lambda_\ell t}a_\ell(A)
\sum_{j=0}^\infty\frac{1}{j!}t^jP_\ell(A)(A-\lambda_\ell I_n)^j.
\]
Now we use the Cayley-Hamilton theorem to observe that 
\[
P_\ell(A)(A-\lambda_\ell I_n)^{m_\ell} = (-1)^{m_\ell}p_A(A) = 0.
\]
Hence $P_\ell(A)(A-\lambda_\ell I_n)^j=0$ for all $j\ge m_\ell$
and therefore
\[
a_\ell(A)P_\ell(A)e^{tA} = e^{\lambda_\ell t}a_\ell(A)
\sum_{j=0}^{m_\ell-1}\frac{1}{j!}t^jP_\ell(A)(A-\lambda_\ell I_n)^j.
\]
Finally, we sum over the index $\ell$ and use \eqref{e:p=AP} to conclude that
\[
e^{tA} = \sum_{\ell=1}^ka_\ell(A)P_\ell(A)e^{tA} = 
\sum_{\ell=1}^k\left[ e^{\lambda_\ell t}a_\ell(A)P_\ell(A)
\sum_{j=0}^{m_\ell-1}\frac{1}{j!}t^j(A-\lambda_\ell I_n)^j \right]
\]
which proves the theorem.  \end{proof}  

As a special case, consider a matrix $A$ that has $n$ distinct
eigenvalues. Then $m_\ell-1=0$ for each $\ell$ implying that the polynomial
$a_\ell(\lambda)$ has degree zero and is a constant independent of $\lambda$. 
Since $0!=1$, $t^0=1$, and $(A-\lambda_\ell I_n)^0=I_n$, we obtain
\begin{corollary} \label{T:etA0}
Let $A$ be an $n\times n$ matrix with distinct eigenvalues
\index{distinct eigenvalues}
$\lambda_1,\ldots,\lambda_n$. Then
\[
e^{tA} = \sum_{\ell=1}^n a_\ell e^{\lambda_\ell t} P_\ell(A).
\]
\end{corollary}

\subsubsection*{An Example with Distinct Eigenvalues}

We revisit the example in \eqref{eq:exsyslinc1}, that is, 
\[
A = \left(\begin{array}{rrr}
     0  &  3  &  1\\
     4  &  1  & -1\\
     2  &  7  & -5
\end{array}\right)
\]
As we discussed previously, the eigenvalues of this matrix are:
$\lambda_1=4$, $\lambda_2=-4+2i$, and $\lambda_3=-4-2i$.
Therefore, the characteristic 
polynomial\index{characteristic polynomial} of $A$ is
\[
p_A(\lambda) = (4-\lambda)(-4+2i-\lambda)(-4-2i-\lambda),
\]
and
\begin{eqnarray*}
P_1(\lambda) & = & (-4+2i-\lambda)(-4-2i-\lambda) \\
P_2(\lambda)& = & (4-\lambda)(-4-2i-\lambda)\\  
P_3(\lambda) & = & (4-\lambda)(-4+2i-\lambda).
\end{eqnarray*}
Moreover, using partial fractions\index{partial fractions}, 
we can write
\[
\frac{1}{p_A(\lambda)} = \frac{\frac{1}{68}}{4-\lambda} + 
\frac{-\frac{1}{8+32i}}{-4+2i-\lambda}
+ \frac{-\frac{1}{8-32i}}{-4-2i-\lambda}.
\]
So
\[
a_1(\lambda)=\frac{1}{68}, \quad a_2(\lambda)=-\frac{1}{8+32i},
 \AND a_3(\lambda)=-\frac{1}{8-32i}.
\]

Next, using \Matlabp, compute
\[
\begin{array}{rcccl}
P_1(A) & = & ((-4+2i)I_3-A)((-4-2i)I_3-A) & = & 
\left(\begin{array}{rrr}  
     34    &       34    &        0 \\     
     34    &       34    &        0 \\     
     34    &       34    &        0      
\end{array}\right) \\
P_2(A) & = & (4I_3-A)((-4-2i)I_3-A) & = &  
\left(\begin{array}{rrr}
     -2  - 8i     &   10  +  6i  &  -8  +  2i \\     
      2  + 8i     &  -10  -  6i  &   8  -  2i \\     
     18  + 4i     &  -22  + 14i  &   4  - 18i       
\end{array}\right) \\
P_3(A) & = & (4I_3-A)((-4+2i)I_3-A) & = & 
\left(\begin{array}{rrr}
     -2  + 8i     &   10  -  6i  &  -8  -  2i \\      
      2  - 8i     &  -10  +  6i  &   8  +  2i \\   
     18  - 4i     &  -22  - 14i  &   4  + 18i    
\end{array}\right)
\end{array}
\]

Corollary~\ref{T:etA0} states that
\[
e^{tA} = e^{4t}a_1(A)P_1(A)+e^{-4t+2it}a_2(A)P_2(A)
	+e^{-4t-2it}a_3(A)P_3(A)
\]
Using \Matlabp, we obtain
\begin{eqnarray*}
e^{tA} & = & \frac{1}{2}e^{4t} 
\left(\begin{array}{rrr}
     1    &      1    &       0  \\    
     1    &      1    &       0  \\    
     1    &      1    &       0      
\end{array}\right)  + 
\frac{1}{4}e^{-4t+2it}
\left(\begin{array}{ccc}
     1     &    -1      +    i     &      -i   \\ 
    -1     &     1      -    i     &       i    \\
 -1  + 2i  &    -1      -   3i    &     2 + i   
\end{array}\right) + \\
& & \frac{1}{4}e^{-4t-2it}
\left(\begin{array}{ccc}
     1     &    -1      -    i     &       i   \\ 
    -1     &     1      +    i     &      -i    \\
 -1  - 2i  &    -1      +    3i    &     2 - i   
\end{array}\right)
\end{eqnarray*}
Therefore,
\[
e^{tA} = \frac{1}{2}e^{4t}
\left(\begin{array}{rrr}
     1    &      1    &       0  \\
     1    &      1    &       0  \\
     1    &      1    &       0
\end{array}\right)  
+ \frac{1}{2}e^{-4t}\left(\cos(2t)\left(\begin{array}{rrr}
1 & -1 & 0\\ -1 & 1 & 0\\ -1 & -1 & 2 \end{array}\right)+\sin(2t)
\left(\begin{array}{rrr}
0 & 1 & -1\\ 0 & -1 & 1\\ 2 & -3 & 1 \end{array}\right)\right)
\]

A distinct advantage of this direct method for computing $e^{tA}$ comes
when solving an initial value problem.  Then we can write the solution 
directly as $X(t)=e^{tA}X_0$.  For example, the solution to the initial
value problem when $X_0=(1,2,3)^t$ is:
\[
X(t) = e^{tA}X_0 = \frac{1}{2}e^{4t}
\left(\begin{array}{c} 3\\3\\3\end{array}\right)
+ \frac{1}{2}e^{-4t}\left(\cos(2t)\left(\begin{array}{r}
-1\\1\\3\end{array}\right)+\sin(2t)
\left(\begin{array}{r} -1\\1\\-1 \end{array}\right)\right) 
\]
Compare this result with \eqref{E:ecp}.


\subsubsection*{An Example with Multiple Eigenvalues}

As an example, recall the matrix \eqref{eq:exsyslin2}
\[
A =
\left(\begin{array}{rrr}
     6  &  6  &  6\\
     5  & 11  & -1\\
     1  & -5  &  7
\end{array}\right).
\]
The eigenvalues of $A$ (using {\tt eig(A)}) are
\[
\lambda_1=0\quad \mbox{and}\quad \lambda_2=\lambda_3=12.
\]
Therefore, the characteristic polynomial is
\[
p_A(\lambda) = (0-\lambda)(12-\lambda)^2=-\lambda(12-\lambda)^2.
\]
Using partial fractions, we write
\[
\frac{1}{p_A(\lambda)} = \frac{1}{-\lambda(12-\lambda)^2}=
\frac{\frac{1}{144}}{-\lambda} + 
\frac{-\frac{1}{6}+\frac{1}{144}\lambda}{(12-\lambda)^2}
\]
It follows that 
\[
a_1(\lambda)=\frac{1}{144} \AND a_2(\lambda) = -\frac{1}{6}+\frac{1}{144}\lambda
\]
and
\[
P_1(\lambda)=(12-\lambda)^2 \AND P_2(\lambda) = -\lambda.
\]

Thus, using \Matlabp, we calculate
\[
a_1(A)= \frac{1}{144}I_3 \AND a_2(A)=-\frac{1}{6}I_3+\frac{1}{144}A=   
\frac{1}{144}\left(\begin{array}{rrr}
    -18  &    6  &       6  \\  
     5   &   -13 &      -1\\   
     1   &    -5 &     -17   
\end{array}\right)
\]
and
\[
P_1(A) = (12I_3-A)^2 =    
\left(\begin{array}{rrr}
     72    &      -72   &       -72  \\    
    -36    &       36   &        36 \\     
    -36    &       36   &        36      
\end{array}\right)   \AND
P_2(A) = -A.
\]
From Theorem~\ref{T:etA} (again with the help of \Matlabp), it follows that 
\begin{eqnarray*}
e^{tA}&  =&  e^{0t}a_1(A)P_1(A) + 
e^{12t}a_2(A)P_2(A)+te^{12t}a_2(A)P_2(A)(A-12I_3)\\
& = & \frac{1}{4}\left(\begin{array}{rrr}
     2     &   -2    &     -2 \\    
    -1     &     1   &       1\\     
    -1     &     1   &       1     
\end{array}\right)  +
\frac{1}{4}e^{12t}\left(\begin{array}{rrr}     
    2 &   2 &   2\\
    1 &   3 &  -1\\
    1 &  -1 &   3
\end{array}\right) + te^{12t}\left(\begin{array}{rrr}
         0   &      0    &     0\\
         2   &     2    &   2\\
        -2   &     -2    &    -2
\end{array}\right).
\end{eqnarray*}
 



\EXER


\TEXER

\noindent  In Exercises~\ref{c12.1.2a} -- \ref{c12.1.2d} consider the system 
of differential equations $\dot X = AX$ corresponding to the given matrix 
$A$ and construct a complete set of linearly independent solutions.
\begin{exercise} \label{c12.1.2a}
$A=\mattwo{2}{1}{0}{2}$.

\begin{solution}
\ans
\[
X_1(t) = e^{2t}\vectwo{1}{0} \AND
X_2(t) = e^{2t}\vectwo{t}{1}
\]
are linearly independent solutions to $\dot{X} = AX$.

\soln First find the eigenvalues of $A$.  Since $A$ is upper triangular,
the eigenvalues are the diagonal entries.  Therefore $\lambda_1 = 2$
is a double eigenvalue.  Then solve $(A - \lambda_1 I_2)w_1 = 0$ to
find that $w_1 = (1,0)^t$ is an eigenvector of $A$.  Find a
generalized eigenvector $w_2$ such that $(A - \lambda_1 I_2)w_2 =
w_1$, in this case, $w_2 = (0,1)^t$.  Now, by \eqref{E:Xj},
there are linearly independent solutions $X_1(t)$ and $X_2(t)$ such that
\[
\begin{array}{rcl}
X_1(t) & = & e^{\lambda_1 t}w_1 = e^{2t}\vectwo{1}{0} \\ X_2(t) & = &
e^{\lambda_1 t}(tw_1 + w_2) = e^{2t}\left(\vectwo{t}{0} +
\vectwo{0}{1}\right).
\end{array}
\]

\end{solution}
\end{exercise}
\begin{exercise} \label{c12.1.2b}
$A=\left(\begin{array}{rrr} 1 & 0 & 0\\ 0 & -1 & 2 \\ 0 & 0 & -1
\end{array}\right)$.

\begin{solution}
\ans
\[
X_1(t) = \cvecthree{e^t}{0}{0}, \quad
X_2(t) = \cvecthree{0}{e^{-t}}{0}, \AND
X_3(t) = e^{-t}\cvecthree{0}{t + 1}{\frac{1}{2}}
\]
are linearly independent solutions to $\dot{X} = AX$.

\soln The eigenvalues of $A$ are the diagonal entries, since $A$ is upper
triangular.  For each eigenvalue, solve $(A - \lambda_j I_2)w_j = 0$
to find that $\lambda_1 = 1$ is an eigenvalue with eigenvector $w_1 =
(1,0,0)^t$, and that $\lambda_2 = -1$ is a double eigenvalue with
single eigenvector $w_2 = (0,1,0)^t$.  Find $w_3$, the generalized
eigenvector for $\lambda_2$, by solving $(A - \lambda_2 I_3)w_3 = w_2$
to obtain $w_3 = (0,1,\frac{1}{2})^t$.  Now, by \eqref{E:Xj},
there are linearly independent solutions $X_j(t)$ such that
\[
\begin{array}{rcl}
X_1(t) & = & e^{\lambda_1 t}w_1 = e^t\vecthree{1}{0}{0} \\
X_2(t) & = & e^{\lambda_2 t}w_2 = e^{-t}\vecthree{0}{1}{0} \\
X_3(t) & = & e^{\lambda_2 t}(tw_2 + w_3) = e^{-t}\left(\vecthree{0}{t}{0}
+ \vecthree{0}{1}{\frac{1}{2}}\right).
\end{array}
\]

\end{solution}
\end{exercise}
\begin{exercise} \label{c12.1.2c}
$A=\mattwo{1}{-1}{1}{3}$.

\begin{solution}
\ans
\[
X_1(t) = e^{2t}\vectwo{1}{-1} \AND
X_2(t) = e^{2t}\cvectwo{t - 1}{-t}
\]
are linearly independent solutions to $\dot{X} = AX$.

\soln First find the eigenvalues of $A$.  In this case, $\lambda_1 = 2$
is a double eigenvalue.  Then solve $(A - \lambda_1 I_2)w_1 = 0$ to
find that $w_1 = (1,-1)^t$ is an eigenvector of $A$.  Find a
generalized eigenvector $w_2$ such that $(A - \lambda_1 I_2)w_2 =
w_1$, in this case, $w_2 = (-1,0)^t$.  Now, by \eqref{E:Xj},
there are linearly independent solutions $X_1(t)$ and $X_2(t)$ such that
\[
\begin{array}{rcl}
X_1(t) & = & e^{\lambda_1 t}w_1 = e^{2t}\vectwo{1}{-1} \\
X_2(t) & = & e^{\lambda_1 t}(tw_1 + w_2) = e^{2t}\left(\vectwo{t}{-t} +
\vectwo{-1}{0}\right).
\end{array}
\]

\end{solution}
\end{exercise}
\begin{exercise} \label{c12.1.2d}
$A=\left(\begin{array}{rrr} -4 & -9 & 0\\ 0 & 5 & 0 \\ -9 & -8 & 5
\end{array}\right)$.

\begin{solution}
\ans
\[
X_1(t) = e^{-4t}\vecthree{1}{0}{1}, \quad
X_2(t) = e^{5t}\vecthree{0}{0}{1}, \AND
X_3(t) = e^{5t}\vecthree{-1}{1}{t}
\]
are linearly independent solutions to $\dot{X} = AX$.

\soln First find the eigenvalues and eigenvectors of $A$.  In this case,
$A$ has an eigenvalue at $\lambda_1 = -4$ with eigenvector $w_1 =
(1,0,1)^t$, and a double eigenvalue at $\lambda_2 = 5$ with single
eigenvector $w_2 = (0,0,1)^t$.  Find $w_3$, the generalized
eigenvector for $\lambda_2$, by solving $(A - \lambda_2 I_3)w_3 = w_2$
to obtain $w_3 = (-1,1,0)^t$.  Now, by \eqref{E:Xj},
there are linearly independent solutions $X_j(t)$ such that
\[
\begin{array}{rcl}
X_1(t) & = & e^{\lambda_1 t}w_1 = e^{-4t}\vecthree{1}{0}{1} \\
X_2(t) & = & e^{\lambda_2 t}w_2 = e^{5t}\vecthree{0}{0}{1} \\
X_3(t) & = & e^{\lambda_2 t}(tw_2 + w_3) = e^{5t}\left(\vecthree{0}{0}{t}
+ \vecthree{-1}{1}{0}\right).
\end{array}
\]

\end{solution}
\end{exercise}

\noindent  In Exercises~\ref{c12.1.6a} -- \ref{c12.1.6d} use Corollary~\ref{T:etA0} to compute $e^{tA}$ for the given matrix.
\begin{exercise} \label{c12.1.6a}
$A = \mattwo{0}{1}{1}{0}$.

\begin{solution}
\ans
\[
e^{tA} = -e^t\frac{1}{2}\mattwo{1}{1}{1}{1} +
e^{-t}\frac{1}{2}\mattwo{-1}{1}{1}{-1}
= -\frac{1}{2}\cmattwo{e^{t} + e^{-t}}{e^t - e^{-t}}{e^t - e^{-t}}
{e^t + e^{-t}}.
\]

\soln First find the eigenvalues of $A$, which are $\lambda_1 = 1$ and
$\lambda_2 = -1$.  By Corollary~\ref{T:etA0},
\[
e^{tA} = \sum_{\ell = 1}^2 e^{\lambda_\ell t}a_\ell(A)P_\ell(A).
\]
The characteristic polynomial of $A$ is $p_A(\lambda) = (\lambda -
1)(\lambda + 1)$, so $P_1(\lambda) = \lambda + 1$ and $P_2(\lambda) =
\lambda - 1$.  Use partial fractions to find that
\[
\frac{1}{p_A(\lambda)} = \frac{-\frac{1}{2}}{1 - \lambda} + \frac{\frac{1}{2}}
{-1 - \lambda}.
\]
Thus, $a_1(\lambda) = -\frac{1}{2}$ and $a_2(\lambda) = \frac{1}{2}$.  So
\[
\begin{array}{rcl}
e^{tA} & = & e^ta_1(A)P_1(A) + e^{-t}a_2(A)P_2(A) \\
& = & \dps e^t\left(-\frac{1}{2}\right)\left(\mattwo{0}{1}{1}{0} +
\mattwo{1}{0}{0}{1}\right) + e^{-t}\left(\frac{1}{2}\right)
\left(\mattwo{0}{1}{1}{0} - \mattwo{1}{0}{0}{1}\right).
\end{array}
\]

\end{solution}
\end{exercise}
\begin{exercise} \label{c12.1.6b}
$A = \mattwo{2}{1}{1}{2}$.

\begin{solution}
\ans
\[
e^{tA} = -e^{3t}\frac{1}{2}\mattwo{1}{1}{1}{1} +
e^t\frac{1}{2}\mattwo{-1}{1}{1}{-1}
= -\frac{1}{2}\cmattwo{e^{3t} + e^{t}}{e^{3t} - e^{t}}{e^{3t} - e^{t}}
{e^{3t} + e^{t}}.
\]

\soln First find the eigenvalues of $A$, which are $\lambda_1 = 3$ and
$\lambda_2 = 1$.  By Corollary~\ref{T:etA0},
\[
e^{tA} = \sum_{\ell = 1}^2 e^{\lambda_\ell t}a_\ell(A)P_\ell(A).
\]
The characteristic polynomial of $A$ is $p_A(\lambda) = (\lambda -
1)(\lambda - 3)$, so $P_1(\lambda) = \lambda - 1$ and $P_2(\lambda) =
\lambda - 3$.  Use partial fractions to find that
\[
\frac{1}{p_A(\lambda)} = \frac{-\frac{1}{2}}{3 - \lambda} + \frac{\frac{1}{2}}
{1 - \lambda}.
\]
Thus, $a_1(\lambda) = -\frac{1}{2}$ and $a_2(\lambda) = \frac{1}{2}$.  So
\[
\begin{array}{rcl}
e^{tA} & = & e^ta_1(A)P_1(A) + e^{-t}a_2(A)P_2(A) \\
& = & \dps e^{3t}\left(-\frac{1}{2}\right)\left(\mattwo{2}{1}{1}{2} -
\mattwo{1}{0}{0}{1}\right) + e^{t}\left(\frac{1}{2}\right)
\left(\mattwo{2}{1}{1}{2} - \mattwo{3}{0}{0}{3}\right).
\end{array}
\]

\end{solution}
\end{exercise}
\begin{exercise} \label{c12.1.6c}
$A = \mattwo{3}{2}{0}{1}$.

\begin{solution}
\ans
\[
e^{tA} = -e^{3t}\frac{1}{2}\mattwo{2}{2}{0}{0} +
e^t\frac{1}{2}\mattwo{0}{2}{0}{-2}
= -\cmattwo{e^{3t}}{e^{3t} - e^{t}}{0}{e^{t}}.
\]

\soln First find the eigenvalues of $A$, which are $\lambda_1 = 3$ and
$\lambda_2 = 1$.  By Corollary~\ref{T:etA0},
\[
e^{tA} = \sum_{\ell = 1}^2 e^{\lambda_\ell t}a_\ell(A)P_\ell(A).
\]
The characteristic polynomial of $A$ is $p_A(\lambda) = (\lambda -
1)(\lambda - 3)$, so $P_1(\lambda) = \lambda - 1$ and $P_2(\lambda) =
\lambda - 3$.  Use partial fractions to find that
\[
\frac{1}{p_A(\lambda)} = \frac{-\frac{1}{2}}{3 - \lambda} + \frac{\frac{1}{2}}
{1 - \lambda}.
\]
Thus, $a_1(\lambda) = -\frac{1}{2}$ and $a_2(\lambda) = \frac{1}{2}$.  So
\[
\begin{array}{rcl}
e^{tA} & = & e^ta_1(A)P_1(A) + e^{-t}a_2(A)P_2(A) \\
& = & \dps e^{3t}\left(-\frac{1}{2}\right)\left(\mattwo{3}{2}{0}{1} -
\mattwo{1}{0}{0}{1}\right) + e^{t}\left(\frac{1}{2}\right)
\left(\mattwo{3}{2}{0}{1} - \mattwo{3}{0}{0}{3}\right).
\end{array}
\]


\end{solution}
\end{exercise}
\begin{exercise} \label{c12.1.6d}
$A = \left(\begin{array}{rrr} 1 & -3 & 1\\ 0 & -2 & -5 \\ 0 & 0 & 3
\end{array}\right)$.

\begin{solution}
\ans
\[
e^{tA} =
\cmatthree{e^t}{-e^t + e^{-2t}}{2e^{3t} - e^t + e^{-2t}}
{0}{e^{-2t}}{-e^{3t} + e^{-2t}}
{0}{0}{e^{3t}}.
\]

\soln The characteristic polynomial of $A$ is
$p_A(\lambda) = (\lambda - 3)(\lambda - 1)(\lambda + 2)$, so the
eigenvalues are $\lambda_1 = 3$, $\lambda_2 = 1$, and $\lambda_3 = -2$.
By Corollary~\ref{T:etA0},
\[
e^{tA} = \sum_{\ell = 3}^2 e^{\lambda_\ell t}a_\ell(A)P_\ell(A).
\]
From the characteristic polynomial $p_A(\lambda)$, compute
\[
P_1(\lambda) = \lambda^2 + \lambda - 2, \quad
P_2(\lambda) = \lambda^2 - \lambda - 6, \AND
P_3(\lambda) = \lambda^2 - 4\lambda + 3.
\]
Use partial fractions to find that
\[
\frac{1}{p_A(\lambda)} =
\frac{\frac{1}{5}\lambda - \frac{1}{2}}{P_1(\lambda)}
+ \frac{-\frac{1}{6}\lambda}{P_2(\lambda)}
+ \frac{-\frac{1}{30}\lambda}{P_3(\lambda)}.
\]
Thus
\[
a_1(\lambda) = \frac{1}{5}\lambda - \frac{1}{2}, \quad
a_2(\lambda) = -\frac{1}{6}\lambda, \AND
a_3(\lambda) = -\frac{1}{30}\lambda.
\]
So,
\[
\begin{array}{rcl}
e^{tA} & = & e^{\lambda_1 t}a_1(A)P_1(A) + e^{\lambda_2 t}a_2(A)P_2(A)
+ e^{\lambda_3 t}a_3(A)P_3(A) \\
& = &
\begin{array}{l} e^{3t}(\frac{1}{5}A - \frac{1}{2})(A^2 + A - 2I_3)
+ e^t(-\frac{1}{6}A)(A^2 - A - 6I_3) \\
+\; e^{-2t}(-\frac{1}{30}A)(A^2 - 4A + 3I_2) \end{array} \\
& = & e^{3t}\matthree{0}{0}{2}{0}{0}{-1}{0}{0}{1}
+ e^t\matthree{1}{-1}{-3}{0}{0}{0}{0}{0}{0}
+ e^{-2t}\matthree{0}{1}{1}{0}{1}{1}{0}{0}{0}.
\end{array}
\]

\end{solution}
\end{exercise}


\CEXER

\noindent  In Exercises~\ref{c12.1.10a} -- \ref{c12.1.10b}  solve the 
initial value problem for the given system of ODE $\dot{X}=AX$ with 
initial condition $X(0)=X_0$.
\begin{exercise}  \label{c12.1.10a}
$A=\left(\begin{array}{rrr} 
     1   &  2  &  -1 \\
    -1   &  1  &   1 \\
     2   &  2  &  -2
\end{array}\right) \AND 
X_0=\left(\begin{array}{r} 1 \\ 2 \\ -1  \end{array}\right)$.

\begin{solution}
\ans The solution to the given initial value problem is
\[
X(t) = -e^{-t}\vecthree{4}{-1}{6} +
3\vecthree{1}{0}{1} + e^t\vecthree{2}{1}{2}
= \cvecthree{-4e^{-t} + 3 + 2e^t}{e^{-t} + e^t}{-6e^{-t} + 3 + 2e^t}.
\]

\soln The eigenvalues of $A$ are $\lambda_1 = -1$, $\lambda_2 = 0$, and
$\lambda_3 = 1$, with associated eigenvectors $w_1 = (4,-1,6)^t$,
$w_2 = (1,0,1)^t$, and $w_3 = (2,1,2)^t$.  So,
\[
\begin{array}{c}
X_1(t) = e^{\lambda_1 t}w_1 = e^{-t}\vecthree{4}{-1}{6}, \quad
X_2(t) = e^{\lambda_2 t}w_2 = \vecthree{1}{0}{1}, \\
X_3(t) = e^{\lambda_3 t}w_3 = e^t\vecthree{2}{1}{2}
\end{array}
\]
are linearly independent solutions.  The general solution is a linear
combination of the $X_j(t)$, and the solution to the initial value
problem $X_0 = X(0)$ satisfies
\[
\vecthree{1}{2}{-1} = X(0) = \alpha_1w_1 + \alpha_2w_2 + \alpha_3w_3
= \alpha_1\vecthree{4}{-1}{6} + \alpha_2\vecthree{1}{0}{1}
+ \alpha_3\vecthree{2}{1}{2}.
\]
Solve this linear system to obtain $\alpha_1 = -1$,
$\alpha_2 = 3$, and $\alpha_3 = 1$.


\end{solution}
\end{exercise}
\begin{exercise}  \label{c12.1.10b}
$A=\left(\begin{array}{rrr}
    -3   &  0   &  2\\
     2   & -1   & -1\\
    -4   &  0   &  3
\end{array}\right) \AND 
X_0=\left(\begin{array}{r} 0 \\ 1 \\ 1  \end{array}\right)$.

\begin{solution}
\ans The solution to the given initial value problem is
\[
X(t) = e^t\vecthree{1}{0}{2} + e^{-t}\vecthree{0}{1}{0} -
e^{-t}\vecthree{1}{t}{1} =
\cvecthree{e^t - e^{-t}}{(1 - t)e^{-t}}{2e^t - e^{-t}}.
\]

\soln The matrix $A$ has an eigenvalue at $\lambda_1 = 1$ with associated
eigenvector $w_1 = (1,0,2)^t$ and a double eigenvalue at $\lambda_2 = -1$
with eigenvector $w_2 = (0,1,0)^t$.  Find a generalized eigenvector at
$\lambda_2$ by solving the equation $(A - \lambda_2I_3)w_3 = w_2$, obtaining
$w_3 = (1,0,1)^t$.  Thus,
\[
\begin{array}{c}
X_1(t) = e^{\lambda_1 t}w_1 = e^{t}\vecthree{1}{0}{2}, \quad
X_2(t) = e^{\lambda_2 t}w_2 = e^{-t}\vecthree{0}{1}{0}, \\
X_3(t) = e^{\lambda_2 t}(w_2t + w_3) = e^{-t}\vecthree{1}{t}{1}
\end{array}
\]
are linearly independent solutions.  The general solution is a linear
combination of the $X_j(t)$, and the solution to the initial value
problem $X_0 = X(0)$ satisfies
\[
\vecthree{0}{1}{1} = X(0) = \alpha_1w_1 + \alpha_2w_2 + \alpha_3w_3
= \alpha_1\vecthree{1}{0}{2} + \alpha_2\vecthree{0}{1}{0}
+ \alpha_3\vecthree{1}{0}{1}.
\]
Solve this linear system to obtain $\alpha_1 = 1$,
$\alpha_2 = 1$, and $\alpha_3 = -1$.

\end{solution}
\end{exercise}

\noindent  In Exercises~\ref{c12.1.8a} -- \ref{c12.1.8d} use
Theorem~\ref{T:etA} and \Matlab to compute $e^{tA}$ for the given matrix.
\begin{exercise} \label{c12.1.8a}
$A = \mattwo{0}{1}{0}{0}$.

\begin{solution}
\ans $e^{tA} = \mattwo{1}{t}{0}{1}$.

\soln The matrix $A = \mattwo{0}{1}{0}{0}$ has an eigenvalue $0$ with
multiplicity two.  Theorem~\ref{T:etA} states that 
\[
e^{tA} = a_1(A)P_1(A)\sum_{j=0}^1t^jA^j= a_1(A)P_1(A)(I_2+tA).
\]
In this case $P_1(\lambda)=1=a_1(\lambda)$.   Therefore,
\[
e^{tA} = I_2+tA = \mattwo{1}{t}{0}{1}.
\]


\end{solution}
\end{exercise}
\begin{exercise} \label{c12.1.8b}
$A = \mattwo{-1}{2}{0}{-1}$.

\begin{solution}
\ans $e^{tA} = e^{-t}\mattwo{1}{2t}{0}{1}$.

\soln The matrix $A = \mattwo{-1}{2}{0}{-1}$ has an eigenvalue $-1$ with
multiplicity two.  Theorem~\ref{T:etA} states that 
\[
e^{tA} = a_1(A)P_1(A)\sum_{j=0}^1e^{-t}t^j(A+I_2)^j=
a_1(A)P_1(A)e^{-t}(I_2+t(A+I_2)).
\]
In this case $P_1(\lambda)=1=a_1(\lambda)$.   Therefore,
\[
e^{tA} = e^{-t}(I_2+t(A+I_2)) = e^{-t}\mattwo{1}{2t}{0}{1}.
\]


\end{solution}
\end{exercise}
\begin{exercise} \label{c12.1.8c}
$A = \left(\begin{array}{rrr} 1 & 1 & 2\\ 1 & 1 & 0 \\ 0 & 0 & 0
\end{array}\right)$.

\begin{solution}
\ans $e^{tA}= \dps\frac{1}{4}\left(\begin{array}{rrc} 
1 & -1 & -2+2t\\ -1 & 1 & -2t \\ 0 & 0 & 2\end{array}\right) +e^{2t}\left(\begin{array}{rrr} 2 & 2 & 2\\  2 & 2 & 2 \\ 0 & 0 & 0 
\end{array}\right)$.

\soln The matrix 
$A = \left(\begin{array}{rrr} 1 & 1 & 2\\ 1 & 1 & 0 \\ 0 & 0 & 0
\end{array}\right)$ has eigenvalues $0$ with multiplicity two and $2$ 
with multiplicity one. Theorem~\ref{T:etA} states that 
\begin{eqnarray*}
e^{tA} & = & a_1(A)P_1(A)\sum_{j=0}^1t^jA^j + e^{2t}a_2(A)P_2(A)\\
 & = & a_1(A)P_1(A)(I_3+tA) + e^{2t}a_2(A)P_2(A).
\end{eqnarray*}
The characteristic polynomial of $A$ is $p_A(\lambda) =(2-\lambda)\lambda^2$.
From \eqref{e:Pj} we see that
\[
P_1(\lambda) = 2-\lambda \AND P_2(\lambda) = \lambda^2.
\]
We recall from \eqref{e:1/p} that 
\[
\frac{1}{p_A(\lambda)} = \frac{a_1(\lambda)}{2-\lambda} + 
\frac{a_2(\lambda)}{\lambda^2} = \frac{\frac{1}{4}}{2-\lambda}
+ \frac{\frac{1}{2}+\frac{1}{4}\lambda}{\lambda^2}.
\]
Therefore
\[
a_1(\lambda) = \frac{1}{4} \AND a_2(\lambda) = \frac{1}{2}+\frac{1}{4}\lambda,
\]
and
\begin{eqnarray*}
e^{tA} & = & \frac{1}{4}(2I_3-A)(I_3+tA) + 
e^{2t}\left(\frac{1}{2}I_3+\frac{1}{4}A\right)A^2\\
& = & \frac{1}{4}\left((2I_3-A)+t(2A-A^2)+e^{2t}(2A^2+A^3)\right).
\end{eqnarray*}
Hence
\[
e^{tA}= \frac{1}{4}\left(\left(\begin{array}{rrr} 1 & -1 & -2\\ -1 & 1 & 0 
\\ 0 & 0 & 2\end{array}\right)+t\left(\begin{array}{rrr} 0 & 0 & 2\\ 0 & 0 & -2
\\ 0 & 0 & 0 \end{array}\right)+e^{2t}\left(\begin{array}{rrr} 8 & 8 & 8\\ 
8 & 8 & 8 \\ 0 & 0 & 0\end{array}\right)\right).
\]


\end{solution}
\end{exercise}
\begin{exercise} \label{c12.1.8d}
$A = \left(\begin{array}{rrr} -1 & 2 & -6\\ 0 & 2 & 0 \\ 0 & -1 & 2
\end{array}\right)$.

\begin{solution}
\ans $e^{tA}= \dps\frac{1}{9}e^{2t}\left(\begin{array}{rcr} 
0 & -2-6t & 6\\ 0 & -3 & 0\\ 0 & 1+3t & -3\end{array}\right) 
+e^{-t}\left(\begin{array}{rrr} -6 & 0 & -12\\  0 & 0 & 0 \\ 0 & 0 & 0 
\end{array}\right)$.


\soln The matrix 
$A = \left(\begin{array}{rrr} -1 & 2 & -6\\ 0 & 2 & 0 \\ 0 & -1 & 2
\end{array}\right)$ has eigenvalues $2$ with multiplicity two and $-1$ 
with multiplicity one. Theorem~\ref{T:etA} states that 
\begin{eqnarray*}
e^{tA} & = & a_1(A)P_1(A)e^{2t}\sum_{j=0}^1t^j(A-2I_3)^j+e^{-t}a_2(A)P_2(A)\\
 & = & a_1(A)P_1(A)e^{2t}(I_3+t(A-2I_3)) + e^{-t}a_2(A)P_2(A).
\end{eqnarray*}
The characteristic polynomial of $A$ is 
$p_A(\lambda) =(2-\lambda)^2(-1-\lambda)$.  From \eqref{e:Pj} we see that
\[
P_1(\lambda) = -1-\lambda \AND P_2(\lambda) = (2-\lambda)^2.
\]
We recall from \eqref{e:1/p} that 
\[
\frac{1}{p_A(\lambda)} = \frac{a_1(\lambda)}{-1-\lambda} + 
\frac{a_2(\lambda)}{(\lambda-2)^2} = \frac{\frac{1}{9}}{-1-\lambda}
+ \frac{-\frac{5}{9}+\frac{1}{9}\lambda}{(\lambda-2)^2}.
\]
Therefore
\[
a_1(\lambda) = \frac{1}{9} \AND a_2(\lambda) = -\frac{5}{9}+\frac{1}{9}\lambda,
\]
and
\begin{eqnarray*}
e^{tA} & = & \frac{1}{9}\left((-I_3-A)e^{2t}(I_3+t(A-2I_3)) + 
e^{-t}(A-5I_3)(A-2I_3)^2\right)\\
& = & 
\frac{1}{9}\left(e^{2t}((-I_3-A)+t(-A^2+A+2I_3)+e^{-t}(A-5I_3)(A-2I_3)^2\right).
\end{eqnarray*}
Hence
\[
e^{tA}= \frac{1}{9}\left(e^{2t}\left(\left(\begin{array}{rrr} 0 & -2 & 6\\ 
0 & -3 & 0\\ 0 & 1 & -3\end{array}\right)+t\left(\begin{array}{rrr} 0& -6 & 0\\ 
0 & 0 & 0\\ 0 & 3 & 0 \end{array}\right)\right)+e^{-t}\left(\begin{array}{rrr} 
-54 & 0 & -108\\ 0 & 0 & 0 \\ 0 & 0 & 0\end{array}\right)\right).
\]




\end{solution}
\end{exercise} 

\end{document}
