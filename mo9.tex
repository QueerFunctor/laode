\documentclass{ximera}

 

\usepackage{epsfig}

\graphicspath{
  {./}
  {figures/}
}

\usepackage{morewrites}
\makeatletter
\newcommand\subfile[1]{%
\renewcommand{\input}[1]{}%
\begingroup\skip@preamble\otherinput{#1}\endgroup\par\vspace{\topsep}
\let\input\otherinput}
\makeatother

\newcommand{\includeexercises}{\directlua{dofile("/home/jim/linearAlgebra/laode/exercises.lua")}}

%\newcounter{ccounter}
%\setcounter{ccounter}{1}
%\newcommand{\Chapter}[1]{\setcounter{chapter}{\arabic{ccounter}}\chapter{#1}\addtocounter{ccounter}{1}}

%\newcommand{\section}[1]{\section{#1}\setcounter{thm}{0}\setcounter{equation}{0}}

%\renewcommand{\theequation}{\arabic{chapter}.\arabic{section}.\arabic{equation}}
%\renewcommand{\thefigure}{\arabic{chapter}.\arabic{figure}}
%\renewcommand{\thetable}{\arabic{chapter}.\arabic{table}}

%\newcommand{\Sec}[2]{\section{#1}\markright{\arabic{ccounter}.\arabic{section}.#2}\setcounter{equation}{0}\setcounter{thm}{0}\setcounter{figure}{0}}

\newcommand{\Sec}[2]{\section{#1}}

\setcounter{secnumdepth}{2}
%\setcounter{secnumdepth}{1} 

%\newcounter{THM}
%\renewcommand{\theTHM}{\arabic{chapter}.\arabic{section}}

\newcommand{\trademark}{{R\!\!\!\!\!\bigcirc}}
%\newtheorem{exercise}{}

\newcommand{\dfield}{{\sf dfield9}}
\newcommand{\pplane}{{\sf pplane9}}

\newcommand{\EXER}{\section*{Exercises}}%\vspace*{0.2in}\hrule\small\setcounter{exercise}{0}}
\newcommand{\CEXER}{}%\vspace{0.08in}\begin{center}Computer Exercises\end{center}}
\newcommand{\TEXER}{} %\vspace{0.08in}\begin{center}Hand Exercises\end{center}}
\newcommand{\AEXER}{} %\vspace{0.08in}\begin{center}Hand Exercises\end{center}}

% BADBAD: \newcommand{\Bbb}{\bf}

\newcommand{\R}{\mbox{$\Bbb{R}$}}
\newcommand{\C}{\mbox{$\Bbb{C}$}}
\newcommand{\Z}{\mbox{$\Bbb{Z}$}}
\newcommand{\N}{\mbox{$\Bbb{N}$}}
\newcommand{\D}{\mbox{{\bf D}}}
\usepackage{amssymb}
%\newcommand{\qed}{\hfill\mbox{\raggedright$\square$} \vspace{1ex}}
%\newcommand{\proof}{\noindent {\bf Proof:} \hspace{0.1in}}

\newcommand{\setmin}{\;\mbox{--}\;}
\newcommand{\Matlab}{{M\small{AT\-LAB}} }
\newcommand{\Matlabp}{{M\small{AT\-LAB}}}
\newcommand{\computer}{\Matlab Instructions}
\newcommand{\half}{\mbox{$\frac{1}{2}$}}
\newcommand{\compose}{\raisebox{.15ex}{\mbox{{\scriptsize$\circ$}}}}
\newcommand{\AND}{\quad\mbox{and}\quad}
\newcommand{\vect}[2]{\left(\begin{array}{c} #1_1 \\ \vdots \\
 #1_{#2}\end{array}\right)}
\newcommand{\mattwo}[4]{\left(\begin{array}{rr} #1 & #2\\ #3
&#4\end{array}\right)}
\newcommand{\mattwoc}[4]{\left(\begin{array}{cc} #1 & #2\\ #3
&#4\end{array}\right)}
\newcommand{\vectwo}[2]{\left(\begin{array}{r} #1 \\ #2\end{array}\right)}
\newcommand{\vectwoc}[2]{\left(\begin{array}{c} #1 \\ #2\end{array}\right)}

\newcommand{\ignore}[1]{}


\newcommand{\inv}{^{-1}}
\newcommand{\CC}{{\cal C}}
\newcommand{\CCone}{\CC^1}
\newcommand{\Span}{{\rm span}}
\newcommand{\rank}{{\rm rank}}
\newcommand{\trace}{{\rm tr}}
\newcommand{\RE}{{\rm Re}}
\newcommand{\IM}{{\rm Im}}
\newcommand{\nulls}{{\rm null\;space}}

\newcommand{\dps}{\displaystyle}
\newcommand{\arraystart}{\renewcommand{\arraystretch}{1.8}}
\newcommand{\arrayfinish}{\renewcommand{\arraystretch}{1.2}}
\newcommand{\Start}[1]{\vspace{0.08in}\noindent {\bf Section~\ref{#1}}}
\newcommand{\exer}[1]{\noindent {\bf \ref{#1}}}
\newcommand{\ans}{}
\newcommand{\matthree}[9]{\left(\begin{array}{rrr} #1 & #2 & #3 \\ #4 & #5 & #6
\\ #7 & #8 & #9\end{array}\right)}
\newcommand{\cvectwo}[2]{\left(\begin{array}{c} #1 \\ #2\end{array}\right)}
\newcommand{\cmatthree}[9]{\left(\begin{array}{ccc} #1 & #2 & #3 \\ #4 & #5 &
#6 \\ #7 & #8 & #9\end{array}\right)}
\newcommand{\vecthree}[3]{\left(\begin{array}{r} #1 \\ #2 \\
#3\end{array}\right)}
\newcommand{\cvecthree}[3]{\left(\begin{array}{c} #1 \\ #2 \\
#3\end{array}\right)}
\newcommand{\cmattwo}[4]{\left(\begin{array}{cc} #1 & #2\\ #3
&#4\end{array}\right)}

\newcommand{\Matrix}[1]{\ensuremath{\left(\begin{array}{rrrrrrrrrrrrrrrrrr} #1 \end{array}\right)}}

\newcommand{\Matrixc}[1]{\ensuremath{\left(\begin{array}{cccccccccccc} #1 \end{array}\right)}}



\renewcommand{\labelenumi}{\theenumi)}
\newenvironment{enumeratea}%
{\begingroup
 \renewcommand{\theenumi}{\alph{enumi}}
 \renewcommand{\labelenumi}{(\theenumi)}
 \begin{enumerate}}
 {\end{enumerate}\endgroup}



\newcounter{help}
\renewcommand{\thehelp}{\thesection.\arabic{equation}}

%\newenvironment{equation*}%
%{\renewcommand\endequation{\eqno (\theequation)* $$}%
%   \begin{equation}}%
%   {\end{equation}\renewcommand\endequation{\eqno \@eqnnum
%$$\global\@ignoretrue}}

%\input{psfig.tex}

\author{Martin Golubitsky and Michael Dellnitz}

%\newenvironment{matlabEquation}%
%{\renewcommand\endequation{\eqno (\theequation*) $$}%
%   \begin{equation}}%
%   {\end{equation}\renewcommand\endequation{\eqno \@eqnnum
% $$\global\@ignoretrue}}

\newcommand{\soln}{\textbf{Solution:} }
\newcommand{\exercap}[1]{\centerline{Figure~\ref{#1}}}
\newcommand{\exercaptwo}[1]{\centerline{Figure~\ref{#1}a\hspace{2.1in}
Figure~\ref{#1}b}}
\newcommand{\exercapthree}[1]{\centerline{Figure~\ref{#1}a\hspace{1.2in}
Figure~\ref{#1}b\hspace{1.2in}Figure~\ref{#1}c}}
\newcommand{\para}{\hspace{0.4in}}

\renewenvironment{solution}{\suppress}{\endsuppress}

\ifxake
\newenvironment{matlabEquation}{\begin{equation}}{\end{equation}}
\else
\newenvironment{matlabEquation}%
{\let\oldtheequation\theequation\renewcommand{\theequation}{\oldtheequation*}\begin{equation}}%
  {\end{equation}\let\theequation\oldtheequation}
\fi

\makeatother


\title{mo9.tex}

\begin{document}
\begin{abstract}
BADBAD
\end{abstract}
\maketitle

\chapter{Linear Maps and Changes of Coordinates}

\subsection*{Section~\protect{\ref{Sect:linmap}} Linear Mappings and Bases}
\rhead{Sect:linmap}{LINEAR MAPPINGS AND BASES}

\exer{c7.2.1}
Compute $A$, the matrix of $L$, using Equation~\Ref{e:defA}:
\[ A = (w_1^t|w_2^t|w_3^t)(v_1^t|v_2^t|v_3^t)^{-1} =
\left(\begin{array}{rrr} -1 & 0 & 3 \\ 0 & 1 & 1 \end{array}\right)
\matthree{1}{2}{-2}{0}{-1}{1}{2}{1}{0}^{-1} =
\left(\begin{array}{rrr} -7 & -11 & 3 \\ -4 & -7 & 2
\end{array}\right). \]

\exer{c7.2.2a}
Let $\frac{d}{dt}$ be a transformation that maps $p(t) \mapsto
\frac{d}{dt}p(t)$.  For $p(t) =  p_1 + p_2t + p_3t^2 + p_4t^3$, then
$\frac{d}{dt}p(t) = p_2 + p_3t + p_4t^2$, so $\frac{d}{dt}$ is indeed
a mapping ${\cal P}_3 \rightarrow {\cal P}_2$.  From calculus, we
know that, for any functions $f$ and $g$:
\[ \frac{d}{dt}(f + g)(t) = \frac{d}{dt}f(t) + \frac{d}{dt}g(t), \]
and that, for any scalar $c$:
\[ \frac{d}{dt}(cf)(t) = c\frac{d}{dt}f(t). \]
Let $f$ and $g$ be elements of ${\cal P}_3$.  Then
$\frac{d}{dt}: {\cal P}_3 \rightarrow {\cal P}_2$ is a linear mapping.

\exer{c7.2.2c}
Let $M = \frac{d}{dt} \circ L$ be a mapping ${\cal P}_2
\rightarrow {\cal P}_2$.  The fundamental
theorem of calculus states that, for any function $g$,
\[ \frac{d}{dt}\int_0^t g(\tau)d\tau = g(t). \]
Thus $M(g) = g$ is valid for all values of $g$, so $M$ is the
identity map.

\para To prove this fact explicitly for this case, note that $M$ is
the identity mapping if it maps every polynomial in ${\cal P}_2$ to
itself.  Lemma~\ref{L:linmapfrombasis} states that this mapping can
be uniquely determined by a basis of ${\cal P}_2$.  According to
Exercise~\ref{c7.2.2}, $\{1,t,t^2\}$ is a basis for ${\cal P}_2$. 
Therefore, $M$ is the identity map if $M(1) = 1$, $M(t) = t$ and
$M(t^2) = t^2$:
\[ \frac{d}{dt} \circ L (1) = \frac{d}{dt}\left(\int_0^tds\right) =
\frac{d}{dt}(t) = 1. \]
\[ \frac{d}{dt} \circ L (t) = \frac{d}{dt}\left(\int_0^tsds\right) =
\frac{d}{dt}\left(\frac{t^2}{2}\right) = t. \]
\[ \frac{d}{dt} \circ L (t^2) = \frac{d}{dt}\left(\int_0^ts^2ds\right) =
\frac{d}{dt}\left(\frac{t^3}{3}\right) = t^2. \]
So $\frac{d}{dt} \circ L$ is indeed the identity map for ${\cal P}_2$.

\exer{c7.2.4}
Let $X$ and $Y$ be elements of ${\cal M}(n)$.  Then,
\[ L(X + Y) = A(X + Y) - (X + Y)A = (AX - XA)
+ (AY - YA) = L(X) + L(Y). \]
For any real scalar $c$,
\[ L(cX) = A(cX) - (cX)A = c(AX) - c(XA) = c(AX - XA) = cL(X). \]
Therefore, $L$ is a linear mapping.

\para The null space of $L$ consists of all matrices $X$ such that
$AX - XA = 0$, or $AX = XA$.  By definition, these are the matrices
such that $X$ commutes with $A$.  Let $X$ and $Y$ be elements of
the null space of $L$.  Then show that $X + Y$ is in the null space
by calculating:
\[ L(X + Y) = A(X + Y) - (X + Y)A = AX + AY - XA - YA = AX + AY - AX
- AY = 0. \]
Show that, for any real scalar $c$, $cX$ is in the null space by
calculating:
\[ L(cX) = A(cX) - (cX)A = cAX - cXA = cAX - cAX = 0. \]
Therefore, the null space of $L$ is a subspace consisting of all
matrices that commute with $A$.

\exer{c7.2.6}
Let $p$ and $q$ be elements of ${\cal P}$.  Then,
\[ \begin{array}{rcl}
L(p + q)(x) & = & \int_0^x(t - 1)(p(t) + q(t))dt \\
& = & \int_0^x((t - 1)p(t)dt + (t - 1)q(t)dt) \\
& = & \int_0^x(t - 1)p(t)dt + \int_0^x(t - 1)q(t)dt \\
& = & L(p(x)) + L(q(x)). \end{array} \]
For any real scalar $c$,
\[ L(cp(x)) = \int_0^x(t - 1)cp(t)dt = c\int_0^x(t - 1)p(t)dt
= cL(p(x)). \]
Therefore, $L$ is a linear mapping.



\newpage
\subsection*{Section~\protect{\ref{S:5.8}} Row Rank Equals Column Rank}
\rhead{S:5.8}{ROW RANK EQUALS COLUMN RANK}

\exer{c5.8.1}
\ans The possible choices for the scalars $\alpha_j$ are
$\alpha = (\alpha_1,\alpha_2,\alpha_3) = \alpha_3(-1,-1,1)$ and
the possible choices for the scalars $\beta_j$ are $\beta = 
(\beta_1,\beta_2,\beta_3) = \beta_3(-\frac{7}{5},-\frac{9}{5},1)$.

\soln Find $A^t$ and solve by row reduction the equation
$A^t\alpha = 0$.  To find the scalars $\beta_j$, solve $A\beta =
0$.  These equations yield
\[
-r_1 - r_2 + r_3 = 0 \AND -7c_1 - 9c_2 + 5c_3 = 0.
\]

\exer{c5.8.3}
(a) \ans The vectors $(1,0,1,0)$, $(0,1,-1,0)$ and $(0,0,0,1)$ form a
basis for the row space of $A$, and the row rank of $A$ is $3$.

\soln Row reduce $A$:
\[
\left(\begin{array}{rrrr} 1 & 1 & 0 & 1 \\ 0 & -1 & 1 & 2 \\1 & 2
& -1 & 3 \end{array}\right) \longrightarrow \left(\begin{array}{rrrr}
1 & 0 & 1 & 0 \\ 0 & 1 & -1 & 0 \\0 & 0 & 0 & 1 \end{array}\right).
\]

(b) \ans The column rank of $A$ is $3$, and the vectors $(1,0,0)$,
$(0,1,0)$, and $(0,0,1)$ form a basis for the column space of $A$.

\soln Row reduce $A^t$:
\[
\left(\begin{array}{rrr} 1 & 0 & 1 \\ 1 & -1 & 2 \\ 0 & 1 & -1
\\ 1 & 2 & 3 \end{array}\right) \longrightarrow \left(\begin{array}{rrr}
1 & 0 & 0 \\ 0 & 1 & 0 \\ 0 & 0 & 1 \\ 0 & 0 & 0 \end{array}\right)
\]

(c) \ans The vector $(-1,1,1,0)$ is a basis for the null space.  Since one
vector forms the basis, the nullity of $A$ is $1$.  

\soln Solve $Ax = 0$ by row reducing $A$, which we have already done.

(d) \ans The null space is trivial and the nullity of $A^t$ is $0$.

\soln Find a basis by solving $A^tx = 0$ by row reduction.  The row
reduced matrix:
\[
\left(\begin{array}{rrr} 1 & 0 & 0 \\ 0 & 1 & 0 \\ 0 & 0 & 1
\\ 0 & 0 & 0 \end{array}\right)
\]
implies $x = (0,0,0)$.

\exer{c5.8.5}
We show that $\rank(A) \leq \min\{\rank(B),\rank(C)\}$ by noting that,
if $A = BC$, then the columns of $A$ are linear combinations of the
columns of $B$, so the span of the column space of $A$ cannot exceed
the span of the column space of $B$.  Therefore, $\rank(A) \leq
\rank(B)$.  Next, note that $A^t = C^tB^t$.  By a similar argument,
$\rank(A^t) \leq \rank(C^t)$.  Since $\rank(A) = \rank(A^t)$,
$\rank(A) \leq \min\{\rank(B),\rank(C)\}$.



\newpage
\subsection*{Section~\protect{\ref{S:coordinates}} Vectors and Matrices in
Coordinates}
\rhead{S:coordinates}{VECTORS AND MATRICES IN COORDINATES}

\exer{c7.1.1}
\ans $[v]_{\cal W} = (7,4)$.

\soln Find the scalars $\alpha_1$ and $\alpha_2$ such that $v = \alpha_1w_1
+ \alpha_2w_2$.  That is, solve the linear system
\[ \begin{array}{rrrrr}
\alpha_1 & - & 2\alpha_2 & = & -1 \\
4\alpha_1 & + & \alpha_2 & = & 32 \end{array} \]
to obtain $(\alpha_1,\alpha_2) = (7,4)$, the coordinates
of $v$ in the ${\cal W}$ basis.

\exer{c7.1.3}
(a) By Theorem~\ref{basis=span+indep},
the subset ${\cal V}$ is a basis for the vector space of $2 \times
3$ matrices if the vectors of ${\cal V}$ are linearly independent and
span the vector space.  Let
\[ B = \left(\begin{array}{rrr} b_{11} & b_{12} & b_{13} \\ b_{21} &
b_{22} & b_{23} \end{array}\right). \]
We show that $B$ is in the span of ${\cal V}$ by noting that
$B = b_{11}E_{11} + b_{12}E_{12} + b_{13}E_{13} + b_{21}E_{21}
+ b_{22}E_{22} + b_{23}E_{23}$.  To show that the matrices $E_{ij}$
are linearly independent, suppose $b_{11}E_{11} + b_{12}E_{12} +
b_{13}E_{13} + b_{21}E_{21} + b_{22}E_{22} + b_{23}E_{23} = 0$.  Then,
\[  B = \left(\begin{array}{rrr} b_{11} & b_{12} & b_{13} \\ b_{21} &
b_{22} & b_{23} \end{array}\right) = 0, \]
so $b_{ij} = 0$.
Therefore, {\cal V} is a basis for the given vector space.

(b) \ans $[A]_{\cal V} = (-1,0,2,3,-2,4)$.

\soln Compute $A = -E_{11} + 2E_{13} + 3E_{21} - 2E_{22} + 4E_{23}$.

\exer{c7.1.6}
\ans $[v]_W = (-2,2,-1)$.

\soln Use \Matlab to row reduce the augmented matrix
$(w_1^t|w_2^t|w_3^t|v)$, obtaining:
\begin{verbatim}
ans = 
     1     0     0    -2
     0     1     0     2
     0     0     1    -1
\end{verbatim}

\exer{c7.3.4}
\ans The matrix $[L]_{\cal W}$ is diagonal in the basis:
\[
{\cal W} = \left\{\vectwo{1}{2},\vectwo{2}{3}\right\}
\]

\soln Theorem~\ref{T:putinform2} states
that the matrix $[L]_{\cal W}$ is diagonal if ${\cal W}$ consists of
eigenvectors of $L$ corresponding to real eigenvalues.  By
computation, we find that $\lambda_1 = 2$ and $\lambda_2 = -1$ are
the eigenvalues of $L$.  We then find that $Lw_1 = 2w_1$ when
$w_1 = (1,2)^t$ and $Lw_2 = -w_2$ when $w_2 = (2,3)^t$, so $w_1$ and
$w_2$ are the eigenvectors of $L$.



\newpage
\subsection*{Section~\protect{\ref{MALT}} Matrices of Linear Maps on a
Vector Space}
\rhead{MALT}{MATRICES OF LINEAR MAPS ON A VECTOR SPACE}

\exer{c7.1.2}
\ans
\[ C_{\cal WZ} = \mattwo{2}{3}{-1}{-2}. \]

\soln Substitute into Equation~\Ref{e:coordformn} as follows:
\[ C_{\cal WZ} = (w_1^t|w_2^t)^{-1}(z_1^t|z_2^t) =
\mattwo{1}{0}{2}{1}^{-1}\mattwo{2}{3}{3}{4} =
\mattwo{2}{3}{-1}{-2}. \]

\exer{c7.3.3}
If there exists a nonzero vector $v$ such that $M\circ L(v) = 0$,
then the nullity of $M \circ L$ is nonzero, so $M \circ L$ is not
invertible.  If $L(v) = 0$, then $M \circ L(v) = 0$.  We know that
$\rm{nullity}(L) = \dim V - \dim W > 0$.  Therefore, $M\circ L(v)$
is not invertible

\exer{c7.5.A}
(a) \ans The matrix $A$ fixes $w_1$, moves $w_2$ to $w_3$, and moves
$w_3$ to $-w_2$.

\soln Load the matrix and vectors into \Matlabp.  Then compute $Aw_j$ for
each vector $w_j$, obtaining $Aw_1 = w_1$, $Aw_2 = w_3$, and $Aw_3 = -w_2$.

(b) \ans $[L_A]_{\cal W} = \matthree{1}{0}{0}{0}{0}{-1}{0}{1}{0}$.

\soln From Section~\ref{S:coordinates}, we know that
\[
[L_A]_{\cal W} = P^{-1}AP,
\]
where $P = (w_1|w_2|w_3)$.  Enter {\tt P} into \Matlab and compute this
matrix.

(c) \ans The matrix $[L_A]_{\cal W}$ fixes $e_1$, moves $e_2$ to
$e_3$, and moves $e_3$ to $-e_2$.

\soln Compute $[L_A]_{\cal W}e_1 = e_1$, $[L_A]_{\cal W}e_2 = e_3$, and
$[L_A]_{\cal W}e_3 = -e_2$.  This result is consistent with part (a),
since $[L_A]_{\cal W}$ maps the standard basis vectors to one another
in the same way that $A$ maps the basis vectors of ${\cal W}$ to one
another.
\end{document}
