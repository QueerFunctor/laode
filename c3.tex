\documentclass{ximera}

 

\usepackage{epsfig}

\graphicspath{
  {./}
  {figures/}
}

\usepackage{morewrites}
\makeatletter
\newcommand\subfile[1]{%
\renewcommand{\input}[1]{}%
\begingroup\skip@preamble\otherinput{#1}\endgroup\par\vspace{\topsep}
\let\input\otherinput}
\makeatother

\newcommand{\includeexercises}{\directlua{dofile("/home/jim/linearAlgebra/laode/exercises.lua")}}

%\newcounter{ccounter}
%\setcounter{ccounter}{1}
%\newcommand{\Chapter}[1]{\setcounter{chapter}{\arabic{ccounter}}\chapter{#1}\addtocounter{ccounter}{1}}

%\newcommand{\section}[1]{\section{#1}\setcounter{thm}{0}\setcounter{equation}{0}}

%\renewcommand{\theequation}{\arabic{chapter}.\arabic{section}.\arabic{equation}}
%\renewcommand{\thefigure}{\arabic{chapter}.\arabic{figure}}
%\renewcommand{\thetable}{\arabic{chapter}.\arabic{table}}

%\newcommand{\Sec}[2]{\section{#1}\markright{\arabic{ccounter}.\arabic{section}.#2}\setcounter{equation}{0}\setcounter{thm}{0}\setcounter{figure}{0}}

\newcommand{\Sec}[2]{\section{#1}}

\setcounter{secnumdepth}{2}
%\setcounter{secnumdepth}{1} 

%\newcounter{THM}
%\renewcommand{\theTHM}{\arabic{chapter}.\arabic{section}}

\newcommand{\trademark}{{R\!\!\!\!\!\bigcirc}}
%\newtheorem{exercise}{}

\newcommand{\dfield}{{\sf dfield9}}
\newcommand{\pplane}{{\sf pplane9}}

\newcommand{\EXER}{\section*{Exercises}}%\vspace*{0.2in}\hrule\small\setcounter{exercise}{0}}
\newcommand{\CEXER}{}%\vspace{0.08in}\begin{center}Computer Exercises\end{center}}
\newcommand{\TEXER}{} %\vspace{0.08in}\begin{center}Hand Exercises\end{center}}
\newcommand{\AEXER}{} %\vspace{0.08in}\begin{center}Hand Exercises\end{center}}

% BADBAD: \newcommand{\Bbb}{\bf}

\newcommand{\R}{\mbox{$\Bbb{R}$}}
\newcommand{\C}{\mbox{$\Bbb{C}$}}
\newcommand{\Z}{\mbox{$\Bbb{Z}$}}
\newcommand{\N}{\mbox{$\Bbb{N}$}}
\newcommand{\D}{\mbox{{\bf D}}}
\usepackage{amssymb}
%\newcommand{\qed}{\hfill\mbox{\raggedright$\square$} \vspace{1ex}}
%\newcommand{\proof}{\noindent {\bf Proof:} \hspace{0.1in}}

\newcommand{\setmin}{\;\mbox{--}\;}
\newcommand{\Matlab}{{M\small{AT\-LAB}} }
\newcommand{\Matlabp}{{M\small{AT\-LAB}}}
\newcommand{\computer}{\Matlab Instructions}
\newcommand{\half}{\mbox{$\frac{1}{2}$}}
\newcommand{\compose}{\raisebox{.15ex}{\mbox{{\scriptsize$\circ$}}}}
\newcommand{\AND}{\quad\mbox{and}\quad}
\newcommand{\vect}[2]{\left(\begin{array}{c} #1_1 \\ \vdots \\
 #1_{#2}\end{array}\right)}
\newcommand{\mattwo}[4]{\left(\begin{array}{rr} #1 & #2\\ #3
&#4\end{array}\right)}
\newcommand{\mattwoc}[4]{\left(\begin{array}{cc} #1 & #2\\ #3
&#4\end{array}\right)}
\newcommand{\vectwo}[2]{\left(\begin{array}{r} #1 \\ #2\end{array}\right)}
\newcommand{\vectwoc}[2]{\left(\begin{array}{c} #1 \\ #2\end{array}\right)}

\newcommand{\ignore}[1]{}


\newcommand{\inv}{^{-1}}
\newcommand{\CC}{{\cal C}}
\newcommand{\CCone}{\CC^1}
\newcommand{\Span}{{\rm span}}
\newcommand{\rank}{{\rm rank}}
\newcommand{\trace}{{\rm tr}}
\newcommand{\RE}{{\rm Re}}
\newcommand{\IM}{{\rm Im}}
\newcommand{\nulls}{{\rm null\;space}}

\newcommand{\dps}{\displaystyle}
\newcommand{\arraystart}{\renewcommand{\arraystretch}{1.8}}
\newcommand{\arrayfinish}{\renewcommand{\arraystretch}{1.2}}
\newcommand{\Start}[1]{\vspace{0.08in}\noindent {\bf Section~\ref{#1}}}
\newcommand{\exer}[1]{\noindent {\bf \ref{#1}}}
\newcommand{\ans}{}
\newcommand{\matthree}[9]{\left(\begin{array}{rrr} #1 & #2 & #3 \\ #4 & #5 & #6
\\ #7 & #8 & #9\end{array}\right)}
\newcommand{\cvectwo}[2]{\left(\begin{array}{c} #1 \\ #2\end{array}\right)}
\newcommand{\cmatthree}[9]{\left(\begin{array}{ccc} #1 & #2 & #3 \\ #4 & #5 &
#6 \\ #7 & #8 & #9\end{array}\right)}
\newcommand{\vecthree}[3]{\left(\begin{array}{r} #1 \\ #2 \\
#3\end{array}\right)}
\newcommand{\cvecthree}[3]{\left(\begin{array}{c} #1 \\ #2 \\
#3\end{array}\right)}
\newcommand{\cmattwo}[4]{\left(\begin{array}{cc} #1 & #2\\ #3
&#4\end{array}\right)}

\newcommand{\Matrix}[1]{\ensuremath{\left(\begin{array}{rrrrrrrrrrrrrrrrrr} #1 \end{array}\right)}}

\newcommand{\Matrixc}[1]{\ensuremath{\left(\begin{array}{cccccccccccc} #1 \end{array}\right)}}



\renewcommand{\labelenumi}{\theenumi)}
\newenvironment{enumeratea}%
{\begingroup
 \renewcommand{\theenumi}{\alph{enumi}}
 \renewcommand{\labelenumi}{(\theenumi)}
 \begin{enumerate}}
 {\end{enumerate}\endgroup}



\newcounter{help}
\renewcommand{\thehelp}{\thesection.\arabic{equation}}

%\newenvironment{equation*}%
%{\renewcommand\endequation{\eqno (\theequation)* $$}%
%   \begin{equation}}%
%   {\end{equation}\renewcommand\endequation{\eqno \@eqnnum
%$$\global\@ignoretrue}}

%\input{psfig.tex}

\author{Martin Golubitsky and Michael Dellnitz}

%\newenvironment{matlabEquation}%
%{\renewcommand\endequation{\eqno (\theequation*) $$}%
%   \begin{equation}}%
%   {\end{equation}\renewcommand\endequation{\eqno \@eqnnum
% $$\global\@ignoretrue}}

\newcommand{\soln}{\textbf{Solution:} }
\newcommand{\exercap}[1]{\centerline{Figure~\ref{#1}}}
\newcommand{\exercaptwo}[1]{\centerline{Figure~\ref{#1}a\hspace{2.1in}
Figure~\ref{#1}b}}
\newcommand{\exercapthree}[1]{\centerline{Figure~\ref{#1}a\hspace{1.2in}
Figure~\ref{#1}b\hspace{1.2in}Figure~\ref{#1}c}}
\newcommand{\para}{\hspace{0.4in}}

\renewenvironment{solution}{\suppress}{\endsuppress}

\ifxake
\newenvironment{matlabEquation}{\begin{equation}}{\end{equation}}
\else
\newenvironment{matlabEquation}%
{\let\oldtheequation\theequation\renewcommand{\theequation}{\oldtheequation*}\begin{equation}}%
  {\end{equation}\let\theequation\oldtheequation}
\fi

\makeatother


\title{c3.tex}

\begin{document}
\begin{abstract}
BADBAD
\end{abstract}
\maketitle

\chapter{Matrices and Linearity}  \label{chap:matrices}

\normalsize

In this chapter we take the first step in abstracting vectors and matrices 
to mathematical objects that are more than just arrays of numbers.  We begin 
the discussion in Section~\ref{S:4.1} by introducing the multiplication of a 
matrix times a vector.  Matrix multiplication simplifies the way in which we 
write systems of linear equations and is the way by which we view matrices as 
mappings.  This latter point is discussed in Section~\ref{s:4.2}.

The mappings that are produced by matrix multiplication are special and
are called {\em linear mappings}.  Some properties of
linear maps are discussed in Section~\ref{S:linearity}.  One consequence of
linearity is the {\em principle of superposition\/} that enables
solutions to systems of linear equations to be built out of simpler
solutions.  This principle is discussed in Section~\ref{S:Superposition}.


In Section~\ref{S:4.6} we introduce 
multiplication of two matrices and discuss properties of this multiplication
in Section~\ref{S:4.7}.  Matrix multiplication is defined in terms of
composition of linear mappings which leads to an explicit formula for matrix
multiplication.   This dual role of multiplication of two matrices ---
first by formula and second as composition --- enables us to solve
linear equations in a conceptual way as well as in an algorithmic way.
The conceptual way of solving linear equations is through the use of
matrix inverses (or inverse mappings) which is described in
Section~\ref{S:SLS}.  In this section we also present
important properties of matrix inversion and a method of computation
of matrix inverses.  There is a simple formula for computing inverses
of $2\times 2$ matrices based on determinants.  The chapter ends with a 
discussion of determinants of $2\times 2$ matrices in Section~\ref{S:det2x2}.


\section{Matrix Multiplication of Vectors} \label{S:4.1}
\index{matrix!multiplication}

In Chapter~\ref{lineq} we discussed how matrices appear when solving systems 
of $m$ linear equations in $n$ unknowns.  Given the system
\begin{equation}   \label{general2}
\arraycolsep 2pt
\begin{array}{rcrcrcrcrl}
a_{11}x_1 & + & a_{12}x_2 & + & \cdots & + & a_{1n}x_n & = & b_1 & \\
a_{21}x_1 & + & a_{22}x_2 & + & \cdots & + & a_{2n}x_n & = & b_2 & \\
    & \vdots &      & \vdots &    & \vdots &     & \vdots &   & \\
a_{m1}x_1 & + & a_{m2}x_2 & + & \cdots & + & a_{mn}x_n & = & b_m & ,
\end{array}
\end{equation}
we saw that all relevant information is contained in the $m\times n$ matrix 
of coefficients
\[
A=\left(
\begin{array}{rrrr}
 a_{11} & a_{12} & \cdots & a_{1n} \\
 a_{21} & a_{22} & \cdots & a_{2n}  \\
 \vdots & \vdots &        & \vdots  \\
 a_{m1} & a_{m2} & \cdots & a_{mn}
\end{array}
\right)
\]
and the $n$ vector
\[
b=\left(
\begin{array}{c}
b_1 \\ \vdots \\ b_n
\end{array}
\right).
\]



\subsection*{Matrices Times Vectors}

We motivate multiplication of a matrix times a vector just as a notational 
advance that simplifies the
presentation of the linear systems.  It is, however, much more
than that.  This concept of multiplication allows us to think of
matrices as mappings and these mappings tell us much about the
structure of solutions to linear systems.  But first we discuss
the notational advantage.

Multiplying an $m\times n$ matrix $A$ times an $n$ vector $x$
\index{matrix vector product}
produces an $m$ vector, as follows:
\begin{equation}  \label{Atimesx}
Ax=\left(
\begin{array}{ccc}
 a_{11}  & \cdots & a_{1n} \\
 \vdots  &        & \vdots  \\
 a_{m1}  & \cdots & a_{mn}
\end{array}
\right)
\left(
\begin{array}{c}
x_1 \\ \vdots \\ x_n
\end{array}
\right)
=
\left(
\begin{array}{c}
a_{11}x_1 + \cdots + a_{1n}x_n \\ \vdots \\
a_{m1}x_1 + \cdots + a_{mn}x_n
\end{array}
\right).
\end{equation}
For example, when $m=2$ and $n=3$, then the product is a $2$-vector
\begin{equation} \label{Atimesx231}
\left(
\begin{array}{ccc}
 a_{11}  & a_{12} & a_{13} \\
 a_{21}  & a_{22} & a_{23}
\end{array}
\right)
\left(
\begin{array}{c}
x_1 \\ x_2 \\ x_3
\end{array}
\right)
=
\left(
\begin{array}{c}
a_{11}x_1 + a_{12}x_2 + a_{13}x_3 \\
a_{21}x_1 + a_{22}x_2 + a_{23}x_3
\end{array}
\right).
\end{equation}
As a specific example, compute
\[
\left(
\begin{array}{rrr}
 2  & 3 & -1 \\
 4  & 1 &  5
\end{array}
\right)
\left(
\begin{array}{r}
2 \\ -3 \\ 4
\end{array}
\right)
=
\left(
\begin{array}{ccccc}
2\cdot 2 & + & 3\cdot(-3) & + & (-1)\cdot 4 \\
4\cdot 2 & + & 1\cdot(-3) & + & 5\cdot 4
\end{array}
\right)
=
\left(
\begin{array}{r}
-9 \\ 25
\end{array}
\right).
\]


Using \Ref{Atimesx} we have a compact notation for writing
systems of linear equations.  For example, using a special
instance of \Ref{Atimesx231},
\[
\left(
\begin{array}{rrr}
 2  & 3 & -1 \\
 4  & 1 &  5
\end{array}
\right)
\left(
\begin{array}{r}
x_1 \\ x_2 \\ x_3
\end{array}
\right)
=
\left(
\begin{array}{c}
2x_1 + 3x_2 - x_3 \\
4x_1 + x_2 + 5x_3
\end{array}
\right).
\]
In this notation we can write the system of two linear equations
in three unknowns
\[
\begin{array}{ccccccr}
2x_1 & + & 3x_2 & - &  x_3 & = &   2 \\
4x_1 & + &  x_2 & + & 5x_3 & = &  -1
\end{array}
\]
as the matrix equation
\[
\left(
\begin{array}{rrr}
 2  & 3 & -1 \\
 4  & 1 &  5
\end{array}
\right)
\left(
\begin{array}{r}
x_1 \\ x_2 \\ x_3
\end{array}
\right)
=
\left(
\begin{array}{r}
2 \\ -1
\end{array}
\right).
\]

Indeed, the general system of linear equations \Ref{general2}
can be written in matrix form using matrix multiplication as
\[
Ax=b
\]
where $A$ is the $m\times n$ matrix of coefficients, $x$ is the
$n$ vector of unknowns, and $b$ is the $m$ vector of constants
on the right hand side of \Ref{general2}.


\subsection*{Matrices Times Vectors in \Matlab}

We have already seen how to define matrices and vectors in
\Matlabp.  Now we show how to multiply a matrix times a vector
using \Matlabp\index{matrix vector product!in \protect\Matlab}.

Load the matrix $A$
\begin{equation*}  \label{eq:5matrix}
A=\left(
\begin{array}{rrrrr}
 5 & -4 &  3 & -6 &  2 \\
 2 & -4 & -2 & -1 &  1 \\
 1 &  2 &  1 & -5 &  3 \\
-2 & -1 & -2 &  1 & -1 \\
 1 & -6 &  1 &  1 &  4
\end{array}
\right)
\end{equation*}
and the vector $x$
\begin{equation*} \label{eq:5rhs}
x=\left(
\begin{array}{r}
 -1 \\
  2 \\
  1 \\
 -1 \\
  3
\end{array}
\right)
\end{equation*}
into \Matlab by typing
\begin{verbatim}
e3_1_4
e3_1_5
\end{verbatim}
The multiplication $Ax$ can be performed by typing
\begin{verbatim}
b = A*x
\end{verbatim}  \index{\computer!*}
and the result should be
\begin{verbatim}
b =
     2
    -8
    18
    -6
    -1
\end{verbatim}
We may verify this result by solving the system of linear
equations $Ax=b$.  Indeed if we type
\begin{verbatim}
A\b
\end{verbatim}  \index{\computer!$\backslash$}
then we get the vector $x$ back as the answer.

\EXER

\TEXER


\begin{exercise} \label{c4.1.1}
Let
\[
A = \mattwo{2}{1}{-1}{4} \AND x = \vectwo{3}{-2}.
\]
Compute $Ax$.
\end{exercise}

\begin{exercise} \label{c4.1.2}
Let
\[
B=\left(\begin{array}{ccc} 3 & 4 & 1\\ 1& 2 & 3 \end{array}\right)
\AND y=\left(\begin{array}{r} 2 \\ 5 \\ -2 \end{array}\right).
\]
Compute $By$.
\end{exercise}

\noindent In Exercises~\ref{c4.1.a3a} -- \ref{c4.1.a3d} decide whether or
not the matrix vector product $Ax$ can be computed; if it can, compute the
product.
\begin{exercise} \label{c4.1.a3a}
$A=\mattwo{1}{2}{0}{-5} \AND x = \vectwo{2}{2}$.
\end{exercise}
\begin{exercise} \label{c4.1.a3b}
$A=\mattwo{1}{2}{0}{-5} \AND
x = \left(\begin{array}{r} 2\\ 2 \\4\end{array}\right)$.
\end{exercise}
\begin{exercise} \label{c4.1.a3c}
$A=\left(\begin{array}{rrr} 1 & 2 & 4 \end{array}\right) \AND
x = \left(\begin{array}{r} -1\\ 1 \\3\end{array}\right)$.
\end{exercise}
\begin{exercise} \label{c4.1.a3d}
$A=\left(5\right) \AND x = \vectwo{1}{0}$.
\end{exercise}

\begin{exercise} \label{c4.1.b3}
Let
\[
A=\left(
\begin{array}{rrrr}
 a_{11} & a_{12} & \cdots & a_{1n} \\
 a_{21} & a_{22} & \cdots & a_{2n}  \\
 \vdots & \vdots &        & \vdots  \\
 a_{m1} & a_{m2} & \cdots & a_{mn}
\end{array}
\right)\quad\mbox{and}\quad
x =
\left( \begin{array}{r} x_1\\ x_2\\ \vdots\\ x_n\end{array}\right).
\]
Denote the columns of the matrix $A$ by
\[
A_1 =
\left(\begin{array}{c} a_{11}\\ a_{21}\\ \vdots\\
a_{m1}\end{array}\right),\quad
A_2 =
\left(\begin{array}{c} a_{12}\\ a_{22}\\ \vdots\\
a_{m2}\end{array}\right),\quad
\cdots\quad
A_n =
\left(\begin{array}{c} a_{1n}\\ a_{2n}\\ \vdots\\ a_{mn}\end{array}\right).
\]
Show that the matrix vector product $Ax$ can be written as
\[
Ax = x_1 A_1 + x_2 A_2 + \cdots + x_n A_n,
\]
where $x_j A_j$ denotes scalar multiplication (see Chapter~\ref{chap:prelim}).
\end{exercise}


\begin{exercise} \label{c4.1.3}
Let
\[
C = \mattwo{1}{1}{2}{-1} \AND b = \vectwo{1}{1}.
\]
Find a $2$-vector $z$ such that $Cz=b$.
\end{exercise}

\begin{exercise} \label{c4.1.4}
Write the system of linear equations
\begin{eqnarray*}
2x_1 + 3x_2 - 2x_3 & = & 4\\
6x_1 -5x_3 & = & 1
\end{eqnarray*}
in the matrix form $Ax=b$.
\end{exercise}


\begin{exercise} \label{c4.1.6}
Find all solutions to
\[
\left(\begin{array}{rrrr} 1 & 3 & -1 & 4 \\ 2 & 1 & 5 & 7 \\
3 & 4 & 4 & 11 \end{array} \right)
\left(\begin{array}{c} x_1 \\ x_2 \\ x_3 \\ x_4\end{array}\right) =
\left(\begin{array}{c} 14 \\ 17 \\31 \end{array}\right).
\]
\end{exercise}

\begin{exercise} \label{c4.1.7}
Let $A$ be a $2\times 2$ matrix.  Find $A$ so that
\begin{eqnarray*}
A\left(\begin{array}{c} 1 \\ 0 \end{array}\right) =
\left(\begin{array}{r} 3 \\ -5 \end{array}\right) \\
A\left(\begin{array}{c} 0 \\ 1 \end{array}\right) =
\left(\begin{array}{r} 1 \\ 4 \end{array}\right).
\end{eqnarray*}
\end{exercise}

\begin{exercise} \label{c4.1.8}
Let $A$ be a $2\times 2$ matrix.  Find $A$ so that
\begin{eqnarray*}
A\left(\begin{array}{r} 1 \\ 1 \end{array}\right) =
\left(\begin{array}{r} 2 \\ -1 \end{array}\right) \\
A\left(\begin{array}{r} 1 \\ -1 \end{array}\right) =
\left(\begin{array}{r} 4 \\ 3 \end{array}\right).
\end{eqnarray*}
\end{exercise}

\begin{exercise} \label{c4.1.9}
Is there an upper triangular $2\times 2$ matrix $A$ such that
\begin{equation}  \label{eq:avect}
A\vectwo{1}{0} = \vectwo{1}{2}?
\end{equation}
Is there a symmetric $2\times 2$ matrix $A$ satisfying \Ref{eq:avect}?
\end{exercise}

\CEXER

\noindent In Exercises~\ref{c4.1.a10a} -- \ref{c4.1.a10b} use \Matlab to
compute $b=Ax$ for the given $A$ and $x$.
\begin{exercise} \label{c4.1.a10a}
\begin{equation*}
A=\left(
\begin{array}{rrrrr}
   -0.2 &   -1.8 &    3.9 &    -6 &   -1.6\\
    6.3 &    8   &    3   &    2.5 &    5.1\\
   -0.8 &   -9.9 &    9.7 &    4.7 &    5.9\\
   -0.9 &   -4.1 &    1.1 &   -2.5 &    8.4\\
   -1 &   -9 &   -2 &  -9.8 &    6.9
\end{array}\right)
\AND
x = \left( \begin{array}{r} -2.6\\  2.4\\  4.6\\   -6.1\\    8.1
\end{array}\right).
\end{equation*}
\end{exercise}
\begin{exercise} \label{c4.1.a10b}
\begin{equation*}
A=\left(
\begin{array}{rrrrrrr}
    14 & -22  &-26 &  -2 & -77 & 100 & -90\\
    26 &  25  &-15 & -63 &  33 &  92 &  14\\
   -53 &  40  & 19 &  40 & -27 & -88 &  40\\
    10 & -21  & 13 &  97 & -72 & -28 &  92\\
    86 & -17  & 43 &  61 &  13 &  10 &  50\\
   -33 &  31  &  2 &  41 &  65 & -48 &  48\\
    31 &  68  & 55 &  -3 &  35 &  19 & -14
\end{array}
\right)\AND
x = \left( \begin{array}{r} 2.7\\   6.1\\   -8.3\\    8.9\\    8.3\\    2\\
  -4.9
\end{array}\right).
\end{equation*}
\end{exercise}



\begin{exercise} \label{c4.1.10}
Let
\begin{equation*}
A = \left(\begin{array}{rrr} 2 & 4 & -1 \\ 1 & 3 & 2\\
-1 & -2 & 5 \end{array}\right) \AND
b = \left(\begin{array}{c} 2 \\ 1 \\ 4 \end{array}\right).
\end{equation*}
Find a $3$-vector $x$ such that $Ax=b$.
\end{exercise}

\begin{exercise} \label{c4.1.11}
Let
\begin{equation*}
A = \left(\begin{array}{rrr} 1.3 & -4.15 & -1.2 \\ 1.6 & -1.2 & 2.4\\
-2.5 & 2.35 & 5.09 \end{array}\right) \AND
b = \left(\begin{array}{c} 1.12 \\ -2.1 \\ 4.36 \end{array}\right).
\end{equation*}
Find a $3$-vector $x$ such that $Ax=b$.
\end{exercise}

\begin{exercise} \label{c4.1.12}
Let $A$ be a $3\times 3$ matrix.  Find $A$ so that
\begin{eqnarray*}
A\left(\begin{array}{r} 2 \\ -1 \\ 1 \end{array}\right) =
\left(\begin{array}{r} 1 \\ 1  \\ -1 \end{array}\right) \\
A\left(\begin{array}{r} 1 \\ -1 \\ 0 \end{array}\right) =
\left(\begin{array}{r} -1 \\ -2 \\ 1 \end{array}\right)\\
A\left(\begin{array}{r} 0 \\ 2 \\ 4 \end{array}\right) =
\left(\begin{array}{r} 5 \\ 1 \\ 1 \end{array}\right).
\end{eqnarray*}
{\bf Hint:} Rewrite these three conditions as a system of linear
equations in the nine entries of $A$.  Then solve this system
using \Matlabp.  (Then pray that there is an easier way.)
\end{exercise}


\section{Matrix Mappings}  \index{matrix!mappings} \label{s:4.2}

Having illustrated the notational advantage of using matrices
and matrix multiplication, we now begin to discuss why there
is also a {\em conceptual advantage\/} to matrix
multiplication, a conceptual advantage that will help
us to understand how systems of linear equations and linear
differential equations may be solved.

Matrix multiplication allows us to view $m\times n$ matrices as
mappings from $\R^n$ to $\R^m$.  Let $A$ be an $m\times n$
matrix and let $x$ be an $n$ vector.  Then
\[
x \mapsto Ax
\]
defines a mapping from $\R^n$ to $\R^m$.

The simplest example of a matrix mapping is given by $1\times 1$
matrices.  Matrix mappings defined from $\R\to\R$ are
\[
x \mapsto ax
\]
where $a$ is a real number.  Note that the graph of this
function is just a straight line through the origin (with slope
$a$).  From this example we see that matrix mappings are very
special mappings indeed. In higher dimensions, matrix mappings
provide a richer set of mappings; we explore here {\em planar\/}
mappings\index{planar mappings} --- mappings of the plane into
itself --- using \Matlab graphics and the program {\sf map}.

The simplest planar matrix mappings are the
{\em dilatations\/}\index{dilatation}.
Let $A=cI_2$ where $c>0$ is a scalar.  When $c<1$ vectors are
contracted by a factor of $c$ and and these mappings are
examples of {\em contractions}\index{contraction}.
When $c>1$ vectors are
stretched or expanded by a factor of $c$ and these dilatations
are examples of {\em expansions}\index{expansion}.
We now explore some more complicated planar matrix mappings.

The next planar motions that we study are those given by the
matrices
\[
A=\mattwo{\lambda}{0}{0}{\mu}.
\]
Here the matrix mapping is given by $(x,y)\mapsto(\lambda x,\mu y)$;
that is, a mapping that independently stretches and/or contracts the
$x$ and $y$ coordinates.  Even these simple looking mappings can move 
objects in the plane in a somewhat complicated fashion.

\subsection*{The Program Map}  \index{\computer!map}

We can use \Matlab to explore planar matrix mappings in an efficient way 
using the program {\tt map}.  In \Matlab type the command
\begin{verbatim}
map
\end{verbatim}
and a menu appears labeled {\sf MAP Setup}.  The $2\times 2$ matrix
\[
\mattwo{0}{-1}{1}{0}
\]
has been pre-entered.  Click on the {\sf Proceed} button.  A
window entitled {\sf MAP Display} appears. Click on {\sf Icons}
and click on an icon --- say {\sf Dog}. \index{\computer!dog}
Then click in the {\sf MAP Display} window and a blue `{\sf Dog}' 
will appear in that window.  Now click on
the {\sf Map} button and a new version of the {\sf Dog} will
appear in yellow --- but the yellow {\sf Dog} is rotated about
the origin counterclockwise by $90^\circ$ from the blue dog.
Indeed, this matrix $A$ just rotates the plane counterclockwise
by $90^\circ$.  To verify this statement just click on {\sf Map}
again and see that the yellow dog rotates $90^\circ$
counterclockwise into the magenta dog.  Of course, the magenta
dog is just rotated $180^\circ$ from the original blue dog.
Clicking on {\sf Map} again produces a fourth dog --- this one
in cyan.  Finally one more click on the {\sf map} button will
rotate the cyan dog into a red dog that exactly covers the
original blue dog.

Choose another icon from the {\sf Icons} menu; a blue version of
this icon appears in the {\sf MAP Display} window.  Now click on
{\sf Map} to see that your chosen icon is just rotated
counterclockwise by $90^\circ$.

Other matrices will produce different motions of the plane.  You
may either type the entries of a matrix in the {\sf Map Setup}
window and click on the {\sf Proceed} button or recall one of
the pre-assigned matrices listed in the menu obtained by
clicking on {\sf Gallery}.  For example, clicking on
the {\sf Contracting rotation} button enters the matrix
\[
\mattwo{0.3}{-0.8}{0.8}{0.3}
\]
This matrix rotates the plane through an angle of approximately
$69.4^\circ$ counterclockwise and contracts the plane by a
factor of approximately $0.85$.  Now click on {\sf Dog} in the
{\sf Icons} menu to bring up the blue dog again.  Repeated
clicking on {\sf map} rotates and contracts the dog so that dogs
in a cycling set of colors slowly converge towards the origin in
a spiral of dogs.

\subsubsection*{Rotations}\index{rotation}

Rotating the plane counterclockwise through an angle $\theta$ is
a motion given by a matrix mapping.  We show that the matrix that
performs this rotation is:
\begin{equation} \label{e:rotmat}
R_\theta = \mattwo{\cos\theta}{-\sin\theta}{\sin\theta}{\cos\theta}.
\end{equation}
To verify that $R_\theta$ rotates the plane counterclockwise
through angle $\theta$, let $v_\varphi$ be the unit vector whose
angle from the horizontal is $\varphi$; that is,
$v_\varphi=(\cos\varphi,\sin\varphi)$.  We can write every vector in
$\R^2$ as $rv_\varphi$ for some number $r\ge 0$.   Using the 
trigonometric identities for the cosine and sine of the sum of two angles, we
have:
\begin{eqnarray*}
R_\theta (rv_\varphi) & = &
\mattwo{\cos\theta}{-\sin\theta}{\sin\theta}{\cos\theta}
\vectwo{r\cos\varphi}{r\sin\varphi} \\
& = & \vectwo{r\cos\theta\cos\varphi -r\sin\theta\sin\varphi}
{r\sin\theta\cos\varphi + r\cos\theta\sin\varphi}\\
& = & r\vectwo{\cos(\theta+\varphi)}{\sin(\theta+\varphi)}  \\
& = & rv_{\varphi+\theta}.
\end{eqnarray*}
This calculation shows that $R_\theta$ rotates every vector in the plane
counterclockwise through angle $\theta$.

It follows from \Ref{e:rotmat} that $R_{180^\circ} = -I_2$.  So rotating a
vector in the plane by $180^\circ$ is the same as reflecting the vector
through the origin.  It also follows that the movement associated with the
linear map $x\mapsto -cx$ where $c>0$ may be thought of as a dilatation
($x\mapsto cx$) followed by rotation through $180^\circ$ ($x\mapsto -x$).

We claim that combining dilatations with general rotations produces spirals.  
Consider the matrix
\[
S = \mattwo{c\cos\theta}{-c\sin\theta}{c\sin\theta}{c\cos\theta} = cR_\theta
\]
where $c<1$.  Then a calculation similar to the previous one shows that
\[
S(rv_\varphi) = c(rv_{\varphi+\theta}).
\]
So $S$ rotates vectors in the plane while contracting them by
the factor $c$.  Thus, multiplying a vector repeatedly by $S$ 
spirals that vector into the origin.  The example that we just 
considered while using {\sf map} is
\[
\mattwo{0.3}{-0.8}{0.8}{0.3}\cong\mattwo{0.85\cos(69.4^\circ)}
{-0.85\sin(69.4^\circ)}{0.85\sin(69.4^\circ)}{0.85\cos(69.4^\circ)},
\]
which has the general form of $S$.

\subsection*{A Notation for Matrix Mappings}
\index{matrix!mappings}

We reinforce the idea that matrices are mappings by introducing a notation 
for the mapping associated with an $m\times n$ matrix $A$.  Define
\[
L_A:\R^n\to\R^m
\]
by
\[
L_A(x) = Ax,
\]
for every $x\in\R^n$.

There are two special matrices:  the $m\times n$ zero matrix 
\index{matrix!zero} $O$ all of whose entries are $0$ and the 
$n\times n$ identity matrix \index{matrix!identity} $I_n$ whose diagonal 
entries are $1$ and whose off diagonal entries are $0$.  For instance,
\[
	I_3 = \left(
\begin{array}{rrr}
 1 & 0 & 0  \\
 0 & 1 & 0  \\
 0 & 0 & 1
\end{array}
\right).
\]

The mappings associated with these special matrices are also special.  
Let $x$ be an $n$ vector.  Then
\begin{equation} \label{multby0}
Ox=0,
\end{equation}
where the $0$ on the right hand side of \Ref{multby0} is the $m$
vector all of whose entries are $0$.  The mapping $L_O$ is the 
{\em zero mapping\/} \index{zero mapping} --- the mapping 
that maps every vector $x$ to $0$.

Similarly,
\[
I_nx=x
\]
for every vector $x$.  It follows that
\[
L_{I_n}(x) = x
\]
is the {\em identity mapping\/}, \index{identity mapping} since
it maps every element to itself.  It is for this reason that the
matrix $I_n$ is called the $n\times n$ {\em identity matrix\/}.

\EXER

\TEXER

\noindent In Exercises~\ref{c4.2.a1a} -- \ref{c4.2.a1c} find a nonzero
vector that is mapped to the origin by the given matrix.
\begin{exercise} \label{c4.2.a1a}
$A=\mattwo{0}{1}{0}{-2}$.
\end{exercise}
\begin{exercise} \label{c4.2.a1b}
$B=\mattwo{1}{2}{-2}{-4}$.
\end{exercise}
\begin{exercise} \label{c4.2.a1c}
$C=\mattwo{3}{-1}{-6}{2}$.
\end{exercise}


\begin{exercise} \label{c4.2.1a}
What $2\times 2$ matrix rotates the plane about the origin counterclockwise
by $30^\circ$?
\end{exercise}
\begin{exercise} \label{c4.2.1b}
What $2\times 2$ matrix rotates the plane clockwise by $45^\circ$?
\end{exercise}
\begin{exercise} \label{c4.2.1c}
What $2\times 2$ matrix rotates the plane clockwise by $90^\circ$ while
dilating it by a factor of $2$?
\end{exercise}

\begin{exercise} \label{c4.2.2a}
Find a $2\times 2$ matrix that reflects vectors in the $(x,y)$ plane across
the $x$ axis.
\end{exercise}
\begin{exercise} \label{c4.2.2b}
Find a $2\times 2$ matrix that reflects vectors in the $(x,y)$ plane across
the $y$ axis.
\end{exercise}
\begin{exercise} \label{c4.2.2c}
Find a $2\times 2$ matrix that reflects vectors in the $(x,y)$ plane across
the line $x=y$.
\end{exercise}

\begin{exercise} \label{c7.8.1}
The matrix
\[
A=\mattwo{1}{K}{0}{1}
\]
is a {\em shear}\index{shear}.  Describe the action of $A$ on the plane
for different values of $K$.
\end{exercise}

\begin{exercise} \label{c7.8.2}
Determine a rotation matrix that maps the vectors $(3,4)$ and
$(1,-2)$ onto the vectors $(-4,3)$ and $(2,1)$ respectively.
\end{exercise}



\begin{exercise} \label{c4.2.3}
Find a $2\times 3$ matrix $P$ that projects three dimensional $xyz$ space onto
the $xy$ plane.  {\bf Hint:} Such a matrix will satisfy
\[
P\left(\begin{array}{c} 0 \\ 0 \\ z \end{array}\right) = \vectwo{0}{0}
\AND
P\left(\begin{array}{c} x \\ y \\ 0 \end{array}\right) = \vectwo{x}{y}.
\]
\end{exercise}

\begin{exercise}	\label{c4.2.3a}
Show that every matrix of the form $\mattwo{a}{-b}{b}{a}$ corresponds to
rotating the plane through the angle $\theta$ followed by a dilatation 
$cI_2$ where
\begin{eqnarray*}
c & = & \sqrt{a^2+b^2}\\
\cos\theta & = & \frac{a}{c} \\
\sin\theta & = & \frac{b}{c}.
\end{eqnarray*}
\end{exercise}

\begin{exercise}  \label{c4.2.3b}
Using Exercise~\ref{c4.2.3a} observe that the matrix
$\mattwo{3}{4}{-4}{3}$ rotates the plane counterclockwise through
an angle $\theta$ and then dilates the planes by a factor of $c$.
Find $\theta$ and $c$.  Use {\tt map} to verify your results.
\end{exercise}


\CEXER


\noindent In Exercises~\ref{c4.2.a4a} -- \ref{c4.2.a4c} use {\sf map} to
find vectors that are stretched and/or contracted to a multiple of
themselves by the given linear mapping.  {\bf Hint:}  Choose a vector in the
{\sf MAP Display} window and apply {\sf Map} several times.
\begin{exercise} \label{c4.2.a4a}
$A=\mattwoc{2}{0}{1.5}{0.5}$.
\end{exercise}
\begin{exercise} \label{c4.2.a4b}
$B=\mattwo{1.2}{-1.5}{-0.4}{1.2}$.
\end{exercise}
\begin{exercise} \label{c4.2.a4c}
$C=\mattwo{2}{-1.25}{0}{-0.5}$.
\end{exercise}

\noindent In Exercises~\ref{c4.2.ba} -- \ref{c4.2.bc} use
Exercise~\ref{c4.2.3a} and {\sf map} to verify that the given matrices
rotate the plane through an angle $\theta$ followed by a dilatation $cI_2$.
Find $\theta$ and $c$ in each case.
\begin{exercise}  \label{c4.2.ba}
$A=\mattwo{1}{-2}{2}{1}$.
\end{exercise}
\begin{exercise}  \label{c4.2.bb}
$B=\mattwo{-2.4}{-0.2}{0.2}{-2.4}$.
\end{exercise}
\begin{exercise}  \label{c4.2.bc}
$C=\mattwoc{2.67}{1.3}{-1.3}{2.67}$.
\end{exercise}


\noindent In Exercises~\ref{c4.2.4a} -- \ref{c4.2.4e} use {\sf map} to
help describe the planar motions of the associated linear mappings for the
given $2\times 2$ matrix.
\begin{exercise} \label{c4.2.4a}
$A=\mattwo{\frac{\sqrt{3}}{2}}{\frac{1}{2}}{-\frac{1}{2}}{\frac{\sqrt{3}}{2}}$.
\end{exercise}
\begin{exercise} \label{c4.2.4b}
$B = \mattwo{\frac{1}{2}}{-\frac{1}{2}}{\frac{1}{2}}{\frac{1}{2}}$.
\end{exercise}
\begin{exercise} \label{c4.2.4c}
$C = \mattwo{0}{1}{1}{0}$.
\end{exercise}
\begin{exercise} \label{c4.2.4d}
$D = \mattwo{1}{0}{0}{0}$.
\end{exercise}
\begin{exercise} \label{c4.2.4e}
$E = \mattwo{\frac{1}{2}}{\frac{1}{2}}{\frac{1}{2}}{\frac{1}{2}}$.
\end{exercise}

\begin{exercise}  \label{c4.2.5}
The matrix 
\[
A = \mattwo{0}{-1}{-1}{0}
\]
reflects the $xy$-plane across the diagonal line $y=-x$ while the matrix
\[
B=\mattwo{-1}{0}{0}{-1}
\]
rotates the plane through an angle of $180^\circ$. Using the program 
{\sf map} verify that both matrices map the vector $(1,1)$ to its negative
$(-1,-1)$.  Now perform two experiments.  First using the {\sf icon} menu in 
{\sf map} place a dog icon at about the point $(1,1)$ and move that dog by 
matrix $A$.  Then replace the dog in its orginal position near $(1,1)$ and 
move that dog using matrix $B$.  Describe the difference in the result.
\end{exercise}   


\section{Linearity}  \label{S:linearity}

We begin by recalling the vector operations of addition and
scalar multiplication.  Given two $n$ vectors, vector addition
\index{vector!addition} is defined by
\[
\vect{x}{n}+\vect{y}{n}=\left(\begin{array}{c} x_1+y_1 \\ \vdots \\
x_n+y_n\end{array}\right).
\]
Multiplication of a scalar \index{scalar multiplication} times a vector
is defined by
\[
c\vect{x}{n} = \vect{cx}{n}.
\]
Using \Ref{Atimesx} we can check that matrix multiplication
satisfies
\begin{eqnarray}
A(x+y) & = & Ax + Ay \label{sum} \\
A(cx) & = & c(Ax). \label{product}
\end{eqnarray}
Using \Matlab we can also verify that the identities \Ref{sum}
and \Ref{product} are valid for some particular choices of $x$,
$y$, $c$ and $A$.  For example, let
\begin{equation*}
A = \left(\begin{array}{cccc} 2 & 3 & 4 & 1\\ 1 & 1 & 2 & 3
\end{array}\right), \quad x = \left(\begin{array}{r} 1 \\ 5 \\ 4 \\
3 \end{array}\right), \quad y = \left(\begin{array}{r} 1 \\ -1 \\ -1 \\
4 \end{array}\right), \AND c=5.
\end{equation*}
Typing {\tt e3\_3\_3} enters this information into \Matlabp.  Now
type
\begin{verbatim}
z1 = A*(x+y)
z2 = A*x + A*y
\end{verbatim}
and compare {\tt z1} and {\tt z2}.  The fact that they are both
equal to
\[
\vectwo{35}{33}
\]
verifies \Ref{sum} in this case.  Similarly, type
\begin{verbatim}
w1 = A*(c*x)
w2 = c*(A*x)
\end{verbatim}
and compare {\tt w1} and {\tt w2} to verify \Ref{product}.


The central idea in linear algebra is the notion of
{\em linearity\/}. \index{linear}
\begin{Def} \label{linearity}
A mapping $L:\R^n\to\R^m$ is {\em linear}\index{linear!mapping}
if
\begin{itemize}
\item[(a)]  $L(x+y) = L(x) + L(y)$ for all $x,y\in\R^n$.
\item[(b)]  $L(cx) = cL(x)$ for all $x\in\R^n$ and all scalars
$c\in\R$.
\end{itemize}
\end{Def}

To better understand the meaning of Definition~\ref{linearity}(a,b),
we verify these conditions for the mapping $L:\R^2\to\R^2$ defined by
\begin{equation}  \label{E:mme}
L(x) = (x_1+3x_2,2x_1-x_2),
\end{equation}
where $x=(x_1,x_2)\in\R^2$.  To verify Definition~\ref{linearity}(a), let
$y=(y_1,y_2)\in\R^2$.  Then
\begin{eqnarray*}
L(x+y) & = & L(x_1+y_1,x_2+y_2)\\
& = & ((x_1+y_1)+3(x_2+y_2), 2(x_1+y_1)-(x_2+y_2)) \\
 & = & (x_1+y_1+3x_2+3y_2, 2x_1+2y_1-x_2-y_2).
\end{eqnarray*}
On the other hand,
\begin{eqnarray*}
L(x)+L(y) & = & (x_1+3x_2,2x_1-x_2) + (y_1+3y_2,2y_1-y_2) \\
 & = & (x_1+3x_2+y_1+3y_2, 2x_1-x_2+2y_1-y_2).
\end{eqnarray*}
Hence
\[
L(x+y) = L(x) + L(y)
\]
for every pair of vectors $x$ and $y $ in $\R^2$.

Similarly, to verify Definition~\ref{linearity}(b), let $c\in\R$ be a scalar
and compute
\[
L(cx) = L(cx_1,cx_2)=((cx_1)+3(cx_2),2(cx_1)-(cx_2)).
\]
Then compute
\[
cL(x) = c(x_1+3x_2,2x_1-x_2)= (c(x_1+3x_2), c(2x_1-x_2)),
\]
from which it follows that
\[
L(cx) = cL(x)
\]
for every vector $x\in\R^2$ and every scalar $c\in\R$.  Thus $L$ is a linear 
mapping.

In fact, the mapping \Ref{E:mme} is a matrix mapping and could
have been written in the form
\[
L(x) = \mattwo{1}{3}{2}{-1}x.
\]
Hence the linearity of $L$ could have been checked using identities
\Ref{sum} and \Ref{product}.  Indeed, matrix mappings are always linear
mappings, as we now discuss.

\subsubsection*{Matrix Mappings are Linear Mappings}

Let $A$ be an $m\times n$ matrix and recall that the matrix mapping
$L_A:\R^n\to\R^m$ is defined by $L_A(x)=Ax$.  We may rewrite \Ref{sum} and
\Ref{product} using this notation as
\begin{eqnarray*}
L_A(x+y) & = & L_A(x) + L_A(y) \\
L_A(cx) & = & cL_A(x).
\end{eqnarray*}
Thus all matrix mappings\index{matrix!mappings} are linear
mappings\index{linear!mapping}.  We will show that all linear
mappings are matrix mappings (see Theorem~\ref{lin-matrices}).  But first
we discuss linearity in the simplest context of mappings from $\R\to\R$.

\subsection*{Linear and Nonlinear Mappings of $\R\to\R$}

Note that $1\times 1$ matrices are just scalars $A=(a)$.  It follows from
\Ref{sum} and \Ref{product} that we have shown that the matrix mappings
$L_A(x)=ax$ are all linear, though this point could have been verified
directly.  Before showing that these are all the linear mappings of
$\R\to\R$, we focus on examples of functions of $\R\to\R$ that are
{\em not\/} linear.

\subsubsection*{Examples of Mappings that are Not Linear}

\begin{itemize}
\item   $f(x)=x^2$.  Calculate
\[
f(x+y) = (x+y)^2 = x^2+2xy+y^2
\]
while
\[
f(x)+f(y) = x^2 + y^2.
\]
The two expressions are not equal and $f(x)=x^2$ is not linear.
\item   $f(x)=e^x$.  Calculate
\[
f(x+y) = e^{x+y} = e^x e^y
\]
while
\[
f(x)+f(y) = e^x + e^y.
\]
The two expressions are not equal and $f(x)=e^x$ is not linear.
\item   $f(x) = \sin x$.  Recall that
\[
f(x+y) =\sin(x+y) = \sin x \cos y +\cos x \sin y
\]
while
\[
f(x)+f(y) = \sin x + \sin y.
\]
The two expressions are not equal and $f(x)=\sin x $ is not
linear.
\end{itemize}

\subsubsection*{Linear Functions of One Variable}

Suppose we take the opposite approach and ask what functions of
$\R\to\R$ are linear.  Observe that if $L:\R\to\R$ is linear,
then
\[
L(x) = L(x\cdot 1).
\]
Since we are looking at the special case of linear mappings on
$\R$, we note that $x$ is a real number as well as a vector.
Thus we can use Definition~\ref{linearity}(b) to observe that
\[
L(x\cdot 1)=xL(1).
\]
So if we let $a=L(1)$, then we see that
\[
L(x)=ax.
\]
Thus linear mappings of $\R$ into $\R$ are very special mappings
indeed; they are all scalar multiples of the identity mapping.

\subsection*{All Linear Mappings are Matrix Mappings}

We end this section by proving that every linear mapping is
given by matrix multiplication. But first we state and prove two
lemmas.  There is a standard set of vectors that is used over
and over again in linear algebra, which we now define.

\begin{Def}  \label{D:canonicalbasis}
Let $j$ be an integer between $1$ and $n$.  The $n$-vector $e_j$ is
the vector that has a $1$ in the $j^{th}$ entry and zeros in all
other entries.
\end{Def} \index{$e_j$}

\begin{lemma}  \label{linequal}
Let $L_1:\R^n\to\R^m$ and $L_2:\R^n\to\R^m$ be linear mappings.
Suppose that $L_1(e_j)=L_2(e_j)$ for every $j=1,\ldots,n$.  Then
$L_1=L_2$.
\end{lemma}

\begin{proof}  Let $x=(x_1,\ldots,x_n)$ be a vector in $\R^n$.  Then
\[
x = x_1e_1 + \cdots + x_ne_n.
\]
Linearity of $L_1$ and $L_2$ implies that
\begin{eqnarray*}
L_1(x) & = & x_1L_1(e_1) + \cdots + x_nL_1(e_n) \\
  & = & x_1L_2(e_1) + \cdots + x_nL_2(e_n) \\
  & = & L_2(x).
\end{eqnarray*}
Since $L_1(x)=L_2(x)$ for all $x\in\R^n$, it follows that
$L_1=L_2$.  \end{proof}

\begin{lemma}  \label{columnsA}
Let $A$ be an $m\times n$ matrix.  Then $Ae_j$ is the $j^{th}$
column of $A$.
\end{lemma}

\begin{proof}  Recall the definition of matrix multiplication given in \Ref{Atimesx}.
In that formula, just set $x_i$ equal to zero for all $i\neq j$ and set
$x_j=1$. \end{proof}

\begin{thm}  \label{lin-matrices}
Let $L:\R^n\to\R^m$ be a linear mapping\index{linear!mapping}.
Then there exists an $m\times n$ matrix $A$ such that $L=L_A$.
\end{thm}

\begin{proof}
There are two steps to the proof: determine the matrix $A$ and
verify that $L_A=L$.

Let $A$ be the matrix whose $j^{th}$ column is $L(e_j)$.  By
Lemma~\ref{columnsA} $L(e_j) = Ae_j$; that is, $L(e_j) = L_A(e_j)$.
Lemma~\ref{linequal} implies that $L=L_A$.  \end{proof}

Theorem~\ref{lin-matrices} provides a simple way of showing that
\[
L(0) = 0
\]
for any linear map $L$.  Indeed, $L(0)=L_A(0)=A0=0$ for some matrix $A$.  
(This fact can also be proved directly from the definition of linear mapping.)

\subsubsection*{Using Theorem~\protect\ref{lin-matrices} to Find Matrices
Associated to Linear Maps}

The proof of Theorem~\ref{lin-matrices} shows that the $j^{th}$ column of the
matrix $A$ associated to a linear mapping $L$ is $L(e_j)$ viewed as a column
vector.  As an example, let $L:\R^2\to\R^2$ be rotation clockwise through
$90^\circ$.  Geometrically, it is easy to see that
\[
L(e_1) = L\left(\vectwo{1}{0}\right) = \vectwo{0}{-1} \AND
L(e_2) = L\left(\vectwo{0}{1}\right) = \vectwo{1}{0}.
\]
Since we know that rotations are linear maps, it follows that the matrix
$A$ associated to the linear map $L$ is:
\[
A = \mattwo{0}{1}{-1}{0}.
\]
Additional examples of linear mappings whose associated matrices can be found
using Theorem~\ref{lin-matrices} are given in Exercises \ref{c4.3.7} --
\ref{c4.3.10}.



\EXER

\TEXER

\begin{exercise} \label{c4.3.1}
Compute $ax+by$ for each of the following:
\begin{itemize}
\item[(a)] $a=2$, $b=-3$, $x=(2,4)$ and $y=(3,-1)$.
\item[(b)] $a=10$, $b=-2$, $x=(1,0,-1)$ and $y=(2,-4,3)$.
\item[(c)] $a=5$, $b=-1$, $x=(4,2,-1,1)$ and $y=(-1,3,5,7)$.
\end{itemize}
\end{exercise}

\begin{exercise} \label{c4.3.2}
Let $x=(4,7)$ and $y=(2,-1)$.  Write the vector $\alpha x+\beta
y$ as a vector in coordinates.
\end{exercise}

\begin{exercise} \label{c4.3.3}
Let $x=(1,2)$, $y=(1,-3)$, and $z=(-2,-1)$.  Show that you can
write
\[
z=\alpha x+ \beta y
\]
for some $\alpha,\beta\in\R$.

\noindent {\bf Hint:} Set up a system of two linear equations in
the unknowns $\alpha$ and $\beta$, and then solve this linear
system.
\end{exercise}

\begin{exercise} \label{c4.3.4}
Can the vector $z=(2,3,-1)$ be written as
\[
z=\alpha x+ \beta y
\]
where $x=(2,3,0)$ and $y=(1,-1,1)$?
\end{exercise}

\begin{exercise} \label{c4.3.5}
Let $x=(3,-2)$, $y=(2,3)$, and $z=(1,4)$.  For which real
numbers $\alpha,\beta,\gamma$ does
\[
\alpha x + \beta y + \gamma z = (1,-2)?
\]
\end{exercise}

\noindent In Exercises~\ref{c4.3.6a} -- \ref{c4.3.6d} determine
whether the given transformation is linear.
\begin{exercise} \label{c4.3.6a}
  $T:\R^3\to\R^2$ defined by $T(x_1,x_2,x_3)=(x_1+2x_2-x_3,x_1-4x_3)$.
\end{exercise}
\begin{exercise} \label{c4.3.6b}
  $T:\R^2\to\R^2$ defined by $T(x_1,x_2)=(x_1+x_1x_2,2x_2)$.
\end{exercise}
\begin{exercise} \label{c4.3.6c}
  $T:\R^2\to\R^2$ defined by $T(x_1,x_2)=(x_1+x_2,x_1-x_2-1)$.
\end{exercise}
\begin{exercise} \label{c4.3.6d}
  $T:\R^2\to\R^3$ defined by $T(x_1,x_2)=(1,x_1+x_2,2x_2)$
\end{exercise}

\begin{exercise} \label{c4.3.7}
Find the $2\times 3$ matrix $A$ that satisfies
\[
Ae_1  =  \vectwo{2}{3},\qquad
Ae_2  =  \vectwo{1}{-1}, \AND
Ae_3  = \vectwo{0}{1}.
\]
\end{exercise}

\begin{exercise} \label{c4.3.8}
The {\em cross product\/} of two $3$-vectors $x=(x_1,x_2,x_3)$
and $y=(y_1,y_2,y_3)$ is the $3$-vector
\[
x\times y = (x_2y_3-x_3y_2,-(x_1y_3-x_3y_1),x_1y_2-x_2y_1).
\]
Let $K=(2,1,-1)$.  Show that the mapping $L:\R^3\to\R^3$ defined by
\[
L(x) = x\times K
\]
is a linear mapping.  Find the $3\times 3$ matrix $A$ such that
\[
L(x) = Ax,
\]
that is, $L=L_A$.
\end{exercise}

\begin{exercise} \label{c4.3.9}
Argue geometrically that rotation of the plane counterclockwise
through an angle of $45^\circ$ is a linear mapping.  Find a
$2\times 2$ matrix $A$ such that $L_A$ rotates the plane
counterclockwise by $45^\circ$.
\end{exercise}

\begin{exercise} \label{c4.3.10}
Let $\sigma$ permute coordinates cyclically in $\R^3$; that is,
\[
\sigma(x_1,x_2,x_3) = (x_2,x_3,x_1).
\]
Find a $3\times 3$ matrix $A$ such that $\sigma = L_A$.
\end{exercise}

\begin{exercise} \label{c4.3.11}
Let $L$ be a linear map.  Using the definition of linearity,
prove that $L(0)=0$.
\end{exercise}

\begin{exercise}  \label{c4.3.12}
Let $L_1:\R^n\to\R^m$ and $L_2:\R^n\to\R^m$ be linear mappings. Prove
that $L:\R^n\to\R^m$ defined by
\[
L(x) = L_1(x) + L_2(x)
\]
is also a linear mapping.  Theorem~\ref{lin-matrices} states that there
are matrices $A$, $A_1$ and $A_2$ such that
\[
L = L_A \AND L_j = L_{A_j}
\]
for $j=1,2$.  What is the relationship between the matrices $A$, $A_1$
and $A_2$?
\end{exercise}

\CEXER

\begin{exercise} \label{c4.3.13}
Let
\[
A = \mattwo{0.5}{0}{0}{2}.
\]
Use {\sf map} to verify that the linear mapping $L_A$ halves
the $x$-component of a point while it doubles the $y$-component.
\end{exercise}

\begin{exercise} \label{c4.3.14}
Let
\[
A = \mattwo{0}{0.5}{-0.5}{0}.
\]
Use {\sf map} to determine how the mapping $L_A$ acts on $2$-vectors.
Describe this action in words.
\end{exercise}

\noindent In Exercises~\ref{c4.3.15A} -- \ref{c4.3.15B} use \Matlab to
verify \Ref{sum} and \Ref{product}.
\begin{exercise} \label{c4.3.15A}
\begin{equation*} \label{eq4.3.15a}
A = \left(
\begin{array}{rrr}
 1 & 2 & 3  \\
 0 & 1 & -2  \\
 4 & 0 & 1
\end{array}
\right),\quad
x=\left(
\begin{array}{r}
 3   \\
 2   \\
 -1
\end{array}
\right),\quad
y=\left(
\begin{array}{r}
 0   \\
 -5   \\
 10
\end{array}
\right),\quad c=21;
\end{equation*}
\end{exercise}
\begin{exercise} \label{c4.3.15B}
\begin{equation*} \label{eq4.3.15b}
A = \left(
\begin{array}{rrrrr}
 4 & 0 & -3 & 2 & 4 \\
 2 & 8 & -4 & -1 & 3 \\
 -1 & 2 & 1 & 10 & -2 \\
 4 & 4 & -2 & 1 & 2 \\
 -2 & 3 & 1 & 1 & -1
\end{array}
\right),\quad
x=\left(
\begin{array}{r}
 1   \\
 3   \\
 -2   \\
 3   \\
 -1
\end{array}
\right),\quad
y=\left(
\begin{array}{r}
 2   \\
 0   \\
 13   \\
 -2   \\
 1
\end{array}
\right),\quad c=-13.
\end{equation*}
\end{exercise}




\section{The Principle of Superposition}
\label{S:Superposition}

The principle of superposition is just a restatement of the fact
that matrix mappings are linear.  Nevertheless, this restatement
is helpful when trying to understand the structure of solutions
to systems of linear equations.

\subsection*{Homogeneous Equations}
\index{homogeneous}

A system of linear equations is {\em homogeneous\/} if it has
the form
\begin{equation} \label{homosys}
Ax=0,
\end{equation}
where $A$ is an $m\times n$ matrix and $x\in\R^n$.  Note that
homogeneous systems are consistent since $0\in\R^n$ is always a
solution, that is, $A(0)=0$.

The {\em principle of superposition\/} \index{principle of
superposition} \index{superposition} makes two assertions:
\begin{itemize}
\item  Suppose that $y$ and $z$ in $\R^n$ are solutions to \Ref{homosys}
(that is, suppose that $Ay=0$ and $Az=0$); then $y+z$ is a solution
to \Ref{homosys}.
\item Suppose that $c$ is a scalar; then $cy$ is a solution to
\Ref{homosys}.
\end{itemize}
The principle of superposition is proved using the linearity of matrix 
multiplication.  Calculate
\[
A(y+z) = Ay + Az = 0+0=0
\]
to verify that $y+z$ is a solution, and calculate
\[
A(cy) = c(Ay) = c\cdot 0 = 0
\]
to verify that $cy$ is a solution.

We see that solutions to homogeneous systems of linear equations
always satisfy the general property of superposition: sums of
solutions are solutions and scalar multiples of solutions are
solutions.

We illustrate this principle by explicitly solving the system of
equations
\[
\left(\begin{array}{rrrr} 1 & 2 & -1 & 1\\ 2 & 5 & -4 & -1
\end{array}\right)\left(\begin{array}{c} x_1\\x_2\\x_3\\x_4
\end{array}\right) = \left(\begin{array}{c} 0\\0
\end{array}\right).
\]
Use row reduction to show that the matrix
\[
\left(\begin{array}{rrrr} 1 & 2 & -1 & 1\\ 2 & 5 & -4 & -1
\end{array}\right)
\]
is row equivalent to
\[
\left(\begin{array}{rrrr} 1 & 0 & 3 & 7\\ 0 & 1 & -2 & -3
\end{array}\right)
\]
which is in reduced echelon form.  Recall, using the methods of
Section~\ref{S:Gauss}, that every
solution to this linear system has the form
\[
\left(\begin{array}{c} -3x_3-7x_4\\ 2x_3+3x_4\\ x_3\\
x_4\end{array}\right) =
x_3\left(\begin{array}{r}-3\\2\\1\\0\end{array}\right) +
x_4\left(\begin{array}{r}-7\\3\\0\\1\end{array}\right).
\]
Superposition is verified again by observing that the form of 
the solutions is preserved under vector addition and scalar
multiplication.  For instance, suppose that
\[
\alpha_1 \left(\begin{array}{r}-3\\2\\1\\0\end{array}\right) +
\alpha_2 \left(\begin{array}{r}-7\\3\\0\\1\end{array}\right)
\AND
\beta_1 \left(\begin{array}{r}-3\\2\\1\\0\end{array}\right) +
\beta_2 \left(\begin{array}{r}-7\\3\\0\\1\end{array}\right)
\]
are two solutions.  Then the sum has the form
\[
\gamma_1 \left(\begin{array}{r}-3\\2\\1\\0\end{array}\right) +
\gamma_2 \left(\begin{array}{r}-7\\3\\0\\1\end{array}\right)
\]
where $\gamma_j = \alpha_j + \beta_j$.


We have actually proved more than superposition.  We have shown
in this example that every solution is a superposition
of just two solutions
\[
\left(\begin{array}{r}-3\\2\\1\\0\end{array}\right) \AND
\left(\begin{array}{r}-7\\3\\0\\1\end{array}\right).
\]

\subsection*{Inhomogeneous Equations}
\index{inhomogeneous}

The linear system of $m$ equations in $n$ unknowns is written as
\[
Ax=b
\]
where $A$ is an $m\times n$ matrix, $x\in\R^n$, and $b\in\R^m$.
This system is {\em inhomogeneous\/} when the vector $b$ is nonzero.
Note that if $y,z\in\R^n$ are solutions to the inhomogeneous
equation (that is, $Ay=b$ and $Az=b$), then $y-z$ is a solution
to the homogeneous equation.  That is,
\[
A(y-z) = Ay - Az = b - b = 0.
\]
For example, let 
\[
A = \left(\begin{array}{rrr}  1 & 2 & 0 \\ -2 & 0 & 1 \end{array}\right)
\AND b = \vectwo{3}{-1}.
\]
Then 
\[
y = \left(\begin{array}{c} 1\\1\\1\end{array}\right) \AND 
z =  \left(\begin{array}{c} 3\\0\\5\end{array}\right)
\]
are both solutions to the linear system $Ax=b$.  It follows that 
\[
y-z = \left(\begin{array}{r} -2\\1\\-4\end{array}\right)
\]
is a solution to the homogeneous system $Ax=0$, which can be checked by 
direct calculation.

 
Thus we can completely solve the inhomogeneous equation by
finding one solution to the inhomogeneous equation and then
adding to that solution every solution of the homogeneous
equation. More precisely, suppose that we know all of the
solutions $w$ to the homogeneous equation $Ax=0$ and one
solution $y$ to the inhomogeneous equation $Ax=b$.  Then $y+w$
is another solution to the inhomogeneous equation and {\em
every\/} solution to the inhomogeneous equation has this form.

\subsubsection*{An Example of an Inhomogeneous Equation}

Suppose that we want to find all solutions of $Ax=b$ where
\[
A = \left(
\begin{array}{rrr}
 3 & 2 & 1  \\
 0 & 1 & -2  \\
 3 & 3 & -1
\end{array}
\right)\quad\mbox{and}\quad
b=\left(
\begin{array}{r}
 -2   \\
 4   \\
 2
\end{array}
\right).
\]
Suppose that you are told that $y=(-5,6,1)^t$ is a solution of the
inhomogeneous equation.  (This fact can be verified by a short calculation
--- just multiply $Ay$ and see that the result equals $b$.)  Next find
all solutions to the homogeneous equation $Ax=0$ by putting $A$ into reduced
echelon form.  The resulting row echelon form matrix is
\[
\left(
\begin{array}{rrr}
 1 & 0 & \frac{5}{3}  \\
 0 & 1 & -2  \\
 0 & 0 & 0
\end{array}
\right).
\]
Hence we see that the solutions of the homogeneous equation $Ax=0$ are
\[
\left(\begin{array}{r} -\frac{5}{3}s\\ 2s\\ s\end{array}\right) =
s\left(\begin{array}{r}-\frac{5}{3}\\2\\1\end{array}\right).
\]
Combining these results, we conclude that all the solutions
of $Ax=b$ are given by
\[
	\left(\begin{array}{r}-5\\6\\1\end{array}\right)+
	s\left(\begin{array}{r}-\frac{5}{3}\\2\\1\end{array}\right).
\]


\EXER

\TEXER

\begin{exercise} \label{c4.4.1}
Consider the homogeneous linear equation
\[
x+y+z = 0
\]
\begin{enumerate}
\item[(a)]  Write all solutions to this equation as a general
superposition of a pair of vectors $v_1$ and $v_2$.
\item[(b)]  Write all solutions as a general superposition of
a second pair of vectors $w_1$ and $w_2$.
\end{enumerate}
\end{exercise}

\begin{exercise} \label{c4.4.2}
Write all solutions to the homogeneous system of linear
equations
\begin{eqnarray*}
x_1+2x_2+x_4-x_5 = 0\\
x_3-2x_4+x_5 = 0
\end{eqnarray*}
as the general superposition of three vectors.
\end{exercise}

\begin{exercise} \label{c4.4.3}
\begin{itemize}
\item[(a)] Find all solutions to the homogeneous equation
$Ax=0$ where
\[
A = \left(\begin{array}{ccc} 2 & 3 & 1 \\ 1 & 1 & 4 \end{array}
\right).
\]
\item[(b)] Find a single solution to the inhomogeneous equation
\begin{equation}  \label{E:inhom}
Ax =\vectwo{6}{6}.
\end{equation}
\item[(c)] Use your answers in (a) and (b) to find all solutions
to \Ref{E:inhom}.
\end{itemize}
\end{exercise}







\section{Composition and Multiplication of Matrices} \label{S:4.6}
\index{composition} \index{matrix!multiplication}

The {\em composition\/} of two matrix mappings leads to another
matrix mapping from which the concept of multiplication of two
matrices follows.  Matrix multiplication can be introduced by
formula, but then the idea is unmotivated and one is left to
wonder why matrix multiplication is defined in such a seemingly
awkward way.

We begin with the example of $2\times 2$ matrices.  Suppose that
\[
A= \left(\begin{array}{rr} 2 & 1\\ 1 & -1\end{array}\right)
\AND
B= \left(\begin{array}{rr} 0 & 3\\ -1 & 4\end{array}\right).
\]
We have seen that the mappings
\[
x\mapsto Ax \AND x\mapsto Bx
\]
map $2$-vectors to $2$-vectors.  So we can ask what happens when
we compose these mappings.  In symbols, we compute
\[
L_A\compose L_B (x) = L_A(L_B(x))= A(Bx).
\]
In coordinates, let $x=(x_1, x_2)$ and compute
\begin{eqnarray*}
A(Bx) & = & A \left(\begin{array}{c} 3x_2 \\ -x_1+4x_2
\end{array}\right)\\
 & = & \left(\begin{array}{c} -x_1+10x_2  \\ x_1-x_2
\end{array}\right).
\end{eqnarray*}
It follows that we can rewrite $A(Bx)$ using multiplication of a
matrix times a vector as
\[
A(Bx) = \left(\begin{array}{rr} -1 & 10 \\ 1 & -1
\end{array}\right)
        \left(\begin{array}{c} x_1 \\ x_2 \end{array} \right).
\]
In particular, $L_A\compose L_B$ is again a linear mapping,
namely $L_C$, where
\[
C =\left(\begin{array}{rr} -1 & 10 \\ 1 & -1 \end{array}\right).
\]
With this computation in mind, we define the product
\[
AB =
\left(\begin{array}{rr} 2 & 1\\ 1 & -1\end{array}\right)
\left(\begin{array}{rr} 0 & 3\\ -1 & 4\end{array}\right)
= \left(\begin{array}{rr} -1 & 10 \\ 1 & -1 \end{array}\right).
\]

Using the same approach we can derive a formula for matrix
multiplication of $2\times 2$ matrices.  Suppose
\[
A= \left(\begin{array}{rr} a_{11} & a_{12}\\ a_{21} &
a_{22}\end{array}\right)
\AND
B= \left(\begin{array}{rr} b_{11} & b_{12}\\ b_{21} &
b_{22}\end{array}\right).
\]
Then
\begin{eqnarray*}
A(Bx) & = & A \left(\begin{array}{c} b_{11}x_1+b_{12}x_2 \\
b_{21}x_1+b_{22}x_2 \end{array}\right)\\
 & = & \left(\begin{array}{c} a_{11}(b_{11}x_1+b_{12}x_2)
+a_{12}(b_{21}x_1+b_{22}x_2)  \\
a_{21}(b_{11}x_1+b_{12}x_2) +
a_{22}(b_{21}x_1+b_{22}x_2) \end{array}\right) \\
 & = & \left(\begin{array}{c} (a_{11}b_{11}+a_{12}b_{21})x_1+
(a_{11}b_{12}+a_{12}b_{22})x_2  \\
(a_{21}b_{11}+a_{22}b_{21})x_1+
(a_{21}b_{12}+a_{22}b_{22})x_2 \end{array}\right) \\
& = & \left(\begin{array}{rr} a_{11}b_{11}+a_{12}b_{21} &
a_{11}b_{12}+a_{12}b_{22}\\
a_{21}b_{11}+a_{22}b_{21} & a_{21}b_{12}+a_{22}b_{22}
\end{array}\right)
        \left(\begin{array}{c} x_1 \\ x_2 \end{array} \right).
\end{eqnarray*}
Hence, for $2\times 2$ matrices, we see that composition of
matrix mappings defines matrix multiplication\index{matrix!multiplication}
as:
\begin{equation}  \label{2x2mult}
\left(\begin{array}{rr} a_{11} & a_{12}\\ a_{21} &
a_{22}\end{array}\right)
\left(\begin{array}{rr} b_{11} & b_{12}\\ b_{21} &
b_{22}\end{array}\right)
=\left(\begin{array}{rr} a_{11}b_{11}+a_{12}b_{21} &
a_{11}b_{12}+a_{12}b_{22}
\\ a_{21}b_{11}+a_{22}b_{21} & a_{21}b_{12}+a_{22}b_{22}
\end{array}\right).
\end{equation}

Formula \Ref{2x2mult} may seem a bit formidable, but it does
have structure.  Suppose $A$ and $B$ are $2\times 2$ matrices,
then the entry of
\[
C=AB
\]
in the $i^{th}$ row, $j^{th}$ column may be written as
\[
a_{i1}b_{1j} + a_{i2}b_{2j} = \sum_{k=1}^2 a_{ik}b_{kj}.
\]
We shall see that an analog of this formula is available for
matrix multiplications of all sizes.  But to derive this
formula, it is easier to develop matrix multiplication
abstractly.


\begin{lemma}  \label{complin}
Let $L_1:\R^n\to\R^m$ and $L_2:\R^p\to\R^n$ be linear mappings.
Then $L=L_1\compose L_2:\R^p\to\R^m$ is a linear mapping.
\end{lemma}

\begin{proof}  Compute
\begin{eqnarray*}
L(x+y) & = & L_1\compose L_2(x+y)\\
 & = & L_1(L_2(x)+L_2(y)) \\
 & = & L_1(L_2(x)) + L_1(L_2(y))\\
 & = & L_1\compose L_2(x) + L_1\compose L_2(y)\\
 & = & L(x) + L(y).
\end{eqnarray*}
Similarly, compute $L_1\compose L_2(cx) = cL_1\compose L_2(x)$.
\end{proof}

We apply Lemma~\ref{complin} in the following way.  Let $A$ be
an $m\times n$ matrix and let $B$ be an $n\times p$
matrix.  Then $L_A:\R^n\to\R^m$ and $L_B:\R^p\to\R^n$ are linear
mappings, and the mapping $L=L_A\compose L_B:\R^p\to\R^m$ is
defined and linear.  Theorem~\ref{lin-matrices} implies that
there is an $m\times p$ matrix $C$ such that $L=L_C$.
Abstractly, we define the {\em matrix product\/} \index{matrix!product}
$AB$ to be $C$.
\begin{quote}
{\em Note that the matrix product $AB$ is defined only when the number of
columns of $A$ is equal to the number of rows of $B$.}
\end{quote}

\subsubsection*{Calculating the Product of Two Matrices}

Next we discuss how to calculate the product of matrices; this
discussion generalizes our discussion of the product of $2\times 2$
matrices.  Lemma~\ref{columnsA} tells how to compute $C=AB$.  The $j^{th}$
column of the matrix product is just
\[
Ce_j = A(Be_j),
\]
where $B_j\equiv Be_j$ is the $j^{th}$ column of the matrix $B$.  Therefore,
\begin{equation}  \label{E:matprod}
C = (AB_1|\cdots|AB_p).
\end{equation}
Indeed, the $(i,j)^{th}$ entry of $C$ is the $i^{th}$ entry of $AB_j$,
that is, the $i^{th}$ entry of
\[
A\left(\begin{array}{c} b_{1j}\\ \vdots\\
b_{nj}\end{array}\right)
=
\left(\begin{array}{c} a_{11}b_{1j} + \cdots + a_{1n}b_{nj} \\
\vdots \\ a_{m1}b_{1j} + \cdots + a_{mn}b_{nj}
\end{array}\right).
\]
It follows that the entry $c_{ij}$ of $C$ in the $i^{th}$ row and
$j^{th}$ column is
\begin{equation} \label{multij}
c_{ij} = a_{i1}b_{1j} + a_{i2}b_{2j} + \cdots + a_{in}b_{nj} =
\sum_{k=1}^n a_{ik}b_{kj}.
\end{equation}

We can interpret \Ref{multij} in the following way.  To calculate $c_{ij}$:
multiply the entries of the $i^{th}$ row of $A$ with the corresponding
entries in the $j^{th}$ column of $B$ and add the results.  This
interpretation reinforces the
idea that for the matrix product $AB$ to be defined, the number of columns
in $A$ must equal the number of rows in $B$.

For example, we now perform the following multiplication:

\begin{eqnarray*}
& & \left(\begin{array}{rrr} 2 & 3 & 1 \\ 3 & -1 & 2 \end{array}\right)
\left(\begin{array}{rr} 1 & -2 \\ 3 & 1 \\ -1 & 4 \end{array}\right)\\
& = & \mattwoc{2\cdot 1+3\cdot 3 + 1\cdot(-1)}{2\cdot(-2)+3\cdot 1+1\cdot 4}
{3\cdot 1 + (-1)\cdot 3+2\cdot(-1)}{3\cdot(-2)+(-1)\cdot 1+2\cdot 4}\\
& = & \mattwo{10}{3}{-2}{1}.
\end{eqnarray*}

\subsubsection*{Some Special Matrix Products}

Let $A$ be an $m\times n$ matrix.  Then
\begin{eqnarray*}
OA & = & O \\
AO & = & O \\
AI_n & = & A \\
I_mA & = & A
\end{eqnarray*}
The first two equalities are easily checked using \Ref{multij}.
It is not significantly more difficult to verify the last two
equalities using \Ref{multij}, but we shall verify these
equalities using the language of linear mappings, as follows:
\[
L_{AI_n}(x)=L_A\compose L_{I_n}(x)=L_A(x),
\]
since $L_{I_n}(x)=x$ is the identity map.  Therefore $AI_n=A$.
A similar proof verifies that $I_mA=A$.  Although the
verification of these equalities using the notions of linear
mappings may appear to be a case of overkill, the next section
contains results where these notions truly simplify the discussion.

\EXER

\TEXER

\noindent In Exercises~\ref{c4.6.-1a} -- \ref{c4.6.-1d} determine whether or 
not the matrix products $AB$ or $BA$ can be computed for each given pair of 
matrices $A$ and $B$.  If the product is possible, perform the computation.
\begin{exercise}  \label{c4.6.-1a}
$A=\mattwo{1}{0}{-2}{1}$ and $B=\mattwo{-2}{0}{3}{-1}$.
\end{exercise}
\begin{exercise}  \label{c4.6.-1b}
$A=\left(\begin{array}{rrr} 0 & -2 & 1\\ 4 & 10 & 0 \end{array}\right)$
and $B=\left(\begin{array}{rr} 0 & 2 \\ 3 & -1 \end{array}\right)$.
\end{exercise}
\begin{exercise}  \label{c4.6.-1c}
$A=\left(\begin{array}{rrrr} 8 & 0 & 2 & 3\\ -3 & 0 & -10 &
3\end{array}\right)$
and $B=\left(\begin{array}{rrr} 0 & 2 & 5\\ -1 & 3 & -1 \\ 0 & 1 &
-5\end{array}\right)$.
\end{exercise}
\begin{exercise}  \label{c4.6.-1d}
$A=\left(\begin{array}{rr} 8 & -1 \\ -3 & 12 \\ 5 & -4\end{array}\right)$
and $B=\left(\begin{array}{rrrr} 2 & 8 & 0 & -3\\ 1 & 4 & 0 & 1\\
-5 & 6 & 7 & -20\end{array}\right)$
\end{exercise}

\noindent In Exercises~\ref{c4.6.0a} -- \ref{c4.6.0d} compute
the given matrix product.
\begin{exercise}  \label{c4.6.0a}
$\mattwo{2}{3}{0}{1}\mattwo{-1}{1}{-3}{2}$.
\end{exercise}
\begin{exercise}  \label{c4.6.0b}
$\left(\begin{array}{rrr} 1 & 2 & 3\\ -2 & 3 & -1 \end{array}\right)
\left(\begin{array}{rr} 2 & 3\\ -2 & 5 \\1 & -1 \end{array}\right)$.
\end{exercise}
\begin{exercise}  \label{c4.6.0c}
$\left(\begin{array}{rr} 2 & 3\\ -2 & 5 \\1 & -1 \end{array}\right)
\left(\begin{array}{rrr} 1 & 2 & 3\\ -2 & 3 & -1 \end{array}\right)$.
\end{exercise}
\begin{exercise}  \label{c4.6.0d}
$\left(\begin{array}{rrr} 2 & -1 &3\\ 1 & 0 & 5\\1 & 5 & -1\end{array}\right)
\left(\begin{array}{rrr} 1 & 7 \\ -2 & -1 \\ -5 & 3\end{array}\right)$.
\end{exercise}


\begin{exercise} \label{c4.6.1}
Determine all the $2\times 2$ matrices $B$ such that $AB=BA$
where $A$ is the matrix
\[
A=\left(\begin{array}{rr} 2 & 0 \\ 0 & -1 \end{array}\right).
\]
\end{exercise}

\begin{exercise} \label{c4.6.2}
Let
\[
A=\mattwo{2}{5}{1}{4} \AND B=\mattwo{a}{3}{b}{2}.
\]
For which values of $a$ and $b$ does $AB=BA$?
\end{exercise}

\begin{exercise} \label{c4.6.3}
Let
\[
A = \left(\begin{array}{rrr} 1 & 0 & -3\\ -2 & 1 & 1 \\ 0 & 1 & -5 \end{array}
\right).
\]
Let $A^t$ is the transpose of the matrix $A$, as defined in
Section~\ref{S:1.3}. Compute $AA^t$.
\end{exercise}

\CEXER

\noindent In Exercises~\ref{c4.7.1a} -- \ref{c4.7.1c} decide for the
given pair of matrices $A$ and $B$ whether or not the products $AB$ or
$BA$ are defined and compute the products when possible.
\begin{exercise} \label{c4.7.1a}
\begin{equation*}
A=\left(\begin{array}{rrr} 2  &  2  & -2 \\ -4  &  4  &  0
\end{array}\right) \AND
B=\left(\begin{array}{rrr} 3 &  -2  &  0\\ 0 &  -1 &   4\\ -2 &  -3
&   5
\end{array}\right)
\end{equation*}
\end{exercise}
\begin{exercise} \label{c4.7.1b}
\begin{equation*}
A=\left(\begin{array}{rrrrr}  -4  &  1  &  0  &  5  & -1\\
5  & -1  & -2  & -4  & -2\\ 1  &  5 &  -4   & 1  &  5
\end{array}\right) \AND
B=\left(\begin{array}{rrrrrr} 1  &  3  & -4  &  3  & -2  &  1\\
0  &  3  &  2  &  3  & -1  &  4\\ 5  &  4  &  4  &  5  & -1  &  0\\
-4  & -3  &  2  &  4  &  1  &  4
\end{array}\right)
\end{equation*}
\end{exercise}
\begin{exercise} \label{c4.7.1c}
\begin{equation*}
A=\left(\begin{array}{rrrr}  -2  & -2  &  4  &  5\\
 0  & -3  & -4  &  3\\ 1  & -3  &  1  &  1\\ 0  &  1  &  0  &  4
\end{array}\right) \AND
B=\left(\begin{array}{rrrr} 2  &  3  & -4  &  5\\
4  & -3  &  0  & -2\\ -3 &  -4  & -4 &  -3\\ -2  & -2  &  3  & -1
\end{array}\right)
\end{equation*}
\end{exercise}



\section{Properties of Matrix Multiplication} \label{S:4.7}
\index{matrix!multiplication}

In this section we discuss the facts that matrix multiplication is
associative (but not commutative) and that certain distributive
properties hold.  We also discuss how matrix multiplication is
performed in \Matlab.

\subsubsection*{Matrix Multiplication is Associative}
\index{associative}

\begin{thm} \label{assoc}
Matrix multiplication is associative.  That is, let $A$ be an
$m\times n$ matrix, let $B$ be a $n\times p$ matrix, and let $C$
be a $p\times q$ matrix.  Then
\[
(AB)C = A(BC).
\]
\end{thm}

\begin{proof} Begin by observing that composition of mappings is always
associative.  In symbols, let $f:\R^n\to\R^m$, $g:\R^p\to\R^n$,
and $h:\R^q\to\R^p$.  Then
\begin{eqnarray*}
f\compose (g\compose h)(x) & = & f[(g\compose h)(x)] \\
  & = & f[g(h(x))] \\
  & = & (f\compose g)(h(x)) \\
  & = & [(f\compose g)\compose h](x).
\end{eqnarray*}
It follows that
\[
f\compose (g\compose h) = (f\compose g)\compose h.
\]

We can apply this result to linear mappings.  Thus
\[
L_A\compose (L_B\compose L_C) = (L_A\compose L_B)\compose L_C.
\]
Since
\[
L_{A(BC)} = L_A\compose L_{BC} = L_A\compose (L_B\compose L_C)
\]
and
\[
L_{(AB)C} = L_{AB}\compose L_C = (L_A\compose L_B)\compose L_C,
\]
it follows that
\[
L_{A(BC)} = L_{(AB)C},
\]
and
\[
A(BC) = (AB)C.
\]
\end{proof}

It is worth convincing yourself that Theorem~\ref{assoc} has
content by verifying by hand that matrix multiplication of
$2\times 2$ matrices is associative.

\subsubsection*{Matrix Multiplication is Not Commutative}
\index{commutative}

Although matrix multiplication is associative, it is {\em not\/}
commutative. This statement is trivially true when the matrix
$AB$ is defined while that matrix $BA$ is not.  Suppose, for
example, that $A$ is a $2\times 3$ matrix and that $B$ is a
$3\times 4$ matrix.  Then $AB$ is a $2\times 4$ matrix, while
the multiplication $BA$ makes no sense whatsoever.

More importantly, suppose that $A$ and $B$ are both $n\times n$
square matrices.  Then $AB=BA$ is generally not valid.  For
example, let
\[
A=\left(\begin{array}{cc} 1 & 0 \\ 0 & 0 \end{array}\right) \AND
B =\left(\begin{array}{cc} 0 & 1 \\ 0 & 0 \end{array}\right).
\]
Then
\[
AB =\left(\begin{array}{cc} 0 & 1 \\ 0 & 0 \end{array}\right)
\AND BA=\left(\begin{array}{cc} 0 & 0 \\ 0 & 0 \end{array}\right).
\]
So $AB\neq BA$.  In certain cases it does happen that $AB=BA$.
For example, when $B=I_n$,
\[
AI_n = A = I_nA.
\]
But these cases are rare.

\subsubsection*{Additional Properties of Matrix Multiplication}
\index{distributive}

Recall that if $A=(a_{ij})$ and $B=(b_{ij})$ are both $m\times n$ 
matrices, then $A+B$ is the $m\times n$ matrix $(a_{ij}+b_{ij})$. 
We now enumerate several properties of matrix multiplication.

\begin{itemize}

\item	Let $A$ and $B$ be $m\times n$ matrices and let $C$ be an $n\times p$
matrix.  Then
\[
(A+B)C = AC + BC.
\]
Similarly, if $D$ is a $q\times m$ matrix, then
\[
D(A+B) = DA + DB.
\]
So matrix multiplication distributes across matrix addition.

\item	If $\alpha$ and $\beta$ are scalars, then
\[
(\alpha+\beta)A = \alpha A + \beta A.
\]
So addition distributes with scalar multiplication.

\item	Scalar multiplication and matrix multiplication satisfy:
\[
(\alpha A)C = \alpha (AC).
\]
\end{itemize}

\subsubsection*{Matrix Multiplication and Transposes}
\index{matrix!transpose}

Let $A$ be an $m\times n$ matrix and let $B$ be an $n\times p$
matrix, so that the matrix product $AB$ is defined and $AB$ is an
$m\times p$ matrix.  Note that $A^t$
is an $n\times m$ matrix and that $B^t$ is a $p\times n$ matrix, so
that in general the product $A^tB^t$ is {\em not\/} defined.  However,
the product $B^tA^t$ is defined and is an $p\times m$ matrix, as is
the matrix $(AB)^t$.  We claim that
\begin{equation}  \label{e:transposeprod}
(AB)^t = B^tA^t.
\end{equation}
We verify this claim by direct computation.  The $(i,k)^{th}$ entry
in $(AB)^t$ is the $(k,i)^{th}$ entry in $AB$.   That entry is:
\[
\sum_{j=1}^{n} a_{kj}b_{ji}.
\]
The $(i,k)^{th}$ entry in $B^tA^t$ is:
\[
\sum_{j=1}^n b^t_{ij}a^t_{jk},
\]
where $a^t_{jk}$ is the $(j,k)^{th}$ entry in $A^t$ and $b^t_{ij}$
is the $(i,j)^{th}$ entry in $B^t$.  It follows from the definition
of transpose that the $(i,k)^{th}$ entry in $B^tA^t$ is:
\[
\sum_{j=1}^n b_{ji}a_{kj} = \sum_{j=1}^n a_{kj}b_{ji},
\]
which verifies the claim.


\subsubsection*{Matrix Multiplication in \Matlab}
\index{matrix!multiplication!in \protect\Matlab}

Let us now explain how matrix multiplication works in \Matlabp.
We load the matrices
\begin{equation*}  \label{examp_AB}
A=\left(\begin{array}{rrr} -5  &  2  &  0\\
               -1  &  1  & -4\\
               -4  &  4  &  2\\
               -1  &  3  & -1 \end{array}\right) \AND
B =\left(\begin{array}{rrrrr}      2  & -2  & -2  &  5  &  5\\
                    4  & -5  &  1  & -1  &  2\\
                    3  &  2  &  3  & -3  &  3
 \end{array}\right)
\end{equation*}%
by typing
\begin{verbatim}
e3_6_2
\end{verbatim}
Now the command {\tt C = A*B} \index{\computer!*} asks \Matlab to compute
the matrix $C$ as the product of $A$ and $B$.  We obtain
\begin{verbatim}
C =
    -2     0    12   -27   -21
   -10   -11    -9     6   -15
    14    -8    18   -30    -6
     7   -15     2    -5    -2
\end{verbatim}
Let us confirm this result by another computation.  As we have
seen above the $4^{th}$ column of $C$ should be given by the
product of $A$ with the $4^{th}$ column of $B$.  Indeed, if we
perform this computation and type
\begin{verbatim}
A*B(:,4)
\end{verbatim}
the result is
\begin{verbatim}
ans =
   -27
     6
   -30
    -5
\end{verbatim}
which is precisely the $4^{th}$ column of $C$.

\Matlab also recognizes when a matrix multiplication of two
matrices is not defined.  For example, the product of
the $3\times 5$ matrix $B$ with the $4\times 3$ matrix $A$
is not defined, and if we type {\tt B*A} then we obtain the
error message
\begin{verbatim}
??? Error using ==> *
Inner matrix dimensions must agree.
\end{verbatim}
We remark that the size of a matrix $A$ can be seen using
the \Matlab command {\tt size}\index{\computer!size}.
For example, the command {\tt size(A)} leads to
\begin{verbatim}
ans =
     4     3
\end{verbatim}
reflecting the fact that $A$ is a matrix with four
rows and three columns.



\EXER


\TEXER


\begin{exercise} \label{c4.7.2.2}
Let $A$ be an $m\times n$ matrix.  Show that the matrices $A A^t$ and
$A^t A$ are symmetric.
\end{exercise}


\begin{exercise} \label{c4.7.3}
Let
\[
A=\mattwo{1}{2}{-1}{-1} \AND B =\mattwo{2}{3}{1}{4}.
\]
Compute $AB$ and $B^tA^t$.  Verify that $(AB)^t=B^tA^t$ for these
matrices $A$ and $B$.
\end{exercise}

\begin{exercise} \label{c4.7.4}
Let
\[
A = \left(\begin{array}{ccc} 0 & 1 & 0\\ 0 & 0 & 1 \\ 0 & 0 & 0 \end{array}
\right).
\]
Compute $B=I+A+\frac{1}{2}A^2$ and $C=I+tA+\frac{1}{2}(tA)^2$.
\end{exercise}

\begin{exercise} \label{c4.7.5}
Let
\[
I = \mattwo{1}{0}{0}{1} \AND J =\mattwo{0}{-1}{1}{0}.
\]
\begin{itemize}
\item[(a)] Show that $J^2=-I$.
\item[(b)] Evaluate $(aI+bJ)(cI+dJ)$ in terms of $I$ and $J$.
\end{itemize}
\end{exercise}

\begin{exercise} \label{c4.7.8}
Recall that a square matrix $C$ is {\em upper triangular\/} if $c_{ij}=0$
when $i>j$.  Show that the matrix product of two upper triangular $n\times n$
matrices is also upper triangular.
\end{exercise}


\CEXER

\noindent In Exercises~\ref{c4.7.0a} -- \ref{c4.7.0c} use \Matlab to
verify that $(A+B)C = AC+BC$ for the given matrices.
\begin{exercise}  \label{c4.7.0a}
$A=\mattwo{0}{2}{2}{1}$, $B=\mattwo{-2}{1}{3}{0}$ and $C=\mattwo{2}{-1}{1}{5}$
\end{exercise}
\begin{exercise}  \label{c4.7.0b}
$A=\mattwo{12}{-2}{3}{1}$,
$B=\mattwo{8}{-20}{3}{10}$ and
$C=\left(\begin{array}{rrr} 10 & 2 & 4\\ 2 & 13 & -4 \end{array}\right)$
\end{exercise}
\begin{exercise}  \label{c4.7.0c}
$A=\left(\begin{array}{rr} 6 & 1 \\ 3 & 20 \\ -5 & 3\end{array}\right)$,
$B=\left(\begin{array}{rr} 2 & -10 \\ 5 & 0 \\ 3 & 1\end{array}\right)$ and
$C=\mattwo{-2}{10}{12}{10}$
\end{exercise}

\begin{exercise} \label{c4.7.2}
Use the {\tt rand(3,3)} command in \Matlab to choose five pairs of
$3\times 3$ matrices $A$ and $B$ at random.  Compute $AB$ and $BA$
using \Matlab to see that in general these matrix products are unequal.
\end{exercise}

\begin{exercise} \label{c4.7.2.1}
Experimentally, find two symmetric $2\times 2$ matrices $A$ and $B$ for
which the matrix product $AB$ is {\em not\/} symmetric.
\end{exercise}



\section{Solving Linear Systems and Inverses} \label{S:SLS}

When we solve the simple equation
\[
ax=b,
\]
we do so by dividing by $a$ to obtain
\[
x=\frac{1}{a}b.
\]
This division works as long as $a\neq 0$.

Writing systems of linear equations as
\[
Ax=b
\]
suggests that solutions should have the form
\[
x=\frac{1}{A} b
\]
and the \Matlab command for solving linear systems
\begin{verbatim}
x=A\b
\end{verbatim}
suggests that there is some merit to this analogy.

The following is a better analogy.  Multiplication by $a$ has the
inverse operation: division by $a$; multiplying a number $x$ by
$a$ and then multiplying the result by $a\inv=1/a$ leaves the number
$x$ unchanged (as long as $a\neq 0$).  In this sense we should
write the solution to $ax=b$ as
\[
x=a\inv b.
\]
For systems of equations $Ax=b$ we wish to write solutions as
\[
x=A\inv b.
\]
In this section we consider the questions: What does $A\inv$
mean and when does $A\inv$ exist? (Even in one dimension, we
have seen that the inverse does not always exist, since
$0\inv=\frac{1}{0}$ is undefined.)

\subsection*{Invertibility}

We begin by giving a precise definition of invertibility for square matrices.
\begin{Def} \label{inverse} \index{inverse} \index{invertible}
The $n\times n$ matrix $A$ is {\em invertible\/} if there is an $n\times n$
matrix $B$ such that
\[
AB=I_n \AND BA=I_n.
\]
The matrix $B$ is called an {\em inverse\/} of $A$.  If $A$ is not invertible,
then $A$ is {\em noninvertible\/} or {\em singular\/}. \index{noninvertible}
\index{singular}
\end{Def}

Geometrically, we can see that some matrices are invertible.  For example, 
the matrix
\[
R_{90} = \mattwo{0}{-1}{1}{0}
\]
rotates the plane counterclockwise through $90^\circ$ and is
invertible.  The inverse matrix of $R_{90}$ is the matrix that rotates the
plane clockwise through $90^\circ$.  That matrix is:
\[
R_{-90} = \mattwo{0}{1}{-1}{0}.
\]
This statement can be checked algebraically by verifying that 
$R_{90}R_{-90}=I_2$ and that $R_{-90}R_{90} = I_2$.

Similarly,  
\[
B = \mattwo{5}{3}{2}{1}
\]
is an inverse of 
\[
A = \mattwo{-1}{3}{2}{-5},
\]
as matrix multiplication shows that $AB=I_2$ and $BA=I_2$. In fact, there is 
an elementary formula for finding inverses of $2\times 2$ matrices (when they 
exist); see \Ref{e:formAinv} in Section~\ref{S:det2x2}.

On the other hand, not all matrices are invertible.  For example, the zero 
matrix is noninvertible, since $0B=0$ for any matrix $B$.

\begin{lemma} \label{B=C}
If an $n\times n$ matrix $A$ is invertible, then its inverse is unique
and is denoted by $A\inv$.
\end{lemma}

\begin{proof}
Let $B$ and $C$ be $n\times n$ matrices that are inverses of $A$.  Then
\[
BA=I_n \AND AC=I_n.
\]
We use the associativity of matrix multiplication to
prove that $B=C$.  Compute
\[
B = BI_n = B(AC) = (BA)C = I_nC = C.
\]
\end{proof}

We now show how to compute inverses for products of invertible
matrices.

\begin{prop} \label{P:invprod} \index{matrix!multiplication}
\index{inverse}
Let $A$ and $B$ be two invertible $n\times n$ matrices.  Then
$AB$ is also invertible and
\[
(AB)\inv = B\inv A\inv.
\]
\end{prop}

\begin{proof}  Use associativity of matrix multiplication to compute
\[
(AB)(B\inv A\inv) = A(BB\inv)A\inv = AI_nA\inv = AA\inv=I_n.
\]
Similarly,
\[
(B\inv A\inv)(AB) = B\inv(A\inv A)B = B\inv B = I_n.
\]
Therefore $AB$ is invertible with the desired inverse. \end{proof}

\begin{prop} \label{L:transposeinv} \index{matrix!transpose}
Suppose that $A$ is an invertible $n\times n$ matrix.  Then
$A^t$ is invertible and
\[
(A^t)\inv = (A\inv)^t.
\]
\end{prop}

\begin{proof}  We must show that $(A\inv)^t$ is the inverse of $A^t$.  Identity
\Ref{e:transposeprod} implies that
\[
(A\inv)^tA^t = (AA\inv)^t = (I_n)^t = I_n,
\]
and
\[
A^t(A\inv)^t = (A\inv A)^t = (I_n)^t = I_n.
\]
Therefore, $(A\inv)^t$ is the inverse of $A^t$, as claimed.  \end{proof}

\subsection*{Invertibility and Unique Solutions}

Next we discuss the implications of invertibility for the
solution of the inhomogeneous linear system:  \index{inhomogeneous}
\begin{equation}  \label{squarematrix}
Ax=b,
\end{equation}
where $A$ is an $n\times n$ matrix and $b\in\R^n$.

\begin{prop} \label{P:inv=>unique}\index{uniqueness of solutions}
Let $A$ be an invertible $n\times n$ matrix and let $b$ be in $\R^n$.
Then the system of linear equations \Ref{squarematrix} has a unique solution.
\end{prop}

\begin{proof}  We can solve the linear system \Ref{squarematrix} by setting
\begin{equation}  \label{soln}
x=A\inv b.
\end{equation}
This solution is easily verified by calculating
\[
Ax=A(A\inv b) = (AA\inv)b = I_nb = b.
\]
Next, suppose that $x$ is a solution to \Ref{squarematrix}.  Then
\[
x = I_n x = (A\inv A)x = A\inv(Ax) = A\inv b.
\]
So $A\inv b$ is the only possible solution.  \end{proof}

\begin{cor} \label{C:inv=>In}
An invertible matrix is row equivalent to $I_n$.
\end{cor}

\begin{proof}  Let $A$ be an invertible $n\times n$ matrix.
Proposition~\ref{P:inv=>unique} states that the system of linear equations
$Ax=b$ has a unique solution.  Chapter~\ref{lineq}, Corollary~\ref{consistent}
states that $A$ is row equivalent to $I_n$.  \end{proof}

The converse of Corollary~\ref{C:inv=>In} is also valid.

\begin{prop}  \label{P:row=>inv}
An $n\times n$ matrix $A$ that is row equivalent to $I_n$ is invertible.
\end{prop}

\begin{proof}  Form the $n\times 2n$ matrix $M=(A|I_n)$.  Since $A$ is row equivalent
to $I_n$, there is a sequence of elementary row operations so that $M$ is
row equivalent to $(I_n|B)$.  Eliminating all columns from the right half
of $M$ except the $j^{th}$ column yields the matrix $(A|e_j)$.  The same
sequence of elementary row operations states that the matrix $(A|e_j)$ is
row equivalent to $(I_n|B_j)$ where $B_j$ is the $j^{th}$ column of $B$.  It
follows that $B_j$ is the solution to the system of linear equations
$Ax=e_j$ and that the matrix product
\[
AB = (AB_1|\cdots|AB_n) = (e_1|\cdots|e_n) = I_n.
\]
So $AB=I_n$.

We claim that $BA=I_n$ and hence that $A$ is invertible.  To verify this claim
form the $n\times 2n$ matrix $N=(I_n|A)$.  Using the same sequence of
elementary row operations again shows that $N$ is row equivalent to $(B|I_n)$.
By construction
the matrix $B$ is row equivalent to $I_n$.  Therefore, there is a unique
solution to the system of linear equations $Bx=e_j$.  Now eliminating all
columns except the $j^{th}$ from the right hand side of the matrix $(B|I_n)$
shows that the solution to the system of linear equations $Bx=e_j$ is just
$A_j$, where $A_j$ is the $j^{th}$ column of $A$.  It follows that
\[
BA = (BA_1|\cdots|BA_n) = (e_1|\cdots|e_n) = I_n.
\]
Hence $BA=I_n$.  \end{proof}

\begin{thm} \label{invertequiv}
Let $A$ be an $n\times n$ matrix.  Then the following are
equivalent:
\begin{itemize}
\item[(a)]  $A$ is invertible. \index{invertible}
\item[(b)] The equation $Ax=b$ has a unique solution for each
$b\in\R^n$.
\item[(c)]  The only solution to $Ax=0$ is $x=0$.
\item[(d)]  $A$ is row equivalent to $I_n$.
\end{itemize}
\end{thm}

\begin{proof} $(a) \Rightarrow (b)$ This implication
is just Proposition~\ref{P:inv=>unique}.

$(b) \Rightarrow (c)$ This implication is straightforward --- just
take $b=0$ in \Ref{squarematrix}.

$(c) \Rightarrow (d)$  This implication is just a restatement of
Chapter~\ref{lineq}, Corollary~\ref{consistent}.

$(d) \Rightarrow (a)$.  This implication is just
Proposition~\ref{P:row=>inv}. \end{proof}


\subsection*{A Method for Computing Inverse Matrices}

The proof of Proposition~\ref{P:row=>inv} gives a constructive method
for finding the inverse of any invertible square matrix.

\begin{thm}  \label{T:AIn}\index{inverse!computation}
Let $A$ be an $n\times n$ matrix that is row equivalent to
$I_n$ and let $M$ be the $n\times 2n$ augmented matrix
\begin{equation}  \label{e:M}
M = (A | I_n).
\end{equation}
Then the matrix $M$ is row equivalent to $(I_n | A\inv)$.
\end{thm}

\subsubsection*{An Example}

Compute the inverse of the matrix
\[
A = \left(\begin{array}{rrr} 1 & 2 & 0 \\ 0 & 1 & 3 \\ 0 & 0 & 1
\end{array}\right).
\]
Begin by forming the $3\times 6$ matrix
\[
M = \left(\begin{array}{rrr|rrr} 1 & 2 & 0 & 1 & 0 & 0 \\
0 & 1 & 3 & 0 & 1 & 0 \\ 0 & 0 & 1 & 0 & 0 & 1
\end{array}\right).
\]
To put $M$ in row echelon form by row reduction, first subtract
3 times the $3^{rd}$ row from the $2^{nd}$ row, obtaining
\[
\left(\begin{array}{rrr|rrr} 1 & 2 & 0 & 1 & 0 & 0 \\
0 & 1 & 0 & 0 & 1 & -3 \\ 0 & 0 & 1 & 0 & 0 & 1
\end{array}\right).
\]
Second, subtract 2 times the $2^{nd}$ row from the $1^{st}$ row, obtaining
\[
\left(\begin{array}{rrr|rrr} 1 & 0 & 0 & 1 & -2 & 6 \\
0 & 1 & 0 & 0 & 1 & -3 \\ 0 & 0 & 1 & 0 & 0 & 1
\end{array}\right).
\]
Theorem~\ref{T:AIn} implies that
\[
A\inv = \left(\begin{array}{rrr} 1 & -2 & 6 \\ 0 & 1 & -3 \\ 0 & 0 & 1
\end{array}\right),
\]
which can be verified by matrix multiplication.

\subsubsection*{Computing the Inverse Using \Matlab}

There are two ways that we can compute inverses using \Matlab.
Either we can perform the row reduction of \Ref{e:M} directly
or we can use the \Matlab the command {\tt inv}.  We illustrate
both of these methods. First type {\tt e3\_7\_4} to recall the matrix
\begin{equation*}
A = \left(\begin{array}{rrr} 1 & 2 & 4 \\ 3 & 1 & 1 \\ 2 & 0 & -1
\end{array}\right).
\end{equation*}

To perform the row reduction of \Ref{e:M} we need to form the matrix
$M$. The \Matlab command for generating an $n\times n$ identity
matrix is {\tt eye(n)}.  Therefore, typing
\begin{verbatim}
M = [A eye(3)]
\end{verbatim} \index{\computer!eye}
in \Matlab yields the result
\begin{verbatim}
M =
     1     2     4     1     0     0
     3     1     1     0     1     0
     2     0    -1     0     0     1
\end{verbatim}
Now row reduce $M$ to reduced echelon form as follows.  Type
\begin{verbatim}
M(3,:) = M(3,:) - 2*M(1,:)
M(2,:) = M(2,:) - 3*M(1,:)
\end{verbatim}
obtaining
\begin{verbatim}
M =
     1     2     4     1     0     0
     0    -5   -11    -3     1     0
     0    -4    -9    -2     0     1
\end{verbatim}
Next type
\begin{verbatim}
M(2,:) = M(2,:)/M(2,2)
M(3,:) = M(3,:) + 4*M(2,:)
M(1,:) = M(1,:) - 2*M(2,:)
\end{verbatim}
to obtain
\begin{verbatim}
M =
    1.0000         0   -0.4000   -0.2000    0.4000         0
         0    1.0000    2.2000    0.6000   -0.2000         0
         0         0   -0.2000    0.4000   -0.8000    1.0000
\end{verbatim}
Finally, type
\begin{verbatim}
M(3,:) = M(3,:)/M(3,3)
M(2,:) = M(2,:) - M(2,3)*M(3,:)
M(1,:) = M(1,:) - M(1,3)*M(3,:)
\end{verbatim}
to obtain
\begin{verbatim}
M =
    1.0000         0         0   -1.0000    2.0000   -2.0000
         0    1.0000         0    5.0000   -9.0000   11.0000
         0         0    1.0000   -2.0000    4.0000   -5.0000
\end{verbatim}
Thus $C=A\inv$ is obtained by extracting the last three columns
of $M$ by typing
\begin{verbatim}
C = M(:,[4 5 6])
\end{verbatim}
which yields
\begin{verbatim}
C =
   -1.0000    2.0000   -2.0000
    5.0000   -9.0000   11.0000
   -2.0000    4.0000   -5.0000
\end{verbatim}
You may check that $C$ is the inverse of $A$ by typing {\tt A*C}
and {\tt C*A}.

In fact, this entire scheme for computing the inverse of a
matrix has been preprogrammed into \Matlab.  Just type
\begin{verbatim}
inv(A)
\end{verbatim} \index{\computer!inv}
to obtain
\begin{verbatim}
ans =
   -1.0000    2.0000   -2.0000
    5.0000   -9.0000   11.0000
   -2.0000    4.0000   -5.0000
\end{verbatim}

We illustrate again this simple method for computing the inverse
of a matrix $A$.  For example, reload the matrix in \Ref{eq:5matrix}
by typing \verb+ e3_1_4+ and obtaining:
\begin{verbatim}
A =
     5    -4     3    -6     2
     2    -4    -2    -1     1
     1     2     1    -5     3
    -2    -1    -2     1    -1
     1    -6     1     1     4
\end{verbatim}
The command {\tt B = inv(A)} stores the inverse of the matrix $A$
in the matrix $B$, and we obtain the result
\begin{verbatim}
B =
   -0.0712    0.2856   -0.0862   -0.4813   -0.0915
   -0.1169    0.0585    0.0690   -0.2324   -0.0660
    0.1462   -0.3231   -0.0862    0.0405    0.0825
   -0.1289    0.0645   -0.1034   -0.2819    0.0555
   -0.1619    0.0810    0.1724   -0.1679    0.1394
\end{verbatim}
This computation also illustrates the fact that even when the matrix
$A$ has integer entries, the inverse of $A$ usually has noninteger
entries.

Let $b=(2,-8,18,-6,-1)$.  Then we may use the inverse $B=A\inv$
to compute the solution of $Ax=b$.  Indeed if we type
\begin{verbatim}
b = [2;-8;18;-6;-1];
x = B*b
\end{verbatim}
then we obtain
\begin{verbatim}
x =
   -1.0000
    2.0000
    1.0000
   -1.0000
    3.0000
\end{verbatim}
as desired (see \Ref{eq:5rhs}).  With this computation we have confirmed
the analytical results of the previous subsections.





\EXER

\TEXER

\begin{exercise} \label{c4.8.1}
Verify by matrix multiplication that the following matrices are inverses
of each other:
\[
\left( \begin{array}{rrr}
1  &  0  &  2\\
0  & -1  &  2\\
1  &  0  &  1
\end{array} \right)\AND
\left( \begin{array}{rrr}
-1 &   0 &   2\\
 2 &  -1 &  -2\\
 1 &   0 &  -1
\end{array} \right).
\]
\end{exercise}

\begin{exercise} \label{c4.8.2}
Let $\alpha \not=0$ be a real number and let $A$ be an invertible
matrix.  Show that the inverse of the matrix $\alpha A$ is given by
$\frac{1}{\alpha}A^{-1}$.
\end{exercise}

\begin{exercise} \label{c4.8.3}
Let $A=\mattwo{a}{0}{0}{b}$ be a $2\times 2$ diagonal matrix.
For which values of $a$ and $b$ is $A$ invertible?
\end{exercise}

\begin{exercise} \label{c4.8.4}
Let $A,B,C$ be general $n\times n$ matrices.  Simplify the expression
$A\inv(BA\inv)\inv(CB\inv)\inv$.
\end{exercise}

\noindent In Exercises~\ref{c4.9.3a} -- \ref{c4.9.3b} use row reduction
to find the inverse of the given matrix.
\begin{exercise} \label{c4.9.3a}
$\left(\begin{array}{rrr} 1 & 4 & 5\\ 0 & 1 & -1\\ -2 & 0 & -8
\end{array}\right)$.
\end{exercise}
\begin{exercise} \label{c4.9.3b}
$\left(\begin{array}{rrr} 1 & -1 & -1\\ 0 & 2 & 0\\ 2 & 0 & -1
\end{array}\right)$.
\end{exercise}

\begin{exercise} \label{c4.8.5}
Let $A$ be an $n\times n$ matrix that satisfies
\[
A^3 + a_2A^2 + a_1A + I_n = 0,
\]
where $A^2=AA$ and $A^3=AA^2$.  Show that $A$ is invertible. \\
{\bf Hint:}  Let $B = -(A^2+a_2A+a_1I_n)$ and verify that $AB=BA=I_n$.
\end{exercise}

\begin{exercise} \label{c4.8.6}
Let $A$ be an $n\times n$ matrix that satisfies
\[
A^m + a_{m-1}A^{m-1} + \cdots + a_1A + I_n = 0.
\]
Show that $A$ is invertible.
\end{exercise}

\begin{exercise} \label{c4.9.6}
For which values of $a,b,c$ is the matrix
\[
A =\left(\begin{array}{rrr} 1 & a & b\\ 0 & 1 & c\\ 0 & 0 & 1
\end{array}\right)
\]
invertible?  Find $A\inv$ when it exists.
\end{exercise}


\CEXER

\noindent In Exercises~\ref{c4.9.7a} -- \ref{c4.9.7b} use row reduction
to find the inverse of the given matrix and confirm your results using the 
command {\tt inv}.
\begin{exercise} \label{c4.9.7a}
\begin{equation*}
A=\left(\begin{array}{ccc} 2 & 1 & 3\\ 1 & 2 & 3 \\ 5& 1 & 0
\end{array}\right).
\end{equation*}
\end{exercise}
\begin{exercise} \label{c4.9.7b}
\begin{equation*}
B= \left(\begin{array}{cccr} 0 & 5 & 1 & 3\\ 1 & 5 & 3 & -1\\ 2 & 1 &
0 & -4 \\ 1 & 7 & 2 & 3 \end{array}\right).
\end{equation*}
\end{exercise}

\begin{exercise} \label{c4.9.8}
Try to compute the inverse of the matrix
\begin{equation*}
C = \left(\begin{array}{rrr} 1 & 0 & 3\\ -1 & 2 & -2 \\ 0& 2 & 1
\end{array}\right)
\end{equation*}
in \Matlab using the command {\tt inv}.  What happens --- can you
explain the outcome?

Now compute the inverse of the matrix
\[
\left(\begin{array}{rrr} 1 & \epsilon & 3\\ -1 & 2 & -2 \\ 0& 2 & 1
\end{array}\right)
\]
for some nonzero numbers $\epsilon$ of your choice.  What can be observed
in the inverse if $\epsilon$ is very small?  What happens when $\epsilon$
tends to zero?
\end{exercise}


\section{Determinants of $2\times 2$ Matrices}
\label{S:det2x2}

There is a simple way for determining whether a $2\times 2$ matrix $A$ is
invertible and there is a simple formula for finding $A\inv$.
\index{inverse}  First, we present the formula.  Let
\[
A=\mattwo{a}{b}{c}{d}.
\]
and suppose that $ad-bc\neq0$.  Then
\begin{equation}  \label{e:formAinv}
A\inv = \frac{1}{ad-bc} \mattwo{d}{-b}{-c}{a}.
\end{equation}
This is most easily verified by directly applying the formula for
matrix multiplication.  So $A$ is invertible when $ad-bc\neq 0$. We
shall prove below that $ad-bc$ must be nonzero when $A$ is invertible.

From this discussion it is clear that the number $ad-bc$ must be an
important quantity for $2\times 2$ matrices.  So we define:
\begin{Def}
The {\em determinant\/} \index{determinant} of the $2\times 2$
matrix $A$ is
\begin{equation}  \label{D:determinant}
\det(A) = ad - bc.
\end{equation}
\end{Def}

\begin{prop} \label{propdet}
As a function on $2\times 2$ matrices, the determinant satisfies
the following properties.
\begin{itemize}
\item[(a)] The determinant of an upper triangular matrix is the
product of the diagonal elements. \index{matrix!upper triangular}
\item[(b)] The determinants of a matrix and its transpose are
equal. \index{matrix!transpose}
\item[(c)] $\det(AB) = \det(A)\det(B)$.
\end{itemize}
\end{prop}

\begin{proof} Both (a) and (b) are easily verified by direct
calculation.  Property (c) is also verified by direct
calculation --- but of a more extensive sort.  Note that
\[
\left(\begin{array}{cc} a & b\\ c & d \end{array}\right)
\left(\begin{array}{cc} \alpha & \beta \\ \gamma & \delta
\end{array}\right) =
\left(\begin{array}{cc} a\alpha+b\gamma & a\beta+b\delta \\
c\alpha+d\gamma & c\beta+d\delta\end{array}\right).
\]
Therefore,
\begin{eqnarray*}
\det(AB) & = & (a\alpha+b\gamma)(c\beta+d\delta) -
     (a\beta+b\delta)(c\alpha+d\gamma)\\
& = & (ac\alpha\beta+bc\beta\gamma+ad\alpha\delta+bd\gamma\delta)
-(ac\alpha\beta+bc\alpha\delta+ad\beta\gamma+bd\gamma\delta)\\
& = & bc(\beta\gamma-\alpha\delta) +
ad(\alpha\delta-\beta\gamma) \\
& = & (ad-bc)(\alpha\delta-\beta\gamma) \\
& = & \det(A)\det(B),
\end{eqnarray*}
as asserted.   \end{proof}

\begin{cor}  \label{C:2x2invert} \index{invertible}
A $2\times 2$ matrix $A$ is invertible if and only if $\det(A)\neq 0$.
\end{cor}

\begin{proof}  If $A$ is invertible, then $AA\inv = I_2$.
Proposition~\ref{propdet} implies that
\[
\det(A)\det(A\inv) = \det(I_2) = 1.
\]
Therefore, $\det(A)\neq 0$.  Conversely, if $\det(A)\neq 0$, then
\Ref{e:formAinv} implies that $A$ is invertible.  \end{proof}


\subsection*{Determinants and Area}

Suppose that $v$ and $w$ are two vectors in $\R^2$ that point in different
directions.  Then, the set of points
\[
z=\alpha v + \beta w \quad\mbox{where } 0\leq\alpha,\beta\leq 1
\]
is a parallelogram\index{parallelogram}, that we denote by $P$.
We denote the area of $P$ by $|P|$.  For example, the unit square $S$, whose
corners are $(0,0)$, $(1,0)$, $(0,1)$, and $(1,1)$, is the parallelogram
generated by the unit vectors $e_1$ and $e_2$.

Next let $A$ be a $2\times 2$ matrix and let
\[
A(P) = \{Az:z\in P\}.
\]
It follows from linearity (since $Az=\alpha Av+\beta Aw$) that $A(P)$ is the
parallelogram generated by $Av$ and $Aw$.

\begin{prop}  \label{P:det&area}
Let $A$ be a $2\times 2$ matrix and let $S$ be the unit square.  Then 
\begin{equation}  \label{e:det&area2}
|A(S)| = |\det A|.
\end{equation}
\end{prop}

\begin{proof}   Note that $A(S)$ is the parallelogram generated by $u_1=Ae_1$ and 
$u_2=Ae_2$, and $u_1$ and $u_2$ are the columns of $A$.  It follows that
\[
(\det A)^2=\det(A^t)\det(A)=\det(A^tA) =
\det\mattwo{u_1^tu_1}{u_1^tu_2}{u_2^tu_1}{u_2^tu_2}.
\]
Hence
\[
(\det A)^2=
\det\mattwo{||u_1||^2}{u_1\cdot u_2}{u_1\cdot u_2}{||u_2||^2} =
||u_1||^2||u_2||^2-(u_1\cdot u_2)^2.
\]
Recall that \Ref{e:areaP} of Chapter~\ref{chap:prelim} states that
\[
|P|^2 = ||v||^2||w||^2 - (v\cdot w)^2.
\]
where $P$ is the parallelogram generated by $v$ and $w$.  Therefore, 
$(\det A)^2 = |A(S)|^2$ and \Ref{e:det&area2} is verified. \end{proof}


\begin{thm}  \label{T:det&area}
Let $P$ be a parallelogram in $\R^2$ and let $A$ be a $2\times 2$
matrix.  Then
\begin{equation} \label{e:det&area}
|A(P)| = |\det A||P|.
\end{equation}
\end{thm}

\begin{proof}  First note that \Ref{e:det&area2} a special case of \Ref{e:det&area}, 
since $|S|=1$.   Next, let $P$ be the parallelogram generated by the (column)
vectors $v$ and $w$, and let $B=(v|w)$.  Then $P=B(S)$.  It
follows from \Ref{e:det&area2} that $|P|=|\det B|$.  Moreover,
\begin{eqnarray*}
|A(P)| & = & |(AB)(S)| \\
& = & |\det(AB)| \\
& = & |\det A||\det B| \\
& = & |\det A||P|,
\end{eqnarray*}
as desired.  \end{proof}

\EXER

\TEXER

\begin{exercise} \label{c4.9.1}
Find the inverse of the matrix
\[
\mattwo{2}{1}{3}{2}.
\]
\end{exercise}

\begin{exercise} \label{c7.8.4}
Find the inverse of the shear matrix $\mattwo{1}{K}{0}{1}$.
\end{exercise}\index{shear}

\begin{exercise} \label{c4.9.4}
Show that the $2\times 2$ matrix $A=\mattwo{a}{b}{c}{d}$ is row
equivalent to $I_2$ if and only if $ad-bc\neq 0$.  {\bf Hint:}
Prove this result separately in the two cases $a\neq 0$ and
$a=0$.
\end{exercise}

\begin{exercise} \label{c4.9.5}
Let $A$ be a $2\times 2$ matrix having integer entries.  Find a
condition on the entries of $A$ that guarantees that $A\inv$ has
integer entries.
\end{exercise}

\begin{exercise} \label{c6.4.4}
Let $A$ be a $2\times 2$ matrix and assume that $\det(A)\neq 0$.
Then use the explicit form for $A\inv$ given in \Ref{e:formAinv}
to verify that
\[
\det(A\inv) = \frac{1}{\det(A)}.
\]
\end{exercise}

\begin{exercise} \label{c7.8.3}
Sketch the triangle whose vertices are $0$, $p=(3,0)^t$, and
$q=(0,2)^t$; and find the area of this triangle.  Let
\[
M=\mattwo{-4}{-3}{5}{-2}.
\]
Sketch the triangle whose vertices are $0$, $Mp$, and $Mq$; and
find the area of this triangle.
\end{exercise}

\begin{exercise} \label{c7.8.4A}
Cramer's rule \index{Cramer's rule} provides a method based on determinants 
for finding the unique solution to the linear equation $Ax=b$ when $A$ is 
an invertible matrix.  More precisely, let $A$ be an invertible $2\times 2$ 
matrix and let $b\in\R^2$ be a column vector. Let $B_j$ be the $2\times 2$ 
matrix obtained from $A$ by replacing the $j^{th}$ column of $A$ by the vector 
$b$.  Let $x=(x_1,x_2)^t$ be the unique solution to $Ax=b$. Then Cramer's rule
states that 
\begin{equation}  \label{E:cramer}
x_j = \frac{\det(B_j)}{\det(A)}.
\end{equation}
Prove Cramer's rule.  {\bf Hint:}  Write the general system of two equations in 
two unknowns as
\begin{eqnarray*}
a_{11}x_1+a_{12}x_2 & = & b_1\\
a_{21}x_1+a_{22}x_2 & = & b_2.
\end{eqnarray*}
Subtract $a_{11}$ times the second equation from $a_{21}$ times the first 
equation to eliminate $x_1$; then solve for $x_2$, and verify \Ref{E:cramer}.  
Use a similar calculation to solve for $x_1$. 
\end{exercise}

\noindent In Exercises~\ref{c7.8.4B} -- \ref{c7.8.4C} use Cramer's 
rule~\Ref{E:cramer} to solve the given system of linear 
equations.
\begin{exercise} \label{c7.8.4B}
Solve \qquad $\begin{array}{rcl} 2x+3y & = & 2 \\ 3x - 5y & = & 1 \end{array}$ 
\qquad for $x$.
\end{exercise} 
\begin{exercise} \label{c7.8.4C}
Solve \qquad
$\begin{array}{rcl} 4x-3y & = & -1 \\ x + 2y & = & 7 \end{array}$ 
\qquad for $y$.
\end{exercise}

\CEXER

\begin{exercise} \label{c4.9.9}
Use \Matlab to choose five $2\times 2$ matrices at random and compute
their inverses.  Do you get the impression that `typically'
$2\times 2$ matrices are invertible?  Try to find a reason for
this fact using the determinant of $2\times 2$ matrices.
\end{exercise}

\noindent  In Exercises~\ref{c3.8.AA} -- \ref{c3.8.AD} use the {\sf unit 
square} icon in the program {\sf map} to test Proposition~\ref{P:det&area}, as 
follows. Enter the given matrix $A$ into {\sf map} and map the {\sf unit 
square} icon.  Compute $\det(A)$ by estimating the area of $A(S)$ --- given 
that $S$ has unit area.  For each matrix, use this numerical experiment to 
decide whether or not the matrix is invertible.
\begin{exercise}  \label{c3.8.AA}
$A=\mattwo{0}{-2}{2}{0}$.
\end{exercise}
\begin{exercise}  \label{c3.8.AB}
$A=\mattwo{-0.5}{-0.5}{0.7}{0.7}$.
\end{exercise}
\begin{exercise}  \label{c3.8.AC}
$A=\mattwo{-1}{-0.5}{-2}{-1}$.
\end{exercise}
\begin{exercise}  \label{c3.8.AD}
$A=\mattwo{0.7071}{0.7071}{-0.7071}{0.7071}$.
\end{exercise}

\end{document}
