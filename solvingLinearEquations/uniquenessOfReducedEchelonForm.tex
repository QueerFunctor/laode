\documentclass{ximera}

 

\usepackage{epsfig}

\graphicspath{
  {./}
  {figures/}
}

\usepackage{morewrites}
\makeatletter
\newcommand\subfile[1]{%
\renewcommand{\input}[1]{}%
\begingroup\skip@preamble\otherinput{#1}\endgroup\par\vspace{\topsep}
\let\input\otherinput}
\makeatother

\newcommand{\includeexercises}{\directlua{dofile("/home/jim/linearAlgebra/laode/exercises.lua")}}

%\newcounter{ccounter}
%\setcounter{ccounter}{1}
%\newcommand{\Chapter}[1]{\setcounter{chapter}{\arabic{ccounter}}\chapter{#1}\addtocounter{ccounter}{1}}

%\newcommand{\section}[1]{\section{#1}\setcounter{thm}{0}\setcounter{equation}{0}}

%\renewcommand{\theequation}{\arabic{chapter}.\arabic{section}.\arabic{equation}}
%\renewcommand{\thefigure}{\arabic{chapter}.\arabic{figure}}
%\renewcommand{\thetable}{\arabic{chapter}.\arabic{table}}

%\newcommand{\Sec}[2]{\section{#1}\markright{\arabic{ccounter}.\arabic{section}.#2}\setcounter{equation}{0}\setcounter{thm}{0}\setcounter{figure}{0}}

\newcommand{\Sec}[2]{\section{#1}}

\setcounter{secnumdepth}{2}
%\setcounter{secnumdepth}{1} 

%\newcounter{THM}
%\renewcommand{\theTHM}{\arabic{chapter}.\arabic{section}}

\newcommand{\trademark}{{R\!\!\!\!\!\bigcirc}}
%\newtheorem{exercise}{}

\newcommand{\dfield}{{\sf dfield9}}
\newcommand{\pplane}{{\sf pplane9}}

\newcommand{\EXER}{\section*{Exercises}}%\vspace*{0.2in}\hrule\small\setcounter{exercise}{0}}
\newcommand{\CEXER}{}%\vspace{0.08in}\begin{center}Computer Exercises\end{center}}
\newcommand{\TEXER}{} %\vspace{0.08in}\begin{center}Hand Exercises\end{center}}
\newcommand{\AEXER}{} %\vspace{0.08in}\begin{center}Hand Exercises\end{center}}

% BADBAD: \newcommand{\Bbb}{\bf}

\newcommand{\R}{\mbox{$\Bbb{R}$}}
\newcommand{\C}{\mbox{$\Bbb{C}$}}
\newcommand{\Z}{\mbox{$\Bbb{Z}$}}
\newcommand{\N}{\mbox{$\Bbb{N}$}}
\newcommand{\D}{\mbox{{\bf D}}}
\usepackage{amssymb}
%\newcommand{\qed}{\hfill\mbox{\raggedright$\square$} \vspace{1ex}}
%\newcommand{\proof}{\noindent {\bf Proof:} \hspace{0.1in}}

\newcommand{\setmin}{\;\mbox{--}\;}
\newcommand{\Matlab}{{M\small{AT\-LAB}} }
\newcommand{\Matlabp}{{M\small{AT\-LAB}}}
\newcommand{\computer}{\Matlab Instructions}
\newcommand{\half}{\mbox{$\frac{1}{2}$}}
\newcommand{\compose}{\raisebox{.15ex}{\mbox{{\scriptsize$\circ$}}}}
\newcommand{\AND}{\quad\mbox{and}\quad}
\newcommand{\vect}[2]{\left(\begin{array}{c} #1_1 \\ \vdots \\
 #1_{#2}\end{array}\right)}
\newcommand{\mattwo}[4]{\left(\begin{array}{rr} #1 & #2\\ #3
&#4\end{array}\right)}
\newcommand{\mattwoc}[4]{\left(\begin{array}{cc} #1 & #2\\ #3
&#4\end{array}\right)}
\newcommand{\vectwo}[2]{\left(\begin{array}{r} #1 \\ #2\end{array}\right)}
\newcommand{\vectwoc}[2]{\left(\begin{array}{c} #1 \\ #2\end{array}\right)}

\newcommand{\ignore}[1]{}


\newcommand{\inv}{^{-1}}
\newcommand{\CC}{{\cal C}}
\newcommand{\CCone}{\CC^1}
\newcommand{\Span}{{\rm span}}
\newcommand{\rank}{{\rm rank}}
\newcommand{\trace}{{\rm tr}}
\newcommand{\RE}{{\rm Re}}
\newcommand{\IM}{{\rm Im}}
\newcommand{\nulls}{{\rm null\;space}}

\newcommand{\dps}{\displaystyle}
\newcommand{\arraystart}{\renewcommand{\arraystretch}{1.8}}
\newcommand{\arrayfinish}{\renewcommand{\arraystretch}{1.2}}
\newcommand{\Start}[1]{\vspace{0.08in}\noindent {\bf Section~\ref{#1}}}
\newcommand{\exer}[1]{\noindent {\bf \ref{#1}}}
\newcommand{\ans}{}
\newcommand{\matthree}[9]{\left(\begin{array}{rrr} #1 & #2 & #3 \\ #4 & #5 & #6
\\ #7 & #8 & #9\end{array}\right)}
\newcommand{\cvectwo}[2]{\left(\begin{array}{c} #1 \\ #2\end{array}\right)}
\newcommand{\cmatthree}[9]{\left(\begin{array}{ccc} #1 & #2 & #3 \\ #4 & #5 &
#6 \\ #7 & #8 & #9\end{array}\right)}
\newcommand{\vecthree}[3]{\left(\begin{array}{r} #1 \\ #2 \\
#3\end{array}\right)}
\newcommand{\cvecthree}[3]{\left(\begin{array}{c} #1 \\ #2 \\
#3\end{array}\right)}
\newcommand{\cmattwo}[4]{\left(\begin{array}{cc} #1 & #2\\ #3
&#4\end{array}\right)}

\newcommand{\Matrix}[1]{\ensuremath{\left(\begin{array}{rrrrrrrrrrrrrrrrrr} #1 \end{array}\right)}}

\newcommand{\Matrixc}[1]{\ensuremath{\left(\begin{array}{cccccccccccc} #1 \end{array}\right)}}



\renewcommand{\labelenumi}{\theenumi)}
\newenvironment{enumeratea}%
{\begingroup
 \renewcommand{\theenumi}{\alph{enumi}}
 \renewcommand{\labelenumi}{(\theenumi)}
 \begin{enumerate}}
 {\end{enumerate}\endgroup}



\newcounter{help}
\renewcommand{\thehelp}{\thesection.\arabic{equation}}

%\newenvironment{equation*}%
%{\renewcommand\endequation{\eqno (\theequation)* $$}%
%   \begin{equation}}%
%   {\end{equation}\renewcommand\endequation{\eqno \@eqnnum
%$$\global\@ignoretrue}}

%\input{psfig.tex}

\author{Martin Golubitsky and Michael Dellnitz}

%\newenvironment{matlabEquation}%
%{\renewcommand\endequation{\eqno (\theequation*) $$}%
%   \begin{equation}}%
%   {\end{equation}\renewcommand\endequation{\eqno \@eqnnum
% $$\global\@ignoretrue}}

\newcommand{\soln}{\textbf{Solution:} }
\newcommand{\exercap}[1]{\centerline{Figure~\ref{#1}}}
\newcommand{\exercaptwo}[1]{\centerline{Figure~\ref{#1}a\hspace{2.1in}
Figure~\ref{#1}b}}
\newcommand{\exercapthree}[1]{\centerline{Figure~\ref{#1}a\hspace{1.2in}
Figure~\ref{#1}b\hspace{1.2in}Figure~\ref{#1}c}}
\newcommand{\para}{\hspace{0.4in}}

\renewenvironment{solution}{\suppress}{\endsuppress}

\ifxake
\newenvironment{matlabEquation}{\begin{equation}}{\end{equation}}
\else
\newenvironment{matlabEquation}%
{\let\oldtheequation\theequation\renewcommand{\theequation}{\oldtheequation*}\begin{equation}}%
  {\end{equation}\let\theequation\oldtheequation}
\fi

\makeatother


\title{Uniqueness of Reduced Echelon Form}

\begin{document}
\begin{abstract}
\end{abstract}
\maketitle


\label{S:uniquerowechelon}


In this section we prove Theorem~\ref{uniquerowechelon}, which
states that every matrix is row equivalent to precisely one
reduced echelon form matrix.  

\noindent {\bf Proof of Theorem~\ref{uniquerowechelon}:}
Suppose that $E$ and $F$ are two $m\times n$ reduced echelon
matrices that are row equivalent to $A$.  Since elementary row
operations are invertible, the two matrices $E$ and $F$ are
row equivalent.  Thus, the systems of linear equations associated to 
the $m\times (n+1)$ matrices $(E|0)$ and $(F|0)$ must have exactly the
same set of solutions.  It is the fact that the solution sets
of the linear equations associated to $(E|0)$ and $(F|0)$ are
identical that allows us to prove that $E=F$.

Begin by renumbering the variables $x_1,\ldots,x_n$ so that the equations 
associated to $(E|0)$ have the form: 
\begin{equation}  \label{e1-ell2a}
\begin{array}{rcl}
  x_1  & = &  - a_{1,\ell+1}x_{\ell+1} - \cdots - a_{1,n}x_n\\
  x_2  & = &  - a_{2,\ell+1}x_{\ell+1} - \cdots - a_{2,n}x_n\\
\vdots &   &    \vdots \\
x_\ell & = &  - a_{\ell,\ell+1}x_{\ell+1} - \cdots - a_{\ell,n}x_n.
\end{array}
\end{equation}
In this form, pivots of $E$ occur in the columns $1,\ldots,\ell$.  We begin 
by showing that the matrix $F$ also has pivots in columns $1,\ldots,\ell$. 
Moreover, there is a unique solution to these equations for {\em every\/}
choice of numbers $x_{\ell+1},\ldots,x_n$.  

Suppose that the pivots of $F$ do not occur in columns $1,\ldots,\ell$.  Then
there is a row in $F$ whose first nonzero entry occurs in a column $k>\ell$. 
This row corresponds to an equation
\[
x_k = c_{k+1}x_{k+1} + \cdots + c_nx_n.
\]
Now, consider solutions that satisfy
\[
x_{\ell+1} = \cdots = x_{k-1} = 0 \AND x_{k+1} = \cdots = x_n = 0.
\]
In the equations associated to the matrix $(E|0)$, there is a unique solution
associated with every number $x_k$; while in the equations associated to the 
matrix $(F|0)$, $x_k$ must be zero to be a solution.  This argument
contradicts the fact that the $(E|0)$ equations and the $(F|0)$
equations have the same solutions.  So the pivots of $F$ must also occur in 
columns $1,\ldots,\ell$, and the equations associated to $F$ must have the 
form:
\begin{equation} \label{e1-ell2b}
\begin{array}{rcl}
x_1  & = & - \hat{a}_{1,\ell+1}x_{\ell+1} - \cdots - \hat{a}_{1,n}x_n \\
x_2  & = & - \hat{a}_{2,\ell+1}x_{\ell+1} - \cdots - \hat{a}_{2,n}x_n \\
\vdots &   &    \vdots  \\
x_\ell & = & - \hat{a}_{\ell,\ell+1}x_{\ell+1} - \cdots - \hat{a}_{\ell,n}x_n
\end{array}
\end{equation}
where $\hat{a}_{i,j}$ are scalars.

To complete this proof, we show that $a_{i,j}=\hat{a}_{i,j}$.  These 
equalities are verified as follows.  There is just one solution to each 
system \Ref{e1-ell2a} and \Ref{e1-ell2b} of the form
\[
x_{\ell+1}=1,\; x_{\ell+2}=\cdots=x_n=0.
\]
These solutions are
\[
(-a_{1,\ell+1}, \ldots, -a_{\ell,\ell+1},1,0,\cdots,0)
\]
for \Ref{e1-ell2a} and
\[
(-\hat{a}_{1,\ell+1},\ldots,-\hat{a}_{\ell,\ell+1},1,0\cdots,0)
\]
for \Ref{e1-ell2b}. It follows that $a_{j,\ell+1}=\hat{a}_{j,\ell+1}$
for $j=1,\ldots,\ell$.  Complete this proof by repeating this argument. 
Just inspect solutions of the form
\[
x_{\ell+1}=0,\; x_{\ell+2}=1,\; x_{\ell+3}=\cdots= x_n=0
\]
through
\[
x_{\ell+1}=\cdots = x_{n-1}=0,\; x_n=1.
\]



\end{document}
