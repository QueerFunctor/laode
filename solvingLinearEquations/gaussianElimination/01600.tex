\documentclass{ximera}
 

\usepackage{epsfig}

\graphicspath{
  {./}
  {figures/}
}

\usepackage{morewrites}
\makeatletter
\newcommand\subfile[1]{%
\renewcommand{\input}[1]{}%
\begingroup\skip@preamble\otherinput{#1}\endgroup\par\vspace{\topsep}
\let\input\otherinput}
\makeatother

\newcommand{\includeexercises}{\directlua{dofile("/home/jim/linearAlgebra/laode/exercises.lua")}}

%\newcounter{ccounter}
%\setcounter{ccounter}{1}
%\newcommand{\Chapter}[1]{\setcounter{chapter}{\arabic{ccounter}}\chapter{#1}\addtocounter{ccounter}{1}}

%\newcommand{\section}[1]{\section{#1}\setcounter{thm}{0}\setcounter{equation}{0}}

%\renewcommand{\theequation}{\arabic{chapter}.\arabic{section}.\arabic{equation}}
%\renewcommand{\thefigure}{\arabic{chapter}.\arabic{figure}}
%\renewcommand{\thetable}{\arabic{chapter}.\arabic{table}}

%\newcommand{\Sec}[2]{\section{#1}\markright{\arabic{ccounter}.\arabic{section}.#2}\setcounter{equation}{0}\setcounter{thm}{0}\setcounter{figure}{0}}

\newcommand{\Sec}[2]{\section{#1}}

\setcounter{secnumdepth}{2}
%\setcounter{secnumdepth}{1} 

%\newcounter{THM}
%\renewcommand{\theTHM}{\arabic{chapter}.\arabic{section}}

\newcommand{\trademark}{{R\!\!\!\!\!\bigcirc}}
%\newtheorem{exercise}{}

\newcommand{\dfield}{{\sf dfield9}}
\newcommand{\pplane}{{\sf pplane9}}

\newcommand{\EXER}{\section*{Exercises}}%\vspace*{0.2in}\hrule\small\setcounter{exercise}{0}}
\newcommand{\CEXER}{}%\vspace{0.08in}\begin{center}Computer Exercises\end{center}}
\newcommand{\TEXER}{} %\vspace{0.08in}\begin{center}Hand Exercises\end{center}}
\newcommand{\AEXER}{} %\vspace{0.08in}\begin{center}Hand Exercises\end{center}}

% BADBAD: \newcommand{\Bbb}{\bf}

\newcommand{\R}{\mbox{$\Bbb{R}$}}
\newcommand{\C}{\mbox{$\Bbb{C}$}}
\newcommand{\Z}{\mbox{$\Bbb{Z}$}}
\newcommand{\N}{\mbox{$\Bbb{N}$}}
\newcommand{\D}{\mbox{{\bf D}}}
\usepackage{amssymb}
%\newcommand{\qed}{\hfill\mbox{\raggedright$\square$} \vspace{1ex}}
%\newcommand{\proof}{\noindent {\bf Proof:} \hspace{0.1in}}

\newcommand{\setmin}{\;\mbox{--}\;}
\newcommand{\Matlab}{{M\small{AT\-LAB}} }
\newcommand{\Matlabp}{{M\small{AT\-LAB}}}
\newcommand{\computer}{\Matlab Instructions}
\newcommand{\half}{\mbox{$\frac{1}{2}$}}
\newcommand{\compose}{\raisebox{.15ex}{\mbox{{\scriptsize$\circ$}}}}
\newcommand{\AND}{\quad\mbox{and}\quad}
\newcommand{\vect}[2]{\left(\begin{array}{c} #1_1 \\ \vdots \\
 #1_{#2}\end{array}\right)}
\newcommand{\mattwo}[4]{\left(\begin{array}{rr} #1 & #2\\ #3
&#4\end{array}\right)}
\newcommand{\mattwoc}[4]{\left(\begin{array}{cc} #1 & #2\\ #3
&#4\end{array}\right)}
\newcommand{\vectwo}[2]{\left(\begin{array}{r} #1 \\ #2\end{array}\right)}
\newcommand{\vectwoc}[2]{\left(\begin{array}{c} #1 \\ #2\end{array}\right)}

\newcommand{\ignore}[1]{}


\newcommand{\inv}{^{-1}}
\newcommand{\CC}{{\cal C}}
\newcommand{\CCone}{\CC^1}
\newcommand{\Span}{{\rm span}}
\newcommand{\rank}{{\rm rank}}
\newcommand{\trace}{{\rm tr}}
\newcommand{\RE}{{\rm Re}}
\newcommand{\IM}{{\rm Im}}
\newcommand{\nulls}{{\rm null\;space}}

\newcommand{\dps}{\displaystyle}
\newcommand{\arraystart}{\renewcommand{\arraystretch}{1.8}}
\newcommand{\arrayfinish}{\renewcommand{\arraystretch}{1.2}}
\newcommand{\Start}[1]{\vspace{0.08in}\noindent {\bf Section~\ref{#1}}}
\newcommand{\exer}[1]{\noindent {\bf \ref{#1}}}
\newcommand{\ans}{}
\newcommand{\matthree}[9]{\left(\begin{array}{rrr} #1 & #2 & #3 \\ #4 & #5 & #6
\\ #7 & #8 & #9\end{array}\right)}
\newcommand{\cvectwo}[2]{\left(\begin{array}{c} #1 \\ #2\end{array}\right)}
\newcommand{\cmatthree}[9]{\left(\begin{array}{ccc} #1 & #2 & #3 \\ #4 & #5 &
#6 \\ #7 & #8 & #9\end{array}\right)}
\newcommand{\vecthree}[3]{\left(\begin{array}{r} #1 \\ #2 \\
#3\end{array}\right)}
\newcommand{\cvecthree}[3]{\left(\begin{array}{c} #1 \\ #2 \\
#3\end{array}\right)}
\newcommand{\cmattwo}[4]{\left(\begin{array}{cc} #1 & #2\\ #3
&#4\end{array}\right)}

\newcommand{\Matrix}[1]{\ensuremath{\left(\begin{array}{rrrrrrrrrrrrrrrrrr} #1 \end{array}\right)}}

\newcommand{\Matrixc}[1]{\ensuremath{\left(\begin{array}{cccccccccccc} #1 \end{array}\right)}}



\renewcommand{\labelenumi}{\theenumi)}
\newenvironment{enumeratea}%
{\begingroup
 \renewcommand{\theenumi}{\alph{enumi}}
 \renewcommand{\labelenumi}{(\theenumi)}
 \begin{enumerate}}
 {\end{enumerate}\endgroup}



\newcounter{help}
\renewcommand{\thehelp}{\thesection.\arabic{equation}}

%\newenvironment{equation*}%
%{\renewcommand\endequation{\eqno (\theequation)* $$}%
%   \begin{equation}}%
%   {\end{equation}\renewcommand\endequation{\eqno \@eqnnum
%$$\global\@ignoretrue}}

%\input{psfig.tex}

\author{Martin Golubitsky and Michael Dellnitz}

%\newenvironment{matlabEquation}%
%{\renewcommand\endequation{\eqno (\theequation*) $$}%
%   \begin{equation}}%
%   {\end{equation}\renewcommand\endequation{\eqno \@eqnnum
% $$\global\@ignoretrue}}

\newcommand{\soln}{\textbf{Solution:} }
\newcommand{\exercap}[1]{\centerline{Figure~\ref{#1}}}
\newcommand{\exercaptwo}[1]{\centerline{Figure~\ref{#1}a\hspace{2.1in}
Figure~\ref{#1}b}}
\newcommand{\exercapthree}[1]{\centerline{Figure~\ref{#1}a\hspace{1.2in}
Figure~\ref{#1}b\hspace{1.2in}Figure~\ref{#1}c}}
\newcommand{\para}{\hspace{0.4in}}

\renewenvironment{solution}{\suppress}{\endsuppress}

\ifxake
\newenvironment{matlabEquation}{\begin{equation}}{\end{equation}}
\else
\newenvironment{matlabEquation}%
{\let\oldtheequation\theequation\renewcommand{\theequation}{\oldtheequation*}\begin{equation}}%
  {\end{equation}\let\theequation\oldtheequation}
\fi

\makeatother

\begin{document}
\begin{computerExercise} \label{c2.3.4}
{\bf Comment:} {\rm To understand the point of this exercise you
must begin by typing the \Matlab command {\tt format short e}.
This command will set a format in which you can see the
difficulties that sometimes arise in numerical computations.}

Consider the following two $3\times 3$-matrices:
\begin{matlabEquation}\label{MATLAB:14}
A=\left( \begin{array}{rrr}
     1  &  3  &  4\\
     2  &  1  &  1\\
    -4  &  3  &  5
\end{array}\right) \AND
B=\left( \begin{array}{rrr}
     3  &  1  &  4\\
     1  &  2  &  1\\
     3  & -4  &  5
\end{array}\right).
\end{matlabEquation}
Note that matrix $B$ is obtained from matrix $A$ by interchanging the
first two columns.
\begin{itemize}
\item[(a)] Use \Matlab to put $A$ into row echelon form using the
transformations
\begin{enumerate}
\item Subtract $2$ times the $1^{st}$ row from the $2^{nd}$.
\item Add $4$ times the $1^{st}$ row to the $3^{rd}$.
\item Divide the $2^{nd}$ row by $-5$.
\item Subtract $15$ times the $2^{nd}$ row from the $3^{rd}$.
\end{enumerate}
\item[(b)] Put $B$ by hand into row echelon form using the
transformations
\begin{enumerate}
\item Divide the $1^{st}$ row by $3$.
\item Subtract the $1^{st}$ row from the $2^{nd}$.
\item Subtract $3$ times the $1^{st}$ row from the $3^{rd}$.
\item Multiply the $2^{nd}$ row by $3/5$.
\item Add $5$ times the $2^{nd}$ row to the $3^{rd}$.
\end{enumerate}
\item[(c)] Use \Matlab to put $B$ into row echelon form using the
same transformations as in part (b).
\item[(d)] Discuss the outcome of the three transformations.  Is
there a difference in the results?  Would you expect to see a
difference?  Could the difference be crucial when solving a system
of linear equations?
\end{itemize}

\begin{solution}

(a)
\begin{verbatim}
A =
   1.0000e+00   3.0000e+00   4.0000e+00
            0   1.0000e+00   1.4000e+00
            0            0            0
\end{verbatim}

(b)
$B = \left(\begin{array}{rrr}
1 & \frac{1}{3} & \frac{4}{3} \\
0 & 1 & -\frac{1}{5} \\
0 & 0 & 0\end{array} \right)$

(c)
\begin{verbatim}
B =
   1.0000e+00   3.3333e-01   1.3333e+00
            0   1.0000e+00  -2.0000e-01
            0            0   2.2204e-16
\end{verbatim}


(d) Note that switching the first two columns of matrix {\tt A} produces
matrix {\tt B}.  Suppose that {\tt A} and {\tt B} represent the left-hand
sides of linear systems with the same vector representing the right-hand
sides.  If the solution of system {\tt A} is $(x,y,z) = (a,b,c)$, then the
solution of system {\tt B} should be $(x,y,z) = (b,a,c)$.  However,
according to the row reduced matrices produced by \Matlab, the system
corresponding to {\tt B} has a unique solution, and the system
corresponding to {\tt A} does not.  Row reducing {\tt B} by hand shows
that there is not a unique solution.  \Matlab calculations provide one because of 
roundoff error.  The division by 3 in the first step of the row reduction
for {\tt B} causes a rounding inaccuracy.  Because of this, \Matlab
eventually computes a very small nonzero value for {\tt B}$(3,3)$
rather than the correct answer of 0.

\end{solution}
\end{computerExercise}
\end{document}
