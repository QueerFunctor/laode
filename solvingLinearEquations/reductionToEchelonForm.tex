\documentclass{ximera}

 

\usepackage{epsfig}

\graphicspath{
  {./}
  {figures/}
}

\usepackage{morewrites}
\makeatletter
\newcommand\subfile[1]{%
\renewcommand{\input}[1]{}%
\begingroup\skip@preamble\otherinput{#1}\endgroup\par\vspace{\topsep}
\let\input\otherinput}
\makeatother

\newcommand{\includeexercises}{\directlua{dofile("/home/jim/linearAlgebra/laode/exercises.lua")}}

%\newcounter{ccounter}
%\setcounter{ccounter}{1}
%\newcommand{\Chapter}[1]{\setcounter{chapter}{\arabic{ccounter}}\chapter{#1}\addtocounter{ccounter}{1}}

%\newcommand{\section}[1]{\section{#1}\setcounter{thm}{0}\setcounter{equation}{0}}

%\renewcommand{\theequation}{\arabic{chapter}.\arabic{section}.\arabic{equation}}
%\renewcommand{\thefigure}{\arabic{chapter}.\arabic{figure}}
%\renewcommand{\thetable}{\arabic{chapter}.\arabic{table}}

%\newcommand{\Sec}[2]{\section{#1}\markright{\arabic{ccounter}.\arabic{section}.#2}\setcounter{equation}{0}\setcounter{thm}{0}\setcounter{figure}{0}}

\newcommand{\Sec}[2]{\section{#1}}

\setcounter{secnumdepth}{2}
%\setcounter{secnumdepth}{1} 

%\newcounter{THM}
%\renewcommand{\theTHM}{\arabic{chapter}.\arabic{section}}

\newcommand{\trademark}{{R\!\!\!\!\!\bigcirc}}
%\newtheorem{exercise}{}

\newcommand{\dfield}{{\sf dfield9}}
\newcommand{\pplane}{{\sf pplane9}}

\newcommand{\EXER}{\section*{Exercises}}%\vspace*{0.2in}\hrule\small\setcounter{exercise}{0}}
\newcommand{\CEXER}{}%\vspace{0.08in}\begin{center}Computer Exercises\end{center}}
\newcommand{\TEXER}{} %\vspace{0.08in}\begin{center}Hand Exercises\end{center}}
\newcommand{\AEXER}{} %\vspace{0.08in}\begin{center}Hand Exercises\end{center}}

% BADBAD: \newcommand{\Bbb}{\bf}

\newcommand{\R}{\mbox{$\Bbb{R}$}}
\newcommand{\C}{\mbox{$\Bbb{C}$}}
\newcommand{\Z}{\mbox{$\Bbb{Z}$}}
\newcommand{\N}{\mbox{$\Bbb{N}$}}
\newcommand{\D}{\mbox{{\bf D}}}
\usepackage{amssymb}
%\newcommand{\qed}{\hfill\mbox{\raggedright$\square$} \vspace{1ex}}
%\newcommand{\proof}{\noindent {\bf Proof:} \hspace{0.1in}}

\newcommand{\setmin}{\;\mbox{--}\;}
\newcommand{\Matlab}{{M\small{AT\-LAB}} }
\newcommand{\Matlabp}{{M\small{AT\-LAB}}}
\newcommand{\computer}{\Matlab Instructions}
\newcommand{\half}{\mbox{$\frac{1}{2}$}}
\newcommand{\compose}{\raisebox{.15ex}{\mbox{{\scriptsize$\circ$}}}}
\newcommand{\AND}{\quad\mbox{and}\quad}
\newcommand{\vect}[2]{\left(\begin{array}{c} #1_1 \\ \vdots \\
 #1_{#2}\end{array}\right)}
\newcommand{\mattwo}[4]{\left(\begin{array}{rr} #1 & #2\\ #3
&#4\end{array}\right)}
\newcommand{\mattwoc}[4]{\left(\begin{array}{cc} #1 & #2\\ #3
&#4\end{array}\right)}
\newcommand{\vectwo}[2]{\left(\begin{array}{r} #1 \\ #2\end{array}\right)}
\newcommand{\vectwoc}[2]{\left(\begin{array}{c} #1 \\ #2\end{array}\right)}

\newcommand{\ignore}[1]{}


\newcommand{\inv}{^{-1}}
\newcommand{\CC}{{\cal C}}
\newcommand{\CCone}{\CC^1}
\newcommand{\Span}{{\rm span}}
\newcommand{\rank}{{\rm rank}}
\newcommand{\trace}{{\rm tr}}
\newcommand{\RE}{{\rm Re}}
\newcommand{\IM}{{\rm Im}}
\newcommand{\nulls}{{\rm null\;space}}

\newcommand{\dps}{\displaystyle}
\newcommand{\arraystart}{\renewcommand{\arraystretch}{1.8}}
\newcommand{\arrayfinish}{\renewcommand{\arraystretch}{1.2}}
\newcommand{\Start}[1]{\vspace{0.08in}\noindent {\bf Section~\ref{#1}}}
\newcommand{\exer}[1]{\noindent {\bf \ref{#1}}}
\newcommand{\ans}{}
\newcommand{\matthree}[9]{\left(\begin{array}{rrr} #1 & #2 & #3 \\ #4 & #5 & #6
\\ #7 & #8 & #9\end{array}\right)}
\newcommand{\cvectwo}[2]{\left(\begin{array}{c} #1 \\ #2\end{array}\right)}
\newcommand{\cmatthree}[9]{\left(\begin{array}{ccc} #1 & #2 & #3 \\ #4 & #5 &
#6 \\ #7 & #8 & #9\end{array}\right)}
\newcommand{\vecthree}[3]{\left(\begin{array}{r} #1 \\ #2 \\
#3\end{array}\right)}
\newcommand{\cvecthree}[3]{\left(\begin{array}{c} #1 \\ #2 \\
#3\end{array}\right)}
\newcommand{\cmattwo}[4]{\left(\begin{array}{cc} #1 & #2\\ #3
&#4\end{array}\right)}

\newcommand{\Matrix}[1]{\ensuremath{\left(\begin{array}{rrrrrrrrrrrrrrrrrr} #1 \end{array}\right)}}

\newcommand{\Matrixc}[1]{\ensuremath{\left(\begin{array}{cccccccccccc} #1 \end{array}\right)}}



\renewcommand{\labelenumi}{\theenumi)}
\newenvironment{enumeratea}%
{\begingroup
 \renewcommand{\theenumi}{\alph{enumi}}
 \renewcommand{\labelenumi}{(\theenumi)}
 \begin{enumerate}}
 {\end{enumerate}\endgroup}



\newcounter{help}
\renewcommand{\thehelp}{\thesection.\arabic{equation}}

%\newenvironment{equation*}%
%{\renewcommand\endequation{\eqno (\theequation)* $$}%
%   \begin{equation}}%
%   {\end{equation}\renewcommand\endequation{\eqno \@eqnnum
%$$\global\@ignoretrue}}

%\input{psfig.tex}

\author{Martin Golubitsky and Michael Dellnitz}

%\newenvironment{matlabEquation}%
%{\renewcommand\endequation{\eqno (\theequation*) $$}%
%   \begin{equation}}%
%   {\end{equation}\renewcommand\endequation{\eqno \@eqnnum
% $$\global\@ignoretrue}}

\newcommand{\soln}{\textbf{Solution:} }
\newcommand{\exercap}[1]{\centerline{Figure~\ref{#1}}}
\newcommand{\exercaptwo}[1]{\centerline{Figure~\ref{#1}a\hspace{2.1in}
Figure~\ref{#1}b}}
\newcommand{\exercapthree}[1]{\centerline{Figure~\ref{#1}a\hspace{1.2in}
Figure~\ref{#1}b\hspace{1.2in}Figure~\ref{#1}c}}
\newcommand{\para}{\hspace{0.4in}}

\renewenvironment{solution}{\suppress}{\endsuppress}

\ifxake
\newenvironment{matlabEquation}{\begin{equation}}{\end{equation}}
\else
\newenvironment{matlabEquation}%
{\let\oldtheequation\theequation\renewcommand{\theequation}{\oldtheequation*}\begin{equation}}%
  {\end{equation}\let\theequation\oldtheequation}
\fi

\makeatother


\title{Reduction to Echelon Form}

\begin{document}
\begin{abstract}
\end{abstract}
\maketitle


\label{S:2.4}

In this section, we formalize our previous numerical
experiments.  We define more precisely the notions of echelon
form and reduced echelon form matrices, and we prove that every
matrix can be put into reduced echelon form using a sequence of
elementary row operations.  Consequently, we will have developed
an algorithm for determining whether a system of linear
equations is consistent or inconsistent, and for determining all
solutions to a consistent system.

\begin{definition}
A matrix $E$ is in (row) {\em echelon form\/} \index{echelon
form} if two conditions hold.
\begin{enumerate}
\item[(a)] The first nonzero entry in each row of $E$ is equal
to $1$.  This leading entry $1$ is called a {\em pivot\/}. \index{pivot}
\item[(b)] A pivot in the $(i+1)^{st}$ row of $E$ occurs in a column to
the right of the column where the pivot in the $i^{th}$ row occurs.
\end{enumerate}
\end{definition}

Here are three examples of matrices that are in echelon form.
The pivot in each row (which is always equal to $1$) is preceded
by a $*$.
\[
\left(\begin{array}{rrrrrrr}
    *1   &  0   &  -1  &  0  &  -6  &   4  &  -6\\
     0   & *1   &  4   &  0  &   0  &  -2  &   0\\
     0   &  0   &  0   & *1  &  -5  &   5  &  -2\\
     0   &  0   &  0   &  0  &   0  &  *1  &   0
\end{array} \right)
\]
\[
\left(\begin{array}{rrrrr}
    *1 &   0  &  -1  &   0  &  -6 \\
     0 &  *1  &   0  &   3  &   0  \\
     0 &   0  &   0  &  *1  &  -5   \\
     0 &   0  &   0  &   0  &   0
\end{array} \right)
\]
\[
\left(\begin{array}{rrrrr}
     0  &  *1  &  -1  &  14  &  -6 \\
     0  &   0  &   0  &  *1  &  15  \\
     0  &   0  &   0  &   0  &   0   \\
     0  &   0  &   0  &   0  &   0
\end{array} \right)
\]
Here are three examples of matrices that are {\em not\/} in echelon form.
\[
\left(\begin{array}{rrrr}
     0  &   0  &   1  &  15  \\
     1  &  -1  &  14  &  -6 \\
     0  &   0  &   0  &   0   \\
\end{array} \right)
\AND
\left(\begin{array}{rrrr}
     1  &  -1  &  14  &  -6 \\
     0  &   0  &   3  &  15  \\
     0  &   0  &   0  &   0   \\
\end{array} \right)
\AND
\left(\begin{array}{rrrr}
     1  &  -1  &  14  &  -6 \\
     0  &   0  &   0  &   0   \\
     0  &   0  &   1  &  15  \\
\end{array} \right)
\]


\begin{definition} \label{D:roweq}\index{row!equivalent}
Two $m\times n$ matrices are {\em row equivalent\/} if one can be
transformed to the other by a sequence of elementary row
operations.
\end{definition}
\index{row!equivalent}

Let $A=(a_{ij})$ be a matrix with $m$ rows and $n$ columns.  We
want to show that we can perform row operations on $A$ so that
the transformed matrix is in echelon form; that is, $A$ is row
equivalent to a matrix in echelon form.  If $A=0$, then we
are finished.  So we assume that some entry in $A$ is nonzero
and that the $1^{st}$ column where that nonzero entry occurs is in
the $k^{th}$ column.  By swapping rows we can assume that
$a_{1k}$ is nonzero.  Next, divide the $1^{st}$ row by $a_{1k}$,
thus setting $a_{1k}=1$.  Now, using
\Matlab notation, perform the row operations
\begin{verbatim}
A(i,:) = A(i,:) - A(i,k)*A(1,:)
\end{verbatim}
for each $i\geq 2$.  This sequence of row operations leads to a
matrix whose first nonzero column has a $1$ in the $1^{st}$ row
and a zero in each row below the $1^{st}$ row.

Now we look for the next column that has a nonzero entry below
the $1^{st}$ row and call that column $\ell$.  By construction
$\ell>k$.  We can swap rows so that the entry in the $2^{nd}$
row, $\ell^{th}$ column is nonzero.  Then we divide the $2^{nd}$
row by this nonzero element, so that the pivot in the $2^{nd}$
row is $1$.  Again we perform elementary row operations so that
all entries below the $2^{nd}$ row in the $\ell^{th}$ column are
set to $0$.  Now proceed inductively until we run out of nonzero
rows.

This argument proves:
\begin{proposition}  \label{P:echform}
Every matrix is row equivalent to a matrix in echelon form.
\end{proposition}

More importantly, the previous argument provides an algorithm for
transforming matrices into echelon form.

\subsection*{Reduction to Reduced Echelon Form}

\begin{definition}\index{echelon form!reduced}
A matrix $E$ is in {\em reduced echelon form\/} if
\begin{enumerate}
\item[(a)]        $E$ is in echelon form, and
\item[(b)] in every column of $E$ having a pivot, every entry
in that column other than the pivot is $0$.
\end{enumerate}
\end{definition}

We can now prove
\begin{theorem}  \label{T:redechform}
Every matrix is row equivalent \index{row!equivalent} to a matrix
in reduced echelon form.
\end{theorem}

\begin{proof} Let $A$ be a matrix.  Proposition~\ref{P:echform} states
that we can transform $A$ by elementary row operations to a matrix
$E$ in echelon form.  Next we transform $E$ into reduced echelon
form by some additional elementary row operations, as follows.
Choose the pivot in the last nonzero row of $E$.  Call that row
$\ell$, and let $k$ be the column where the pivot occurs.  By
adding multiples of the $\ell^{th}$ row to the rows above, we can
transform each entry in the $k^{th}$ column above the pivot to $0$.
Note that none of these row operations alters the matrix before
the $k^{th}$ column.  (Also note that this process is identical to
the process of back substitution.)

Again we proceed inductively by choosing the pivot in the
$(\ell-1)^{st}$ row, which is $1$, and zeroing out all entries
above that pivot using elementary row operations.  \end{proof}

\subsection*{Reduced Echelon Form in \Matlab}

Preprogrammed into \Matlab is a routine to row reduce any
matrix to reduced echelon form.  The command is
{\tt rref}\index{\computer!rref}.
For example, recall the $4\times 7$ matrix $A$ in \Ref{examp5}
by typing {\tt e2\_3\_12}.  Put $A$ into reduced row echelon
form by typing {\tt rref(A)} and obtaining
\begin{verbatim}
ans =
     1     0     0     0    -6     4    -6
     0     1     0     0    10   -20     0
     0     0     1     0    -5     5    -2
     0     0     0     1     0     2     1
\end{verbatim}
Compare the result with the system of equations \Ref{e:refexamp5}.

\subsection*{Solutions to Systems of Linear Equations}

Originally, we introduced elementary row
operations as operations that do not change solutions to the
linear system.  More precisely, we discussed how solutions to
the original system are still solutions to the transformed
system and how no new solutions are introduced by elementary
row operations.  This argument is most easily seen by observing
that
\begin{center}
all elementary row operations are invertible
\end{center}
--- they can be undone.

For example, swapping two rows is undone by just swapping these
rows again.  Similarly, multiplying a row by a nonzero number
$c$ is undone by just dividing that same row by $c$.  Finally,
adding $c$ times the $j^{th}$ row to the $i^{th}$ row is undone
by subtracting $c$ times the $j^{th}$ row from the $i^{th}$ row.

Thus, we can make several observations about solutions to linear
systems.  Let $E$ be an augmented matrix corresponding to a system
of linear equations having $n$ variables.  Since an augmented
matrix is formed from the matrix of coefficients by adding a
column, we see that the augmented matrix has $n+1$ columns.

\begin{theorem} \label{number}
Suppose that $E$ is an $m\times(n+1)$ augmented matrix that is in 
reduced echelon form.  Let $\ell$ be the number of nonzero rows in $E$
\begin{enumerate}
\item[(a)] The system of linear equations corresponding to $E$
is inconsistent if and only if the $\ell^{th}$ row in $E$ has a
pivot in the $(n+1)^{st}$ column.
\item[(b)] If the linear system corresponding to $E$ is consistent,
then the set of all solutions is parameterized by $n-\ell$
parameters.
\end{enumerate} \index{inconsistent} \index{consistent}
\end{theorem}

\begin{proof}  Suppose that the last nonzero row in $E$ has its
pivot in the $(n+1)^{st}$ column. Then the corresponding
equation is:
\[
0x_1 + 0x_2 + \cdots + 0x_n = 1,
\]
which has no solutions.  Thus the system is inconsistent.

Conversely, suppose that the last nonzero row has its pivot
before the last column.  Without loss of generality, we can
renumber the columns --- that is, we can renumber the variables
$x_j$ --- so that the pivot in the $i^{th}$ row occurs in the
$i^{th}$ column, where $1\leq i\leq\ell$.  Then the associated
system of linear equations has the form:
\begin{eqnarray*}
x_1 + a_{1,\ell+1}x_{\ell+1} + \cdots + a_{1,n}x_n & = &  b_1 \\
x_2 + a_{2,\ell+1}x_{\ell+1} + \cdots + a_{2,n}x_n & = &  b_2 \\
\vdots &   &    \vdots  \\
x_\ell + a_{\ell,\ell+1}x_{\ell+1} + \cdots + a_{\ell,n}x_n & = & b_\ell.
\end{eqnarray*}
This system can be rewritten in the form:
\begin{eqnarray}
  x_1  & = &  b_1 - a_{1,\ell+1}x_{\ell+1} - \cdots - a_{1,n}x_n
\nonumber\\
  x_2  & = &  b_2 - a_{2,\ell+1}x_{\ell+1} - \cdots - a_{2,n}x_n
\label{e1-ell}\\
\vdots &   &    \vdots \nonumber \\
x_\ell & = &  b_\ell - a_{\ell,\ell+1}x_{\ell+1} - \cdots -
a_{\ell,n}x_n.
\nonumber
\end{eqnarray}

Thus, each choice of the $n-\ell$ numbers
$x_{\ell+1},\ldots,x_n$ uniquely determines values of
$x_1,\ldots,x_\ell$ so that $x_1,\ldots,x_n$ is a solution to
this system.  In particular, the system is consistent, so (a) is
proved; and the set of all solutions is parameterized by
$n-\ell$ numbers, so (b) is proved.  \end{proof}

\subsubsection*{Two Examples Illustrating Theorem~\ref{number}}

The reduced echelon form matrix
\[
E = \left(\begin{array}{ccc|c} 1 & 5 & 0 & 0 \\ 0 & 0 & 1 & 0\\
	0 & 0 & 0 & 1 \end{array}\right)
\]
is the augmented matrix of an inconsistent system of three equations
in three unknowns.

The reduced echelon form matrix
\[
E = \left(\begin{array}{ccc|c} 1 & 5 & 0 & 2 \\ 0 & 0 & 1 & 5\\
	0 & 0 & 0 & 0 \end{array}\right)
\]
is the augmented matrix of a consistent system of three equations
in three unknowns $x_1,x_2,x_3$.  For this matrix $n=3$ and $\ell=2$.
It follows from Theorem~\ref{number} that the solutions to this system
are specified by one parameter.  Indeed, the solutions are
\begin{eqnarray*}
x_1 & = & 2 - 5x_2\\
x_3 & = & 5
\end{eqnarray*}
and are specified by the one parameter $x_2$.


\subsubsection*{Consequences of Theorem~\ref{number}}

It follows from Theorem~\ref{number} that linear systems of equations
with fewer equations than unknowns and with zeros on the right hand
side always have nonzero solutions.  More precisely:
\begin{corollary}  \label{existencehomo}
Let $A$ be an $m\times n$ matrix where $m<n$.  Then the system of
linear equations whose augmented matrix is $(A|0)$ has a nonzero
solution.
\end{corollary}

\begin{proof} Perform elementary row operations on the augmented matrix $(A|0)$
to arrive at the reduced echelon form matrix $(E|0)$.  Since the zero
vector is a solution, the associated system of equations is consistent.
Now the number of nonzero rows $\ell$ in $(E|0)$ is less than or equal to
the number of rows $m$ in $E$.  By assumption $m<n$ and hence $\ell<n$.
It follows from Theorem~\ref{number} that solutions to the linear system
are parametrized by $n-\ell \ge 1$ parameters and that there are nonzero
solutions.   \end{proof}

Recall that two $m\times n$ matrices are row equivalent if one
can be transformed to the other by elementary row operations.

\begin{corollary}  \label{consistent}
Let $A$ be an $n\times n$ square matrix and let $b$ be in
$\R^n$.  Then $A$ is row equivalent to the identity matrix $I_n$
if and only if the system of linear equations whose augmented
matrix is $(A|b)$ has a unique solution.
\end{corollary} \index{matrix!identity}  \index{matrix!augmented}

\begin{proof}	Suppose that $A$ is row equivalent to $I_n$.  Then, by using
the same sequence of elementary row operations, it follows that
the $n\times (n+1)$ augmented matrix $(A|b)$ is row equivalent
to $(I_n|c)$ for some vector $c\in\R^n$.  The system of linear
equations that corresponds to $(I_n|c)$ is:
\[
\begin{array}{ccc}
x_1 & = & c_1 \\
\vdots & \vdots & \vdots \\
x_n & = & c_n,
\end{array}
\]
which transparently has the unique solution $x=(c_1,\ldots,c_n)$.
Since elementary row operations do not change the solutions of
the equations, the original augmented system $(A|b)$ also has a
unique solution.

Conversely, suppose that the system of linear equations
associated to $(A|b)$ has a unique solution.  Suppose that
$(A|b)$ is row equivalent to a reduced echelon form matrix $E$.
Suppose that the last nonzero row in $E$ is the $\ell^{th}$ row.
Since the system has a solution, it is consistent.  Hence
Theorem~\ref{number}(b) implies that the solutions to the system
corresponding to $E$ are parameterized by $n-\ell$ parameters.
If $\ell<n$, then the solution is not unique.  So $\ell=n$.

Next observe that since the system of linear equations is
consistent, it follows from Theorem~\ref{number}(a) that the
pivot in the $n^{th}$ row must occur in a column before the
$(n+1)^{st}$.  It follows that the reduced echelon matrix
$E=(I_n|c)$ for some $c\in\R^n$.  Since $(A|b)$ is row
equivalent to $(I_n|c)$, it follows, by using the same sequence
of elementary row operations, that $A$ is row equivalent to
$I_n$.  \end{proof}

\subsection*{Uniqueness of Reduced Echelon Form and Rank}

Abstractly, our discussion of reduced echelon form has one point
remaining to be proved.  We know that every matrix $A$ can be
transformed by elementary row operations to reduced echelon
form.  Suppose, however, that we use two different sequences of
elementary row operations to transform $A$ to two reduced
echelon form matrices $E_1$ and $E_2$.  Can $E_1$ and $E_2$ be
different?  The answer is: No.

\begin{theorem} \label{uniquerowechelon}
For each matrix $A$, there is precisely one reduced echelon form
matrix $E$ that is row equivalent\index{row!equivalent}  to $A$.
\end{theorem}\index{echelon form!reduced!uniqueness}

The proof of Theorem~\ref{uniquerowechelon} is given in 
Section~\ref{S:uniquerowechelon}.  Since every matrix is row equivalent 
to a unique matrix in reduced echelon form, we can define the rank 
of a matrix as follows.
\begin{definition}  \label{D:rank}
Let $A$ be an $m\times n$ matrix that is row equivalent to a
reduced echelon form matrix $E$.  Then the {\em rank\/} of $A$,
denoted $\rank(A)$, is the number of nonzero rows in $E$.
\end{definition}  \index{rank}

We make three remarks concerning the rank of a matrix.
\begin{itemize}
\item An echelon form matrix is always row equivalent to a
reduced echelon form matrix with the same number of nonzero
rows.  Thus, to compute the rank of a matrix, we need only
perform elementary row operations until the matrix is in echelon
form.
\item	The rank of any matrix is easily computed in \Matlabp.
Enter a matrix $A$ and type {\tt rank(A)}\index{\computer!rank}.
\item The number $\ell$ in the statement of Theorem~\ref{number}
is just the rank of $E$.
\end{itemize}
In particular, if the rank of the augmented matrix corresponding to
a consistent system of linear equations in $n$ unknowns has rank $\ell$,
then the solutions to this system are parametrized by $n-\ell$ parameters.


\EXER

\TEXER

\noindent In Exercises~\ref{c2.4.1} -- \ref{c2.4.1b} row reduce the given 
matrix to reduced echelon form and determine the rank of $A$.
\begin{exercise} \label{c2.4.1}
$A=\left(\begin{array}{rrrr}
1 &  2 & 1 & 6\\
3 &  6 & 1 & 14\\
1 &  2 & 2 & 8
         \end{array}\right)$
       \begin{prompt}
       The reduced echelon form of the matrix is:
\[
A = \left(\begin{array}{rrrr} 1 & \answer{2} & \answer{0} & \answer{4} \\ 0 & \answer{0} & \answer{1} & \answer{2} \\ 0 & 0 & \answer{0}
& 0\end{array}\right)
\]
The rank of $A$ is $\answer{2}$, since the reduced echelon matrix has $\answer{2}$ nonzero
rows.         
       \end{prompt}
\end{exercise}
\begin{exercise} \label{c2.4.1b}
$B=\left(\begin{array}{rrr}
1 &  -2 & 3\\
3 &  -6 & 9 \\
1 &  -8 & 2
         \end{array}\right)$
       \begin{prompt}
       The reduced echelon form of the matrix is:
\[
B=\left(\begin{array}{rrr} 1 & 0 & \answer{\frac{10}{3}} \\ 
0 & 1 & \answer{\frac{1}{6}} \\ 0 & 0 & \answer{0} \end{array}\right)
\]
The rank of $B$ is $\answer{2}$, since the reduced echelon matrix has $\answer{2}$ nonzero
rows.
       \end{prompt}
\end{exercise}

\begin{exercise} \label{c2.4.2}
The augmented matrix of a consistent system of five equations in seven
unknowns has rank equal to three.  How many parameters are needed to
specify all solutions?
\begin{prompt}
  There are $\answer{4}$ parameters needed to specify all solutions.
\end{prompt}
\begin{hint}
  According to Theorem \ref{number}, $n - \ell$
parameters are needed to parameterize the set of all solutions of a
linear system, where $n$ is the number of unknowns, and $\ell$ is the
rank of the reduced echelon matrix.  In this case, $n = 7$ and $\ell =
3$.
\end{hint}
\end{exercise}

\begin{exercise} \label{c2.4.2b}
The augmented matrix of a consistent system of nine equations in twelve
unknowns has rank equal to five.  How many parameters are needed to
specify all solutions?
\begin{prompt}
  There are $\answer{7}$ parameters needed to specify all solutions.
\end{prompt}
\begin{hint}
  According to Theorem \ref{number}, $n - \ell$
parameters are needed to parameterize the set of all solutions of a
linear system, where $n$ is the number of unknowns, and $\ell$ is the
rank of the reduced echelon matrix.  In this case, $n = 12$ and $\ell =
5$.
\end{hint}
\end{exercise}

\CEXER

\noindent In Exercises~\ref{c2.4.3a} -- \ref{c2.4.3d}, use {\tt rref} on
the given augmented matrices to determine whether the associated system of
linear equations is consistent or inconsistent.  If the equations are
consistent, then determine how many parameters are needed to enumerate
all solutions.
\begin{exercise} \label{c2.4.3a}
\begin{equation*}
A = \left(\begin{array}{rrrrr|r}
2 & 1 & 3 & -2 & 4 & 1 \\
5 & 12 & -1 & 3 & 5 & 1 \\
-4  &  -21 &    11  &  -12  &    2  &    1  \\
23  &  59  &  -8   & 17  &  21  &   4
\end{array}\right) \quad
\end{equation*}
\end{exercise}
\begin{exercise} \label{c2.4.3b}
\begin{equation*}
B = \left(\begin{array}{rrrr|r}
     2   &  4  & 6  &  -2   &  1 \\
     0   &  0   &  4  &   1  &  -1\\
     2   &  4   &  0   &  1  &   2
\end{array}\right)\end{equation*}
\end{exercise}
\begin{exercise} \label{c2.4.3c}
\begin{equation*}
C = \left(\begin{array}{rrr|r}
     2  &   3  &  -1  &   4 \\
     8  &  11  &  -7  &   8\\
     2  &   2  &  -4  &  -3
\end{array}\right) \quad
\end{equation*}
\end{exercise}
\begin{exercise} \label{c2.4.3d}
\begin{equation*}
D = \left(\begin{array}{rrrr|r}
    2.3 &  4.66  & -1.2   & 2.11  & -2 \\
         0  &   0  &  1.33  &   0  &  1.44\\
    4.6  &  9.32  & -7.986   & 4.22  & -10.048\\
    1.84  &  3.728 & -5.216   & 1.688 & -6.208
\end{array}\right)
\end{equation*}
\end{exercise}

\noindent In Exercises~\ref{c2.4.4a} -- \ref{c2.4.4c} compute the rank of
the given matrix.
\begin{exercise} \label{c2.4.4a}
$\mattwo{1}{-2}{-3}{6}$.
\end{exercise}
\begin{exercise} \label{c2.4.4b}
$\left(\begin{array}{rrrr} 2 & 1 & 0 & 1\\
	-1 & 3 & 2 & 4\\ 5 & -1 & 2 & -2\end{array}\right)$.
\end{exercise}
\begin{exercise} \label{c2.4.4c}
$\left(\begin{array}{rrr} 3 & 1 & 0 \\
	-1 & 2 & 4\\ 2 & 3 & 4 \\ 4 & -1 & -4 \end{array}\right)$.
\end{exercise}




\end{document}
