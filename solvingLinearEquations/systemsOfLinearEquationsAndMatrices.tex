\documentclass{ximera}

 

\usepackage{epsfig}

\graphicspath{
  {./}
  {figures/}
}

\usepackage{morewrites}
\makeatletter
\newcommand\subfile[1]{%
\renewcommand{\input}[1]{}%
\begingroup\skip@preamble\otherinput{#1}\endgroup\par\vspace{\topsep}
\let\input\otherinput}
\makeatother

\newcommand{\includeexercises}{\directlua{dofile("/home/jim/linearAlgebra/laode/exercises.lua")}}

%\newcounter{ccounter}
%\setcounter{ccounter}{1}
%\newcommand{\Chapter}[1]{\setcounter{chapter}{\arabic{ccounter}}\chapter{#1}\addtocounter{ccounter}{1}}

%\newcommand{\section}[1]{\section{#1}\setcounter{thm}{0}\setcounter{equation}{0}}

%\renewcommand{\theequation}{\arabic{chapter}.\arabic{section}.\arabic{equation}}
%\renewcommand{\thefigure}{\arabic{chapter}.\arabic{figure}}
%\renewcommand{\thetable}{\arabic{chapter}.\arabic{table}}

%\newcommand{\Sec}[2]{\section{#1}\markright{\arabic{ccounter}.\arabic{section}.#2}\setcounter{equation}{0}\setcounter{thm}{0}\setcounter{figure}{0}}

\newcommand{\Sec}[2]{\section{#1}}

\setcounter{secnumdepth}{2}
%\setcounter{secnumdepth}{1} 

%\newcounter{THM}
%\renewcommand{\theTHM}{\arabic{chapter}.\arabic{section}}

\newcommand{\trademark}{{R\!\!\!\!\!\bigcirc}}
%\newtheorem{exercise}{}

\newcommand{\dfield}{{\sf dfield9}}
\newcommand{\pplane}{{\sf pplane9}}

\newcommand{\EXER}{\section*{Exercises}}%\vspace*{0.2in}\hrule\small\setcounter{exercise}{0}}
\newcommand{\CEXER}{}%\vspace{0.08in}\begin{center}Computer Exercises\end{center}}
\newcommand{\TEXER}{} %\vspace{0.08in}\begin{center}Hand Exercises\end{center}}
\newcommand{\AEXER}{} %\vspace{0.08in}\begin{center}Hand Exercises\end{center}}

% BADBAD: \newcommand{\Bbb}{\bf}

\newcommand{\R}{\mbox{$\Bbb{R}$}}
\newcommand{\C}{\mbox{$\Bbb{C}$}}
\newcommand{\Z}{\mbox{$\Bbb{Z}$}}
\newcommand{\N}{\mbox{$\Bbb{N}$}}
\newcommand{\D}{\mbox{{\bf D}}}
\usepackage{amssymb}
%\newcommand{\qed}{\hfill\mbox{\raggedright$\square$} \vspace{1ex}}
%\newcommand{\proof}{\noindent {\bf Proof:} \hspace{0.1in}}

\newcommand{\setmin}{\;\mbox{--}\;}
\newcommand{\Matlab}{{M\small{AT\-LAB}} }
\newcommand{\Matlabp}{{M\small{AT\-LAB}}}
\newcommand{\computer}{\Matlab Instructions}
\newcommand{\half}{\mbox{$\frac{1}{2}$}}
\newcommand{\compose}{\raisebox{.15ex}{\mbox{{\scriptsize$\circ$}}}}
\newcommand{\AND}{\quad\mbox{and}\quad}
\newcommand{\vect}[2]{\left(\begin{array}{c} #1_1 \\ \vdots \\
 #1_{#2}\end{array}\right)}
\newcommand{\mattwo}[4]{\left(\begin{array}{rr} #1 & #2\\ #3
&#4\end{array}\right)}
\newcommand{\mattwoc}[4]{\left(\begin{array}{cc} #1 & #2\\ #3
&#4\end{array}\right)}
\newcommand{\vectwo}[2]{\left(\begin{array}{r} #1 \\ #2\end{array}\right)}
\newcommand{\vectwoc}[2]{\left(\begin{array}{c} #1 \\ #2\end{array}\right)}

\newcommand{\ignore}[1]{}


\newcommand{\inv}{^{-1}}
\newcommand{\CC}{{\cal C}}
\newcommand{\CCone}{\CC^1}
\newcommand{\Span}{{\rm span}}
\newcommand{\rank}{{\rm rank}}
\newcommand{\trace}{{\rm tr}}
\newcommand{\RE}{{\rm Re}}
\newcommand{\IM}{{\rm Im}}
\newcommand{\nulls}{{\rm null\;space}}

\newcommand{\dps}{\displaystyle}
\newcommand{\arraystart}{\renewcommand{\arraystretch}{1.8}}
\newcommand{\arrayfinish}{\renewcommand{\arraystretch}{1.2}}
\newcommand{\Start}[1]{\vspace{0.08in}\noindent {\bf Section~\ref{#1}}}
\newcommand{\exer}[1]{\noindent {\bf \ref{#1}}}
\newcommand{\ans}{}
\newcommand{\matthree}[9]{\left(\begin{array}{rrr} #1 & #2 & #3 \\ #4 & #5 & #6
\\ #7 & #8 & #9\end{array}\right)}
\newcommand{\cvectwo}[2]{\left(\begin{array}{c} #1 \\ #2\end{array}\right)}
\newcommand{\cmatthree}[9]{\left(\begin{array}{ccc} #1 & #2 & #3 \\ #4 & #5 &
#6 \\ #7 & #8 & #9\end{array}\right)}
\newcommand{\vecthree}[3]{\left(\begin{array}{r} #1 \\ #2 \\
#3\end{array}\right)}
\newcommand{\cvecthree}[3]{\left(\begin{array}{c} #1 \\ #2 \\
#3\end{array}\right)}
\newcommand{\cmattwo}[4]{\left(\begin{array}{cc} #1 & #2\\ #3
&#4\end{array}\right)}

\newcommand{\Matrix}[1]{\ensuremath{\left(\begin{array}{rrrrrrrrrrrrrrrrrr} #1 \end{array}\right)}}

\newcommand{\Matrixc}[1]{\ensuremath{\left(\begin{array}{cccccccccccc} #1 \end{array}\right)}}



\renewcommand{\labelenumi}{\theenumi)}
\newenvironment{enumeratea}%
{\begingroup
 \renewcommand{\theenumi}{\alph{enumi}}
 \renewcommand{\labelenumi}{(\theenumi)}
 \begin{enumerate}}
 {\end{enumerate}\endgroup}



\newcounter{help}
\renewcommand{\thehelp}{\thesection.\arabic{equation}}

%\newenvironment{equation*}%
%{\renewcommand\endequation{\eqno (\theequation)* $$}%
%   \begin{equation}}%
%   {\end{equation}\renewcommand\endequation{\eqno \@eqnnum
%$$\global\@ignoretrue}}

%\input{psfig.tex}

\author{Martin Golubitsky and Michael Dellnitz}

%\newenvironment{matlabEquation}%
%{\renewcommand\endequation{\eqno (\theequation*) $$}%
%   \begin{equation}}%
%   {\end{equation}\renewcommand\endequation{\eqno \@eqnnum
% $$\global\@ignoretrue}}

\newcommand{\soln}{\textbf{Solution:} }
\newcommand{\exercap}[1]{\centerline{Figure~\ref{#1}}}
\newcommand{\exercaptwo}[1]{\centerline{Figure~\ref{#1}a\hspace{2.1in}
Figure~\ref{#1}b}}
\newcommand{\exercapthree}[1]{\centerline{Figure~\ref{#1}a\hspace{1.2in}
Figure~\ref{#1}b\hspace{1.2in}Figure~\ref{#1}c}}
\newcommand{\para}{\hspace{0.4in}}

\renewenvironment{solution}{\suppress}{\endsuppress}

\ifxake
\newenvironment{matlabEquation}{\begin{equation}}{\end{equation}}
\else
\newenvironment{matlabEquation}%
{\let\oldtheequation\theequation\renewcommand{\theequation}{\oldtheequation*}\begin{equation}}%
  {\end{equation}\let\theequation\oldtheequation}
\fi

\makeatother


\title{Systems of Linear Equations and Matrices}

\begin{document}
\begin{abstract}
\end{abstract}
\maketitle


\label{S:2.1}

It is a simple exercise to solve the system of two equations
\begin{equation} \label{small}
\arraycolsep 2pt
\begin{array}{rcrcr}
 x & + & y & = & 7 \\
-x & + & 3y & = & 1
\end{array}
\end{equation}
to find that $x=5$ and $y=2$.  One way to solve
system \eqref{small} is to add the two equations, obtaining
\[
4y=8;
\]
hence $y=2$.  Substituting $y=2$ into the $1^{st}$ equation in
\eqref{small} yields $x=5$.

This system of equations can be solved in a more algorithmic
fashion by solving the $1^{st}$ equation in \eqref{small} for $x$
as
\[
x = 7 - y,
\]
and substituting this answer into the $2^{nd}$ equation in
\eqref{small}, to obtain
\[
-(7-y) +3y = 1.
\]
This equation simplifies to:
\[
4y = 8.
\]
Now proceed as before.

\subsection*{Solving Larger Systems by Substitution}

In contrast to solving the simple system of two equations,
it is less clear how to solve a complicated system of five
equations such as:
\begin{equation}    \label{big}
\arraycolsep 2pt
\begin{array}{rcrcrcrcrcrl}
 5x_1 & - & 4x_2 & + & 3x_3 & - & 6x_4 & + & 2x_5 & = &   4  & \\
 2x_1 & + &  x_2 & - &  x_3 & - &  x_4 & + &  x_5 & = &   6  & \\
  x_1 & + & 2x_2 & + &  x_3 & + &  x_4 & + & 3x_5 & = &  19  & \\
-2x_1 & - &  x_2 & - &  x_3 & + &  x_4 & - &  x_5 & = & -12  & \\
  x_1 & - & 6x_2 & + &  x_3 & + &  x_4 & + & 4x_5 & = &   4  & \!
.
\end{array}
\end{equation}
The algorithmic method used to solve \eqref{small} can be expanded
to produce a method, called {\em substitution\/},
\index{substitution} for solving larger systems. We describe the
substitution method as it applies to \eqref{big}.  Solve the
$1^{st}$ equation in \eqref{big} for $x_1$, obtaining
\begin{equation} \label{x1}
x_1 = \frac{4}{5}  + \frac{4}{5}x_2 - \frac{3}{5}x_3
   + \frac{6}{5}x_4 - \frac{2}{5}x_5.
\end{equation}
Then substitute the right hand side of \eqref{x1} for $x_1$ in the
remaining four equations in \eqref{big} to obtain a new system of
four equations in the four variables $x_2$,$x_3$,$x_4$,$x_5$.
This procedure eliminates the variable $x_1$.  Now proceed
inductively --- solve the $1^{st}$ equation in the new system
for $x_2$ and substitute this expression into the remaining
three equations to obtain a system of three equations in three
unknowns.  This step eliminates the variable $x_2$.  Continue by
substitution to eliminate the variables $x_3$ and $x_4$, and
arrive at a simple equation in $x_5$ --- which can be solved.
Once $x_5$ is known, then $x_4$, $x_3$, $x_2$, and $x_1$ can be
found in turn.

\subsection*{Two Questions}

\begin{itemize}
\item Is it realistic to expect to complete the substitution
procedure without making a mistake in arithmetic?
\item Will this procedure work --- or will some unforeseen
difficulty arise?
\end{itemize}

Almost surely, attempts to solve \eqref{big} by hand,
using the substitution procedure, will lead to arithmetic
errors.  However, computers and software have developed to the
point where solving a system such as \eqref{big} is routine.  In
this text, we use the software package \Matlab to illustrate
just how easy it has become to solve equations such as \eqref{big}.

The answer to the second question requires knowledge of the
{\em theory\/} of linear algebra.  In fact, no difficulties will
develop when trying to solve the particular system \eqref{big}
using the substitution algorithm.  We discuss why later.

\subsection*{Solving Equations by \Matlab}

We begin by discussing the information that is needed by \Matlab
to solve \eqref{big}.  The computer needs to know that there are
five equations in five unknowns --- but it does not need to keep
track of the unknowns $(x_1,x_2,x_3,x_4,x_5)$ by name.  Indeed,
the computer just needs to know the {\em matrix of
coefficients\/} \index{matrix!coefficient} in \eqref{big}
\begin{matlabEquation}  \label{bigmatrix}
\left(
\begin{array}{rrrrr}
 5 & -4 &  3 & -6 &  2 \\
 2 &  1 & -1 & -1 &  1 \\
 1 &  2 &  1 &  1 &  3 \\
-2 & -1 & -1 &  1 & -1 \\
 1 & -6 &  1 &  1 &  4
\end{array}
\right)
\end{matlabEquation}
and the {\em vector\/} on the right hand side of \eqref{big}
\begin{matlabEquation} \label{bigRHS}
\left(
\begin{array}{r}
  4 \\
  6 \\
 19 \\
-12 \\
  4
\end{array}
\right).
\end{matlabEquation}

We now describe how we enter this information into \Matlabp.  To
reduce the drudgery and to allow us to focus on ideas, the entries
in equations having a $*$ after their label,
such as \eqref{bigmatrix}, have been entered in the {\tt laode}
toolbox. This information can be accessed as follows.  After
starting your \Matlab session, type
\begin{verbatim}
e2_1_4
\end{verbatim}
followed by a carriage return.  This instruction tells \Matlab to
load equation \eqref{bigmatrix} of Chapter~\ref{lineq}.  The matrix of
coefficients is now available in \Matlabp; note that this matrix is
stored in the $5\times 5$ array {\tt A}.  What should appear is:
\begin{verbatim}
A =
     5    -4     3    -6     2
     2     1    -1    -1     1
     1     2     1     1     3
    -2    -1    -1     1    -1
     1    -6     1     1     4
\end{verbatim}
Indeed, comparing this result with \eqref{bigmatrix}, we see that
{\tt A} contains precisely the same information.

Since the label \eqref{bigRHS} is followed by a `$*$', we can enter
the vector in \eqref{bigRHS} into \Matlab by typing
\begin{verbatim}
e2_1_5
\end{verbatim}
Note that the right hand side of \eqref{big} is stored in the vector {\tt b}.
\Matlab should have responded with
\begin{verbatim}
b =
     4
     6
    19
   -12
     4
\end{verbatim}
Now \Matlab has all the information it needs to solve the system
of equations given in \eqref{big}.  To have \Matlab solve this
system, type
\begin{verbatim}
x = A\b
\end{verbatim}
\index{\computer!$\backslash$}to obtain
\begin{verbatim}
x =
    5.0000
    2.0000
    3.0000
    4.0000
    1.0000
\end{verbatim}
This answer is interpreted as follows: the five values of the
unknowns $x_1$,$x_2$,$x_3$,$x_4$,$x_5$ are stored in the vector
$x$; that is,
\begin{equation} \label{answer1}
 x_1 = 5,\quad x_2 = 2,\quad x_3 = 3,\quad x_4 = 4,\quad x_5 = 1.
\end{equation}
The reader may verify that \eqref{answer1} is indeed a solution of
\eqref{big} by substituting the values in \eqref{answer1} into the
equations in \eqref{big}.

\subsection*{Changing Entries in \Matlab}

\Matlab also permits access to single components of $x$.  For
instance, type
\begin{verbatim}
x(5)
\end{verbatim}
and the $5^{th}$ entry of $x$ is displayed,
\begin{verbatim}
ans =
    1.0000
\end{verbatim}
We see that the component {\tt x(i)} of {\tt x} corresponds to
the component $x_i$ of the vector $x$ where $i=1,2,3,4,5$.
Similarly, we can access the entries of the coefficient matrix
\index{matrix!coefficient} {\tt A}.
For instance, by typing
\begin{verbatim}
A(3,4)
\end{verbatim}
\Matlab responds with
\begin{verbatim}
ans =
    1
\end{verbatim}

It is also possible to change an individual entry in either a vector
or a matrix.  For example, if we enter
\begin{verbatim}
A(3,4) = -2
\end{verbatim}  \index{\computer!A(3,4)}
we obtain a new matrix {\tt A} which when displayed is:
\begin{verbatim}
A =
     5    -4     3    -6     2
     2     1    -1    -1     1
     1     2     1    -2     3
    -2    -1    -1     1    -1
     1    -6     1     1     4
\end{verbatim}
Thus the command {\tt A(3,4) = -2} changes the entry in the
$3^{rd}$ row, $4^{th}$ column of {\tt A} from $1$ to $-2$.
In other words, we have now entered into \Matlab the
information that is needed to solve the system of equations
\[
\arraycolsep 2pt
\begin{array}{rcrcrcrcrcrl}
 5x_1 & - & 4x_2 & + & 3x_3 & - &  6x_4 & + & 2x_5 & = &   4  & \\
 2x_1 & + &  x_2 & - &  x_3 & - &   x_4 & + &  x_5 & = &   6  & \\
  x_1 & + & 2x_2 & + &  x_3 & - &  2x_4 & + & 3x_5 & = &  19  & \\
-2x_1 & - &  x_2 & - &  x_3 & + &   x_4 & - &  x_5 & = & -12  & \\
x_1 & - & 6x_2 & + & x_3 & + & x_4 & + & 4x_5 & = & 4 & \! .
\end{array}
\]
As expected, this change in the coefficient matrix results in a
change in the solution of system \eqref{big}, as well.  Typing
\begin{verbatim}
x = A\b
\end{verbatim}
now leads to the solution
\begin{verbatim}
x =
    1.9455
    3.0036
    3.0000
    1.7309
    3.8364
\end{verbatim}
that is displayed to an accuracy of four decimal places.

In the next step, change {\tt A} as follows:
\begin{verbatim}
A(2,3) = 1
\end{verbatim}
The new system of equations is:
\begin{equation}  \label{incon}
\arraycolsep 2pt
\begin{array}{rcrcrcrcrcrl}
 5x_1 & - & 4x_2 & + & 3x_3 & - &  6x_4 & + & 2x_5 & = &   4  & \\
 2x_1 & + &  x_2 & + &  x_3 & - &   x_4 & + &  x_5 & = &   6  & \\
  x_1 & + & 2x_2 & + &  x_3 & - &  2x_4 & + & 3x_5 & = &  19  & \\
-2x_1 & - &  x_2 & - &  x_3 & + &   x_4 & - &  x_5 & = & -12  & \\
  x_1 & - & 6x_2 & + &  x_3 & + &   x_4 & + & 4x_5 & = &   4  & \!
.
\end{array}
\end{equation}
The command
\begin{verbatim}
x = A\b
\end{verbatim}  \index{\computer!$\backslash$}
now leads to the message
\begin{verbatim}
Warning: Matrix is singular to working precision.

x =
   Inf
   Inf
   Inf
   Inf
   Inf
\end{verbatim}  \index{\computer!inf}
Obviously, something is {\em wrong\/}; \Matlab cannot find a
solution to this system of equations!  Assuming that \Matlab is
working correctly, we have shed light on one of our previous
questions: the method of substitution described by \eqref{x1} need
{\em not\/} always lead to a solution, even though the method
does work for system \eqref{big}.  Why?  As we will see, this is
one of the questions that is answered by the theory of linear
algebra.  In the case of \eqref{incon}, it is fairly easy to see
what the difficulty is: the second and fourth equations
have the form $y=6$ and $-y=-12$, respectively.

\vspace{0.1in}

\noindent {\bf Warning:}  The \Matlab command
\begin{verbatim}
x = A\b
\end{verbatim}
may give an error message similar to the previous one.  When
this happens, one must approach the answer with caution.



\includeexercises



\end{document}

%%% Local Variables:
%%% mode: latex
%%% TeX-master: "../linearAlgebra"
%%% End:
