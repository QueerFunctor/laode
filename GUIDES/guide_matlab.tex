\documentclass{article}
\usepackage{hyperref}
\usepackage{fullpage}
\usepackage{enumerate}

\setlength{\parindent}{0in}
\setlength{\parskip}{1ex}


\title{Guide: MATLAB installation}
\author{Jim Fowler \and Marty Golubitsky}

\usepackage{xcolor,verbatimbox}
\catcode`>=\active %
\catcode`<=\active %
\def\openesc{\color{red}}
\def\closeesc{\color{black}}
\def\vbdelim{\catcode`<=\active\catcode`>=\active%
\def<{\openesc}
\def>{\closeesc}}
\catcode`>=12 %
\catcode`<=12 %

\begin{document}

\maketitle
\tableofcontents

\section{For Ohio State faculty and
students}\label{for-ohio-state-faculty-and-students}

Computers not only enable the completion of complicated calculations,
but they also improve the conceptual understanding of
mathematics. This course relies on MATLAB, a popular software package
for matrix computations. MATLAB is available free-of-charge to Ohio
State students and faculty.  To download and activate your free copy
of MATLAB, please follow the steps below.

\begin{enumerate}
\def\labelenumi{\arabic{enumi})}
\item
  Launch Ohio State's IT Self-Service. Links to an external site. tool
  and click Login at the upper right.
\item
  Click Order Services. It is marked with a shopping cart but MATLAB
  will not cost you anything.
\item
  Click the Software Request link located under the ``Software
  Services'' heading.
\item
  Complete the Software Request Form, choosing MATLAB. Agree to terms
  and conditions.
\item
  Click Check Out.
\item
  Click Submit Order at the lower right.
\item
  Your order confirmation page lists your selections and their costs, as
  well as provides a Request Number (i.e., REQ12345) for tracking
  purposes.
\end{enumerate}

The rest of the instructions for installing MATLAB will automatically be
emailed to your Ohio State email account. Instructions will vary as to
whether you are on Windows, Mac OS, or Linux.

\section{Getting the m\_files}\label{getting-the-m_files}

To produce m\_files with filenames coming from the numeric labels in the
textbook, run

\begin{verbatim}
cd ~/linear-algebra/laode/m_files
ruby numeric-labels.rb
\end{verbatim}

Then to see the m\_files, run

\begin{verbatim}
cd ~/linear-algebra/laode/m_files/linearAlgebra
ls
\end{verbatim}

\end{document}
