\documentclass{article}
\usepackage{hyperref}
\usepackage{fullpage}
\usepackage{enumerate}

\setlength{\parindent}{0in}
\setlength{\parskip}{1ex}


\title{Guide: Installing MATLAB and Course \texttt{.m}-files}
\author{Jim Fowler \and Marty Golubitsky}

\usepackage{xcolor,verbatimbox}
\catcode`>=\active %
\catcode`<=\active %
\def\openesc{\color{red}}
\def\closeesc{\color{black}}
\def\vbdelim{\catcode`<=\active\catcode`>=\active%
\def<{\openesc}
\def>{\closeesc}}
\catcode`>=12 %
\catcode`<=12 %

\begin{document}

\maketitle

\section{MATLAB}\label{for-ohio-state-faculty-and-students}

Computers not only enable the completion of complicated calculations,
but they also improve the conceptual understanding of
mathematics. This course relies on MATLAB, a popular software package
for matrix computations. MATLAB is available free-of-charge to Ohio
State students and faculty.

To download and activate your free copy of MATLAB, please follow the
steps below.  For support, faculty should email
\href{mailto:support@math.osu.edu}{\texttt{support@math.osu.edu}} and
students should email
\href{mailto:8help@osu.edu}{\texttt{8help@osu.edu}}.

\begin{enumerate}
\def\labelenumi{\arabic{enumi})}
\item Launch \href{https://osuitsm.service-now.com/selfservice/}{Ohio
    State's IT Self-Service} and click Login at the upper right.
\item Click Order Services. It is marked with a shopping cart but
  MATLAB will be free.
\item Click the Software Request link located under the ``Software
  Services'' heading.
\item Complete the Software Request Form, choosing MATLAB. Agree to
  terms and conditions.
\item Click Check Out.
\item Click Submit Order at the lower right.
\item Your order confirmation page lists your selections and their
  costs, as well as provides a Request Number (i.e., REQ12345) for
  tracking purposes.
\end{enumerate}

The remaining instructions for installing MATLAB will be emailed to
your Ohio State email account. Instructions are OS dependent.

\section{Course \texttt{.m}-files}

The course files~\verb|*.m| can be downloaded with the textbook.
They will need to be stored in a directory that MATLAB searches.
There are two options.

Start MATLAB.  You will be in the MATLAB home directory.

\begin{enumerate}[(a)]
\item One option is to store these files in the MATLAB home
  directory.  To do this, enter the following commands in the MATLAB
  command window.
  
\begin{verbatim}
cd(userpath)
websave('mfiles.zip','http://go.osu.edu/mfiles')
unzip('mfiles.zip')
movefile('m_files/*', userpath)
rmdir m_files
\end{verbatim}
  
\item Another option is to store these files in a directory called
  \verb|m_files| To achieve this, start MATLAB and enter the following
  commands.
\begin{verbatim}
cd(userpath)
websave('mfiles.zip','http://go.osu.edu/mfiles')
unzip('mfiles.zip')
\end{verbatim}
  To access the files in the future, navigate to the \verb|m_files|
  directory with
\begin{verbatim}
cd(userpath)
cd m_files
\end{verbatim}
\end{enumerate}

As with many large software systems, updating to the latest version
may break compatibility with older software.  After you have installed
MATLAB and loaded the course files, enter \verb|pplane9| into the
MATLAB command window to verify that \verb|pplane9| works.

\end{document}
